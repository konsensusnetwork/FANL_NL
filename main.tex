% Options for packages loaded elsewhere
\PassOptionsToPackage{unicode}{hyperref}
\PassOptionsToPackage{hyphens}{url}
%
\documentclass[
  a5paper,
  smalldemyvopaper,10pt,twoside,onecolumn,openright,extrafontsizes,hidelinks]{memoir}

\usepackage{amsmath,amssymb}
\usepackage{iftex}
\ifPDFTeX
  \usepackage[T1]{fontenc}
  \usepackage[utf8]{inputenc}
  \usepackage{textcomp} % provide euro and other symbols
\else % if luatex or xetex
  \usepackage{unicode-math}
  \defaultfontfeatures{Scale=MatchLowercase}
  \defaultfontfeatures[\rmfamily]{Ligatures=TeX,Scale=1}
\fi
\usepackage{lmodern}
\ifPDFTeX\else  
    % xetex/luatex font selection
\fi
% Use upquote if available, for straight quotes in verbatim environments
\IfFileExists{upquote.sty}{\usepackage{upquote}}{}
\IfFileExists{microtype.sty}{% use microtype if available
  \usepackage[]{microtype}
  \UseMicrotypeSet[protrusion]{basicmath} % disable protrusion for tt fonts
}{}
\makeatletter
\@ifundefined{KOMAClassName}{% if non-KOMA class
  \IfFileExists{parskip.sty}{%
    \usepackage{parskip}
  }{% else
    \setlength{\parindent}{0pt}
    \setlength{\parskip}{6pt plus 2pt minus 1pt}}
}{% if KOMA class
  \KOMAoptions{parskip=half}}
\makeatother
\usepackage{xcolor}
\setlength{\emergencystretch}{3em} % prevent overfull lines
\setcounter{secnumdepth}{5}
% Make \paragraph and \subparagraph free-standing
\makeatletter
\ifx\paragraph\undefined\else
  \let\oldparagraph\paragraph
  \renewcommand{\paragraph}{
    \@ifstar
      \xxxParagraphStar
      \xxxParagraphNoStar
  }
  \newcommand{\xxxParagraphStar}[1]{\oldparagraph*{#1}\mbox{}}
  \newcommand{\xxxParagraphNoStar}[1]{\oldparagraph{#1}\mbox{}}
\fi
\ifx\subparagraph\undefined\else
  \let\oldsubparagraph\subparagraph
  \renewcommand{\subparagraph}{
    \@ifstar
      \xxxSubParagraphStar
      \xxxSubParagraphNoStar
  }
  \newcommand{\xxxSubParagraphStar}[1]{\oldsubparagraph*{#1}\mbox{}}
  \newcommand{\xxxSubParagraphNoStar}[1]{\oldsubparagraph{#1}\mbox{}}
\fi
\makeatother


\providecommand{\tightlist}{%
  \setlength{\itemsep}{0pt}\setlength{\parskip}{0pt}}\usepackage{longtable,booktabs,array}
\usepackage{calc} % for calculating minipage widths
% Correct order of tables after \paragraph or \subparagraph
\usepackage{etoolbox}
\makeatletter
\patchcmd\longtable{\par}{\if@noskipsec\mbox{}\fi\par}{}{}
\makeatother
% Allow footnotes in longtable head/foot
\IfFileExists{footnotehyper.sty}{\usepackage{footnotehyper}}{\usepackage{footnote}}
\makesavenoteenv{longtable}
\usepackage{graphicx}
\makeatletter
\def\maxwidth{\ifdim\Gin@nat@width>\linewidth\linewidth\else\Gin@nat@width\fi}
\def\maxheight{\ifdim\Gin@nat@height>\textheight\textheight\else\Gin@nat@height\fi}
\makeatother
% Scale images if necessary, so that they will not overflow the page
% margins by default, and it is still possible to overwrite the defaults
% using explicit options in \includegraphics[width, height, ...]{}
\setkeys{Gin}{width=\maxwidth,height=\maxheight,keepaspectratio}
% Set default figure placement to htbp
\makeatletter
\def\fps@figure{htbp}
\makeatother

% typographical packages
\usepackage{microtype}
\usepackage{setspace}
\tolerance=6000
\hyphenpenalty=1000

% typographical settings for the body text
\setlength{\parskip}{0em}
\setlength{\parindent}{1em}
\linespread{1.3}

% DEFINITIONS TITLE PAGE / COPYRIGHT
\newcommand{\titleoriginal}{Title}
\newcommand{\subtitleoriginal}{Subtitle}
\newcommand{\yearoriginal}{Year}
\newcommand{\subtitletranslation}{Translation Subtitle}
\newcommand{\yeartranslation}{Translation Year}
\newcommand{\stringtranslation}{Translation String}
\newcommand{\stringlicense}{Translation License String.}
\newcommand{\stringpublisher}{Published by String}
\newcommand{\ISBNHC}{978-9916-}
\newcommand{\ISBNSC}{978-9916-}
\newcommand{\ISBNEBOOK}{978-9916-}
%\newcommand{\ISBNAUDIO}{978-9916-}
\newcommand{\press}{Konsensus Network}
\newcommand{\translatorone}{Translator 1}
\newcommand{\translators}{
\large\textit{\stringtranslation:}\\
\translatorone\\
}

% PHYSICAL DOCUMENT SETUP
\setstocksize{210mm}{148mm}
\settrimmedsize{210mm}{148mm}{*}
\setbinding{5mm}
\setlrmarginsandblock{8mm}{15mm}{*}
\setulmarginsandblock{25mm}{26mm}{*}

% FONTS
\usepackage{fontspec}
\setmainfont{stone-serif}[
    Path=./fonts/stone-serif-itc-pro/,
    Scale=0.86,
    Extension=.OTF,
    UprightFont=*-Regular,
    BoldFont=*-SemiBd,
    ItalicFont=*-MediumIt,
    BoldItalicFont=*-SemiBdIt
    ]

\setsansfont{stone-sans}[
    Path=./fonts/stone-sans/,
    Scale=0.82,
    Extension=.otf,
    UprightFont=*-Medium,
    BoldFont=*-Semibold,
    ItalicFont=*-MediumItalic,
    BoldItalicFont=*-SemiBoldItalic
    ]

\usepackage{lettrine}
\setcounter{DefaultLines}{3}
\renewcommand{\DefaultLoversize}{0.1}
\renewcommand{\DefaultLraise}{0}
\renewcommand{\LettrineTextFont}{}
\setlength{\DefaultFindent}{\fontdimen2\font}
\setlength{\DefaultNindent}{0em}

%\usepackage[footskip=8mm]{geometry}

% custom second title page
\makeatletter
\newcommand*\halftitlepage{\begingroup % Misericords, T&H p 153
  \setlength\drop{0.1\textheight}
  \begin{center}
  \vspace*{\drop}
  \rule{\textwidth}{0in}\par
  {\Large\sffamily\thetitle\par}
  \rule{\textwidth}{0in}\par
  \vfill
  \end{center}
\endgroup}
\makeatother

% custom title page
\makeatletter
\newlength\drop
\newcommand*\titleM{\begingroup % Misericords, T&H p 153
  \setlength\drop{0.15\textheight}
  \begin{center}
  \vspace*{\drop}
  {\huge\sffamily\thetitle\par}
  \vspace{2em}
  %{\normalsize\sffamily\textit\subtitletranslation\par}
  %\vspace{2em}
  \rule{5.5cm}{0.3mm}\par
  \vspace{2em}
  {\Large\sffamily\textit\theauthor\par}
  \vspace{3em}
  %{\footnotesize\sffamily\textit\translators\par}
  \vfill
  \includegraphics[width=3.5cm]{figures/knw.png}\par
  \end{center}
\endgroup}
\makeatother

% copyright page
\makeatletter
\newcommand*\copyrightpage{\begingroup
  \setlength\drop{0.1\textheight}
  \vphantom{just for the drop}
  \vfill
  \begin{scriptsize}
  \noindent \copyright\space \yearoriginal: \theauthor
  \par\noindent \textit{\titleoriginal}
  \vspace{0.5\baselineskip}
  %\par\noindent \copyright\space \yeartranslation\space \stringtranslation: \translatorone
  %\par\noindent \textit{\thetitle: \subtitletranslation}
  \vspace{\baselineskip}
  \par\noindent \textit{\stringlicense}
  \vspace{0.5\baselineskip}
  \par\noindent \stringpublisher: \href{https://konsensus.network}{\textit{konsensus.network}}
  \vspace{0.5\baselineskip}
  \par\noindent v1.0.0
  \vspace{0.5\baselineskip}
  \setlength{\parindent}{2em}% default 20pt
  \par\noindent ISBN \ISBNHC \:Hardcover
  \par\hspace{0.28\parindent}\ISBNSC \:Paperback
  \par\hspace{0.28\parindent}\ISBNEBOOK \:E-book\par
  \setlength{\parindent}{0pt}
  \end{scriptsize}
  \vspace{3em}
  \par\noindent \href{https://konsensus.network}{\large\MakeUppercase \press \hspace{3em} \includegraphics[width=1cm]{figures/freestarfish.png}}
  \setcounter{footnote}{0}
  \clearpage
\endgroup}
\makeatother

% HEADER AND FOOTER MANIPULATION
% for normal pages
\nouppercaseheads
\headsep = 10mm
\makepagestyle{mystyle} 
\makeevenhead{mystyle}{\scriptsize\sffamily\mdseries\thepage}{}{}
\makeoddhead{mystyle}{{\scriptsize\sffamily\mdseries\leftmark}}{}{\scriptsize\sffamily\mdseries\thepage}
\makeevenfoot{mystyle}{}{}{}
\makeoddfoot{mystyle}{}{}{}
\makeatletter

% for pages where chapters begin
\makepagestyle{plain}
\makerunningwidth{plain}{\headwidth}
\makeevenfoot{plain}{}{}{}
\makeoddfoot{plain}{}{\scriptsize\sffamily\mdseries\thepage}{}
\pagestyle{mystyle}

\newif\ifmainmatter
\appto\mainmatter{\mainmattertrue}
\appto\backmatter{\mainmatterfalse}
\appto\appendix{\mainmatterfalse}

\renewcommand\chaptermark[1]{%
  \markboth{\MakeUppercase{%
    \ifmainmatter~\oldstylenums\thechapter.~\fi#1}}{}}%

% TOC
\usepackage[]{tocloft}
\renewcommand{\cftsectiondotsep}{\cftnodots}
\renewcommand{\cftpartfont}{\Large\sffamily\MakeUppercase}
\renewcommand{\cftchapterfont}{\small\sffamily}
\renewcommand{\cftsectionfont}{\Small\sffamily}
\renewcommand{\cftpartpagefont}{\Large\sffamily}
\renewcommand{\cftchapterpagefont}{\small}
\renewcommand{\cftchapterpresnum}{HOOFDSTUK\space}
\renewcommand{\cftchapternumwidth}{7em}
\setlength{\cftchapterindent}{0em}
\setlength{\cftsectionindent}{7em}
\setlength{\cftbeforechapterskip}{-0.2em}
\setsecnumdepth{chapter}
\setcounter{tocdepth}{0}


% Redefine footnote presentation
\makeatletter
\renewcommand\@makefntext[1]{%
  \noindent\hb@xt@2em{% <-- Box of fixed size for footnote number and space
    \@thefnmark\quad}% <-- Footnote number followed by a quad space
  \parbox[t]{\dimexpr\linewidth-2em}{#1}% <-- Parbox to control the width of footnote content
}
\makeatother

% layout check and fix
\checkandfixthelayout

% COUNTERS FOOTNOTES
\usepackage{chngcntr}
\counterwithout*{footnote}{chapter}

% TITLE FORMATTING
\usepackage{titlesec}
\titleformat
    {\chapter}[display]
    {\huge\sffamily}
    {\Large\sffamily\chaptertitlename\space\thechapter}
    {0pt}
    {\vspace{28pt}}

\titleformat
  {\section}[block]
  {\sffamily\large\bfseries}
  {}
  {0pt}
  {}
  
\titlespacing*{\section}{0pt}{2em}{0.5em}

\titleformat{\subsection}{\sffamily\bfseries}{}{}{}
\titlespacing*{\subsection}{0pt}{2em}{0em}

% QUOTE FORMATTING
\renewenvironment{quote}%
               {\list{}{\rightmargin=.6cm\leftmargin=.6cm}%
                \itshape \item[]}% <- The effect of \samepage is local!!!
               {\endlist}

% LAYOUT CHECK AND FIX
\checkandfixthelayout

% CUSTOM TITLE PAGE
\makeatletter
\def\@maketitle{%
  % the half title page
  \pagestyle{empty}
  \halftitlepage
  \cleardoublepage

  % the title page
  \titleM
  \clearpage

  % the copyright page
  \copyrightpage
  \cleardoublepage
  \pagestyle{mystyle}
}
\makeatother
% END PREAMBLE
\makeatletter
\@ifpackageloaded{bookmark}{}{\usepackage{bookmark}}
\makeatother
\makeatletter
\@ifpackageloaded{caption}{}{\usepackage{caption}}
\AtBeginDocument{%
\ifdefined\contentsname
  \renewcommand*\contentsname{Inhoudsopgave}
\else
  \newcommand\contentsname{Inhoudsopgave}
\fi
\ifdefined\listfigurename
  \renewcommand*\listfigurename{Lijst van figuren}
\else
  \newcommand\listfigurename{Lijst van figuren}
\fi
\ifdefined\listtablename
  \renewcommand*\listtablename{Lijst van tabellen}
\else
  \newcommand\listtablename{Lijst van tabellen}
\fi
\ifdefined\figurename
  \renewcommand*\figurename{Figuur}
\else
  \newcommand\figurename{Figuur}
\fi
\ifdefined\tablename
  \renewcommand*\tablename{Tabel}
\else
  \newcommand\tablename{Tabel}
\fi
}
\@ifpackageloaded{float}{}{\usepackage{float}}
\floatstyle{ruled}
\@ifundefined{c@chapter}{\newfloat{codelisting}{h}{lop}}{\newfloat{codelisting}{h}{lop}[chapter]}
\floatname{codelisting}{Listing}
\newcommand*\listoflistings{\listof{codelisting}{Lijst van listings}}
\makeatother
\makeatletter
\makeatother
\makeatletter
\@ifpackageloaded{caption}{}{\usepackage{caption}}
\@ifpackageloaded{subcaption}{}{\usepackage{subcaption}}
\makeatother

\ifLuaTeX
\usepackage[bidi=basic]{babel}
\else
\usepackage[bidi=default]{babel}
\fi
\babelprovide[main,import]{dutch}
% get rid of language-specific shorthands (see #6817):
\let\LanguageShortHands\languageshorthands
\def\languageshorthands#1{}
\ifLuaTeX
  \usepackage{selnolig}  % disable illegal ligatures
\fi
\usepackage{bookmark}

\IfFileExists{xurl.sty}{\usepackage{xurl}}{} % add URL line breaks if available
\urlstyle{same} % disable monospaced font for URLs
\hypersetup{
  pdftitle={Voor een nieuwe vrijheid},
  pdfauthor={Murray N. Rothbard},
  pdflang={nl},
  hidelinks,
  pdfcreator={LaTeX via pandoc}}


\title{Voor een nieuwe vrijheid}
\author{Murray N. Rothbard}
\date{+0100-01-01}

\begin{document}
\frontmatter
\maketitle

\renewcommand*\contentsname{Inhoudsopgave}
{
\setcounter{tocdepth}{0}
\tableofcontents
}

\mainmatter
\bookmarksetup{startatroot}

\chapter*{Over dit boek}\label{over-dit-boek}
\addcontentsline{toc}{chapter}{Over dit boek}

\markboth{Over dit boek}{Over dit boek}

\bookmarksetup{startatroot}

\chapter*{Inleiding}\label{inleiding}
\addcontentsline{toc}{chapter}{Inleiding}

\markboth{Inleiding}{Inleiding}

In deze tijd van verandering en groeienden controle is het essentieel om
de fundamenten van onze vrijheid opnieuw te onderzoeken. We leven in een
wereld waarin onze keuzes en beslissingen steeds vaker worden beperkt
door regels en voorschriften. Dit roept de vraag op: wat betekent
vrijheid voor ons, en hoe kunnen we ervoor zorgen dat deze niet verder
afbrokkelt? Het Libertarisch Manifest biedt een frisse kijk op deze
belangrijke kwesties. Het stelt de individuele vrijheid centraal en
pleit voor de noodzaak om persoonlijke verantwoordelijkheden te omarmen.
Het is tijd om terug te keren naar de basisprincipes die onze
samenleving hebben gevormd: autonomie, zelfbeschikking en respect voor
elkaar. Laten we ons verzetten tegen de steeds aanhoudende invloed van
externe machten en samen op zoek gaan naar een nieuwe definitie van
vrijheid die ons in staat stelt onze eigen weg te kiezen. Dit manifest
is niet alleen een oproep tot actie, maar ook een uitnodiging om na te
denken over de toekomst die we willen creëren.

Er zijn tegenwoordig veel varianten van het libertarisme, maar het
Rothbardianisme blijft het centrum van de intellectuele
aantrekkingskracht. Het fungeert als de voornaamste inspiratiebron en
als ons geweten; het vormt de strategische en morele kern van het debat,
zelfs als de naam zelf niet altijd erkend wordt. De reden hiervoor is
dat Murray Rothbard de grondlegger is van het moderne libertarisme. Dit
politieke en ideologische systeem biedt een definitieve ontsnapping aan
de valstrikken van zowel links als rechts, evenals aan hun centrale
plannen voor het gebruik van staatsmacht. Het libertarisme stelt een
radicale benadering voor, die stelt dat staatsmacht onwerkbaar en
immoreel is.

`Meneer Libertariër' was de bijnaam van Murray N. Rothbard, die ook wel
`De Grootste Levende Vijand van de Staat' werd genoemd. En dat blijft
hij. Hij putte uit een rijke traditie van voorgangers: de
klassiek-liberale denkers, de Oostenrijkse economen, de Amerikaanse
anti-oorlogstraditie en de traditie van het natuurrecht. Maar Rothbard
was degene die al deze losse ideeën heeft samengevoegd tot een
samenhangend systeem. Op het eerste gezicht is dat systeem
ongeloofwaardig, maar zodra Rothbard het definieert en verdedigt, lijkt
het onvermijdelijk. De afzonderlijke onderdelen zijn eenvoudig:
zelfeigenaarschap, strikte eigendomsrechten, vrije markten en een
duidelijk tegen de staat gerichte houding. De implicaties van dit
systeem zijn echter wereldschokkend.

Als je eenmaal het volledige plaatje hebt gezien --- en \emph{For a New
Liberty} is al meer dan een kwart eeuw het belangrijkste middel om dit
te onthullen --- kun je het niet meer vergeten. Het wordt de onmisbare
lens waardoor we de gebeurtenissen in de echte wereld met de grootste
helderheid kunnen waarnemen.

Dit boek legt uit waarom Rothbard elk jaar meer aanzien lijkt te
genieten. Zijn invloed is sinds zijn dood enorm toegenomen. Bovendien
onthult het waarom het Rothbardianisme zoveel vijanden heeft, zowel aan
de linkerzijde als aan de rechterzijde en in het midden. De reden is
eenvoudig: de wetenschap van vrijheid die hij met duidelijkheid heeft
uiteengezet, is net zo opwindend in de hoop die het biedt voor een vrije
wereld, als het onverbiddelijk is in de kritiek op fouten. De logische
en morele consistentie van zijn inzichten, samen met de empirische
verklaringskracht, vormen een bedreiging voor elke intellectuele visie
die gericht is op het gebruik van de staat om de wereld te hervormen
volgens een voorgeprogrammeerd plan. Tegelijkertijd biedt het de lezer
een hoopvolle visie op wat mogelijk is.

Rothbard begon met het schrijven van dit boek nadat hij een telefoontje
kreeg van Tom Mandel, een redacteur bij Macmillan. Mandel had een
opiniestuk van Rothbard gezien in de New York Times dat in het voorjaar
van 1971 was gepubliceerd. Dit was de enige opdracht die Rothbard ooit
ontving van een commerciële uitgeverij. Als je naar het originele
manuscript kijkt, dat zo consistent in lettertype is en bijna compleet
is na de eerste opzet, lijkt het alsof hij het bijna moeiteloos heeft
geschreven. Het is naadloos, onverbiddelijk en energiek.

De historische context laat een vaak over het hoofd gezien punt zien:
modern libertarisme is niet ontstaan als reactie op socialisme of links
--- hoewel het zeker anti-links en antisocialistisch is, zoals die
termen doorgaans worden begrepen. In de Amerikaanse geschiedenis is
libertarisme eerder ontstaan als een reactie op het statisme van het
conservatisme en de selectieve viering van centrale planning in
conservatieve kringen. Amerikaanse conservatieven zijn misschien niet
blij met de verzorgingsstaat of overmatig bedrijfsregulering, maar ze
waarderen de macht die in naam van nationalisme, oorlogszucht,
`pro-familie' beleid en de inbreuk op persoonlijke vrijheid en privacy
wordt uitgeoefend. In de periode na LBJ (Lyndon B. Johnson) waren het
vooral Republikeinse presidenten die verantwoordelijk waren voor de
grootste uitbreidingen van de uitvoerende en rechterlijke macht, en niet
de Democraten. De verdediging van pure vrijheid tegen de compromissen en
corrupties van het conservatisme --- te beginnen met Nixon en voortgezet
onder Reagan en de Bush-presidenten --- inspireerde de ontwikkeling van
de Rothbardiaanse politieke economie.

Het is opvallend hoe Rothbard er bewust voor koos om geen blad voor de
mond te nemen in zijn argumentatie. Veel andere intellectuelen in
dezelfde situatie zouden de neiging hebben om hun standpunten af te
zwakken om ze aanvaardbaarder te maken. Waarom zou hij bijvoorbeeld
pleiten voor staatloosheid of anarchisme, als een pleidooi voor een
beperkte overheid meer mensen naar de beweging zou kunnen aantrekken?
Waarom het imperialisme van de VS veroordelen, als dit de
aantrekkingskracht van het boek op anti-Sovjetconservatieven die anders
misschien openstonden voor een vrijemarkbenadering zou kunnen beperken?
Waarom zo diep ingaan op het privatiseren van rechtbanken, wegen en
water, als dat sommige mensen zou kunnen afschrikken? Waarom het
gevoelige onderwerp van consumptieregels en persoonlijke moraal
aansnijden - en dat doen met zo'n verwarrende consistentie - als het
zeker een breder publiek had kunnen trekken als het buiten beschouwing
was gelaten? En waarom zo uitvoerig ingaan op monetaire kwesties en
centraal bankieren, als een afgezwakt pleidooi voor vrij ondernemerschap
veel conservatieven van de Kamer van Koophandel tevreden had kunnen
stellen?

Maar inkorten en compromissen sluiten omwille van de tijd of het publiek
was voor hem geen optie. Hij wist dat hij eens in zijn leven de kans
kreeg om het volledige libertarisme in al zijn glorie te presenteren, en
die kans wilde hij niet laten liggen. Daarom lezen we hier niet alleen
een pleidooi voor het terugdringen van de overheid, maar voor het
helemaal afschaffen ervan. Hij pleit niet alleen voor het toekennen van
eigendomsrechten, maar ook voor het overlaten aan de markt, zelfs als
het gaat om het afdwingen van contracten. En het is niet alleen een
oproep om uitkeringen te verminderen, maar om de hele verzorgingsstaat
af te schaffen.

Waar andere pogingen, zowel vóór als na dit boek, om een libertarisch
pleidooi te houden, meestal oproepen tot overgangsmaatregelen of halve
maatregelen, of bereid zijn veel toe te geven aan statisten, is dat niet
wat we van Murray krijgen. Voor hem geen programma's zoals `school
vouchers' of de privatisering van overheidsprogramma's die eigenlijk
helemaal niet zouden moeten bestaan. Hij presenteert in plaats daarvan
een volwaardige visie op wat vrijheid kan zijn en zet deze krachtig
door. Daarom hebben zoveel vergelijkbare pogingen om het Libertarisch
Manifest te schrijven de tand des tijds niet doorstaan, terwijl dit boek
toch bijzonder gewild blijft.

In de tussenliggende jaren zijn er veel boeken over het libertarisme
verschenen die zich uitsluitend richtten op filosofie, politiek,
economie of geschiedenis. De boeken die al deze onderwerpen combineren,
zijn meestal verzamelwerken van verschillende auteurs. Alleen Rothbard
beheerst alle gebieden, waardoor hij in staat was een geïntegreerd
manifest te schrijven - een manifest dat nog nooit is vervangen. Toch is
zijn benadering typisch bescheiden. Hij verwijst voortdurend naar andere
schrijvers en denkers, zowel uit het verleden als uit zijn eigen
generatie.

Bovendien zijn sommige inleidingen van dit soort geschreven om de lezer
makkelijker toegang te geven tot een moeilijk boek, maar dat is hier
niet het geval. Hij praat nooit neerbuigend tegen zijn lezers, maar
steeds helder. Rothbard spreekt voor zichzelf. Ik zal de lezer besparen
op te sommen wat mijn favoriete delen zijn of te speculeren over welke
passages Rothbard misschien had verduidelijkt als hij een nieuwe editie
had kunnen uitbrengen. De lezer zal zelf ontdekken dat elke pagina vol
energie en passie zit, dat de logica van zijn argumenten onweerstaanbaar
overtuigend is en dat het intellectuele vuur dat dit werk inspireerde,
nog net zo fel brandt als all die jaren geleden.

Het boek wordt nog steeds als `gevaarlijk' beschouwd, omdat wanneer je
eenmaal bent blootgesteld aan het Rothbardianisme, je geen enkel ander
werk over politiek, economie of sociologie nog op dezelfde manier kunt
lezen. Wat ooit een commercieel fenomeen was, is inmiddels een klassiek
statement geworden waarvan ik voorspel dat het nog generaties lang
gelezen zal worden.

Llewellyn H. Rockwell, Jr.

Auburn, Alabama.

Het boek wordt nog steeds als `gevaarlijk' beschouwd. Want zodra je in
aanraking komt met het Rothbardianisme, kun je andere boeken over
politiek, economie of sociologie niet meer op dezelfde manier lezen. Wat
ooit een commercieel succes was, is inmiddels een klassiek statement
geworden waarvan ik verwacht dat het generaties lang gelezen zal worden.
Llewellyn H. Rockwell, Jr.~Auburn, Alabama

\bookmarksetup{startatroot}

\chapter{De libertarische erfenis: De Amerikaanse Revolutie en klassiek
liberalisme}\label{de-libertarische-erfenis-de-amerikaanse-revolutie-en-klassiek-liberalisme}

Op de dag van de verkiezingen in 1976 ontving het kandidaatschap van de
Libertarische Partij, met Roger L. MacBride als president en David P.
Bergland als vice-president, 174.000 stemmen in tweeëndertig staten door
het hele land. Het nuchtere Congressional Quarterly classificeerde de
jonge Libertarische Partij als de derde grote politieke partij in
Amerika. De opmerkelijke groei van deze nieuwe partij is zichtbaar,
aangezien ze pas in 1971 van start ging met een handjevol leden die
bijeenkwamen in een huiskamer in Colorado. Het jaar daarop organiseerden
ze een presidentieel kandidaatschap, dat in twee staten op het
stembiljet verscheen. En nu is de Libertarische Partij de derde grote
partij van Amerika.

Nog opmerkelijker is dat de Libertarische Partij deze groei wist te
bereiken terwijl ze trouw bleef aan een nieuw ideologisch credo:
`libertarisme'. Hiermee bracht ze voor het eerst in een eeuw een partij
op het Amerikaanse politieke toneel die zich richtte op principes, in
plaats van enkel op het vergaren van banen en geld uit de publieke kas.
Deskundigen en politicologen hebben ons vaak verteld dat het unieke van
Amerika en haar partijstelsel te vinden is in het gebrek aan ideologie
en in het `pragmatisme' (een eufemisme voor het uitsluitend mikken op
het veroveren van belastinggeld en banen van de belastingbetalers). Hoe
kunnen we dan de verbazingwekkende groei verklaren van een nieuwe partij
die zich openlijk en enthousiast inzet voor een ideologie?

Eén verklaring is dat Amerikanen niet altijd pragmatisch en
niet-ideologisch waren. Historici realiseren zich inmiddels dat de
Amerikaanse Revolutie niet alleen ideologisch was, maar ook voortkwam
uit een sterke toewijding aan het credo en de principes van het
libertarisme. De Amerikaanse revolutionairen waren doordrongen van het
libertarisme, een ideologie die hen ertoe aanzette zich met hun leven,
hun bezit en hun eer te verzetten tegen de schendingen van hun rechten
en vrijheden door de Britse imperiale regering. Historici hebben lang
gediscussieerd over de exacte oorzaken van de Amerikaanse Revolutie:
waren deze constitutioneel, economisch, politiek of ideologisch?
Tegenwoordig begrijpen we dat de revolutionairen, als libertariërs, geen
tegenstelling zagen tussen morele en politieke rechten aan de ene kant
en economische vrijheid aan de andere kant. Sterker nog, zij beschouwden
burgerlijke en morele vrijheid, politieke onafhankelijkheid en de
vrijheid om te handelen en te produceren als onderdelen van één
onmiskenbaar systeem. Adam Smith zou dit systeem in hetzelfde jaar dat
de Onafhankelijkheidsverklaring werd geschreven, omschrijven als het
`voor de hand liggende en eenvoudige systeem van natuurlijke vrijheid.'

Het Libertarische credo is ontstaan uit de `klassiek liberale'
bewegingen van de zeventiende en achttiende eeuw in de Westerse wereld,
met name vanuit de Engelse Revolutie van de zeventiende eeuw. Deze
radicale libertarische beweging was, ondanks dat ze slechts gedeeltelijk
succesvol was in haar geboorteplaats Groot-Brittannië, in staat om de
Industriële Revolutie in gang te zetten. Hierdoor werd de industrie
bevrijd van de verstikkende beperkingen van staatscontrole en gilden die
door de overheid gesteund werden. De klassieke liberale beweging
vertegenwoordigde een krachtige libertarische `revolutie' tegen wat we
de Oude Orde kunnen noemen: het ancien régime dat zijn onderdanen
eeuwenlang had gedomineerd. Vanaf de zestiende eeuw had dit regime,
bovenop een ouderwets netwerk van feodale landmonopolies en stedelijke
gildebeperkingen, een absolute centrale staat en een koning met
goddelijk recht opgelegd. Het resultaat was een Europa dat gevangen zat
in een verlammend web van controles, belastingen en monopolistische
privileges, die door centrale en lokale overheden aan hun favoriete
producenten werden verleend. De combinatie van de nieuwe
bureaucratische, oorlogszuchtige centrale staat met bevoorrechte
kooplieden - een verbinding die door latere historici `mercantilisme'
zou worden genoemd - en een klasse van heersende feodale landheren,
vormde de Oude Orde waartegen de nieuwe beweging van klassieke liberalen
en radicalen in de zeventiende en achttiende eeuw opkwam en in opstand
kwam.

Het doel van de klassieke liberalen was om individuele vrijheid in al
haar facetten te bevorderen. In de economie moesten belastingen flink
omlaag, moesten controles en regels verdwijnen, en moest menselijke
energie, ondernemingszin en marktwerking de ruimte krijgen om te creëren
en te produceren. Dit zou zowel iedereen als de massa van consumenten
ten goede komen. Ondernemers moesten de vrijheid hebben om te
concurreren, zich te ontwikkelen en innovaties te creëren. De ketenen
van controle rondom grond, arbeid en kapitaal moesten worden verbroken.
Persoonlijke en burgerlijke vrijheid moesten beschermd worden tegen de
plunderingen en tirannie van de koning of zijn aanhangers. Religie, dat
eeuwenlang de oorsprong was van bloedige oorlogen tussen verschillende
sekten die om de controle over de staat streden, moest bevrijd worden
van staatsinmenging. Hierdoor konden alle religies - of geen religie -
vreedzaam naast elkaar bestaan. Ook vrede moest de leidraad vormen voor
het buitenlands beleid van de nieuwe klassiek-liberalen. Het eeuwenoude
regime van imperiale expansie en machtsoverheersing moest plaatsmaken
voor een beleid van vrede en vrije handel met alle naties. Aangezien
oorlog vaak werd toegeschreven aan permanente legers en marines, en aan
militaire macht die altijd op zoek was naar uitbreiding, moesten deze
militaire instellingen worden vervangen door vrijwillige lokale
milities. Dit zouden burgers zijn die enkel bereid waren om hun eigen
huizen en buurten te verdedigen.

Het bekende thema `scheiding van kerk en staat' was slechts één van de
vele met elkaar verbonden motieven. Deze motieven kunnen worden
samengevat als `scheiding van de economie van de staat', `scheiding van
meningsuiting en pers van de staat', `scheiding van land van de staat'
en `scheiding van oorlog en militaire zaken van de staat'. Eigenlijk
gaat het om de scheiding van de staat van vrijwel alles.

De staat moest kortom uiterst klein blijven, met een budget dat zo laag
mogelijk was. De klassiek-liberalen hebben nooit een duidelijke theorie
over belastingen ontwikkeld, maar zij bestreden elke belastingverhoging
en elk nieuw soort belasting met grote tegenstand. In Amerika leidde dit
zelfs twee keer tot spanningen die bijna tot de Revolutie resulteerden:
de belasting op zegels en de belasting op thee.

De eerste denkers van het libertarische klassieke liberalisme waren de
Levelers tijdens de Engelse Revolutie en de filosoof John Locke aan het
eind van de zeventiende eeuw. Zij werden gevolgd door de `True Whig', de
radicale libertarische oppositie tegen de `Whig Settlement', het regime
van achttiende-eeuws Groot-Brittannië. John Locke beschreef de
natuurlijke rechten van elk individu met betrekking tot zijn persoon en
eigendom. De rol van de overheid was strikt beperkt tot het beschermen
van deze rechten. In de woorden van de door Locke geïnspireerde
Onafhankelijkheidsverklaring: `Om deze rechten veilig te stellen, zijn
er onder de mensen regeringen ingesteld, die hun rechtvaardige macht
ontlenen aan de instemming van de geregeerden. Wanneer een regeringsvorm
destructief wordt voor deze doeleinden, heeft het volk het recht om deze
te veranderen of af te schaffen.'

Hoewel Locke in de Amerikaanse koloniën veel werd gelezen, was zijn
abstracte filosofie niet echt gericht op het aanzetten tot revolte. Die
taak werd opgepakt door radicale Lockeanen in de achttiende eeuw. Zij
schreven op een meer toegankelijke, directe en gepassioneerde manier en
passten de basisfilosofie toe op de concrete problemen van de ontdekke
regering, en vooral de Britse. Het belangrijkste artikel in deze lijn
was `Cato's Brieven', een serie krantenartikelen die begin 1720 in
Londen werden gepubliceerd door de `True Whigs', John Trenchard en
Thomas Gordon. Terwijl Locke had geschreven over de revolutionaire druk
die kan ontstaan wanneer de overheid de vrijheid ondermijnt, wijzen
Trenchard en Gordon erop dat de overheid altijd de neiging heeft om
individuele rechten te schenden. Volgens de `Brieven van Cato' is de
geschiedenis van de mensheid een verslag van een onvermijdelijk conflict
tussen Macht en Vrijheid. De Macht (de regering) staat altijd klaar om
haar invloed uit te breiden ten koste van de rechten van mensen en hun
vrijheden. Daarom, zo betoogt Cato, moet de macht klein worden gehouden
en geconfronteerd met voortdurende waakzaamheid en tegenstand van het
publiek, om ervoor te zorgen dat deze altijd binnen zijn nauwe grenzen
blijft.

\begin{quote}
Wij weten uit talloze voorbeelden en ervaringen dat mensen met macht er
alles aan doen om deze te behouden, zelfs als dat de meest afschuwelijke
middelen vereist. Bijna niemand op aarde heeft ooit afstand gedaan van
zijn macht, zolang hij er maar op zijn eigen manier gebruik van kon
maken. Het lijkt dan ook zeker dat het welzijn van de wereld of van hun
volk niet een van hun motieven is om aan de macht te blijven of om deze
op te geven.

Het is de aard van macht om zich voortdurend op te dringen. Elke
buitengewone bevoegdheid, verleend op specifieke momenten en bij
bijzondere gelegenheden, wordt omgevormd tot een algemene bevoegdheid
die altijd gebruikt kan worden, zelfs als er geen uitzonderlijke
situatie is. Macht doet nooit vrijwillig afstand van enig voordeel.

Helaas! Macht overschrijdt dagelijks de grenzen van Vrijheid, en doet
dat met verontrustend succes. Het evenwicht tussen beiden lijkt bijna
verloren. Tirannie heeft bijna de hele aarde in zijn greep en bedreigt
de mensheid, wortel en tak. Het maakt de wereld tot een slachthuis en
zal zeker doorgaan met vernietigen, totdat het ofwel zelf wordt
vernietigd, of, wat het waarschijnlijker maakt, niets anders meer heeft
om te vernietigen.1
\end{quote}

Dergelijke waarschuwingen werden met veel enthousiasme overgenomen door
de Amerikaanse kolonisten. Zij herdrukten `Cato's Brieven' vele malen in
de koloniën, tot aan de tijd van de Revolutie. Deze diepgewortelde
houding leidde tot wat de historicus Bernard Bailyn treffend het
`transformerende radicale libertarisme' van de Amerikaanse Revolutie
noemde.

De revolutie was niet alleen de eerste succesvolle moderne poging om het
juk van het westerse imperialisme van zich af te werpen - op dat moment
de machtigste grootmacht ter wereld. Belangrijker was dat de Amerikanen
voor het eerst in de geschiedenis hun nieuwe regeringen beperkten met
talloze limieten en restricties, vastgelegd in grondwetten en vooral in
zogenaamde `bills of rights' (een verklaring van burgerrechten). Kerk en
staat werden in de nieuwe staten strikt van elkaar gescheiden en
godsdienstvrijheid werd gewaarborgd. De overblijfselen van het
feodalisme werden in alle staten uitgeroeid door de afschaffing van de
feodale privileges van erfopvolging en eerstgeboorterecht. (In het
eerste geval kan een overleden voorouder land voor altijd aan zijn
familie toevertrouwen, waardoor zijn erfgenamen geen deel van het land
kunnen verkopen. In het tweede geval bepaalt de wet dat alleen de oudste
zoon het eigendom erft.)

De nieuwe federale regering, gevormd op basis van de Artikelen van de
Confederatie, had geen recht om belastingen te heffen op het volk. Voor
elke fundamentele uitbreiding van haar bevoegdheden was unanieme
toestemming van alle deelstaatregeringen vereist. Bovenal was de
militaire en oorlogszuchtige macht van de nationale regering beperkt,
voortkomend uit terughoudendheid en wantrouwen. De libertariërs van de
achttiende eeuw begrepen namelijk dat oorlog, permanente legers en
militarisme lange tijd de belangrijkste manieren waren om de staatsmacht
uit te breiden.2

Bernard Bailyn heeft de prestaties van de Amerikaanse revolutionairen
als volgt samengevat:

\begin{quote}
De modernisering van de Amerikaanse politiek en regering tijdens en na
de Revolutie vormde een plotselinge, radicale realisatie van het
programma dat voor het eerst volledig was uiteengezet door de
oppositie-intelligentsia tijdens het bewind van George de Eerste.
Terwijl de Engelse oppositie, die zich een weg baande tegen een
zelfgenoegzame sociale en politieke orde, alleen had gestreefd en
gedroomd, konden de Amerikanen nu handelen. Gedreven door dezelfde
aspiraties, maar deel uitmakend van een maatschappij die in veel
opzichten modern was en nu politiek bevrijd, grepen zij hun kans. Waar
de Engelse oppositie tevergeefs had gepleit voor gedeeltelijke
hervormingen, gingen Amerikaanse leiders snel en met weinig sociale
ontwrichting over tot de systematische uitvoering van radicale
bevrijdingsideeën.

In dit proces hebben ze de belangrijkste thema's van het
achttiende-eeuwse radicale libertarisme, die hier werkelijkheid zijn
geworden, verankerd in de Amerikaanse politieke cultuur. Het eerste
thema is de overtuiging dat macht een kwaad is; misschien een
noodzakelijkheid, maar wel een kwalijke. Macht corrumpeert eindeloos en
moet op elke mogelijke manier gecontroleerd, beperkt en begrensd worden,
zolang dit verenigbaar is met een minimum aan burgerlijke orde.
Geschreven grondwetten, de scheiding der machten, verklaringen van
burgerrechten, en beperkingen voor de uitvoerende macht, de wetgevende
macht en de rechtbanken: dit alles drukt het diepe wantrouwen uit tegen
macht dat in het ideologische hart van de Amerikaanse Revolutie ligt.
Deze erfenis heeft ons sindsdien blijvend beïnvloed.3
\end{quote}

Hoewel het klassieke liberale denken in Engeland begon, bereikte het
zijn meest consistente en radicale ontwikkeling -- en zijn grootste
belichaming -- in Amerika. Dit kwam doordat de Amerikaanse koloniën vrij
waren van het feodale landmonopolie en de aristocratische elite die in
Europa zo diep geworteld was. In Amerika waren de heersers Britse
koloniale ambtenaren en een handvol bevoorrechte kooplieden, die
relatief gemakkelijk aan de kant konden worden geschoven toen de
Revolutie uitbrak en de Britse regering werd omvergeworpen. Hierdoor
vond het klassieke liberalisme meer steun onder de bevolking en
ondervond het veel minder opstandige institutionele weerstand in de
Amerikaanse koloniën dan in Europa. Daarnaast hoefden de Amerikaanse
opstandelingen, vanwege hun geografische isolatie, zich geen zorgen te
maken over aanvallen van nabijgelegen, contrarevolutionaire regeringen,
zoals dat bijvoorbeeld wel het geval was in Frankrijk.

\section{NA DE REVOLUTIE}\label{na-de-revolutie}

De modernisering van de Amerikaanse politiek en regering tijdens en na
de Revolutie leidde tot een plotselinge en radicale uitvoering van het
programma dat voor het eerst volledig was uiteengezet door de
oppositie-intelligentsia tijdens het bewind van George de Eerste. Waar
de Engelse oppositie, die zich een weg baande tegen een zelfgenoegzame
sociale en politieke orde, alleen maar had gedroomd en gestreefd, konden
de Amerikanen nu daadwerkelijk handelen. Gedreven door dezelfde
aspiraties, maar deel uitmakend van een maatschappij die op veel vlakken
modern was en politiek bevrijd, grepen zij hun kans. In tegenstelling
tot de Engelse oppositie, die tevergeefs had gepleit voor gedeeltelijke
hervormingen, gingen Amerikaanse leiders snel en met weinig sociale
ontwrichting over tot de systematische uitvoering van radicale
bevrijdingsideeën. In dit proces hebben zij de belangrijkste thema's van
het achttiende-eeuwse radicale libertarisme, die hier werkelijkheid zijn
geworden, verankerd in de Amerikaanse politieke cultuur. Het eerste
thema is het geloof dat macht een kwaad is; misschien een
noodzakelijkheid, maar wel een kwalijke. Macht corrumpeert eindeloos en
moet op elke mogelijke manier gecontroleerd, beperkt en begrensd worden,
zolang dit mogelijk is bij een minimum aan burgerlijke orde. Geschreven
grondwetten, de scheiding der machten, verklaringen van burgerrechten,
en beperkingen voor de uitvoerende macht, de wetgevende macht en de
rechtbanken drukken het diepe wantrouwen tegen macht uit dat in het
ideologische hart van de Amerikaanse Revolutie ligt. Deze erfenis heeft
ons sindsdien blijvend beïnvloed. Hoewel het klassieke liberale denken
in Engeland begon, bereikte het zijn meest consistente en radicale
ontwikkeling -- en zijn grootste belichaming -- in Amerika. De
Amerikaanse koloniën waren vrij van het feodale landmonopolie en de
aristocratische elite die in Europa zo diep geworteld was. In Amerika
waren de heersers Britse koloniale ambtenaren en een handvol
bevoorrechte kooplieden, die relatief gemakkelijk aan de kant konden
worden geschoven toen de Revolutie uitbrak en de Britse regering werd
omvergeworpen. Hierdoor vond het klassieke liberalisme meer steun onder
de bevolking en ondervond het veel minder institutionele weerstand in de
Amerikaanse koloniën dan zijn Europese tegenhanger. Bovendien hoefden de
Amerikaanse opstandelingen, vanwege hun geografische isolatie, zich geen
zorgen te maken over aanvallen van nabijgelegen, contrarevolutionaire
regeringen, zoals dat bijvoorbeeld wel het geval was in Frankrijk.

Zo werd Amerika, meer dan enig ander land, geboren uit een expliciete
libertarische revolutie. Het was een revolutie tegen het imperium, tegen
belastingen, handelsmonopolies en regulering, en tegen militarisme en
uitvoerende macht. Deze revolutie leidde tot regeringen met ongekende
beperkingen op hun macht.

Hoewel er in Amerika weinig institutioneel verzet was tegen de opmars
van het liberalisme, waren er vanaf het begin machtige elitekrachten
actief. Vooral grote kooplieden en planters wilden het Britse
`mercantilistische' systeem met zijn hoge belastingen, controles en
overheidsmonopolieprivileges behouden. Deze groepen verzochten om een
sterke centrale regering; ze wilden eigenlijk het Britse systeem zonder
Groot-Brittannië. Deze conservatieve en reactionaire krachten dienden
zich voor het eerst aan tijdens de Revolutie en vormden later in de
jaren 1790 de Federalistische partij en de Federalistische regering.

In de negentiende eeuw zette de libertarische impuls zich voort. De
Jeffersoniaanse en Jacksoniaanse bewegingen, evenals de
Democratisch-Republikeinse en later de Democratische partijen, streefden
naar een bijna volledige afschaffing van de overheid in het Amerikaanse
leven. Het ideale doel was een regering zonder een staand leger of
marine; een regering zonder schulden en zonder directe federale of
accijnsbelastingen, met vrijwel geen importtarieven. Dit resulteerde in
een verwaarloosbaar niveau van belastingen en uitgaven. De overheid
moest zich niet bemoeien met openbare werken of interne verbeteringen.
Er was geen controle of regulering, en geld en bankieren moesten vrij,
betrouwbaar en ongeïnflateerd blijven. Kortom, in de woorden van H.L.
Mencken: `een regering die er ternauwernood aan ontkomt helemaal geen
regering te zijn.'

Het Jeffersoniaanse streven naar bijna geen regering kwam tot stilstand
nadat Jefferson aan de macht was gekomen. Eerst maakte hij concessies
aan de Federalisten, mogelijk het resultaat van een deal om
Federalistische stemmen te krijgen en zo een gelijkspel in het
kiescollege te doorbreken. Daarna volgde de ongrondwettelijke aankoop
van het Louisiana Territory. Maar vooral de imperialistische drang naar
oorlog met Groot-Brittannië in Jeffersons tweede ambtstermijn zorgde
voor deze mislukking. Deze neiging leidde tot oorlog en tot een
eenpartijstelsel dat vrijwel het volledige federalistische programma
invoerde: hoge militaire uitgaven, een centrale bank, een beschermend
invoerheffingstarief, directe federale belastingen en openbare werken.
Een gepensioneerde Jefferson bleef in Monticello gruwen van de
resultaten en inspireerde jonge politici als Martin Van Buren en Thomas
Hart Benton om een nieuwe partij op te richten: de Democratische partij.
Zij wilden Amerika terugwinnen van het nieuwe Federalisme en de geest
van het oude Jeffersoniaanse programma heroveren. Toen de twee jonge
leiders zich blindelings aan Andrew Jackson vasthielden als hun redder,
was de nieuwe Democratische partij geboren.

De Jacksoniaanse libertariërs hadden een plan: acht jaar Andrew Jackson
als president, gevolgd door acht jaar Van Buren en daarna nog acht jaar
Benton. Na vierentwintig jaar van bloeiende Jacksoniaanse Democratie zou
het Menckeniaanse ideaal van vrijwel geen overheid werkelijkheid zijn
geworden. Het leek geen onmogelijke droom, want de Democratische partij
was snel de belangrijkste meerderheidspartij van het land geworden. De
massa achter het volk was gewonnen voor de libertarische zaak. Jackson
had zijn acht jaar in functie achter de rug, waarin de centrale bank
werd afgeschchaft en de staatsschuld werd kwijtgescholden. Van Buren had
ook vier jaar gediend, waarin de federale overheid onafhankelijk werd
van het banksysteem. Maar de verkiezingen van 1840 vormden een
uitzondering. Van Buren werd verslagen door een ongekend demagogische
campagne, geleid door de eerste grote moderne campagnevoorzitter,
Thurlow Weed. Hij was een pionier in alle campagne-uitingen, zoals
pakkende slogans, buttons, liedjes en parades, die we nu kennen. De
tactieken van Weed zorgden ervoor dat de onbekende Whig, generaal
William Henry Harrison, aan de macht kwam. Dit was echter duidelijk een
toevallige overwinning. In 1844 waren de Democraten vastbesloten om met
dezelfde campagnetactieken terug te vechten en ze waren van plan om dat
jaar het presidentschap terug te winnen. Van Buren zou natuurlijk de
triomfantelijke koers van de Jacksoniaanse beweging hervatten. Maar toen
deed zich een noodlottige gebeurtenis voor: de Democratische partij werd
verdeeld over de cruciale kwestie van de slavernij, of beter gezegd over
de uitbreiding van de slavernij naar nieuwe gebieden. De makkelijke
herverkiezing van Van Buren kwam ten einde door een breuk binnen de
partij over de toelating van Texas als slavenstaat. Van Buren was tegen,
Jackson was voor, en deze verdeeldheid symboliseerde de bredere scheuren
binnen de Democratische partij. Slavernij, de ernstige antiliberale fout
binnen het libertarisme van het Democratische programma, stond op om de
partij en haar libertarisme volledig te ondermijnen.

De Burgeroorlog ging gepaard met ongekend bloedvergieten en
vernietigingen. Het triomferende, bijna eenpartijstelsel van het
Republikeinse regime gebruikte deze situatie om zijn statistische,
voorheen Whig-programma door te voeren. Dit omvatte nationale
regeringsmacht, beschermende invoerrechten, subsidies voor grote
bedrijven, inflatoir papiergeld, een hernieuwde controle van de federale
overheid over het bankwezen, grootschalige interne verbeteringen, hoge
accijnzen en, tijdens de oorlog, dienstplicht en inkomstenbelasting.
Bovendien verloren de staten hun eerdere recht op secessie en andere
bevoegdheden ten opzichte van de federale overheid. Na de oorlog
hervatte de Democratische partij haar libertarische koers. Maar nu stond
er een veel langere en moeilijkere weg naar vrijheid voor hen open dan
voorheen.

We hebben gezien hoe Amerika de diepste libertarische traditie heeft
ontwikkeld. Deze traditie is nog altijd terug te vinden in veel van onze
politieke retoriek en wordt weerspiegeld in de pittige,
individualistische houding van een groot deel van het Amerikaanse volk
ten opzichte van de overheid. In dit land is er veel meer vruchtbare
grond voor een heropleving van het libertarisme dan ergens anders.

\section{WEERSTAND TEGEN VRIJHEID}\label{weerstand-tegen-vrijheid}

De Jacksoniaanse libertariërs hadden een duidelijk plan: acht jaar
Andrew Jackson als president, gevolgd door acht jaar Van Buren en
vervolgens nog eens acht jaar Benton. Na vierentwintig jaar van
bloeiende Jacksoniaanse Democratie zou het Menckeniaanse ideaal van
bijna geen overheid werkelijkheid zijn geworden. Het was geen
onmogelijke droom; de Democratische partij was snel de grootste
meerderheidspartij in het land geworden. De massa achter het volk
steunde de libertarische zaak. Jackson had zijn acht jaar in functie
achter de rug, waarin hij de centrale bank afschafte en de staatsschuld
kwijtgescholden werd. Van Buren had vier jaar gediend, waarin de
federale overheid onafhankelijk werd van het banksysteem. Echter, de
verkiezingen van 1840 waren een uitzondering. Van Buren werd verslagen
door een ongekend demagogische campagne, geleid door de eerste grote
moderne campagnevoorzitter, Thurlow Weed. Hij was een pionier op het
gebied van campagne-uitingen zoals pakkende slogans, buttons, liedjes en
parades, die ons nu bekend zijn. Weeds tactieken leidden ertoe dat de
onbekende Whig, generaal William Henry Harrison, aan de macht kwam. Dit
was echter een toevallige overwinning. In 1844 waren de Democraten
vastbesloten om met dezelfde campagnetactieken terug te vechten en van
plan om dat jaar het presidentschap terug te winnen. Van Buren zou
natuurlijk de triomfantelijke koers van de Jacksoniaanse beweging
hervatten. Maar toen deed zich een noodlottige gebeurtenis voor: de
Democratische partij werd verdeeld over de cruciale kwestie van de
slavernij, of beter gezegd, de uitbreiding van de slavernij naar nieuwe
gebieden. De veronderstelde makkelijke herverkiezing van Van Buren
strandde door een breuk binnen de partij over de toelating van Texas als
slavenstaat. Van Buren was tegen, Jackson was voor, en deze verdeeldheid
symboliseerde de bredere scheuren binnen de Democratische partij.
Slavernij, een ernstige antiliberale fout binnen het libertarisme van
het Democratische programma, ondermijnde de partij en haar libertarisme
volledig. De Burgeroorlog ging gepaard met ongekend bloedvergieten en
verwoestingen. Het triomferende, bijna eenpartijstelsel van het
Republikeinse regime gebruikte deze situatie om zijn statistische,
voorheen Whig-programma door te voeren. Dit omvatte nationale
regeringsmacht, beschermende invoerrechten, subsidies voor grote
bedrijven, inflatoir papiergeld, een hernieuwde controle van de federale
overheid over het bankwezen, grootschalige interne verbeteringen, hoge
accijnzen en, tijdens de oorlog, dienstplicht en inkomstenbelasting.
Bovendien verloren de staten hun vroegere recht op secessie en andere
bevoegdheden ten opzichte van de federale overheid. Na de oorlog
hervatte de Democratische partij haar libertarische koers. Maar ze
stonden nu voor een veel langere en moeilijkere weg naar vrijheid dan
voorheen. We hebben gezien hoe Amerika de diepste libertarische traditie
heeft ontwikkeld. Deze traditie is nog steeds terug te vinden in veel
van onze politieke retoriek en wordt weerspiegeld in de pittige,
individualistische houding van een groot deel van het Amerikaanse volk
ten opzichte van de overheid. In dit land is er meer vruchtbare grond
voor een heropleving van het libertarisme dan ergens anders.

We kunnen nu vaststellen dat de snelle groei van de libertarische
beweging en de Libertarische Partij in de jaren zeventig stevig
geworteld is in wat Bernard Bailyn beschreef als een krachtige
`permanente erfenis' van de Amerikaanse Revolutie. Maar als deze erfenis
zo essentieel is voor de Amerikaanse traditie, wat is er dan misgegaan?
Waarom is er nu een nieuwe libertarische beweging nodig om de
Amerikaanse droom terug te winnen?

Om deze vraag te beantwoorden, moeten we eerst begrijpen dat het
klassieke liberalisme een serieuze bedreiging vormde voor de politieke
en economische belangen van de heersende klassen. Deze klassen,
waaronder koningen, edelen, aristocraten, bevoorrechte kooplieden,
militaire machtsstructuren en staatsbureaucratieën, profiteerden van de
Oude Orde. Ondanks drie grote gewelddadige revoluties die door liberalen
zijn ontketend --- de Engelse in de 17e eeuw en de Amerikaanse en Franse
in de 18e eeuw --- waren de overwinningen in Europa slechts
gedeeltelijk. Het verzet was sterk en slaagde erin om monopolies op
grondbezit, religieuze instellingen en een oorlogszuchtig buitenlands en
militair beleid in stand te houden. Ook wisten ze het kiesrecht enige
tijd te beperken tot de rijke elite. De liberalen moesten zich richten
op het verbreden van het kiesrecht. Voor beide partijen was immers
duidelijk dat de objectieve economische en politieke belangen van de
meerderheid van de bevolking lagen in individuele vrijheid. Het is
interessant om te vermelden dat aan het begin van de 19e eeuw de
laissez-faire krachten bekend stonden als `liberalen' en `radicalen'
(voor de puristen en consequentere onder hen). De oppositie die de Oude
Orde wilde behouden of ernaar terug wilde keren, werd algemeen aangeduid
als `conservatieven'.

Het conservatisme ontstond in het begin van de negentiende eeuw als een
bewuste poging om het verfoeide werk van de nieuwe klassiek liberale
geest, voortkomend uit de Amerikaanse, Franse en industriële revoluties,
ongedaan te maken. Geïnspireerd door de reactionaire Franse denkers
Bonald en Maistre, streefde het conservatisme ernaar gelijke rechten en
gelijkheid voor de wet te vervangen door de gestructureerde en
hiërarchische heerschappij van bevoorrechte elites. Daarnaast wilden zij
individuele vrijheid en een minimale overheid vervangen door absolute
heerschappij en een grote overheid. Religieuze vrijheid moest wijken
voor de theocratische dominantie van een staatskerk. Vrede en vrije
handel zouden plaatsmaken voor militarisme, mercantilistische
beperkingen en oorlog in het belang van de natiestaat. Ook de opkomst
van industrie en productie moest plaatsmaken voor de oude feodale en
agrarische orde. Het conservatisme wilde bovendien de nieuwe wereld van
massaconsumptie en stijgende levensstandaarden voor iedereen vervangen
door de Oude Orde, waarin de massa een kaal bestaan leidde en de
heersende elite in luxe consumeerde.

Tegen het midden van de negentiende eeuw begonnen de conservatieven in
te zien dat hun zaak gedoemd was als ze bleven vasthouden aan de
volledige afschaffing van de Industriële Revolutie en de enorme stijging
van de levensstandaard van de massa. Ook hun verzet tegen de verruiming
van het kiesrecht plaatste hen openlijk tegenover de belangen van het
publiek. Daarom besloot de `rechtervleugel' --- een term die verwijst
naar de plaats waar de vertegenwoordigers van de Oude Orde tijdens de
Franse Revolutie in de vergaderzaal zaten --- van koers te veranderen.
Ze actualiseerden hun statische credo door hun directe verzet tegen
industrialisme en democratisch stemrecht op te geven. In plaats van de
openlijke haat en minachting van het oude conservatisme voor de massa,
gebruikten de nieuwe conservatieven dubbelhartigheid en demagogie. Ze
maakten de massa het hof met de boodschap: `Ook wij zijn voorstander van
industrialisatie en een hogere levensstandaard. Maar om deze doelen te
bereiken moeten we de industrie reguleren voor het algemeen belang; we
moeten georganiseerde samenwerking opstellen in plaats van het
natuurwetmatige 'eet-of-wordt-gegeten' van de vrije en concurrerende
markt. En bovenal moeten we de natievernietigende liberale
leerstellingen van vrede en vrije handel vervangen door de
natieverheerlijkende maatregelen van oorlog, protectionisme, imperium en
militaire dapperheid.' Voor al deze veranderingen was natuurlijk een
grote overheid nodig in plaats van een minimale overheid.

En zo keerden aan het eind van de negentiende eeuw het statisme en de
grote overheid terug, maar nu met een pro-industrieel en
pro-welzijnsbeleid. De Oude Orde kwam terug, dit keer echter met een
andere samenstelling van begunstigden. Het waren niet meer de
aristocratie, de feodale landheren, het leger, de bureaucratie en de
bevoorrechte kooplieden, maar nu vooral het leger, de bureaucratie, de
verzwakte feodale landheren en vooral de bevoorrechte fabrikanten. Nieuw
Rechts, onder leiding van Bismarck in Pruisen, vormde een rechts
collectivisme dat was gebaseerd op oorlog, militarisme, protectionisme
en verplichte kartelvorming binnen het bedrijfsleven en de industrie.
Dit leidde tot een gigantisch netwerk van controles, voorschriften,
subsidies en privileges, dat een groot partnerschap vormde tussen de
Grote Overheid en specifieke bevoorrechte elementen in de bedrijfswereld
en de industrie.

Er moest iets gedaan worden aan het nieuwe fenomeen van een massaal
aantal industriële loonarbeiders, het zogenaamde `proletariaat'.
Gedurende de achttiende en vroege negentiende eeuw, en zelfs tot het
einde van de negentiende eeuw, steunde de meerderheid van de arbeiders
het laissez-faire-principe en de vrije concurrerende markt. Zij zagen
dit als de beste optie voor hun lonen en arbeidsomstandigheden, en voor
een groeiend aanbod van goedkope consumptiegoederen. Zelfs de vroege
vakbonden in Groot-Brittannië waren sterke voorstanders van
laissez-faire. Nieuwe conservatieven, geleid door Bismarck in Duitsland
en Disraeli in Groot-Brittannië, verzwakten de libertarische wil van de
arbeiders. Ze deden dit door krokodillentranen te laten om de
omstandigheden van de industriële arbeiders, terwijl ze tegelijk de
industrie karteliseerden en reguleren, wat efficiënte concurrentie niet
per ongeluk belemmerde. In het begin van de twintigste eeuw werd de
nieuwe conservatieve `bedrijfsstaat' --- het dominante politieke systeem
in de Westerse wereld, toen en nu --- verrijkt met zogenaamde
`verantwoordelijke' en corporatistische vakbonden. Deze vakbonden
fungeerden als junior partners van de Grote Overheid en bevoordeelden
grote bedrijven binnen het nieuwe statistische en corporatistische
besluitvormingssysteem.

Om dit nieuwe systeem te vestigen en een Nieuwe Orde te creëren die een
gemoderniseerde versie was van het ancien régime van voor de Amerikaanse
en Franse revoluties, moesten de nieuwe heersende elites een gigantisch
bedrog plegen op het misleide publiek. Dit bedrog houdt tot op de dag
van vandaag stand. Terwijl het bestaan van elke regering, of het nu een
absolute monarchie of een militaire dictatuur is, steunt op de
instemming van de meerderheid van het publiek, moet een democratische
regering deze instemming op een meer directe en dagelijkse manier zien
te verwerven. Om dat te bereiken, moesten de nieuwe conservatieve elites
het publiek op cruciale en fundamentele manieren zien te paaien. De
massa's moesten ervan overtuigd worden dat tirannie beter was dan
vrijheid, dat een gekartelliseerd en geprivilegieerd industrieel
feodalisme voordeliger was voor consumenten dan een vrije, concurrerende
markt, dat een gekartelliseerd monopolie moest worden opgelegd in naam
van antimonopolie, en dat oorlog en militaire verrijking, ten goede van
de heersende elites, werkelijk in het belang waren van het
dienstplichtige, belaste en vaak afgeslachte publiek. Hoe zou dit alles
gerealiseerd moeten worden?

In alle samenlevingen wordt de publieke opinie gevormd door de
intellectuele klassen, de opiniemakers van de maatschappij. De meeste
mensen bedenken of verspreiden zelf geen ideeën en concepten; in
tegendeel, ze nemen vaak de ideeën over die door professionele
intellectuelen en handelaars in gedachten worden gepromoot. Door de
geschiedenis heen, zoals we verderop zullen zien, hebben despoten en
heersende elites veel meer behoefte gehad aan de diensten van
intellectuelen dan aan vreedzame burgers in een vrije samenleving.
Staten hebben altijd opiniërende intellectuelen nodig gehad om het
publiek te laten geloven dat hun heerschappij wijs, goed en
onvermijdelijk is; ze moeten geloven dat de `keizer kleren heeft'. Tot
aan de moderne tijd waren dergelijke intellectuelen doorgaans
kerkelijken of toverdokters, de hoeders van de religie. Dit eeuwenoude
partnerschap tussen kerk en staat was een hechte samenwerking. De kerk
vertelde haar misleide volgelingen dat de koning regeerde op goddelijke
aanwijzing en daarom gehoorzaamd moest worden. In ruil daarvoor gaf de
koning een aanzienlijk deel van de belastinginkomsten aan de kerk.
Daarom was het van groot belang voor de libertarische klassiek-liberalen
om successen te behalen in de scheiding van Kerk en Staat. De nieuwe
liberale wereld was er een waarin intellectuelen seculier konden zijn en
in hun eigen levensonderhoud konden voorzien, los van staatsubsidies.

Om hun nieuwe statistische orde en neomercantilistische corporatieve
staat te vestigen, moesten de nieuwe conservatieven een nieuwe alliantie
smeden tussen intellectuelen en de staat. In een steeds seculierder
tijdperk betekende dit vooral samenwerking met seculiere intellectuelen
in plaats van met kerkmensen. Specifiek zochten ze hierbij de
samenwerking van een nieuwe generatie professoren, promovendi,
historici, leraren, technocratische economen, maatschappelijk werkers,
sociologen, artsen en ingenieurs. Deze nieuwe alliantie bestond uit twee
delen. In het begin van de negentiende eeuw vertrouwden de
conservatieven, die zich lieten overtreffen door hun liberale
tegenstanders, sterk op de vermeende deugden van irrationaliteit,
romantiek, traditie en theocratie. Door de nadruk te leggen op traditie
en irrationele symbolen konden de conservatieven het publiek verleiden
om de bevoorrechte hiërarchische heerschappij voort te zetten en om de
natiestaat en zijn oorlogsmachine te blijven aanbidden. In het laatste
deel van de negentiende eeuw nam het nieuwe conservatisme de schijn van
rede en `wetenschap' aan. Nu was het de wetenschap die beweerde dat de
economie en de maatschappij bestuurd moesten worden door technocratische
`experts'. In ruil voor het verspreiden van deze boodschap aan het
publiek werden de nieuwe generatie intellectuelen beloond met banen en
prestige als verdedigers van de Nieuwe Orde, evenals als planners en
regelaars van de nieuw gekarteliseerde economie en maatschappij.

Om de dominantie van het nieuwe statisme over de publieke opinie te
waarborgen en de instemming van het publiek te vormen, begonnen de
regeringen in de Westerse wereld aan het eind van de negentiende en het
begin van de twintigste eeuw met het beheersen van het onderwijs. Ze
wilden de ideeën en opvattingen van de mensen beïnvloeden, en dit deden
ze door middel van leerplichtwetten en een netwerk van openbare scholen.
De openbare scholen werden doelbewust ingezet om jonge leerlingen
gehoorzaamheid aan de staat en andere burgerdeugden bij te brengen. Deze
statisering van het onderwijs leidde er bovendien toe dat leraren en
professionele onderwijskundigen uitgroeiden tot een van de belangrijkste
belangen in de uitbreiding van het statisme.

Een van de manieren waarop de nieuwe statistische intellectuelen te werk
gingen, was door de betekenis van oude etiketten te veranderen. Op deze
manier konden ze de emotionele connotaties die aan deze etiketten
verbonden waren, manipuleren in de hoofden van het publiek. De
laissez-faire libertariërs stonden lange tijd bekend als `liberalen',
terwijl de meest radicale onder hen als `radicalen' werden aangeduid. Ze
werden ook als `progressieven' gezien, omdat zij de aanhangers waren van
industriële vooruitgang, vrijheid en een stijgende levensstandaard voor
consumenten. De nieuwe generatie statistische academici en
intellectuelen maakte zich de termen `liberaal' en `progressief' eigen.
Ze slaagden erin om hun laissez-faire tegenstanders als ouderwets,
`neanderthaler' en `reactionair' te bestempelen. Zelfs de term
`conservatief' werd aan de klassiek liberalen opgelegd. Zoals we hebben
gezien, konden de nieuwe statisten zich ook het begrip `rede'
toe-eigenen.

Als de laissez-faire liberalen verward waren door de opkomst van het
statisme en mercantilisme, dat zich als `progressief' corporatief
statisme manifesteerde, was er een andere reden voor de achteruitgang
van het klassieke liberalisme aan het eind van de negentiende eeuw: de
groei van het socialisme. Dit idee deed zijn intrede in de jaren 1830 en
kende na de jaren 1880 een enorme uitbreiding. Het bijzondere aan het
socialisme was dat het een verwarrende, hybride beweging was, beïnvloed
door de twee al bestaande polaire ideologieën: liberalisme en
conservatisme. Van de klassiek-liberalen namen de socialisten een
openlijke acceptatie van industrialisme en de Industriële Revolutie
over. Ze eerden de `wetenschap' en de `rede' en toonden ten minste een
retorische toewijding aan klassieke liberale idealen zoals vrede,
individuele vrijheid en een stijgende levensstandaard. Inderdaad, de
socialisten waren, lang vóór de latere corporatisten, pioniers in het
omarmen van wetenschap, rede en industrialisme. Ze namen niet alleen de
klassiek-liberale toewijding aan democratie over, maar breidden deze uit
met de eis voor een `uitgebreide democratie', waarin `het volk' de
economie en elkaar zou besturen.

Aan de andere kant namen de socialisten van de conservatieven de
toewijding aan dwang en statistische middelen over om hun liberale
doelen te bereiken. Industriële harmonie en groei moesten worden
gerealiseerd door de staat te laten uitgroeien tot een almachtig
instituut dat de economie en de maatschappij regeerde in naam van de
`wetenschap'. Een voorhoede van technocraten zou de volledige controle
over ieders leven en bezit overnemen, zogenaamd in naam van het `volk'
en de `democratie'. Niet tevreden met de liberale verworvenheden van
rede en vrijheid voor wetenschappelijk onderzoek, zou de socialistische
staat de heerschappij van wetenschappers over de rest invoeren. Evenmin
waren de socialisten tevreden met liberalen die arbeiders de vrijheid
gaven om onvoorstelbare welvaart te bereiken. De socialistische staat
zou in plaats daarvan de heerschappij van arbeiders over anderen
instellen - of liever, de heerschappij van politici, bureaucraten en
technocraten in hun naam. Ook het liberale credo van gelijkheid van
rechten en gelijkheid voor de wet zou door de socialistische staat met
voeten getreden worden. In plaats daarvan streefde men naar het
monsterlijke en onmogelijke doel van gelijkheid of uniformiteit in
resultaten. Of liever gezegd, er zou een nieuwe bevoorrechte elite en
nieuwe klasse worden opgericht, zogenaamd in naam van het realiseren van
deze onmogelijke gelijkheid.

Het socialisme was een verwarrende en hybride beweging. Het probeerde de
liberale idealen van vrijheid, vrede en industriële harmonie en groei te
realiseren. Deze doelen kunnen echter alleen worden bereikt door
vrijheid en de scheiding van de overheid van vrijwel alle aspecten van
het leven. De socialisten hanteerden hiervoor oude conservatieve
middelen zoals statisme, collectivisme en hiërarchische privileges. Dit
was een beweging die gedoemd was om te falen, en dat gebeurde ook in
veel landen waar socialisten in de twintigste eeuw aan de macht kwamen.
De massa's werden alleen geconfronteerd met ongekend wanbeleid,
hongersnood en armoede.

Maar het ergste van de opkomst van de socialistische beweging was dat
deze in staat was de klassiek-liberalen `ter linkerzijde' te
overvleugelen. Hiermee bedoel ik dat ze de partij van hoop, radicalisme
en revolutie in de Westerse wereld werden. Zoals de verdedigers van het
ancien régime tijdens de Franse Revolutie rechts in de zaal zaten, zo
bevonden de liberalen en radicalen zich aan de linkerkant. Van dat
moment tot de opkomst van het socialisme werden de libertarische
klassiek-liberalen als `links', zelfs `extreem links', beschouwd in het
ideologische spectrum. Zelfs in 1848 zaten militante laissez-faire
Franse liberalen zoals Frédéric Bastiat aan de linkerkant in de
nationale vergadering. De klassiek-liberalen waren begonnen als de
radicale, revolutionaire partij in het Westen, als de partij van hoop en
verandering voor vrijheid, vrede en vooruitgang. Het was een ernstige
strategische fout om zich te laten overvleugelen en de socialisten de
indruk te geven dat zij de `partij van links' waren. Dit leidde ertoe
dat de liberalen ten onrechte in een verwarde middenpositie werden
geplaatst, met socialisme en conservatisme als de tegengestelde polen.
Aangezien het libertarisme niets anders is dan een partij van
verandering en vooruitgang in de richting van vrijheid, betekende het
opgeven van die rol het verlies van een groot deel van hun bestaansrecht
- zowel in de realiteit als in de perceptie van het publiek.

Maar dit alles had niet kunnen gebeuren als de klassiek-liberalen
zichzelf niet van binnenuit hadden laten verzwakken. Ze hadden kunnen
benadrukken -- zoals sommigen van hen inderdaad deden -- dat het
socialisme een verwarde en intern inconsistente beweging was, een
quasi-conservatieve stroming die absolute monarchie en feodalisme een
modern gezicht gaf. Zijzelf waren nog steeds de enige echte radicalen,
onverschrokken mensen die zich vasthielden aan niets minder dan de
volledige overwinning van het libertarische ideaal.

\section{Verval van binnenuit}\label{verval-van-binnenuit}

Het socialisme was een verwarrende en hybride beweging. Het probeerde de
liberale idealen van vrijheid, vrede en industriële harmonie te
realiseren. Deze doelen kunnen echter alleen worden bereikt door
vrijheid en de scheiding van de overheid van vrijwel alles. De
socialisten maakten gebruik van oude conservatieve methoden zoals
statisme, collectivisme en hiërarchische privileges. Dit was een
gedoemde beweging die in veel landen waar ze in de twintigste eeuw aan
de macht kwamen, alleen maar leidde tot wanbeleid, hongersnood en
armoede. Het meest problematische aan de opkomst van het socialisme was
dat deze beweging de klassiek-liberalen `ter linkerzijde' overtrof.
Daarmee bedoel ik dat ze de partij werden van hoop, radicalisme en
revolutie in de Westerse wereld. Tijdens de Franse Revolutie zaten de
verdedigers van het ancien régime rechts in de zaal, terwijl de
liberalen en radicalen aan de linkerkant zaten. Van dat moment totdat
het socialisme opkwam, werden de libertarische klassiek-liberalen als
`links', zelfs `extreem links', beschouwd in het ideologische spectrum.
In 1848 bijvoorbeeld zaten actieve laissez-faire Franse liberalen zoals
Frédéric Bastiat aan de linkerkant van de nationale vergadering. De
klassiek-liberalen waren in het Westen begonnen als de radicale,
revolutionaire partij voor vrijheid, vrede en vooruitgang. Het was een
ernstige strategische fout om de socialisten de kans te geven zich als
de `partij van links' te presenteren. Hierdoor kwamen de liberalen in
een verwarrende middenpositie terecht, met socialisme en conservatisme
als tegengestelde polen. Aangezien het libertarisme niets meer is dan
een beweging voor verandering en vooruitgang richting vrijheid,
betekende het opgeven van die rol het verlies van een groot deel van hun
bestaansrecht -- zowel in de werkelijkheid als in de perceptie van het
publiek. Dit alles had echter niet kunnen gebeuren als de
klassiek-liberalen zichzelf niet van binnenuit hadden verzwakt. Ze
hadden kunnen benadrukken -- en sommigen deden dat ook -- dat het
socialisme een verwarde en intern inconsistente beweging was, een
quasi-conservatieve stroming die absolute monarchie en feodalisme een
modern gezicht gaf. Zij zelf waren nog steeds de enige echte radicalen,
onverschrokken mensen die streefden naar niets minder dan de volledige
overwinning van het libertarische ideaal.

Maar na het behalen van indrukwekkende gedeeltelijke overwinningen op
het statisme, begonnen de klassiek-liberalen hun radicalisme te
verliezen. Hun vasthoudendheid in de strijd tegen het conservatieve
statisme, gericht op een definitieve overwinning, nam af. In plaats van
deze gedeeltelijke overwinningen te gebruiken als basis voor verdere
druk, raakten de klassiek-liberalen hun ijver voor verandering en het
zuiveren van principes kwijt. Ze begrepen zich tevreden met het behouden
van hun bestaande overwinningen en transformeerden van een radicale naar
een conservatieve beweging. Met `conservatief' doel ik op het tevreden
zijn met het handhaven van de status quo. Kortom, de liberalen lieten
het veld open voor het socialisme om de partij van hoop en radicalisme
te worden. Zelfs latere corporatisten konden zich voordoen als
`liberalen' en `progressieven', in tegenstelling tot de `extreem
rechtse' en `conservatieve' klassiek-liberalen. De laatsten lieten zich
in een positie dwingen waarin ze niets meer hoopten dan op stilstand,
een afwezigheid van verandering. Zo'n strategie is dwaas en onhoudbaar
in een wereld die voortdurend verandert.

De degeneratie van het liberalisme betrof niet alleen de houding en
strategie, maar ook de principes. De liberalen namen het namelijk voor
lief dat de staat de macht om oorlog te voeren, het onderwijs, het geld
en bankieren, en de wegen in handen hield. Kortom, ze lieten de controle
over alle cruciale machtsfactoren in de samenleving aan de staat over.
In tegenstelling tot de totale vijandigheid van de achttiende-eeuwse
liberalen jegens de uitvoerende macht en de bureaucratie, tolereerden en
verwelkomden de negentiende-eeuwse liberalen zelfs de opschaling van de
macht van de uitvoerende macht en de oprichting van een diepgewortelde
oligarchische ambtenarenbureaucratie.

Bovendien versmolten principe en strategie in het verval van de
achttiende-eeuwse en vroeg-negentiende-eeuwse liberale inzet voor het
`abolitionisme.' Dit was de overtuiging dat, ongeacht of het ging om
slavernij of een ander aspect van het statisme, deze instituties zo snel
mogelijk moesten worden afgeschaft. De onmiddellijke afschaffing van het
statisme, hoewel in de praktijk onwaarschijnlijk, moest worden
nagestreefd als de enige morele optie. Het kiezen voor een geleidelijke
afschaffing in plaats van een onmiddellijke afschaffing van een
kwaadwillende en dwingende instelling, betekent het goedkeuren van dat
kwaad en daarmee het schenden van libertarische principes. Zoals de
grote abolitionist en libertariër William Lloyd Garrison zei: `Hoe
oprecht we ook aandringen op onmiddellijke afschaffing, het zal, helaas,
uiteindelijk een geleidelijke afschaffing zijn. We hebben nooit gezegd
dat slavernij in één klap omvergeworpen zou worden; dat het zo zou
moeten zijn, zullen we altijd beweren.'

Er waren twee belangrijke veranderingen in de filosofie en ideologie van
het klassieke liberalisme. Deze veranderingen vormden zowel een
voorbeeld van als een bijdrage aan het verval van het liberalisme als
een vitale, progressieve en radicale kracht in de Westerse wereld. De
eerste en meest significante verandering was het loslaten van de
filosofie van natuurlijke rechten, die vervangen werd door
technocratisch utilitarisme. In plaats van vrijheid te baseren op de
dwingende moraal van het recht van elk individu op vrijheid en eigendom,
werd vrijheid volgens het utilitarisme gezien als de beste manier om een
vaag gedefinieerd algemeen welzijn te bereiken. Deze verschuiving van
natuurrechten naar utilitarisme had twee ernstige gevolgen. Ten eerste
werd de zuiverheid van het doel en de consistentie van het principe
onvermijdelijk aangetast. De natuurrechtlibertariër, die zich inzet voor
moraliteit en gerechtigheid, strijdt hardnekkig voor het pure principe.
De utilitarist daarentegen waardeert vrijheid slechts als een tijdelijk
nuttig middel. Wat tijdelijk nuttig is, kan echter met de wind
meewaaien, en dat doet het ook. Hierdoor wordt het voor de utilitarist
in zijn koele berekening van kosten en baten gemakkelijk om in tal van
gevallen voor statisme te kiezen en zo principes op te geven. Dit
gebeurde precies met de Benthamitische utilitaristen in Engeland.
Aanvankelijk begonnen ze met ad hoc-libertarisme en laissez-faire, maar
al snel vonden ze het steeds gemakkelijker om naar statisme af te
glijden. Een voorbeeld hiervan was hun streven naar een `efficiënt'
ambtenarenapparaat en uitvoerende macht. Deze efficiëntie kreeg
prioriteit en verving zelfs elk begrip van rechtvaardigheid of recht.

Ten tweede, en net zo belangrijk, is het zeldzaam om een utilitariër te
vinden die ook radicaal is en ijvert voor de onmiddellijke afschaffing
van kwaad en dwang. Utilitaristen, met hun focus op doelmatigheid,
verzetten zich bijna onvermijdelijk tegen verstoringen of radicale
veranderingen. Er zijn geen utilitaristische revolutionairen geweest.
Daarom zijn utilitaristen nooit onmiddellijk abolitionisten. De
abolitionist streeft ernaar misstanden en onrecht zo snel mogelijk uit
te bannen. Bij het nastreven van dit doel is er geen ruimte voor koele,
ad hoc afwegingen van kosten en baten. Hierdoor lieten de klassieke
liberale utilitaristen het radicalisme achter zich en werden ze slechts
geleidelijke hervormers. Maar door hervormers te worden, plaatsten ze
zichzelf ook onvermijdelijk in de rol van adviseurs en
efficiëntiedeskundigen voor de staat. Met andere woorden, ze lieten
zowel het libertarische principe als een principiële libertarische
strategie los. De utilitaristen eindigden als apologeten van de
bestaande orde, van de status quo, en waren daardoor kwetsbaar voor de
beschuldigingen van socialisten en progressieve corporatisten dat ze
enkel bekrompen en conservatieve tegenstanders van verandering waren. Zo
begonnen ze als radicalen en revolutionairen, de tegenhangers van de
conservatieven, en eindigden ze als het toonbeeld van waartegen ze
hadden gevochten.

Deze utilitaristische verlamming van het libertarisme is nog steeds aan
de gang. In de begindagen van het economische denken heeft het
utilitarisme, onder invloed van Bentham en Ricardo, de vrije
markteconomie veroverd. Deze invloed is vandaag de dag nog steeds sterk
aanwezig. De huidige vrije markteconomie zit vol met oproepen tot
gradualisme, met minachting voor ethiek, rechtvaardigheid en consistente
principes. Daarnaast is er een bereidheid om de principes van de vrije
markt op te geven zodra de kosten-batenverhouding in het geding komt.
Daarom wordt de huidige vrije markteconomie door intellectuelen vaak
gezien als een verdediging van een enigszins gewijzigde status quo. En
die beschuldiging is te vaak terecht.

Een tweede, versterkende verandering in de ideologie van
klassiek-liberalen vond plaats aan het eind van de negentiende eeuw.
Toen namen ze, althans voor een paar decennia, de theorieën van het
sociaal evolutionisme over, vaak aangeduid als `sociaal Darwinisme'.
Statistische historici hebben sociaal-darwinistische
laissez-faire-liberalen zoals Herbert Spencer en William Graham Sumner
vaak neergezet als wrede voorvechters van de uitroeiing of in ieder
geval de verdwijning van de sociaal `ongeschikten'. Veel van deze ideeën
waren simpelweg een versiering van de gezonde economische en
sociologische opvattingen over de vrije markt, gecast in de populaire
mode van het evolutionisme. Maar het werkelijk belangrijke en
verlammende aspect van hun sociaal-darwinisme was de onwettige
overdracht naar de sociale sfeer van de opvatting dat soorten, of later
genen, zeer langzaam veranderen over duizenden jaren. De
sociaal-darwinistische liberaal liet dus het idee van revolutie of
radicale verandering varen. In plaats daarvan koos hij ervoor om
achterover te leunen en te wachten op de onvermijdelijke, kleine
evolutionaire veranderingen die zich over eeuwen zouden voortzetten.
Kortom, negerend dat het liberalisme de macht van de heersende elites
had moeten doorbreken door een reeks radicale veranderingen en
revoluties, werden de sociaal-darwinisten conservatieven. Zij predikten
tegen alle radicale maatregelen en waren alleen voorstander van de meest
minutieuze, geleidelijke veranderingen.

In feite is de grote libertariër Spencer een fascinerend voorbeeld van
de veranderingen binnen het klassieke liberalisme, een ontwikkeling die
in de Verenigde Staten wordt weerspiegeld door William Graham Sumner.
Herbert Spencer belichaamt in zekere zin een groot deel van het verval
van het liberalisme in de negentiende eeuw. Hij begon als een zeer
radicale liberaal, bijna als een pure libertariër. Maar zodra het virus
van de sociologie en het sociaal-darwinisme zijn geest binnendrong, liet
Spencer het libertarisme als dynamische, radicale historische beweging
achter zich. Hij schudde het echter niet louter theoretisch van zich af.
Terwijl hij uitkeek naar een uiteindelijke overwinning van pure
vrijheid, waarbij `contract' het zou winnen van `status' en industrie
van militarisme, begon Spencer deze overwinning als onvermijdelijk te
beschouwen. Maar hij dacht dat dit pas zou plaatsvinden na millennia van
geleidelijke evolutie. Daarom liet hij het liberalisme als een
strijdbaar en radicaal credo achter zich. In de praktijk beperkte hij
zijn liberalisme tot een vermoeiend en conservatief verzet tegen het
groeiende collectivisme en het statisme van zijn tijd.

Maar als het utilitarisme, gesteund door het sociaal-darwinisme, de
belangrijkste reden voor het filosofische en ideologische verval van de
liberale beweging was, dan was het opgeven van de strikte principes
tegen oorlog, imperialisme en militarisme de meest catastrofale oorzaak
voor haar ondergang. In land na land werd het klassieke liberalisme
vernietigd door de aanlokkelijke roep van de natiestaat en het imperium.
In Engeland verloochenenden liberalen aan het eind van de negentiende en
het begin van de twintigste eeuw het anti-oorlogs- en
anti-imperialistische `Little Englandisme' van Cobden, Bright en de
Manchester School. In plaats daarvan omarmden ze het ongegeneerd
getitelde `Liberal Imperialism' en sloten zich aan bij de conservatieven
bij de uitbreiding van het imperium. Ook gingen ze samenwerken met
conservatieven en rechtse socialisten in het destructieve imperialisme
en collectivisme van de Eerste Wereldoorlog. In Duitsland slaagde
Bismarck erin om de eerder bijna triomferende liberalen te splitsen door
hen te verleiden met de eenwording van Duitsland via bloed en ijzer. In
beide landen leidde dit tot de ondergang van de liberale zaak.

In de Verenigde Staten was de klassieke liberale partij lange tijd de
Democratische Partij. In de tweede helft van de negentiende eeuw werd
deze vaak aangeduid als `de partij van de persoonlijke vrijheid'. Het
was niet alleen een verdediger van persoonlijke vrijheid, maar ook van
economische vrijheid. De partij was een standvastige tegenstander van
het drankverbod, de blauwe zondagswetten en de leerplicht. Daarnaast
streefde ze naar vrije handel, hard geld (het vermijden van inflatie
door de overheid), de scheiding van bankwezen en staat, en een minimaal
overheidsoptreden. De Democratische Partij beschouwde staatsmacht als
verwaarloosbaar en de federale macht als vrijwel onbestaand. Op het
gebied van buitenlands beleid was de Democratische Partij, hoewel minder
rigoureus, doorgaans een voorvechter van vrede, antimilitarisme en
anti-imperialisme. Echter, zowel het persoonlijke als het economische
libertarisme werden verlaten toen de Democratische Partij in 1896 werd
overgenomen door de krachten van Bryan. Twee decennia later werd het
beleid van non-interventie ruw aan de kant geschoven door Woodrow
Wilson. Deze interventie en de daaropvolgende oorlog leidden tot een
eeuw van dood en verwoesting, met voortdurende conflicten en nieuwe
heerschappijen. Ook bevorderden ze in alle oorlogvoerende landen een
nieuw corporatistisch statisme, een welvaartsstaat waarin grote
regeringen, bedrijven, vakbonden en intellectuelen samenwerken.

De laatste adempauze van het oude laissez-faire liberalisme in Amerika
werd gevormd door moedige en ouder wordende libertariërs. Rond de
eeuwwisseling richtten zij samen de Anti-Imperialist League op om de
Amerikaanse oorlog tegen Spanje en de daaropvolgende imperialistische
oorlog tegen de Filippino's te bestrijden. Deze oorlogen hadden als doel
nationale onafhankelijkheid te verwerven, zowel van Spanje als van de
Verenigde Staten. Tegenwoordig lijkt het misschien vreemd dat een
anti-imperialist geen marxist is, maar het verzet tegen imperialisme
begon bij laissez-faire liberalen zoals Cobden en Bright in Engeland en
Eugen Richter in Pruisen. In feite bestond de Anti-Imperialist League,
onder leiding van de industrieel en econoom Edward Atkinson uit Boston
(en inclusief Sumner), grotendeels uit laissez-faire radicalen. Zij
hadden al eerder strijd gevoerd voor de afschaffing van de slavernij en
waren daarna voorstanders van vrije handel, hard geld en een minimale
overheid. Voor hen vormde de strijd tegen het nieuwe Amerikaanse
imperialisme een voortzetting van hun levenslange verzet tegen dwang,
statisme en onrecht. Dit was hun verzet tegen de grote overheid op elk
gebied van het leven, zowel binnenlands als buitenlands.

We hebben het nogal akelige verhaal gevolgd van de ondergang van het
klassieke liberalisme, na zijn opkomst en gedeeltelijke triomf in de
vorige eeuwen. Wat is dan de reden voor de opleving en bloei van
libertarische ideeën en activiteiten in de afgelopen jaren, vooral in de
Verenigde Staten? Hoe konden deze krachtige krachten en coalities voor
het statisme zoveel toegeven aan een herrezen libertarische beweging?
Zou de hernieuwde opmars van het statisme aan het eind van de
negentiende en in de twintigste eeuw niet eerder aanleiding tot
somberheid moeten geven dan een herleving van een schijnbaar op sterven
na dood libertarisme? Waarom bleef het libertarisme niet gewoon dood en
begraven?

We hebben gezien waarom het libertarisme van nature als eerste en het
sterkst kon opkomen in de Verenigde Staten, een land met een rijke
libertarische traditie. Maar we hebben nog niet gekeken naar de vraag:
waarom heeft het libertarisme de afgelopen jaren een renaissance
doorgemaakt? Welke hedendaagse omstandigheden hebben bijgedragen aan
deze verrassende ontwikkeling? We zullen deze vraag uitstellen tot het
einde van het boek, totdat we eerst hebben onderzocht wat het
libertarische credo is en hoe we dat credo kunnen toepassen om de
belangrijkste probleemgebieden in onze samenleving aan te pakken.

\textbf{Deel I}

\subsection{Het libertarische credo We hebben het best een zorgwekkende
verhaal gevolgd over de neergang van het klassieke liberalisme, na de
opkomst en gedeeltelijke triomf in voorgaande eeuwen. Wat veroorzaakt
dan de opleving en bloei van libertarische ideeën en activiteiten in de
afgelopen jaren, vooral in de Verenigde Staten? Hoe konden deze
krachtige krachten en coalities van statisme in hemelsnaam zoveel
toegeven aan een herrezen libertarische beweging? Zou de weer opkomende
invloed van het statisme aan het eind van de negentiende en in de
twintigste eeuw niet eerder reden tot somberheid moeten zijn dan een
herleving van een schijnbaar stervend libertarisme? Waarom bleef het
libertarisme niet gewoon dood en begraven? We hebben gezien waarom het
libertarisme van nature het eerst en het sterkst kon ontstaan in de
Verenigde Staten, een land met een rijke libertarische traditie. Maar we
hebben nog niet bekeken waarom het libertarisme de afgelopen jaren een
renaissance heeft doorgemaakt. Welke hedendaagse omstandigheden hebben
geleid tot deze verrassende ontwikkeling? We stellen het beantwoorden
van deze vraag uit tot het einde van het boek. Eerst moeten we
onderzoeken wat het libertarische credo inhoudt en hoe we dit credo
kunnen toepassen om belangrijke probleemgebieden in onze samenleving aan
te
pakken.}\label{het-libertarische-credo-we-hebben-het-best-een-zorgwekkende-verhaal-gevolgd-over-de-neergang-van-het-klassieke-liberalisme-na-de-opkomst-en-gedeeltelijke-triomf-in-voorgaande-eeuwen.-wat-veroorzaakt-dan-de-opleving-en-bloei-van-libertarische-ideeuxebn-en-activiteiten-in-de-afgelopen-jaren-vooral-in-de-verenigde-staten-hoe-konden-deze-krachtige-krachten-en-coalities-van-statisme-in-hemelsnaam-zoveel-toegeven-aan-een-herrezen-libertarische-beweging-zou-de-weer-opkomende-invloed-van-het-statisme-aan-het-eind-van-de-negentiende-en-in-de-twintigste-eeuw-niet-eerder-reden-tot-somberheid-moeten-zijn-dan-een-herleving-van-een-schijnbaar-stervend-libertarisme-waarom-bleef-het-libertarisme-niet-gewoon-dood-en-begraven-we-hebben-gezien-waarom-het-libertarisme-van-nature-het-eerst-en-het-sterkst-kon-ontstaan-in-de-verenigde-staten-een-land-met-een-rijke-libertarische-traditie.-maar-we-hebben-nog-niet-bekeken-waarom-het-libertarisme-de-afgelopen-jaren-een-renaissance-heeft-doorgemaakt.-welke-hedendaagse-omstandigheden-hebben-geleid-tot-deze-verrassende-ontwikkeling-we-stellen-het-beantwoorden-van-deze-vraag-uit-tot-het-einde-van-het-boek.-eerst-moeten-we-onderzoeken-wat-het-libertarische-credo-inhoudt-en-hoe-we-dit-credo-kunnen-toepassen-om-belangrijke-probleemgebieden-in-onze-samenleving-aan-te-pakken.}

\bookmarksetup{startatroot}

\chapter{Bezit en uitwisseling}\label{bezit-en-uitwisseling}

In onze samenleving hebben bezit en uitwisseling een cruciale rol. Bezit
bepaalt vaak onze status en ons welzijn. We identificeren onszelf met
wat we bezitten, van materiële goederen tot immateriële zaken zoals
kennis en vaardigheden. Dit zorgt ervoor dat we ons onderscheidend
voelen ten opzichte van anderen. Aan de andere kant is uitwisseling een
fundament van sociale interactie. Het delen van middelen en ideeën
verbindt ons met anderen. Door uitwisseling ontstaan nieuwe sociale
netwerken en samenwerkingen. Dit proces bevordert niet alleen
persoonlijke groei, maar ook maatschappelijke ontwikkeling. Het is
interessant om te zien hoe deze twee concepten met elkaar samenhangen.
Aan de ene kant hebben we de behoefte om te bezitten, terwijl we aan de
andere kant ook de drang voelen om te delen. Dit spanningsveld tussen
bezit en uitwisseling vormt de basis voor veel van ons gedrag en onze
overtuigingen. Door deze dynamiek begrijpen we beter hoe ons economisch
en sociaal leven is opgebouwd. Bezit en uitwisseling zijn niet slechts
economische begrippen; ze zijn diepgeworteld in de menselijke ervaring.

\section{HET NON-AGRESSIE PRINCIPE}\label{het-non-agressie-principe}

Het non-agressie principe is een fundamenteel idee binnen de
libertarische filosofie. Dit beginsel stelt dat geweld of dwang nooit
gerechtvaardigd zijn, tenzij het gaat om zelfverdediging. Het omvat de
overtuiging dat elk individu recht heeft op zijn of haar eigen leven,
eigendom en vrijheid. Zodra iemand deze rechten schendt door geweld te
gebruiken tegen een ander, verstoort deze persoon de sociale orde. De
vrijheid van een individu eindigt waar die van een ander begint. Dit
principe is niet alleen van toepassing op individuen, maar ook op
overheden en instellingen. Overheden mogen geen geweld gebruiken tegen
hun burgers, tenzij het om zelfverdediging gaat. Het non-agressie
principe bevordert een samenleving waarin mensen op vreedzame wijze met
elkaar om kunnen gaan. Door deze gedragscode kunnen mensen hun
meningsverschillen oplossen zonder geweld. Dit leidt tot een positieve
en productieve omgeving waar samenwerking en wederzijds respect centraal
staan. In wezen roept het non-agressie principe op tot verantwoordelijk
gedrag. Het vraagt mensen om anderen te respecteren en hun rechten te
waarderen, wat uiteindelijk bijdraagt aan een harmoniën samenleving.

Het libertarische credo is gebaseerd op één centraal principe: geen
enkele persoon of groep mag agressie plegen tegen de persoon of het
eigendom van een ander. Dit principe noemen we het `NON-AGRESSIE
PRINCIPE'. Agressie wordt gedefinieerd als het aanvangen van fysiek
geweld of de dreiging daarvan tegen de persoon of het eigendom van
iemand anders. In dit opzicht kan agressie worden gezien als synoniem
voor invasie.

Als niemand een ander mag aanvallen en iedereen het absolute recht heeft
om vrij te zijn van agressie, betekent dit dat de libertariër volledig
achter wat algemeen bekendstaat als burgerlijke vrijheden staat. Dit
omvat de vrijheid van meningsuiting, publicatie, vergaderen en het zich
bezighouden met zogenoemde `slachtofferloze misdaden' zoals pornografie,
seksuele afwijking en prostitutie. De libertariër beschouwt deze laatste
niet als misdaden, omdat hij een `misdaad' definieert als een
gewelddadige inbreuk op de persoon of het eigendom van iemand anders.
Daarnaast ziet hij dienstplicht als een vorm van slavernij op grote
schaal. Aangezien moderne oorlogen vaak gepaard gaan met het massaal
doden van burgers, beschouwt de libertariër dergelijke conflicten als
massamoord en daarmee als volledig illegitiem.

Al deze standpunten worden tegenwoordig als `links' beschouwd op de
ideologische schaal. Aan de andere kant verzet de libertariër zich ook
tegen inbreuken op de rechten van privébezit. Dit houdt in dat hij zich
net zo krachtig verzet tegen overheidsinmenging in eigendomsrechten en
de vrije markteconomie, bijvoorbeeld door middel van controles,
voorschriften, subsidies of verboden. Als elk individu het recht heeft
op zijn eigen bezit zonder de angst voor plundering, dan heeft hij ook
het recht om zijn bezit weg te geven, bijvoorbeeld via een legaat of
erfenis. Bovendien kan hij zijn bezit ruilen met anderen, zonder sprake
van inmenging, wat de basis is voor vrije contracten en de vrije
markteconomie. De libertariër is voorstander van het recht op onbeperkt
privébezit en vrije ruil. Daarom pleit hij voor een systeem van
`laissez-faire kapitalisme'.

In de huidige terminologie zou het libertarische standpunt over eigendom
en economie als `extreem rechts' worden bestempeld. De libertariër ziet
echter geen tegenstrijdigheid in zijn `linkse' opvattingen over sommige
onderwerpen en `rechtse' opvattingen over andere. Sterker nog, hij
beschouwt zijn positie als bijna de enige consistente richting, die
voortdurend de vrijheid van elk individu benadrukt. Hoe kan een linkse
politicus tegen de gewelddadigheid van oorlog en dienstplicht zijn,
terwijl hij tegelijkertijd belastingheffing en overheidscontrole steunt?
En hoe kan een rechtse politicus zijn toewijding aan privébezit en vrij
ondernemerschap verkondigen, terwijl hij ook voorstander is van oorlog,
dienstplicht en het verbieden van niet-invasieve activiteiten en
praktijken die hij immoreel vindt? Daarnaast hoe kan een rechtse
politicus pleiten voor een vrije markt, terwijl hij geen probleem ziet
in de enorme subsidies, verstoringen en onproductieve inefficiënties van
het militair-industrieel complex?

De libertariër verzet zich tegen elke vorm van agressie door
privépersonen of groepen die de rechten van anderen of hun eigendommen
schenden. Hij ziet echter dat er door de geschiedenis heen, en tot op de
dag van vandaag, één centrale en allesoverheersende agressor is: de
staat. In tegenstelling tot andere denkers, ongeacht of ze links, rechts
of ergens daartussenin staan, weigert de libertariër de staat de morele
goedkeuring te geven om acties uit te voeren die bijna iedereen als
immoreel, illegaal en crimineel zou beschouwen als ze door een individu
of groep in de samenleving zouden worden gepleegd. De libertariër staat
er op dat de algemene morele wet op iedereen van toepassing is, zonder
speciale uitzonderingen voor welke persoon of groep dan ook. Wanneer we
de staat `naakt' bekijken, zien we dat het universeel is toegestaan en
zelfs wordt aangemoedigd om daden te plegen die zelfs door
niet-libertariërs als verwerpelijke misdaden worden erkend. De staat
pleegt doorgaans massamoord, wat hij `oorlog' noemt, of soms
`onderdrukking van ondermijning.' Hij houdt zich bezig met slavernij
binnen zijn strijdkrachten, hetgeen hij `dienstplicht' noemt. Bovendien
leeft de staat voort op gedwongen diefstal, wat hij `belasting' noemt.
De libertariër benadrukt met klem dat het niet relevant is of deze
praktijken wel of niet door de meerderheid van de bevolking worden
gesteund, noch of er instemming van het volk is. Oorlog is massamoord,
dienstplicht is slavernij en belasting is diefstal. Kortom, de
libertariër doet denken aan het kind in de fabel en wijst erop dat de
keizer geen kleren heeft.

Door de eeuwen heen heeft de intellectuele elite van de natie de keizer
voorzien van een reeks pseudokleren. In de afgelopen eeuwen hebben deze
intellectuelen het publiek geleid om te geloven dat de staat en zijn
heersers goddelijk waren, of op zijn minst gekleed in goddelijke
autoriteit. Hierdoor leek wat voor het naïeve en ongeschoolde oog als
despotisme, massamoord en grootschalige diefstal kon worden beschouwd,
slechts het goddelijke dat zijn welwillende en mysterieuze wegen
bewandelde binnen het politieke lichaam. In de afgelopen decennia, toen
de goddelijke legitimering beetje bij beetje verouderd raakte, hebben de
`hofintellectuelen' van de keizer steeds verfijndere verdedigingen
ontwikkeld. Ze hebben het publiek geïnformeerd dat de handelingen van de
overheid in het belang van het `algemeen welzijn' en de `publieke
welvaart' zijn. Ze verklaarden dat het belastingheffings- en
uitgaventraject werkt via het mysterieuze proces van de `hefboom' om de
economie in balans te houden. Bovendien wordt beweerd dat een
verscheidenheid aan overheidsdiensten onmogelijk kan worden uitgevoerd
door burgers die vrijwillig op de markt of in de samenleving handelen.
De libertariër wijst dit alles van de hand. Hij beschouwt de
verschillende verdedigingen als frauduleuze middelen om publieke steun
te verkrijgen voor de heerschappij van de staat. Bovendien houdt hij vol
dat de diensten die de overheid feitelijk levert, veel efficiënter en
deugdzamer kunnen worden aangeboden door particuliere en coöperatieve
ondernemingen.

De libertariër beschouwt het als een van zijn belangrijkste taken om de
staat te ontmaskeren en haar mythische status onder haar ongelukkige
onderdanen te ondermijnen. Het is zijn plicht om keer op keer en grondig
aan te tonen dat niet alleen de keizer, maar ook de `democratische'
staat, zonder kleren staat. Alle overheden handhaven hun bestaan door
middel van een uitbuitende heerschappij over het publiek. Deze
heerschappij staat lijnrecht tegenover wat objectief noodzakelijk zou
moeten zijn. Hij legt uit dat het bestaan van belastingen en de staat
onvermijdelijk een scheiding in klassen creëert: tussen de uitbuitende
heersers en de uitgebuite bevolking. Daarnaast benadrukt hij dat de
hofintellectuelen, die de staat altijd hebben gesteund, de taak hebben
om verwarring te zaaien. Dit doet zij om het publiek te overtuigen de
heerschappij van de staat te accepteren. In ruil daarvoor verkrijgen
deze intellectuelen een aandeel in de macht en de buit die de heersers
van hun misleide onderdanen onttrekken.

Neem bijvoorbeeld het belastinginstituut, waarvan sommigen beweren dat
het in zekere zin `vrijwillig' is. Iedereen die echt gelooft in de
`vrijwillige' aard van belastingheffing, wordt uitgenodigd om belasting
te weigeren en te observeren wat er dan gebeurt. Wanneer we belastingen
analyseren, zien we dat alleen de overheid in de samenleving haar
inkomsten verwerft door middel van dwingend geweld. Alle andere personen
en instellingen verdienen hun inkomen door vrijwillige giften (zoals bij
een loge, liefdadigheidsvereniging of schaakclub) of door de verkoop van
goederen of diensten die consumenten vrijwillig aanschaffen. Als iemand
anders dan de overheid `belasting' zou gaan heffen, zou dit duidelijk
als dwang en zelfs als verhuld banditisme worden beschouwd. Maar de
mystieke opsmuk van `soevereiniteit' heeft het proces zo vertroebeld dat
alleen libertariërs bereid zijn om belasting te benoemen zoals het
werkelijk is: gelegaliseerde en georganiseerde diefstal op grote schaal.

\section{EIGENDOMSRECHTEN}\label{eigendomsrechten}

Het eigendomsrecht is een fundamenteel principe in elke samenleving. Het
waarborgt dat individuen het recht hebben om te beschikken over en
gebruik te maken van hun bezittingen. Dit recht is niet alleen
belangrijk voor persoonlijke vrijheid, maar ook voor economische groei
en ontwikkeling. Historisch gezien zijn eigendomsrechten ontstaan uit de
noodzaak om conflicten te vermijden en het individuele welzijn te
beschermen. Dankzij duidelijke eigendomsrechten kunnen mensen investeren
in hun bezit, weten ze dat hun inspanningen niet voor niets zijn en dat
ze de vruchten ervan kunnen plukken. Eigendomsrechten omvatten meer dan
alleen fysiek bezit. Ze strekt zich ook uit tot intellectueel eigendom,
zoals uitvindingen, ontwerpen en creatieve werken. Deze rechten
stimuleren innovatie en creativiteit, omdat mensen er zeker van zijn dat
ze beloond worden voor hun ideeën en creaties. Desondanks zijn
eigendomsrechten in veel delen van de wereld nog steeds kwetsbaar. In
sommige gevallen worden deze rechten ondermijnd door overheidsbeleid of
door sociale druk. Dit kan leiden tot economische stagnatie en een
gebrek aan motivatie om te investeren en innoveren. Daarom is het
essentieel dat regeringen de eigendomsrechten van individuen beschermen.
Een sterk juridisch systeem en een cultuur die het belang van eigendom
erkent, dragen bij aan een stabiele en bloeiende samenleving. Alleen dan
kunnen mensen echt profiteren van hun bezittingen en bijdragen aan het
algemeen welzijn.

Als het centrale principe van het libertarische credo de non-agressie
tegen iemands persoon en eigendom is, hoe komt men dan tot dit principe?
Wat is de basis of onderbouwing? Op dit punt verschillen libertariërs,
zowel hedendaagse als historische, behoorlijk van mening. In grote
lijnen zijn er drie soorten grondslagen voor het libertarische principe,
die overeenkomen met drie soorten ethische filosofie: het
emotivistische, het utilitaristische en het natuurrechtenstandpunt. De
emotivisten stellen dat ze vrijheid of non-agressie als uitgangspunt
nemen vanuit subjectieve, emotionele overwegingen. Hoewel hun eigen
intense emotie een geldige basis lijkt voor hun politieke filosofie, kan
dit nauwelijks anderen overtuigen. Door zich uiteindelijk buiten het
domein van rationeel discours te plaatsen, zorgen de emotivisten ervoor
dat hun gekoesterde leerstelling weinig algemeen succes heeft.

De utilitaristen stellen, op basis van hun onderzoek naar de gevolgen
van vrijheid in vergelijking met alternatieve systemen, dat vrijheid
waarschijnlijk meer leidt tot algemeen geaccepteerde doelen zoals
harmonie, vrede en welvaart. Niemand betwist dat we de relatieve
gevolgen moeten bestuderen bij het beoordelen van de voordelen of
tekortkomingen van verschillende overtuigingen. Toch zijn er veel
problemen als we ons beperken tot een utilitaristische ethiek. Ten
eerste gaat het utilitarisme ervan uit dat we alternatieven kunnen
afwegen en beleidskeuzes kunnen maken op basis van hun positieve of
negatieve gevolgen. Maar als het legitiem is om waardeoordelen te vellen
over de gevolgen van iets, waarom zou het dan niet ook legitiem zijn om
dergelijke oordelen over de handeling zelf te vellen? Is er niet iets
inherent aan een handeling dat van nature als goed of slecht kan worden
beschouwd?

Een ander probleem met utilitaristen is dat zij zelden een principe als
absoluut en consistent beschouwen, om het toe te passen op de diverse
concrete situaties in de werkelijkheid. Ze gebruiken een principe
hoogstens als een vage richtlijn of aspiratie, een tendens die ze op elk
moment terzijde kunnen schuiven. Dit gebrek was kenmerkend voor de
negentiende-eeuwse Engelse radicalen, die de laissez-faire-visie van de
achttiende-eeuwse liberalen overnamen. In plaats van de natuurrechten
als basis voor hun filosofie, stelden zij een zogenaamd
`wetenschappelijk' utilitarisme voor. Hierdoor gebruikten de
negentiende-eeuwse laissez-faire-liberalen laissez-faire als een vage
tendens in plaats van als een onfeilbare maatstaf, wat het libertarische
credo steeds verder en fatale compromissen opleverde. Misschien klinkt
het hard om te zeggen dat utilitaristen niet `vertrouwd' kunnen worden
met het handhaven van het libertarische principe in elke specifieke
situatie, maar dit is een eerlijke weergave van de zaak. Een opvallend
hedendaags voorbeeld is de vrijemarkteconoom professor Milton Friedman.
Net als zijn klassieke voorgangers behoudt hij de vrijheid tegenover
staatsinterventie als algemene tendens. In de praktijk staat hij echter
een groot aantal schadelijke uitzonderingen toe. Deze uitzonderingen
maken het principe bijna volledig ongeldig, vooral op het gebied van
politie en militaire zaken, onderwijs, belastingen, welzijn,
`buurteffecten', antitrustwetten en geld en bankieren.

Laten we een grimmig voorbeeld bekijken. Stel je een samenleving voor
die alle roodharigen beschouwt als agenten van de duivel die
geëxecuteerd moeten worden zodra ze worden aangetroffen. Laten we
aannemen dat er slechts een klein aantal roodharigen in elke generatie
is, zo weinig dat het statistisch niet significant is. De
utilitariër-libertariër zou kunnen redeneren: `Hoewel de moord op
geïsoleerde roodharigen betreurenswaardig is, zijn de executies in
aantal gering. De overgrote meerderheid van het publiek, als
niet-roodharigen, ervaart enorme psychische voldoening van de publieke
executie van roodharigen. De sociale kosten zijn verwaarloosbaar,
terwijl het psychologische voordeel voor de rest van de samenleving
groot is. Daarom is het juist en gepast dat de samenleving de
roodharigen executeert.' De libertariër die zich baseert op
natuurrechten, en zich druk maakt om de rechtvaardigheid van de daad,
zal met afschuw reageren. Hij verzet zich krachtig en ondubbelzinnig
tegen de executies, die hij als totaal ongerechtvaardigde moord en
agressie tegen niet-agressieve personen beschouwt. Het feit dat het
stoppen van de moorden de meerderheid van de samenleving zou beroven van
een grote bron van psychisch plezier, zal deze `absolutistische'
libertariër totaal niet beïnvloeden. Toegewijd aan rechtvaardigheid en
logische consistentie, geeft de natuurrechtslibertariër met plezier toe
dat hij `doctrinair' is. Kort gezegd beschouwt hij zichzelf als een
ongegeneerde aanhanger van zijn eigen doctrines.

Laten we het hebben over de natuurrechtelijke basis van het
libertarische credo. Deze basis is in de een of andere vorm door de
meeste libertariërs, zowel heden als verleden, geaccepteerd.
`Natuurlijke rechten' vormt de hoeksteen van een politieke filosofie die
is ingebed in een grotere structuur van het `natuurrecht'. De theorie
over natuurrecht is gebaseerd op het inzicht dat we leven in een wereld
vol entiteiten, elk met unieke en specifieke eigenschappen. Dit geeft
elke entiteit een eigen `natuur', die door de menselijke rede, zintuigen
en geestelijke vermogens kan worden onderzocht. Koper heeft een aparte
natuur en gedraagt zich op een bepaalde manier, net als ijzer en zout.
De mens heeft als soort eveneens een specificeerbare natuur, net als de
wereld om hem heen en de interacties daartussen. Kortom, de activiteit
van elke anorganische en organische entiteit wordt bepaald door zijn
eigen aard en door de aard van de andere entiteiten waarmee het in
contact komt. Terwijl het gedrag van planten en ten minste de lagere
dieren wordt beïnvloed door hun biologische aard of instincten, geldt
voor de mens dat elke individuele persoon zijn eigen doelen moet kiezen
en de middelen moet aanwenden om deze te bereiken. Aangezien mensen geen
automatische instincten hebben, moeten ze leren over zichzelf en de
wereld. Ze moeten hun verstand gebruiken om waarden te selecteren,
inzicht krijgen in oorzaak en gevolg, en doelgericht handelen om hun
leven te verbeteren en zichzelf te onderhouden. Omdat mensen alleen als
individuen kunnen denken, voelen, evalueren en handelen, is het van
vitaal belang voor ieders voortbestaan en welvaart dat zij vrij zijn om
te leren, te kiezen, hun vermogens te ontwikkelen en te handelen naar
hun kennis en waarden. Dit is het noodzakelijke pad van de menselijke
natuur. Geweld gebruiken om dit leerproces en deze keuzes te verstoren
of te verlammen, staat lijnrecht tegenover wat de mens voor zijn leven
en welvaart nodig heeft. Gewelddadige inmenging in het leerproces en de
keuzevrijheid van de mens is daarom diep `antimenselijk'; het schendt de
natuurlijke wetten die voortkomen uit de behoeften van de mens.

Individualisten worden door hun tegenstanders vaak beschuldigd
`atomistisch' te zijn. Hiermee wordt bedoeld dat zij zouden beweren dat
elk individu in een soort vacuüm leeft, denkend en kiezend zonder enige
relatie tot anderen in de samenleving. Dit is echter een autoritaire
stroman; weinige, zo niet geen, individualisten zijn ooit echt
`atomisten' geweest. Integendeel, het is duidelijk dat individuen altijd
van elkaar leren, samenwerken en met elkaar omgaan. Dit is zelfs
noodzakelijk voor het overleven van de mensheid. De essentie is dat elk
individu uiteindelijk zelf beslist welke invloeden hij opneemt en welke
hij verwerpt, en in welke volgorde dat gebeurt. De libertariër
verwelkomt het proces van vrijwillige uitwisseling en samenwerking
tussen vrijhandelende individuen. Wat hij verafschuwt, is het gebruik
van geweld om deze vrijwillige samenwerking te verstoren en iemand te
dwingen op een andere manier te kiezen en te handelen dan zijn eigen
verstand aangeeft.

De meest haalbare manier om de natuurrechtelijke verklaring van de
libertarische positie te verduidelijken, is door deze op te splitsen in
verschillende delen. Laten we beginnen met het basisprincipe van het
`recht op zelfeigenaarschap'. Het recht op zelfeigenaarschap bevestigt
het absolute recht van elk mens, simpelweg omdat hij of zij mens is, om
het eigen lichaam te `bezitten'. Dit betekent dat iemand het recht heeft
om zijn of haar lichaam te beheren zonder gedwongen inmenging. Aangezien
elk individu moet denken, leren, waarde moet toekennen en zijn of haar
doelen en middelen moet kiezen om te overleven en zich te ontwikkelen,
biedt het recht op zelfeigenaarschap de mens de vrijheid om deze
essentiële activiteiten uit te voeren zonder belemmeringen door
dwangmatig ingrijpen.

Denk goed na over de gevolgen van het ontzeggen van het recht aan
iedereen om zijn of haar eigen persoon te bezitten. Er blijven dan
slechts twee opties over: ofwel (1) een bepaalde klasse mensen, A, heeft
het recht om een andere klasse, B, te bezitten, ofwel (2) iedereen heeft
het recht om een proportioneel aandeel in alle anderen te bezitten. Het
eerste alternatief houdt in dat klasse A de rechten van het mens-zijn
toekomt, terwijl klasse B in feite als sub-menselijk wordt gezien en
daarom geen rechten verdient. Maar omdat ook klasse B mensen is, valt
het eerste alternatief in tegenspraak met de natuurlijke mensenrechten,
die aan deze groep worden ontzegd. Bovendien, zoals we zullen zien,
betekent het toestaan dat klasse A klasse B bezit, dat de eerste klasse
wordt toegestaan om de tweede te exploiteren en dus parasitair te leven
ten koste van de laatste. Dit parasitisme zelf ondermijnt echter de
essentiële economische voorwaarde voor leven: namelijk productie en
uitwisseling.

Het tweede alternatief, dat we `participatief communalisme' of
`communisme' kunnen noemen, stelt voor dat iedereen het recht heeft om
een gelijk aandeel in anderen te bezitten. Als er twee miljard mensen op
de wereld zijn, betekent dit dat iedereen recht heeft op een
twee-miljardste deel van elke andere persoon. Allereerst kunnen we
stellen dat dit ideaal op een absurditeit is gebaseerd: het verkondigen
dat iedereen recht heeft op een deel van anderen, terwijl men zelf geen
recht heeft op zijn eigen lichaam, is tegenstrijdig. Ten tweede kunnen
we ons de levensvatbaarheid van zo'n wereld voorstellen: een wereld
waarin niemand vrij is om enige actie te ondernemen zonder goedkeuring
of zelfs een bevel van alle anderen in de samenleving. Het is duidelijk
dat in zo'n `communistische' wereld niemand iets kan doen en het
menselijk ras snel ten onder zou gaan. Als een wereld zonder
zelfeigenaarschap en met honderd procent andermans eigendom de ondergang
betekent voor de mensheid, dan zijn alle stappen in die richting in
strijd met de natuurlijke wet die bepaalt wat het beste is voor de mens
en zijn leven op aarde.

Uiteindelijk is het niet mogelijk om een participatieve communistische
wereld in de praktijk te realiseren. Het is fysiek onmogelijk voor
iedereen om voortdurend iedereen in de gaten te houden en zo zijn
proportionele aandeel in het eigendom van anderen uit te oefenen. In
werkelijkheid is het idee van universeel en gelijkwaardig eigenaarschap
dus utopisch en onhaalbaar. Hierdoor komt het toezicht en de controle
over anderen noodzakelijkerwijs in handen van een gespecialiseerde groep
mensen, die zo een heersende klasse vormt. Elke poging tot
communistische heerschappij zal daarom automatisch een vorm van
klassenheerschappij worden, waardoor we weer uitkomen bij ons eerste
alternatief.

De libertariër verwerpt deze alternatieven en kiest uiteindelijk voor
het universele recht op zelfeigenaarschap als primair uitgangspunt. Dit
recht is inherent aan ieder individu, simpelweg omdat hij of zij mens
is. De uitdaging ligt echter bij het ontwikkelen van een theorie over
het eigendom van niet-menselijke objecten, de dingen van deze aarde. Het
is relatief eenvoudig om te zien wanneer iemand de eigendomsrechten van
een andere persoon schaadt. Als A B aanvalt, schendt A het
eigendomsrecht van B op zijn eigen lichaam. Maar bij niet-menselijke
objecten ligt de situatie ingewikkelder. Neem bijvoorbeeld het geval
waarin X een horloge steelt dat Y toebehoort. We kunnen niet automatisch
aannemen dat X agressie pleegt tegen Y's eigendomsrecht op het horloge.
Misschien is X wel de oorspronkelijke, `echte' eigenaar van het horloge,
wat zou betekenen dat hij zijn legitieme eigendom terugneemt. Om dit
soort situaties te beoordelen, hebben we een theorie van
rechtvaardigheid in eigendom nodig. Een dergelijke theorie moet ons
kunnen vertellen wie de rechtmatige eigenaar is: X, Y of iemand anders.

Sommige libertariërs proberen het probleem op te lossen door te stellen
dat degene die volgens de overheid de eigendomstitel heeft, gezien moet
worden als de rechtmatige eigenaar van dat eigendom. Op dit moment
hebben we nog niet diep ingegaan op de aard van de overheid, maar de
tegenstrijdigheid is overduidelijk: het is vreemd om te zien dat een
groep die altijd wantrouwend staat tegenover bijna alle
overheidshandelingen, het ineens aan de overheid overlaat om het
essentiële concept van eigendom te definiëren en toe te passen. Vooral
utilitaristische laissez-fairisten denken dat het het meest haalbaar is
om de nieuwe libertarische wereld te beginnen met de bevestiging van
alle bestaande eigendomstitels. Met andere woorden, zij willen dat de
eigendomstitels en rechten worden erkend zoals vastgesteld door dezelfde
overheid die ze als chronische agressor veroordelen.

Laten we dit verduidelijken met een hypothetisch voorbeeld. Stel je voor
dat de druk vanuit de libertarische beweging zo is toegenomen dat de
regering, inclusief haar verschillende afdelingen, bereid is om af te
treden. Maar ze bedenken een slinkse truc. Net voordat de regering van
de staat New York aftreedt, nemen ze een wet aan waardoor het hele
grondgebied van New York privébezit wordt van de familie Rockefeller. De
wetgevende macht van Massachusetts doet hetzelfde voor de familie
Kennedy. En zo gaat het verder voor elke staat. De regering kan dan wel
aftreden en de afschaffing van belastingen en dwingende wetgeving
aankondigen, maar de zegevierende libertariërs zouden voor een dilemma
komen te staan. Erkennen zij de nieuwe eigendomstitels als legitiem
privébezit? De utilitariërs, die geen theorie hebben over
rechtvaardigheid in eigendomsrechten, zouden, als ze consistent blijven
bij hun acceptatie van de bestaande eigendomstitels zoals bepaald door
de overheid, een nieuwe sociale orde moeten aanvaarden. In die orde
zouden vijftig nieuwe landeigenaren belastingen heffen in de vorm van
eenzijdig opgelegde `huur'. Het punt is dat alleen libertariërs die
geloven in natuurlijke rechten, die een theorie hebben over
rechtvaardigheid in eigendomstitels die niet afhankelijk is van
overheidsbesluiten, in de positie zouden zijn om de nieuwe heersers te
bespotten. Zij zouden de aanspraken op privébezit van het land kunnen
afwijzen als ongeldig. Zoals de grote negentiende-eeuwse liberaal Lord
Acton al aangaf, biedt het natuurrecht de enige zekere basis voor een
blijvende kritiek op de wetten en decreten van de overheid. Wat precies
het natuurrechtelijke standpunt over eigendomstitels zou kunnen zijn, is
de vraag waar we ons nu op richten.

We hebben het recht van elk individu op zelfeigenaarschap vastgesteld,
wat inhoudt dat iedereen eigendomsrecht heeft op zijn eigen lichaam en
persoon. Mensen zijn echter geen zwevende geesten; ze zijn geen op
zichzelf staande entiteiten. Ze kunnen alleen overleven en gedijen door
interactie met de wereld om hen heen. Bijvoorbeeld, ze moeten op land
staan. Daarnaast moeten ze, om te overleven en zichzelf in stand te
houden, de hulpbronnen die de natuur biedt, omzetten in
`consumptiegoederen' --- in voorwerpen die beter geschikt zijn voor hun
gebruik en consumptie. Voedsel moet worden verbouwd en gegeten;
mineralen moeten worden gedolven, omgezet in kapitaal en daarna in
bruikbare consumptiegoederen. Met andere woorden, de mens moet niet
alleen zijn eigen persoon bezitten, maar ook materiële objecten die hij
kan beheren en gebruiken. Maar hoe worden de eigendomstitels van deze
objecten dan toegewezen?

Laten we beginnen met een beeldhouwer die een kunstwerk maakt van klei
en andere materialen. Laten we voor het moment de kwestie van de
oorspronkelijke eigendomsrechten op de klei en het gereedschap van de
beeldhouwer even negeren. De vraag die opkomt is: Wie is de eigenaar van
het kunstwerk dat ontstaat na het modelleren door de beeldhouwer? Het is
eigenlijk de `creatie' van de beeldhouwer, niet omdat hij nieuwe materie
heeft geschapen, maar omdat hij de door de natuur gegeven materie -- de
klei -- heeft getransformeerd in een andere vorm. Deze vorm is bepaald
door zijn eigen ideeën en is gevormd door zijn eigen handen en energie.
Het is zeker een uitzonderlijk persoon die, gegeven deze situatie, zou
beweren dat de beeldhouwer geen eigendomsrecht heeft op zijn eigen
creatie. Als ieder mens het recht heeft om zijn eigen lichaam te
bezitten, en als hij moet omgaan met materiële objecten om te overleven,
dan heeft de beeldhouwer absoluut het recht om zijn product te bezitten.
Hij heeft dit product namelijk, door zijn inspanning en creativiteit,
tot een waar verlengstuk van zijn eigen persoonlijkheid gemaakt. Hij
heeft zijn stempel op de grondstof gedrukt door `zijn arbeid' met de
klei te vermengen, zoals de grote theoreticus John Locke het verwoordde.
Het product dat door zijn energie is getransformeerd, is de materiële
belichaming geworden van de ideeën en visie van de beeldhouwer. John
Locke verwoordde de zaak als volgt:

\begin{quote}
Iedereen heeft eigendom in zijn eigen persoon. Niemand anders heeft daar
recht op dan hijzelf. De arbeid van zijn lichaam en het werk van zijn
handen zijn werkelijk van hem. Wat hij ook uit de natuurlijke staat
haalt, is door zijn arbeid vermengd en verbonden met iets dat van hem
is, en wordt daardoor zijn eigendom. Omdat hij iets uit de algemene
staat heeft gehaald waarin de natuur het heeft geplaatst, heeft hij door
zijn arbeid iets toegevoegd dat het algemene recht van andere mensen
uitsluit. De arbeid is het onbetwiste eigendom van de arbeider, waardoor
niemand anders dan hij recht kan hebben op datgene waarmee het ooit
verbonden is.
\end{quote}

Net als bij het eigendom van mensenlichamen zijn er drie logische
alternatieven: (1) de bewerker, of `schepper', heeft het eigendomsrecht
op zijn creatie; (2) een ander persoon of een groep mensen heeft het
recht om deze creatie zonder toestemming van de beeldhouwer geweld aan
te doen; of (3) ieder individu ter wereld heeft een gelijk,
proportioneel aandeel in het eigendom van het kunstwerk --- de
`gemeenschappelijke' oplossing. Eerlijk gezegd is er maar een handvol
mensen die zou beweren dat het eerlijk is om het eigendom van de
beeldhouwer te confisqueren, hetzij door één of meerdere anderen, hetzij
namens de wereld als geheel. Met welk recht doen ze dat? Met welk recht
eigenen zij zich het product van de geest en de energie van de maker
toe? In dit duidelijke geval zou het recht van de schepper om te
beschikken over datgene waarmee hij zijn persoon en arbeid heeft
vermengd, algemeen erkend moeten worden. Ook hier geldt dat de
gemeenschappelijke wereldoplossing in de praktijk vaak zou neerkomen op
een oligarchie van enkele anderen die het werk van de schepper in naam
van `wereldwijd publiek' eigendom onteigenen.

Het belangrijkste punt is echter dat het geval van de beeldhouwer
kwalitatief niet verschilt van andere gevallen van `productie'. De
persoon of personen die de klei uit de grond haalden en deze aan de
beeldhouwer verkochten, zijn misschien niet zo `creatief' als de
beeldhouwer, maar ook zij zijn `producenten'. Ook zij hebben hun ideeën
en technologische kennis gecombineerd met de door de natuur gegeven
grond om een nuttig product te creëren. Zij zijn ook `producenten' en
hebben hun arbeid samengevoegd met natuurlijke materialen om die
materialen om te zetten in bruikbare goederen en diensten. Ook zij
hebben recht op het eigendom van hun producten. Waar begint dit proces
dan? Laten we ons opnieuw tot Locke wenden:

\begin{quote}
Hij die zich voedt met de eikels die hij onder een eik verzamelt of de
appels die hij van de bomen in het bos plukt, heeft deze zeker aan
zichzelf toegeëigend. Niemand kan ontkennen dat het voedsel van hem is.
Maar wanneer zijn die eikels en appels eigenlijk van hem geworden? Was
dat toen hij ze at? Of toen hij ze kookte? Of toen hij ze mee naar huis
nam? Of misschien al toen hij ze oprapelde? Het is duidelijk dat als de
eerste verzameling ze niet van hem maakte, niets anders dat kon. Zijn
arbeid maakte het verschil tussen die eikels en de `algemene' voorraad.
Hij voegde er iets aan toe dat de natuur, de gemeenschappelijke moeder
van allen, niet had gedaan, en dus werd het zijn privébezit. Zal iemand
beweren dat hij geen recht had op de eikels of appels die hij zich op
die manier toeëigende, omdat hij niet de toestemming van iedereen had om
ze tot de zijne te maken? Was het dan diefstal om zich iets toe te
eigenen dat voor iedereen gemeenschappelijk was? Als zo'n instemming
nodig was, zou de mens verhongeren, ondanks de overvloed die God hem
gegeven heeft. Zo wordt het gras dat mijn paard heeft gegeten, het gras
dat mijn knecht heeft gemaaid en het erts dat ik op mijn grond heb
uitgegraven, mijn eigendom. Dit is zo, zonder enige toewijzing of
instemming van iemand anders. De arbeid die ik heb verricht om ze uit de
gemeenschappelijke staat te halen waarin ze zich bevonden, heeft mijn
eigendom gevestigd.

Door expliciete toestemming van elke gewone man noodzakelijk te maken
voor iemands toe-eigening van enig deel van wat gemeenschappelijk is,
konden kinderen of knechten het vlees dat hun vader of meester hen had
gegeven, niet snijden zonder dat elk zijn eigen deel werd toegewezen.
Hoewel het water dat in de fontein stroomt van iedereen is, wie kan er
nog twijfelen dat het water in de kruik alleen toebehoort aan degene die
het eruit heeft gehaald? Zijn arbeid heeft het uit de handen van de
natuur gehaald, waar het gemeenschappelijk was, en heeft het daardoor
aan zichzelf toegeëigend.

Aldus maakt de wet van de redelijkheid het hert tot eigendom van de
Indiaan die het doodde. Het wordt toegestaan dat hij het als zijn
eigendom beschouwt, omdat hij daar zijn arbeid aan heeft besteed,
terwijl het daarvoor nog het gemeenschappelijke recht van iedereen was.
Zelfs onder degenen die tot het beschaafde deel van de mensheid behoren,
blijft deze oorspronkelijke wet van de natuur van toepassing als basis
voor eigendom. Op grond hiervan is de vis die iemand vangt in de
oceaan---de grote, nog steeds bestaande gemeenschappelijkheid van de
mensheid---ook zijn eigendom. Of de ambergris (de zeldzame
walvisuitscheiding) die iemand ophaalt; door de arbeid die nodig is om
het uit de gemeenschappelijke staat te verwijderen, wordt het eigendom
van degene die er moeite voor doet.
\end{quote}

Als ieder mens eigenaar is van zijn eigen persoon en daarmee van zijn
eigen arbeid, en als hij bovendien eigenaar is van alles wat hij heeft
`gecreëerd' of verzameld uit de voorheen ongebruikte, ongeërfde
`natuurtoestand', wat betekent dit dan voor de grote vraag: het recht om
de aarde zelf te bezitten of te beheersen? Kortom, als de verzamelaar
het recht heeft om de eikels of bessen die hij plukt te bezitten en de
boer het recht om zijn oogst van tarwe of perziken te claimen, wie heeft
dan het recht om het land te bezitten waarop deze zaken groeien? Hieraan
komt Henry George met zijn aanhangers, die verder zijn gegaan dan de
libertariërs, in het geding. Zij ontkennen het recht van het individu om
het stuk land zelf te bezitten, dat wil zeggen de grond waarop deze
activiteiten plaatsvinden. De Georgisten beweren dat, hoewel iedereen
eigenaar zou moeten zijn van de goederen die hij produceert of creëert,
omdat de natuur of God het land heeft geschapen, geen enkel individu het
recht heeft om eigenaar te worden van dat land. Toch, als het land
effectief als grondstof moet worden gebruikt, moet het eigendom zijn van
of gecontroleerd worden door iemand of een groep. Dit brengt ons weer
terug bij onze drie opties: of het land behoort toe aan de eerste
gebruiker, de persoon die het als eerste in productie neemt; of het
behoort toe aan een groep anderen; of het behoort toe aan de wereld als
geheel, waarbij elk individu een proportioneel deel van elke hectare
bezit. George's voorkeur voor die laatste optie lost zijn morele
probleem nauwelijks op. Als het land zelf aan God of de natuur zou
moeten toebehoren, waarom zou het dan moreel beter zijn om elke hectare
wereldwijde grond aan de wereld als geheel toe te kennen dan aan
individuele eigendom? Bovendien is het in de praktijk onmogelijk voor
elke persoon in de wereld om effectief eigenaar te zijn van zijn
viermiljardste deel (stel dat de wereldbevolking vier miljard bedraagt)
van elk stuk landoppervlak. In werkelijkheid zou een kleine oligarchie
het bezit en de controle hebben, in plaats van de wereld als geheel.

Maar naast de moeilijkheden met het Georgistische standpunt is de
natuurrechtelijke rechtvaardiging voor het eigendom van grond hetzelfde
als voor het oorspronkelijke eigendom van andere bezittingen. Want zoals
we hebben gezien, `schept' geen enkele producent daadwerkelijk materie.
Hij neemt materie die door de natuur is gegeven en transformeert deze
door zijn arbeidsinspanningen, in lijn met zijn ideeën en visie. Dit is
precies wat de pionier, of `homesteader', doet wanneer hij onbenut land
in privébezit neemt. Net zoals de man die staal produceert uit
ijzererts, dat hij met zijn kennis en energie transformeert, doet de man
die ijzer uit de grond haalt hetzelfde. De `homesteader' bewerkt,
omheint of bebouwt het land op vergelijkbare wijze. Ook de huiseigenaar
verandert de door de natuur gegeven grond door zijn arbeid en zijn
persoonlijkheid. Hij is net zo legitiem eigenaar van het land als de
beeldhouwer of de fabrikant; hij is net zo goed een `producent' als de
anderen.

Bovendien, als het oorspronkelijke land door de natuur of God is
gegeven, dan geldt datzelfde voor de talenten, gezondheid en schoonheid
van mensen. Net zoals deze eigenschappen aan specifieke individuen zijn
toegekend en niet aan de `samenleving', geldt dat ook voor land en
natuurlijke hulpbronnen. Al deze hulpbronnen zijn in handen van
individuen en niet van de `maatschappij', die een abstract begrip is
zonder werkelijke bestaansgrond. Er is geen bestaande entiteit met de
naam `maatschappij'; er zijn alleen individuen die met elkaar
interageren. Wanneer men zegt dat de `maatschappij' land of enig ander
bezit gemeenschappelijk zou moeten bezitten, betekent dat in feite dat
een groep oligarchen - in de praktijk overheidsbureaucraten - het bezit
zou moeten ontvangen. Dit gaat ten koste van de onteigening van de
schepper of de huiseigenaar die het product oorspronkelijk heeft
voortgebracht.

Bovendien kan niemand iets produceren zonder de medewerking van
oorspronkelijk land, zelfs niet als dat land slechts als staanplaats
dient. Geen enkel mens kan iets produceren of creëren met alleen zijn
arbeid; hij heeft de steun van land en andere natuurlijke hulpbronnen
nodig. Bij zijn geboorte heeft de mens niets meer dan zichzelf en de
wereld om zich heen: het land en de natuurlijke grondstoffen die de
natuur hem biedt. Hij neemt deze grondstoffen en transformeert ze door
zijn arbeid, geest en energie in goederen die nuttiger zijn voor de
mens. Als een individu echter geen eigenaar kan zijn van het
oorspronkelijke land, kan hij ook niet in volle omvang eigenaar zijn van
de vruchten van zijn arbeid. De boer kan zijn tarweoogst niet bezitten
als hij het land waarop de tarwe groeit, niet kan bezitten. Aangezien
zijn arbeid onlosmakelijk verbonden is met het land, kan hij niet van
het een beroofd worden zonder van het ander beroofd te worden.

Bovendien, als een producent geen recht heeft op de vruchten van zijn
arbeid, wie heeft dat dan wel? Het is moeilijk te begrijpen waarom een
pasgeboren baby uit Pakistan moreel gezien aanspraak zou kunnen maken op
een evenredig deel van het eigendom van een stuk land in Iowa, dat
iemand net heeft omgevormd tot een tarweveld. En omgekeerd geldt dit
natuurlijk ook voor een baby in Iowa die aanspraak zou maken op een
Pakistaanse boerderij. Land in zijn oorspronkelijke staat is ongebruikt
en heeft geen eigenaar. Georgisten en andere landcommunalisten kunnen
wel stellen dat de hele wereldbevolking er `eigenaar' van is, maar als
niemand er gebruik van heeft gemaakt, is het in werkelijkheid in beheer
van niemand. De pionier, de huiseigenaar, de eerste gebruiker en
omvormer van dit land is degene die dit eenvoudige, voorlopig waardeloze
stuk grond als eerste in productie en sociaal gebruik brengt. Het is
moeilijk te begrijpen waarom het moreel verantwoord zou zijn om zijn
eigendom te ontnemen ten gunste van mensen die nog nooit in de buurt van
het land zijn geweest en die misschien zelfs niet weten dat het bezit
bestaat, waarop ze verondersteld worden aanspraak te maken.

Het morele vraagstuk rondom natuurrechten wordt nog helderder als we
naar dieren kijken. Dieren kunnen worden gezien als `economische grond',
omdat ze oorspronkeljike hulpbronnen zijn die door de natuur zijn
gegeven. Toch zou iemand een paard volledig ontzeggen aan de man die het
vindt en domesticeert. Is dat echt anders dan de eikels en bessen die
doorgaans worden toegewezen aan de verzamelaar? Ook op het gebied van
land geldt dat een huiseigenaar het voorheen `wilde',
niet-gedomesticeerde land `temt' door het productief te gebruiken. Door
zijn arbeid te vermengen met deze grond, zou hij net zo'n duidelijke
aanspraak moeten hebben als in het geval van dieren. Zoals Locke het
verwoordde: `Zoveel land als een man bewerkt, plant, verbetert,
cultiveert en waarvan hij het product kan gebruiken, zoveel is zijn
eigendom. Door zijn arbeid sluit hij het als het ware af van het
gemeenschappelijke.'{[}\^{}4{]}

De libertarische theorie van eigendom werd treffend samengevat door twee
Franse economen uit de negentiende eeuw die voor laissez-faire pleitten:

\begin{quote}
Als de mens rechten verwerft over dingen, dan komt dat doordat hij
tegelijkertijd actief, intelligent en vrij is. Door zijn activiteit
verspreidt hij zich over de uiterlijke natuur. Zijn intelligentie stelt
hem in staat om de natuur te beheren en deze naar zijn eigen gebruik aan
te passen. Door zijn vrijheid legt hij een relatie van oorzaak en gevolg
tussen zichzelf en de natuur, en maakt hij deze tot zijn eigendom.

Waar in een beschaafd land vind je een kluit aarde of een blad dat niet
het stempel van de mens draagt? In de stad zijn we omringd door
menselijke werken. We lopen op een vlakke stoep of een gebaande weg. Het
is de mens die de voorheen modderige grond heeft vruchtbaar gemaakt. Hij
heeft vuursteen of stenen van verre heuvels gehaald om de omgeving te
bedekken. We wonen in huizen; de mens heeft de steen uit de groeve
gehaald, heeft deze uitgehouwen en het hout geschaafd. Het is de
creativiteit van de mens die de materialen op de juiste manier heeft
gerangschikt tot een gebouw, dat ooit uit rots en bos bestond. Ook op
het platteland is het werk van de mens overal zichtbaar. Hij heeft de
grond bewerkt en generaties arbeiders hebben deze verfijnd en verrijkt.
Menselijke inspanningen hebben de rivieren afgedamd en vruchtbaarheid
gecreëerd waar woestenij was. Overal is een krachtige hand zichtbaar die
materie heeft vormgegeven, en een intelligente wil die deze heeft
aangepast aan de behoeften van ons soort. De natuur heeft haar meester
erkend, en de mens voelt zich thuis in de natuur. Hij heeft deze voor
zijn gebruik toegeëigend; ze is van hem geworden, zijn eigendom. Dit
eigendom is legitiem en het vormt een recht dat voor de mens net zo
heilig is als de vrije uitoefening van zijn vermogens. Het behoort hem
toe omdat het volledig voortkomt uit zijn eigen wezen en niets anders is
dan een uitdrukking ervan. Vóór hem was er nauwelijks iets meer dan
materie. Sinds hij er is, is er een verwisselbare rijkdom ontstaan. Dit
zijn artikelen die waarde hebben gekregen door industrie, fabricage,
behandeling, extractie of gewoon transport. Van het beeld van een grote
meester, waarin materie de kleinste rol speelt, tot de emmer water die
de drager uit de rivier haalt om naar de consument te brengen, verwerft
rijkdom, wat deze ook moge zijn, zijn waarde alleen door overgebrachte
kwaliteiten. Deze kwaliteiten maken deel uit van menselijke activiteit,
intelligentie en kracht. De producent heeft een fragment van zichzelf
achtergelaten in het ding dat zo waardevol is geworden. Daarom kan het
worden gezien als een verlenging van de vermogens van de mens die op de
uiterlijke natuur inwerkt. Als vrij wezen behoort hij aan zichzelf. De
oorzaak, dat wil zeggen de productiekracht, is hijzelf; het gevolg,
ofwel de geproduceerde rijkdom, is ook nog steeds hijzelf. Wie durft de
eigendomstitel, zo duidelijk gemarkeerd door de afdruk van zijn
persoonlijkheid, te betwisten?

Dan moeten we terug naar de mens, de schepper van alle rijkdom. Door
arbeid drukt de mens zijn persoonlijkheid op de materie. Het is arbeid
die de aarde bewerkt en een onbewoonde woestenij omzet in een vruchtbare
akker. Het is arbeid die van een onbetreden bos een goed geordend woud
maakt. Het zijn reeksen van arbeid, vaak uitgevoerd door een grote
opeenvolging van arbeiders, die uit hennep zaad maakt, draad uit hennep,
doek uit draad en kleding uit doek. Het is arbeid die het vormeloze
pyriet, verzameld in de mijn, omzet in elegant brons dat een openbare
ruimte siert en de gedachte van een kunstenaar aan een heel volk
overbrengt.

Eigendom, zichtbaar gemaakt door arbeid, behoort tot de rechten van de
persoon die het heeft voortgebracht. Net als het individu is deze
eigendom onschendbaar, zolang het niet in conflict komt met de rechten
van anderen. Het is ook individueel, omdat het voortkomt uit de
onafhankelijkheid van het individu. Wanneer meerdere personen hebben
bijgedragen aan het ontstaan ervan, heeft de laatste eigenaar de
waardevolle vruchten van zijn persoonlijke arbeid gekocht, evenals het
werk van zijn voorgangers. Dit is meestal het geval bij gefabriceerde
artikelen. Wanneer eigendom door verkoop of vererving van de ene hand
naar de andere gaat, blijven de voorwaarden ongewijzigd. Het blijft de
vrucht van menselijke vrijheid, die zich uitdrukt in arbeid. De houder
heeft dezelfde rechten als de producent die het oorspronkelijk in bezit
nam.
\end{quote}

\section{DE SAMENLEVING EN HET
INDIVIDU}\label{de-samenleving-en-het-individu}

Waar in een beschaafd land vind je een kluit aarde of een blad dat niet
het stempel van de mens draagt? In de stad zijn we omringd door
menselijke werken. We lopen op een vlakke stoep of een gebaande weg. Het
is de mens die de modderige grond heeft omgevormd tot vruchtbare aarde.
Hij heeft vuursteen en stenen van verre heuvels gehaald om de omgeving
mee te bedekken. We wonen in huizen; de mens heeft de steen uit de
groeve gehaald, deze uitgehouwen en het hout bewerkt. Het is de
creativiteit van de mens die materialen op de juiste manier heeft
samengevoegd tot gebouwen, die ooit nog rotsen en bossen waren. Ook op
het platteland is het werk van de mens overal zichtbaar. Hij heeft de
grond bewerkt, en generaties arbeiders hebben die verder verfijnd en
verrijkt. Menselijk werk heeft de rivieren afgedamd en vruchtbaarheid
gecreëerd waar vroeger alleen woestenij was. Overal is een krachtige
hand zichtbaar die materie heeft gevormd, en een intelligente wil die
die materie heeft aangepast aan de behoeften van ons soort. De natuur
heeft haar meester erkend, en de mens voelt zich thuis in de natuur. Hij
heeft deze toegeëigend voor zijn gebruik; ze is van hem geworden, zijn
eigendom. Dit eigendom is legitiem en vormt een recht dat net zo heilig
is voor de mens als de vrije uitoefening van zijn vermogens. Het behoort
hem toe omdat het volledig voortkomt uit zijn eigen wezen en niets
anders is dan een uitdrukking ervan. Vóór hem was er nauwelijks iets
meer dan materie. Sinds zijn komst bestaat er verwisselbare rijkdom, dat
wil zeggen, goederen die waarde hebben gekregen door industrie,
fabricage, behandeling of transport. Van het beeld van een grote
meester, waarin materie de kleinste rol speelt, tot de emmer water die
de drager uit de rivier haalt om naar de consument te brengen, verwerft
rijkdom alleen zijn waarde door overgebrachte kwaliteiten. Deze
kwaliteiten zijn het resultaat van menselijke activiteit, intelligentie
en kracht. De producent heeft een fragment van zichzelf achtergelaten in
het ding dat zo waardevol is geworden en kan daarom worden gezien als
een verlengstuk van de vermogens van de mens die op de natuur ingrijpt.
Als vrij wezen behoort hij aan zichzelf. De oorzaak, ofwel de
productiekracht, is hijzelf; het gevolg, de geproduceerde rijkdom, is
ook nog steeds hijzelf. Wie durft de eigendomstitel, zo duidelijk
gemarkeerd door het bewijs van zijn persoonlijkheid, te betwisten? Laten
we terugkeren naar de mens, de schepper van alle rijkdom. Door arbeid
drukt de mens zijn persoonlijkheid op de materie. Het is arbeid die de
aarde bewerkt en een onbewoonde woestenij omzet in een vruchtbare akker.
Het is arbeid die van een onbetreden bos een goed geordend woud maakt.
Het zijn reeksen van arbeid, vaak uitgevoerd door talloze arbeiders, die
uit hennep zaad maken, draad uit hennep, doek uit draad en kleding uit
doek. Het is arbeid die het vormeloze pyriet, verzameld in de mijn,
omzet in elegant brons dat een openbare ruimte siert en de gedachte van
een kunstenaar aan een heel volk overbrengt. Eigendom, zichtbaar gemaakt
door arbeid, behoort toe aan de persoon die het heeft voortgebracht. Net
als de persoon is deze eigendom onschendbaar, zolang het niet in
conflict komt met de rechten van anderen. Het is ook individueel, omdat
het voortkomt uit de onafhankelijkheid van het individu. Wanneer
meerdere personen hebben bijgedragen aan het ontstaan ervan, heeft de
laatste eigenaar de waardevolle vruchten van zijn persoonlijke arbeid
gekocht, evenals het werk van zijn voorgangers. Dit is meestal het geval
bij gefabriceerde artikelen. Wanneer eigendom door verkoop of vererving
van de ene hand naar de andere gaat, blijven de voorwaarden ongewijzigd.
Het blijft de vrucht van menselijke vrijheid, die zich uitdrukt in
arbeid. De houder heeft dezelfde rechten als de producent die het
oorspronkelijk in bezit nam.

We hebben uitgebreid gesproken over individuele rechten. Maar wat, zo
vraagt men zich af, is er met de `rechten van de samenleving'? Gaan deze
niet boven die van het individu? De libertariër is echter een
individualist; hij gelooft dat een van de grootste fouten in sociale
theorieën is om de `maatschappij' te beschouwen als een werkelijk
bestaande entiteit. Soms wordt de `maatschappij' behandeld als een
superieure of zelfs goddelijke figuur, waar hij zijn eigen overheersende
`rechten' aan ontleent. Op andere momenten wordt ze neergezet als een
bestaand kwaad, verantwoordelijk voor al het onheil in de wereld. De
individualist stelt dat alleen individuen bestaan. Zij denken, voelen,
kiezen en handelen. `De maatschappij' is geen levende entiteit, maar een
label voor een verzameling interacties tussen individuen. Het behandelen
van de maatschappij als iets dat kan kiezen of handelen, verdoezelt de
werkelijke krachten die aan het werk zijn. Stel je voor dat in een
kleine gemeenschap tien mensen zich verenigen om drie anderen te beroven
en te onteigenen. Dit is duidelijk een geval van een groep individuen
die gezamenlijk optreedt tegen een andere groep. Als deze tien zich in
deze situatie `de maatschappij' zouden voordoen, handelend in het belang
van `de maatschappij', zou de redenering in de rechtszaal belachelijk
worden gemaakt. Zelfs de rovers zouden waarschijnlijk te beschaamd zijn
om dergelijke argumenten te gebruiken. Maar als hun aantal toeneemt, kan
deze verdoezeling wijdverspreid worden en het publiek misleiden.

Het misleidende gebruik van een collectief zelfstandig naamwoord zoals
`natie', dat in dit opzicht vergelijkbaar is met `maatschappij', heeft
historicus Parker T. Moon scherp aan de kaak gesteld:

\begin{quote}
Wanneer je het eenvoudige `Frankrijk' gebruikt, denk je aan Frankrijk
als één geheel, een entiteit. Als we zeggen `Frankrijk stuurde haar
troepen om Tunesië te veroveren', geven we het land niet alleen de
status van eenheid, maar ook die van een persoonlijkheid. De woorden
verhullen de feiten en maken van internationale betrekkingen een
glamoureus drama, waarin gepersonifieerde naties de acteurs zijn. We
vergeten al te gemakkelijk de mannen en vrouwen van vlees en bloed die
de échte acteurs zijn. Als we geen woord als `Frankrijk' hadden, zouden
we de Tunesië-expeditie nauwkeuriger moeten beschrijven, bijvoorbeeld
zo: `Een paar van deze achtendertig miljoen mensen stuurden
dertigduizend anderen om Tunesië te veroveren.' Deze manier van
beschrijven roept onmiddellijk vragen op. Wie waren die `weinigen'?
Waarom stuurden zij dertigduizend naar Tunesië? En waarom gehoorzaamden
die dertigduizend? Het opbouwen van een imperium gebeurt niet door
`naties', maar door mensen. De uitdaging is om de mannen te ontdekken,
de actieve en belanghebbende minderheden in elk land, die rechtstreeks
profiteren van imperialisme. Vervolgens moeten we analyseren waarom de
meerderheid de kosten betaalt en de oorlog voert die nodig is voor
imperialistische expansie.
\end{quote}

De individualistische kijk op `de maatschappij' kan worden samengevat
als volgt: `De maatschappij' is iedereen behalve jezelf. Deze analyse
dient als basis om gevallen te onderzoeken waarin `de maatschappij' niet
alleen wordt gezien als een superheld met bijzondere rechten, maar ook
als een superschurk die massaal de schuld krijgt. Neem bijvoorbeeld de
gangbare opvatting dat niet de individuele crimineel, maar `de
maatschappij' verantwoordelijk is voor zijn daden. Stel je voor dat
Smith Jones berooft of vermoordt. De traditionele visie is dat Smith
verantwoordelijk is voor zijn actie. De moderne liberaal daarentegen
beweert dat `de maatschappij' hiervoor verantwoordelijk is. Dit klinkt
zowel verfijnd als humanitair, totdat we het individualistisch
perspectief toepassen. Dan zien we dat liberalen eigenlijk zeggen dat
iedereen behalve Smith, waaronder ook het slachtoffer Jones,
verantwoordelijk is voor de misdaad. Wanneer we het zo formuleren, zou
vrijwel iedereen de absurditeit van dit standpunt inzien. Toch verstoort
het oproepen van de fictieve entiteit `maatschappij' dit proces. Zoals
socioloog Arnold W. Green het verwoordt: `Hieruit volgt dat als de
maatschappij verantwoordelijk is voor misdaad en criminelen niet
verantwoordelijk worden gehouden, alleen die leden van de maatschappij
die geen misdaad plegen verantwoordelijk kunnen zijn voor misdaad. Zo'n
voor de hand liggende onzin kan alleen worden vermeden door de
maatschappij te presenteren als een kwaadwillend wezen, los van de
mensen en hun daden.'

De grote Amerikaanse libertarische auteur Frank Chodorov benadrukte deze
visie op de samenleving door te stellen: `Samenlevingen zijn mensen.'

\begin{quote}
De maatschappij is een collectief begrip en niets meer dan dat; het is
een handige term om een groep mensen aan te duiden. Dit geldt ook voor
woorden als familie, menigte of bende, of voor elke andere naam die we
geven aan een verzameling personen. De samenleving is geen aparte
`persoon'. Als de volkstelling bijvoorbeeld honderd miljoen bedraagt,
dan is dat alles wat er is; er zijn niet meer mensen, want er kan niets
aan de samenleving worden toegevoegd, behalve door voortplanting. Het
idee van de Samenleving als een metafysische entiteit stort in elkaar
wanneer we realiseren dat de Samenleving verdwijnt als de samenstellende
delen uit elkaar vallen. Dit zien we bijvoorbeeld bij een `spookstad' of
bij een beschaving waarover we leren door de artefacten die ze
achterlieten. Wanneer de individuen verdwijnen, verdwijnt ook het
geheel. Het geheel heeft geen eigen bestaan. Het gebruik van het
collectieve zelfstandig naamwoord met een enkelvoudig werkwoord leidt
ons in een denkfout; we zijn geneigd de collectiviteit te personaliseren
en te denken dat zij een eigen lichaam en psyche heeft.
\end{quote}

\section{\texorpdfstring{\textbf{VRIJE UITWISSELING EN VRIJ
CONTRACT}}{VRIJE UITWISSELING EN VRIJ CONTRACT}}\label{vrije-uitwisseling-en-vrij-contract}

De individualistische kijk op `de maatschappij' kan als volgt worden
samengevat: `De maatschappij' is iedereen behalve jezelf. Deze analyse
biedt een basis om te kijken naar situaties waarin `de maatschappij'
niet alleen als een superheld met speciale rechten wordt gezien, maar
ook als een superschurk die massaal de schuld krijgt. Neem bijvoorbeeld
de gangbare opvatting dat niet de individuele crimineel, maar `de
maatschappij' verantwoordelijk is voor zijn daden. Stel dat Smith Jones
berooft of vermoordt. De traditionele visie is dat Smith
verantwoordelijk is voor zijn acties. Tegenover die visie beweert de
moderne liberaal dat de `maatschappij' hiervoor verantwoordelijk is. Dit
klinkt zowel verfijnd als humanitair, maar wanneer we het een
individualistisch perspectief geven, zien we dat liberalen in wezen
zeggen dat iedereen, behalve Smith -- inclusief het slachtoffer Jones --
verantwoordelijk is voor de misdaad. Wanneer we het zo onder woorden
brengen, zou bijna iedereen de absurditeit van dit standpunt inzien.
Toch verstoort het idee van de fictieve entiteit `maatschappij' dit
proces. Zoals socioloog Arnold W. Green zegt: `Hieruit volgt dat als de
maatschappij verantwoordelijk is voor misdaad en criminelen niet
verantwoordelijk worden gehouden, alleen die leden van de maatschappij
die geen misdaad plegen verantwoordelijk kunnen worden gehouden. Zo'n
voor de hand liggende onzin kan alleen worden omzeild door de
maatschappij te presenteren als een kwaadwillend wezen, los van de
mensen en hun daden.' De grote Amerikaanse libertarische schrijver Frank
Chodorov benadrukte deze visie op de samenleving met de uitspraak:
`Samenlevingen zijn mensen.' De maatschappij is namelijk een collectief
begrip en niet meer dan dat; het is een handige term om een groep mensen
aan te duiden. Dit geldt ook voor termen als familie, menigte of bende,
net als voor elke andere benaming voor een verzameling personen. De
samenleving is geen afzonderlijke `persoon'. Als de volkstelling
bijvoorbeeld honderd miljoen bedraagt, is dat alles wat er is. Er zijn
niet meer mensen, want er kan alleen iets aan de samenleving worden
toegevoegd door voortplanting. Het idee van de Samenleving als een
metafysische entiteit faalt wanneer we ons realiseren dat de Samenleving
verdwijnt als de afzonderlijke delen uit elkaar vallen. Dit zien we
bijvoorbeeld bij een `spookstad' of een beschaving waarvan we leren door
de artefacten die zijn achtergelaten. Zodra de individuen verdwijnen,
verdwijnt ook het geheel. Het geheel heeft namelijk geen eigen bestaan.
Het gebruik van een collectief zelfstandig naamwoord met een enkelvoudig
werkwoord leidt ons in een denkfout; we zijn geneigd om de
collectiviteit te personaliseren en te denken dat het een eigen lichaam
en psyche heeft.

De centrale kern van het libertarische credo is het vaststellen van het
absolute recht op privébezit voor iedere persoon. Dit recht betreft ten
eerste het eigen lichaam en ten tweede de tot dan toe ongebruikte
natuurlijke hulpbronnen, die hij transformeert door middel van zijn
arbeid. Deze twee principes -- het recht op zelfbezit en het recht op
`homestead' -- vormen de basis van het libertarische systeem. De gehele
libertarische doctrine draait om het ontwikkelen en toepassen van alle
implicaties van deze centrale opvattingen. Bijvoorbeeld: een man, X,
bezit zijn eigen persoon en arbeid, evenals de boerderij die hij ontgint
en waar hij tarwe verbouwt. Een andere man, Y, bezit de vis die hij
vangt, en een derde man, Z, bezit de kool die hij heeft gekweekt en het
land daaronder. Wanneer iemand iets bezit, heeft hij bovendien het recht
om deze eigendomstitels weg te geven of te ruilen met een ander.
Hierdoor verkrijgt de andere persoon ook een absoluut eigendomsrecht.
Uit dit recht op privébezit volgt de basisprincipes voor vrije
contracten en de vrije markteconomie. Dus wanneer X tarwe verbouwt, zal
hij waarschijnlijk overeenkomen om een deel daarvan te ruilen voor een
deel van de vis die Y vangt, of voor een deel van de kool die Z
verbouwt. Als zowel X als Y vrijwillig overeenkomen om eigendomstitels
te ruilen (of Y en Z, of X en Z), dan wordt het eigendom met gelijke
legitimiteit het eigendom van de andere persoon. Wanneer X tarwe ruilt
voor de vis van Y, wordt die vis eigendom van X, zodat hij ermee kan
doen wat hij wil, en de tarwe wordt op dezelfde manier eigendom van Y.

Daarnaast kan een man niet alleen de materiële objecten die hij bezit
ruilen, maar ook zijn eigen arbeid, die hij uiteraard ook in zijn bezit
heeft. Zo kan Z zijn arbeid aanbieden door les te geven aan de kinderen
van boer X, in ruil voor een deel van de opbrengst van de boerderij.

Het toeval wil dat de vrije markteconomie, met de specialisatie en
arbeidsverdeling die deze met zich meebrengt, de productiefste vorm van
economie is die de mens kent. Deze economie heeft dan ook bijgedragen
aan de industrialisatie en de moderne beschaving. Hoewel dit een
positief utilitair resultaat van de vrije markt is, is het voor de
libertariër niet de belangrijkste reden om dit systeem te steunen. Die
centrale reden is moreel van aard en is gebaseerd op de
natuurrechtelijke verdediging van privébezit die we eerder hebben
besproken. Zelfs als aangetoond zou kunnen worden dat een maatschappij
die door despotisme wordt gekenmerkt en waar systematisch rechten worden
geschonden, productiever is dan wat Adam Smith `het systeem van
natuurlijke vrijheid' noemde, zou de libertariër dit systeem nog steeds
verwerpen. Gelukkig gaan, net als op veel andere gebieden, het
utilitarisme en de morele, natuurlijke rechten evenals de algemene
welvaart hand in hand.

Hoe complex het systeem ook lijkt, de ontwikkelde markteconomie is in
essentie niets meer dan een uitgebreid netwerk van vrijwillige en
wederzijds overeengekomen uitwisselingen tussen twee personen. Dit
hebben we al aangetoond met de voorbeelden van de tarwe- en koolboeren,
of de boer en de leraar. Neem bijvoorbeeld de situatie waarin ik voor
een dubbeltje een krant koop. Hier vindt een uitwisseling plaats die
voordelig is voor beide partijen. Ik draag mijn eigendom van het
dubbeltje over aan de krantenhandelaar, en hij draagt het eigendom van
de krant aan mij over. We doen dit omdat, volgens de arbeidsverdeling,
ik inschat dat de krant me meer waard is dan het dubbeltje, terwijl de
krantenverkoper liever het dubbeltje heeft dan de krant. Vergelijkbaar,
wanneer ik lesgeef aan een universiteit, vind ik het salaris dat ik
ontvang aantrekkelijker dan het niet besteden van mijn tijd aan
lesgeven. De universitaire autoriteiten berekenen ondertussen dat ze
liever mijn diensten als docent hebben dan mij geen geld te betalen. Als
de krantenhandelaar bijvoorbeeld 50 cent voor de krant zou vragen, zou
ik kunnen besluiten dat de prijs niet rechtvaardig is. Evenzo, als ik
zou aandringen op een verdrievoudiging van mijn huidige salaris, zou de
universiteit wel eens kunnen besluiten om geen gebruik meer te maken van
mijn diensten.

Veel mensen erkennen de rechtvaardigheid van eigendomsrechten en de
vrije markteconomie. Ze zijn het ermee eens dat de boer moet kunnen
vragen wat zijn graan opbrengt bij de consument, of dat de arbeider moet
kunnen ontvangen wat anderen bereid zijn te betalen voor zijn diensten.
Maar er is één punt waar ze op botsen: erfenis. Als Willie Stargell tien
keer zo goed en `productief' is als Joe Jack, vinden ze het terecht dat
Stargell tien keer zoveel verdient. Maar wat rechtvaardigt, vragen ze
zich af, dat iemand met als enige verdienste dat hij als Rockefeller is
geboren, veel meer rijkdom erft dan iemand die als Rothbard is geboren?
Het libertarische antwoord is dat men zich niet moet richten op de
ontvanger, of dat nu het kind Rockefeller of het kind Rothbard is, maar
op de gever: de man die de erfenis bekostigt. Want als Smith, Jones en
Stargell recht hebben op hun arbeid en eigendom en de mogelijkheid om
dit eigendom te ruilen voor vergelijkbaar eigendom van anderen, dan
hebben ze ook het recht om hun eigendom te schenken aan wie ze willen.
De meeste schenkingen bestaan namelijk uit de overdracht van eigendom
van ouders aan hun kinderen -- kortom, erfenis. Als Willie Stargell
eigenaar is van het geld dat hij verdient met zijn arbeid, heeft hij het
recht om dat geld aan de baby Stargell te geven.

In de vrije markteconomie ruilt de boer zijn tarwe voor geld. De
molenaar koopt de tarwe, verwerkt deze en verandert het in meel.
Vervolgens verkoopt de molenaar het meel aan de bakker, die er brood van
maakt. De bakker verkoopt dat brood weer aan de groothandelaar, die het
op zijn beurt doorverkoopt aan de detailhandelaar. Tenslotte verkoopt de
detailhandelaar het brood aan de consument. Bij elke stap kan de
producent arbeidskrachten inhuren in ruil voor geld. Hoe geld in dit
proces wordt gebruikt, is vrij ingewikkeld. Maar het is belangrijk te
begrijpen dat geld, conceptueel gezien, gelijkstaat aan elk individueel
product of elke groep nuttige producten die in ruil voor tarwe, meel,
enzovoort worden verhandeld. In plaats van geld zou er net zo goed stof,
ijzer of iets anders geruild kunnen worden. Bij iedere stap worden
wederzijds voordelige uitwisselingen van eigendomsovereenkomsten gemaakt
en uitgevoerd.

We kunnen nu kijken naar hoe de libertariër het begrip `vrijheid'
definieert. Vrijheid is een toestand waarin de eigendomsrechten van een
persoon over zijn lichaam en zijn legitieme materiële bezittingen niet
worden aangetast of aangevallen. Iemand die het eigendom van een ander
steelt, schendt de vrijheid van het slachtoffer en beperkt deze, net
zoals iemand die een ander op zijn hoofd slaat. Vrijheid en onbeperkt
eigendomsrecht gaan hand in hand. Aan de andere kant beschouwt de
libertariër `misdaad' als een daad van agressie tegen iemands
eigendomsrechten, hetzij op zijn persoon, hetzij op zijn bezittingen.
Misdaad is een inbreuk door middel van geweld op iemands eigendom en
daardoor op zijn vrijheid. `Slavernij' -- het tegenovergestelde van
vrijheid -- is een toestand waarin de slaaf weinig of geen recht heeft
op zelfeigendom. Zijn persoon en bezittingen worden door zijn meester
systematisch met geweld onteigend.

De libertariër is dus duidelijk een individualist, maar geen strijder
voor gelijkheid. De enige `gelijkheid' die hij zal voorstaan, is het
gelijke recht van elke mens op het eigendom van zijn eigen persoon, op
het eigendom van ongebruikte middelen die hem `toebehoren' en op het
eigendom van anderen dat hij heeft verkregen door vrijwillige ruil of
schenking.

\section{EIGENDOMSRECHTEN EN
`MENSENRECHTEN'}\label{eigendomsrechten-en-mensenrechten}

In een vrije markteconomie ruilt de boer zijn tarwe voor geld. De
molenaar koopt deze tarwe, verwerkt het en maakt er meel van. Daarna
verkoopt de molenaar het meel aan de bakker, die er brood van bakt. De
bakker verkoopt het brood aan de groothandelaar, die het weer
doorverkoopt aan de detailhandelaar. Tot slot verkoopt de
detailhandelaar het brood aan de consument. Bij elke stap kan de
producent arbeidskrachten inhuren in ruil voor geld. Het proces van hoe
geld hierin wordt gebruikt is vrij complex. Belangrijk is echter te
begrijpen dat geld, conceptueel gezien, gelijkstaat aan elk individueel
product of elke categorie nuttige producten die in ruil voor tarwe,
meel, enzovoort worden verhandeld. In plaats van geld kan er ook met
stof, ijzer of iets anders worden geruild. Bij elke stap worden
wederzijds voordelige uitwisselingen van eigendomsovereenkomsten gemaakt
en uitgevoerd. Nu kunnen we kijken naar de manier waarop libertariërs
het begrip `vrijheid' definiëren. Vrijheid is een toestand waarin de
eigendomsrechten van een persoon over zijn lichaam en legitieme
materiële bezittingen niet worden aangetast of aangevallen. Iemand die
het eigendom van een ander steelt, schendt de vrijheid van het
slachtoffer en beperkt deze. Dit is vergelijkbaar met iemand die een
ander verwondt. Vrijheid en onbeperkt eigendomsrecht zijn nauw met
elkaar verbonden. Voor een libertariër betekent `misdaad' een daad van
agressie tegen iemand zijn eigendomsrechten, hetzij op zijn persoon,
hetzij op zijn bezittingen. Misdaad is dus een geweldpleging tegen
iemand zijn eigendom en daarmee ook tegen zijn vrijheid. `Slavernij' --
het tegenovergestelde van vrijheid -- is de staat waarin de slaaf weinig
of geen recht heeft op zelfeigendom. Zijn persoon en bezittingen worden
door zijn meester met geweld onteigend. De libertariër is dus duidelijk
een individualist, maar geen voorvechter van gelijkheid. De enige soort
`gelijkheid' die hij ondersteunt, is het gelijke recht van elke mens op
het eigendom van zijn eigen persoon, op ongebruikte middelen die hem
`toebehoren', en op het eigendom van anderen dat hij heeft verkregen via
vrijwillige ruil of schenking.

Liberalen erkennen over het algemeen het recht van elk individu op
`persoonlijke vrijheid'. Dit omvat de vrijheid om te denken, te spreken,
te schrijven en deel te nemen aan persoonlijke `uitwisselingen', zoals
seksuele activiteiten tussen `instemmende volwassenen'. Kortom, de
liberaal probeert het recht van het individu op het eigendom van zijn
eigen lichaam te waarborgen, maar ontkent vervolgens zijn recht op
`eigendom', oftewel eigendom van materiële zaken. Dit leidt tot de
typisch liberale tegenstellingen tussen de `mensenrechten' die hij
verdedigt en de `eigendomsrechten' die hij verwerpt. Voor de libertariër
zijn deze twee echter onlosmakelijk met elkaar verbonden; ze staan of
vallen samen.

Neem bijvoorbeeld de liberale socialist die pleit voor overheidseigendom
van alle `productiemiddelen', terwijl hij tegelijkertijd het
`menselijke' recht op vrijheid van meningsuiting en persvrijheid
verdedigt. Hoe kan dit `menselijke' recht worden uitgeoefend als de
individuen die het publiek vormen, hun recht op eigendom van goederen
wordt ontzegd? Als de overheid bijvoorbeeld al het krantenpapier en alle
drukkerijen in handen heeft, hoe kan het recht op persvrijheid dan
worden waargemaakt? Zodra de overheid al het krantenpapier bezit, heeft
ze de macht om te bepalen wie dat papier krijgt. Dit maakt iemands
`recht op een vrije pers' een schijnvertoning, mocht de overheid
besluiten om hem geen krantenpapier toe te wijzen. Bovendien moet de
overheid het schaarse krantenpapier op een of andere manier verdelen,
wat betekent dat het recht op een vrije pers voor bijvoorbeeld
minderheden of `subversieve' antisocialisten waarschijnlijk niet
gerespecteerd zal worden. Dit geldt ook voor het `recht op vrije
meningsuiting'. Als de overheid eigenaar is van alle vergaderzalen en
deze naar eigen inzicht toewijst, hebben individuen weinig ruimte om hun
mening te uiten. Stel je voor dat de atheïstische regering van
Sovjet-Rusland besluit om belangrijke middelen niet toe te wijzen aan de
productie van matzohs. Dan wordt de `vrijheid van godsdienst' voor
orthodoxe Joden een schijnvertoning. De Sovjetregering kan echter altijd
argumenteren dat orthodoxe Joden een kleine minderheid vormen, en dat
het niet noodzakelijk is om kapitaalgoederen aan de productie van
matzohs te besteden.

De fundamentele fout in de liberale scheiding tussen `mensenrechten' en
`eigendomsrechten' is dat mensen worden gezien als etherische
abstracties. Als een persoon recht heeft op zelfeigenaarschap en
controle over zijn leven, dan moet hij in de echte wereld ook het recht
hebben om zijn leven te onderhouden door middelen te benutten en te
transformeren. Hij moet de grond en de hulpbronnen waarop hij staat en
die hij nodig heeft, kunnen bezitten. Kortom, om zijn `mensenrecht' --
of zijn eigendomsrecht op zijn eigen persoon -- te waarborgen, moet hij
ook het eigendomsrecht hebben op de materiële wereld, op de objecten die
hij produceert. Eigendomsrechten zijn mensenrechten en zijn essentieel
voor de mensenrechten die liberalen trachten te verdedigen. Het
mensenrecht van een vrije pers is afhankelijk van het mensenrecht op
privé-eigendom van krantenpapier.

In wezen kunnen mensenrechten niet los gezien worden van
eigendomsrechten. Het mensenrecht op vrijheid van meningsuiting komt
simpelweg neer op het eigendomsrecht om een vergaderzaal te huren van de
eigenaren of om er zelf een te bezitten. Het mensenrecht op persvrijheid
houdt in dat je het recht hebt om materialen aan te schaffen en
vervolgens pamfletten of boeken te drukken en deze te verkopen aan
geïnteresseerden. Er bestaat geen extra `recht op vrije meningsuiting'
of persvrijheid dat losstaat van de eigendomsrechten die in een bepaalde
situatie van toepassing zijn. Bovendien zal het onderzoeken en
identificeren van de relevante eigendomsrechten eventuele schijnbare
conflicten tussen rechten oplossen.

Neem bijvoorbeeld het klassieke voorbeeld waarin liberalen over het
algemeen toegeven dat iemands `recht op vrijheid van meningsuiting' moet
worden ingeperkt in het belang van het `algemeen belang'. Denk aan het
beroemde oordeel van rechter Holmes, waarin staat dat niemand het recht
heeft om onterecht `brand' te roepen in een overvol theater. Holmes en
zijn volgelingen hebben deze illustratie keer op keer gebruikt om de
veronderstelde noodzaak aan te tonen dat alle rechten relatief en
voorlopig zijn, in plaats van precies en absoluut.

Maar het probleem hier is niet dat rechten niet te ver kunnen worden
doorgedreven. Het zit `m erin dat de kwestie wordt besproken in vage en
wollige termen van 'vrijheid van meningsuiting', in plaats van vanuit
het perspectief van de rechten van privébezit. Laten we het probleem
analyseren aan de hand van eigendomsrechten. De man die een rel
veroorzaakt door vals `brand' te roepen in een overvol theater, is ofwel
de eigenaar van het theater (of de vertegenwoordiger van de eigenaar) of
een betalende bezoeker. Als hij de eigenaar is, dan heeft hij zijn
klanten misleid. Hij heeft hun geld aangenomen in ruil voor de belofte
dat er een film of toneelstuk zou worden opgevoerd. Nu verstoort hij de
voorstelling door vals `brand' te roepen en deze onderbreking aan te
richten. Hierdoor heeft hij zijn contractuele verplichtingen verzaakt en
daarmee het eigendom -- het geld -- van zijn klanten gestolen en hun
eigendomsrechten geschonden.

Stel je voor dat de schreeuwer een bezoeker is en geen eigenaar. In dat
geval schendt hij het eigendomsrecht van de eigenaar, evenals dat van de
andere gasten die voor hun toegang hebben betaald. Als gast heeft hij
onder bepaalde voorwaarden toegang gekregen tot het eigendom. Dit omvat
de verplichting om het eigendom van de eigenaar niet te schenden en de
voorstelling niet te verstoren. Zijn opzettelijke actie schaadt daarom
de eigendomsrechten van de theatereigenaar en van alle andere bezoekers.

Het is dus niet nodig om individuele rechten te beperken in het geval
van de man die vals `brand' roept. De rechten van het individu blijven
absoluut; het zijn eigendomsrechten. De persoon die kwaadwillig `brand'
roept in een overvol theater is inderdaad een crimineel, maar niet omdat
zijn zogenaamde `recht op vrije meningsuiting' pragmatisch ingeperkt
moet worden in het belang van het `algemeen belang'. Hij is een
crimineel omdat hij duidelijk en onmiskenbaar de eigendomsrechten van
een ander heeft geschonden.

\bookmarksetup{startatroot}

\chapter{De staat}\label{de-staat}

\section{De staat als agressor}\label{de-staat-als-agressor}

De kerngedachte van het libertarische denken is om zich te verzetten
tegen elke vorm van agressie tegen de eigendomsrechten van individuen,
zowel in hun persoon als in de materiële voorwerpen die zij vrijwillig
hebben verworven. Hoewel individuele criminelen en bendes
vanzelfsprekend worden bestreden, is er niets unieks aan het
libertarische credo. Bijna alle mensen en stromingen verzetten zich
immers tegen willekeurig geweld tegen personen en eigendom.

Er is echter een verschil in nadruk bij libertariërs, zelfs op het
gebied van het algemeen aanvaarde principe om mensen te beschermen tegen
criminaliteit. In een libertarische samenleving zou er geen `officier
van justitie' zijn die criminelen vervolgt in naam van een
niet-bestaande `samenleving', en dit zelfs tegen de wil van het
slachtoffer. Het slachtoffer zou zelf beslissen of hij aangifte doet.
Aan de andere kant zou het slachtoffer in een libertarische wereld ook
een aanklacht kunnen indienen tegen een overtreder, zonder dat hij eerst
diezelfde officier van justitie moet overtuigen van de noodzaak om
verder te gaan. Bovendien zou in het strafsysteem van de libertarische
wereld nooit, zoals nu, de nadruk liggen op het opsluiten van de
misdadiger door de `maatschappij'. De focus zou in plaats daarvan liggen
op het dwingen van de misdadiger om het slachtoffer schadeloos te
stellen. Het huidige systeem, waarin het slachtoffer geen
schadevergoeding krijgt en in plaats daarvan belasting betaalt om de
opsluiting van zijn eigen aanvaller te bekostigen, zou in een wereld die
gericht is op de bescherming van eigendomsrechten en het welzijn van
slachtoffers klinkklare onzin zijn.

Bovendien, hoewel de meeste libertariërs geen pacifisten zijn, zullen ze
het huidige systeem niet ondersteunen door in te grijpen in het recht
van mensen om pacifist te zijn. Stel je voor dat Jones, een pacifist,
wordt aangevallen door Smith, een crimineel. Als Jones, vanwege zijn
overtuigingen, tegen het verdedigen van zichzelf door middel van geweld
is en ook niet wil dat er vervolging van de misdaad plaatsvindt, dan zal
hij eenvoudigweg geen aangifte doen. Daarmee is de kous af. Er zal geen
overheidsapparaat zijn dat de pacifist kan vervolgen. Evenmin zal er een
overheidsorgaan zijn dat criminelen vervolgt en berecht, zelfs als het
slachtoffer dat niet wil.

Het belangrijkste verschil tussen libertariërs en anderen ligt niet bij
privécriminaliteit, maar in hun visie op de rol van de staat, oftewel de
overheid. Libertariërs beschouwen de staat als de grootste, meest
blijvende en best georganiseerde agressor tegen zowel de personen als de
eigendommen van het publiek. Dit geldt voor alle staten, ongeacht of ze
nu democratisch, dictatoriaal of monarchaal zijn, en ongeacht hun
politieke kleur.

De Staat! Altijd zijn de regering en haar heersers boven de algemene
morele wet geplaatst. De `Pentagon Papers' zijn slechts één van de vele
voorbeelden uit de geschiedenis van mannen die in hun privéleven
volkomen eerbaar zijn, maar zich tegenover het publiek als leugenaar
gedragen. Waarom? Om `staatsredenen'. Dienstbaarheid aan de staat wordt
gezien als een excuus voor handelingen die als immoreel of crimineel
beschouwd zouden worden als ze door gewone burgers waren gepleegd. Wat
libertariërs onderscheidt, is dat ze de algemene morele wet strak en
compromisloos toepassen op mensen die handelen als leden van het
staatsapparaat. Libertariërs maken geen uitzonderingen. Eeuwenlang heeft
de staat, of beter gezegd, individuen die als `lid van de regering'
handelen, zijn criminele activiteiten omhuld met hoogdravende retoriek.
Eeuwenlang heeft de staat massamoord gepleegd, dat `oorlog' genoemd, en
de massaslachting die daarmee gepaard gaat verheerlijkt. Eindeloos heeft
de staat mensen tot slaven gemaakt voor zijn gewapende bataljons en dit
`dienstplicht' of `nationale dienst' genoemd. Eeuwenlang heeft de staat
mensen met bajonetten beroofd en dit `belastingen' genoemd. Als je wilt
begrijpen hoe libertariërs naar de staat en haar daden kijken, zie de
staat dan simpelweg als een criminele bende. Dan vallen alle
libertarische standpunten vanzelf op hun plaats.

Laten we eens kijken naar wat de overheid onderscheidt van andere
organisaties in de samenleving. Veel politicologen en sociologen
vervagen dit belangrijke verschil door naar alle organisaties en groepen
te verwijzen als hiërarchisch, gestructureerd of `gouvernementeel'.
Linkse anarchisten, bijvoorbeeld, verzetten zich zowel tegen de overheid
als tegen particuliere organisaties zoals bedrijven, omdat zij beiden
als even `elitair' en `dwingend' beschouwen. De `rechtse' libertariër
daarentegen is niet tegen ongelijkheid, en zijn idee van `dwang' geldt
alleen voor geweld. Libertariërs maken een cruciaal onderscheid tussen
de overheid --- of die nu centraal, staats- of lokaal is --- en andere
instellingen in de samenleving. Of liever gezegd: twee cruciale
verschillen. Ten eerste ontvangt elke andere persoon of groep zijn
inkomen via vrijwillige betalingen, zoals bijdragen of giften
(bijvoorbeeld aan een lokale gemeenschap of bridgeclub), of door
vrijwillige verkoop van goederen of diensten op de markt (bijvoorbeeld
een kruidenier, honkbalspeler of staalfabrikant). Alleen de overheid
verkrijgt haar middelen door dwang en geweld, dat wil zeggen via de
dreiging van inbeslagname of gevangenisstraf als betaling uitblijft. Dit
gedwongen innen van geld noemen we `belasting'. Een tweede belangrijk
verschil is dat, behalve criminele misdadigers, alleen de overheid haar
middelen kan gebruiken om geweld uit te oefenen tegen haar eigen of
andere burgers. Alleen de overheid kan bijvoorbeeld pornografie
verbieden, religieuze verplichtingen afdwingen of mensen opsluiten voor
het verkopen van goederen tegen een hogere prijs dan de overheid
toestaat. Deze verschillen kunnen samengevat worden als: alleen de
overheid is bevoegd om de eigendomsrechten van haar burgers te schenden,
of dit nu is om inkomsten te verwerven, om haar morele code op te leggen
of om mensen te doden met wie ze het niet eens is. Bovendien verkrijgen
alle regeringen, zelfs de minst heerszuchtige, het grootste deel van hun
inkomsten uit de dwingende macht van belastingheffing. Historisch gezien
komt het merendeel van alle slavernij en moord in de wereldgeschiedenis
uit de overheid. Aangezien de centrale drijfveer van libertariërs is om
zich te verzetten tegen elke agressie tegen de rechten van persoon en
eigendom, verzetten zij zich noodzakelijkerwijs tegen de staat als de
inherente en belangrijkste vijand van die kostbare rechten.

Er is nog een andere reden waarom staatsagressie veel belangrijker is
dan privé-agressie. Deze reden heeft te maken met de grotere organisatie
en centrale mobilisatie van middelen die heersers van de staat kunnen
opleggen. Het belangrijkste is echter de afwezigheid van enige controle
op staatsdiefstal, terwijl die controle wel bestaat wanneer we te maken
hebben met overvallers of de maffia. Om ons te beschermen tegen
privécriminelen kunnen we ons wenden tot de staat en de politie. Maar
wie beschermt ons tegen de staat zelf? Niemand. Een cruciaal kenmerk van
de staat is namelijk dat deze het monopolie op de bescherming afdwingt.
De staat eigent zich een vrijwel monopolie toe op geweld en op de
uiteindelijke besluitvorming binnen de samenleving. Als we bijvoorbeeld
niet tevreden zijn met de beslissingen van de staatsrechtbanken, zijn er
geen andere beschermingsinstanties waar we terechtkunnen.

Het klopt dat we, althans in de Verenigde Staten, een grondwet hebben
die strikte beperkingen oplegt aan bepaalde bevoegdheden van de
overheid. Maar zoals we in de afgelopen eeuw hebben gezien, kan geen
enkele grondwet zichzelf interpreteren of handhaven; mensen moeten dat
doen. Wanneer de uiteindelijke macht om een grondwet te interpreteren
wordt toegewezen aan het Hooggerechtshof van de eigen regering, ontstaat
er al snel de neiging dat het Hof steeds ruimere bevoegdheden voor de
regering zal legitimeren. Bovendien zijn de vaak geprezen `checks and
balances' en de `scheiding der machten' in de Amerikaanse regering
nauwelijks effectief, omdat al deze onderdelen uiteindelijk deel
uitmaken van dezelfde overheid en bestuurd worden door dezelfde
machthebbers.

Een van Amerika's meest briljante politieke theoretici, John C. Calhoun,
schreef op profetische wijze over de inherente neiging van een staat om
de grenzen van zijn grondwet te overschrijden:

\begin{quote}
Een geschreven grondwet biedt zeker veel voordelen, maar het is een
grote vergissing om te denken dat het simpelweg opnemen van bepalingen
die de bevoegdheden van de regering beperken voldoende is. Het is
essentieel dat degenen voor wie deze bepalingen zijn bedoeld, ook de
middelen hebben om de naleving ervan af te dwingen. Zonder deze middelen
zal de dominante partij niet terugschrikken voor misbruik van haar
bevoegdheden. De partij die aan de macht is, zal zich altijd inzetten
voor de bevoegdheden die de grondwet toekent en tegen de beperkingen die
bedoeld zijn om deze bevoegdheden te reguleren. Als dominante partij
heeft ze deze beperkingen niet nodig om zichzelf te beschermen.

De kleinere of zwakkere partij zou daarentegen de tegenovergestelde
richting inslaan en deze beperkingen essentieel achten voor hun
bescherming tegen de dominante partij. Als er echter geen middelen zijn
om de grote partij te dwingen de regels na te leven, is hun enige hoop
een strikte interpretatie van de grondwet. De dominante partij zou zich
verzetten tegen een liberale interpretatie -- een benadering die de
woorden van de machtiging de ruimst mogelijke betekenis zou geven. Het
zou dan gaan om constructie versus constructie: de ene om te beperken en
de andere om de bevoegdheden van de regering tot het uiterste uit te
breiden. Maar wat heeft de strikte interpretatie van de zwakkere partij
voor nut tegenover de liberale interpretatie van de sterke partij? Hangt
de uitkomst niet af van de partij die de meeste macht heeft om haar
visie door te voeren? In zo'n ongelijke strijd is het resultaat niet
twijfelachtig. De partij die pleit voor beperkingen zou worden
overwonnen. Het einde van dit conflict zou de ondermijning van de
grondwet inhouden. De beperkingen zouden uiteindelijk nietig verklaard
worden en de regering zou uitgroeien tot een organisatie met onbeperkte
macht.

Noch zou de verdeling van de regering in afzonderlijke en, ten opzichte
van elkaar, onafhankelijke departementen dit resultaat voorkomen.
Aangezien elk departement -- en in feite de hele regering -- onder
controle staat van de numerieke meerderheid, is het overduidelijk dat
een loutere verdeling van bevoegdheden onder haar agenten of
vertegenwoordigers weinig of niets zal bijdragen om onderdrukking en
machtsmisbruik tegen te gaan.¹
\end{quote}

Maar waarom zou je je zorgen maken over de zwakte van de grenzen aan de
macht van de overheid? Vooral in een `democratie', zoals Amerikaanse
liberalen het vaak zeiden in hun hoogtijdagen voor het midden van de
jaren zestig, toen twijfels de liberale utopie begonnen te doordringen:
`Zijn wij niet de regering?' In de zin `wij zijn de regering' verbergt
de handige collectieve term `wij' de keiharde werkelijkheid van het
politieke leven. Als wij echt de overheid zijn, dan is alles wat een
overheid een individu aandoet niet alleen rechtvaardig en niet
tiranniek, maar ook `vrijwillig' van de kant van het betrokken individu.
Als de overheid een enorme staatsschuld heeft opgebouwd die betaald moet
worden door de één groep te belasten voor de andere, wordt deze last
gemakkelijk verdoezeld met de uitspraak `we hebben het aan onszelf te
danken' (maar wie zijn `wij' en wie is `onszelf'?). Als de overheid een
man oproept of zelfs in de gevangenis steekt vanwege een afwijkende
mening, dan `doet hij het alleen zichzelf aan' en is er dus niets
ongepasts gebeurd. Volgens deze redenering zijn de Joden die door de
Nazi-regering werden vermoord niet vermoord; zij zouden `zelfmoord
hebben gepleegd' omdat zij de regering waren (die democratisch gekozen
was). Daarom was alles wat de regering met hen deed alleen maar
vrijwillig van hun kant. Maar voor die aanhangers van de overheid die de
staat zien als een welwillende en vrijwillige vertegenwoordiger van het
publiek, is er geen ontsnapping uit zulke groteske situaties.

We moeten daarom concluderen dat `wij' niet de regering zijn; de
regering is niet `wij'. De regering vertegenwoordigt op geen enkele
manier de meerderheid van het volk. Zelfs als 90 procent van de
bevolking zou besluiten om de andere 10 procent te vermoorden of tot
slaaf te maken, blijft dit moord en slavernij. Het zou geen vrijwillige
zelfmoord of slavernij door de onderdrukte minderheid zijn. Misdaad is
misdaad en agressie tegen rechten is agressie, ongeacht het aantal
burgers dat instemt met de onderdrukking. Er is niets heiligs aan de
meerderheid; zelfs een lynchpartij kan in haar eigen gebied als
meerderheid worden beschouwd.

Maar terwijl de meerderheid, zoals bij een lynchpartij, actief tiranniek
en agressief kan optreden, is de normale en voortdurende toestand van de
staat doorgaans een oligarchische heerschappij. Dit betekent dat een
dwingende elite controle heeft gekregen over het staatsapparaat. Hier
zijn twee fundamentele redenen voor: ten eerste de ongelijkheid en
arbeidsverdeling die inherent zijn aan de menselijke aard, wat leidt tot
een `IJzeren Wet van Oligarchie' in alle menselijke activiteiten; en ten
tweede de parasitaire aard van de staat zelf.

We hebben gesteld dat de individualist geen strijder voor gelijkheid is.
Een van de redenen hiervoor is het besef van de individualist dat er een
enorme diversiteit en individualiteit binnen de mensheid bestaat. Deze
diversiteit heeft de kans om te bloeien en zich uit te breiden naarmate
de beschaving en de levensstandaard verbeteren. Individuen verschillen
in vaardigheden en interesses, zowel binnen als tussen beroepen. Daarom
zal het leiderschap in elke beroepsgroep en sociale laag, of het nu gaat
om staalproductie of het organiseren van een bridgeclub, onvermijdelijk
worden overgenomen door een relatief klein aantal van de meest bekwame
en energieke mensen, terwijl de rest zich zal schikken als gewone
volgelingen. Deze waarheid geldt voor alle activiteiten, ongeacht of ze
heilzaam of kwaadaardig zijn, zoals in het geval van criminele
organisaties. De ontdekking van de IJzeren Wet van de Oligarchie werd
gedaan door de Italiaanse socioloog Robert Michels. Hij ontdekte dat de
Sociaal-Democratische Partij van Duitsland, ondanks haar retorische
toewijding aan gelijkheid, in de praktijk een rigide oligarchische en
hiërarchische structuur had.

Een tweede fundamentele reden voor de oligarchische heerschappij van de
staat is de parasitaire aard ervan. De staat leeft namelijk gedwongen
van de productie van burgers. Voor de beoefenaars van deze parasitaire
uitbuiting moet de opbrengst beperkt blijven tot een relatief kleine
groep. Anders zou het leiden tot een zinloze plundering waarbij niemand
profijt heeft. Nergens wordt de dwingende en parasitaire aard van de
staat zo duidelijk toegelicht als door de grote Duitse socioloog Franz
Oppenheimer aan het eind van de negentiende eeuw. Oppenheimer stelde dat
er twee en slechts twee manieren zijn waarop mensen rijkdom kunnen
verwerven. De eerste is de methode van productie en vrijwillige ruil,
wat hij de `economische middelen' noemde. De tweede is de methode van
roof door middel van geweld, die hij de `politieke middelen' noemde. Het
politieke middel is duidelijk parasitair, omdat het vereist dat er
voorafgaand geproduceerd is, zodat de uitbuiters iets kunnen
confisqueren. Het trekt waarde weg uit de totale productie in de
samenleving in plaats van hier iets aan toe te voegen. Oppenheimer
definieerde de staat vervolgens als de `organisatie van de politieke
middelen' - een systematisering van dit roofzuchtige proces binnen een
bepaald territoriaal gebied.

Kortom, particuliere misdaad is hooguit sporadisch en onvoorspelbaar.
Het parasitisme is van korte duur, en de dwingende, parasitaire
levenslijn kan op elk moment worden doorgesneden door het verzet van de
slachtoffers. De staat biedt een legaal, ordelijk en systematisch kanaal
voor het roven van het eigendom van producenten. Hierdoor wordt de
levenslijn van de parasitaire elite in de samenleving zeker, veilig en
relatief `vredig'. De grote libertarische schrijver Albert Jay Nock
beschreef treffend dat `de staat het monopolie op misdaad opeist en
uitoefent\ldots{} Hij verbiedt privé-moord, maar organiseert zelf moord
op kolossale schaal. Hij bestraft privédiefstal, maar bestrijkt
gewetenloos alles wat hij wil, of het nu gaat om het eigendom van een
burger of van een vreemdeling.'

In het begin kan het schokkend zijn om belasting te beschouwen als roof
en de staat als een bende rovers. Toch kan iedereen die blijft denken
dat belasting in zekere zin een `vrijwillige' betaling is, de gevolgen
ervaren als hij ervoor kiest om niet te betalen. De beroemde econoom
Joseph Schumpeter, die zeker geen libertariër was, stelde dat `de staat
leeft van inkomsten die in de privésfeer voor privédoeleinden worden
geproduceerd en door politiek geweld van deze doeleinden moesten worden
afgewend. De theorie die belastingen vergelijkt met clubgelden of de
betaling voor bijvoorbeeld een dokter, toont alleen maar aan hoe ver dit
deel van de sociale wetenschappen verwijderd is van werkelijk
wetenschappelijk denken.'4 De vooraanstaande Weense rechtspositivist
Hans Kelsen probeerde in zijn essay \emph{The General Theory of Law and
the State} een politieke theorie en rechtvaardiging voor de staat op te
stellen op een strikt 'wetenschappelijke' en waardevrije basis. Wat
gebeurde er? Al vroeg in het boek stuitte hij op het cruciale knelpunt,
de pons asinorum van de politieke filosofie: wat onderscheidt de
verordeningen van de staat van de bevelen van een bende? Kelsen's
antwoord was simpelweg dat de decreten van de staat `geldig' zijn,
waarna hij vrolijk verder ging zonder te proberen dit concept van
`geldigheid' te definiëren of uit te leggen. Het zou inderdaad een
nuttige oefening zijn voor niet-libertariërs om over deze vraag na te
denken: hoe kun je belasting zo definiëren dat het anders is dan roof?

Voor de grote negentiende-eeuwse individualistische anarchist en
grondwettelijk jurist Lysander Spooner was het geen probleem om het
antwoord te vinden. Spooners analyse van de staat als roversgroep is
misschien wel de meest vernietigende die ooit is geschreven:

\begin{quote}
Het klopt dat de theorie van onze Grondwet stelt dat alle belastingen
vrijwillig zijn; dat onze regering functioneert als een onderlinge
verzekeringsmaatschappij, waartoe de mensen vrijwillig zijn toegetreden.

Maar deze theorie over onze regering wijkt totaal af van de
werkelijkheid. De feitelijke situatie is dat de regering, net als een
struikrover, tegen mensen zegt: `Je geld of je leven.' Veel
belastingbetalingen, zo niet de meeste, gebeuren onder druk van dat
dreigement.

Het klopt dat de overheid iemand niet op een afgelegen plek aanhoudt,
hem vanaf de kant van de weg bespringt en met een pistool tegen zijn
hoofd zijn zakken doorzoekt. Maar daardoor is het geen beroving, en het
is misschien zelfs geniepiger en schandelijker.

De struikrover neemt de verantwoordelijkheid, het gevaar en de misdaad
van zijn daad volledig op zich. Hij doet niet alsof hij recht heeft op
je geld of dat hij van plan is dit voor jouw eigen voordeel te
gebruiken. Hij is eerlijk over zijn rol en presenteert zich niet als
iets anders dan een rover. Hij is niet zo schaamteloos om te beweren dat
hij slechts een `beschermer' is en dat hij het geld van mensen tegen hun
wil aanneemt om die verdwaasde reizigers te `beschermen', die zichzelf
eigenlijk heel goed kunnen beschermen of zijn zogenaamde
beschermingssysteem niet waarderen. Hij is veel te slim om zulke onzin
te verkondigen. Bovendien laat hij je, nadat hij je geld heeft
aangenomen, met rust. Hij blijft je niet tegen je wil volgen; hij gaat
er niet vanuit dat hij jouw `soeverein' is vanwege de `bescherming' die
hij zegt te bieden. Hij probeert je niet `te beschermen' door je te
dwingen om hem te gehoorzamen en hem te dienen. Hij zal je niet
verplichten dit te doen of je verbieden dat te doen. Hij berooft je niet
steeds opnieuw van je geld, zolang het hem niet uitkomt of het voor zijn
eigen plezier is. En als je zijn autoriteit betwist of je tegen zijn
eisen verzet, zal hij je niet zonder genade neerschieten en je aanduiden
als een rebel, verrader of vijand van je land. Hij is te veel een
gentleman om zich schuldig te maken aan dergelijke bedrog, beledigingen
en schurkenstreken. Kortom, hij berooft je niet alleen, maar probeert
ook niet om jou tot zijn dienaar of slaaf te maken.
\end{quote}

Als de staat een groep plunderaars is, wie vormt de staat dan? Het is
duidelijk dat de heersende elite op elk moment bestaat uit (a) het
fulltime apparaat -- koningen, politici en bureaucraten die de staat
bemannen en bedienen; en (b) de groepen die hebben gemanoeuvreerd om
privileges, subsidies en voordelen van de staat te verkrijgen. De rest
van de samenleving bestaat uit de geregeerden. Het was weer John C.
Calhoun die helder zag dat, ongeacht hoe klein de macht van de overheid
is, hoe laag de belastingdruk ook is of hoe gelijkmatig deze wordt
verdeeld, de aard van de overheid twee ongelijkheid en inherent
conflicterende klassen creëert in de maatschappij: degenen die netto
belasting betalen (de `belastingbetalers') en degenen die netto van de
belastingen leven (de `belastingconsumenten'). Stel je voor dat de
overheid een lage en ogenschijnlijk gelijkmatig verdeelde belasting
oplegt om de bouw van een dam te financieren. Deze actie onttrekt geld
aan het grote publiek om uit te betalen aan de netto
`belastingconsumenten': de bureaucraten die de operatie leiden, de
aannemers en arbeiders die de dam bouwen, en dergelijke. En naarmate de
reikwijdte van de overheidsbesluiten toeneemt, worden de fiscale lasten
zwaarder, vervolgde Calhoun. Dit leidt tot meer druk en een kunstmatige
ongelijkheid tussen deze twee klassen.

\begin{quote}
Hoewel de agenten en werknemers van de overheid relatief klein in aantal
zijn, vormen zij het deel van de gemeenschap dat exclusief de
opbrengsten van de belastingen ontvangt. Welk bedrag er ook in de vorm
van belastingen van de gemeenschap wordt afgenomen, als dat niet
verloren gaat, komt uiteindelijk bij hen terecht in de vorm van uitgaven
of uitbetalingen. Deze twee elementen -- uitbetalingen en belastingen --
vormen samen de fiscale actie van de overheid. Ze zijn met elkaar
verbonden. Wat de overheid van de gemeenschap afneemt als belastingen,
wordt vervolgens overgedragen aan het deel dat de ontvangers zijn in de
vorm van uitbetalingen. Omdat de ontvangers slechts een deel van de
gemeenschap vormen, volgt hieruit dat de werking van het fiscale systeem
ongelijk is tussen de belastingbetalers en de ontvangers. Dit kan niet
anders, tenzij elk individu hetgeen hij in de vorm van belastingen
betaalt, ook in gelijke mate terugkrijgt in de vorm van uit betalingen.
Dit zou het proces namelijk zinloos en absurd maken.

Het resultaat van de ongelijke fiscale acties van de overheid is dat de
gemeenschap in twee grote klassen wordt verdeeld. De eerste klasse
bestaat uit degenen die daadwerkelijk de belastingen betalen en dus de
last dragen van het onderhouden van de overheid. De tweede klasse
bestaat uit degenen die de uitbetalingen ontvangen en in feite door de
overheid worden ondersteund. Kort gezegd, we hebben belastingbetalers en
belastingconsumenten.

Maar het gevolg hiervan is dat ze tegenovergestelde relaties aangaan met
betrekking tot de fiscale maatregelen van de overheid en het beleid dat
daarmee samenhangt. Hoe hoger de belastingen en uitbetalingen, hoe
groter de winst voor de één en het verlies voor de ander, en vice
versa\ldots{} Elke verhoging leidt er dus toe dat de één zich verrijkt
en versterkt, terwijl de ander verarmt en verzwakt.
\end{quote}

Als staten overal worden geleid door een oligarchische groep rovers, hoe
kunnen zij dan hun heerschappij over de bevolking handhaven? Het
antwoord, zoals de filosoof David Hume meer dan twee eeuwen geleden al
opmerkte, is dat op de lange termijn elke regering, hoe dictatorial ook,
afhankelijk is van de steun van de meerderheid van de onderdanen. Dit
maakt deze regeringen echter niet `vrijwillig', want het bestaan van
belastingen en andere dwangmiddelen toont aan hoeveel druk de staat moet
uitoefenen. Bovendien hoeft de steun van de meerderheid geen
enthousiaste goedkeuring te zijn; het kan ook gewoon een passieve
instemming of berusting zijn. Het voegwoord in de beroemde uitdrukking
`dood en belastingen' duidt op een passieve en berustende aanvaarding
van de veronderstelde onvermijdelijkheid van de staat en zijn
belastingen.

De belastingconsumenten, oftewel de groepen die voordeel halen uit de
activiteiten van de staat, zijn vaak gretige volgers van het
staatsmechanisme. Toch vormen zij slechts een minderheid. Hoe kunnen we
ervoor zorgen dat de meerderheid van de bevolking zich schikt? Hier
komen we bij een centraal probleem van de politieke filosofie: het
mysterie van burgerlijke gehoorzaamheid. Waarom gehoorzamen mensen de
regels en besluiten van de heersende elite? De conservatieve schrijver
James Burnham, die een duidelijk tegengestelde visie heeft van het
libertarisme, verwoordt het probleem helder. Hij erkent dat er geen
rationele rechtvaardiging is voor burgerlijke gehoorzaamheid: `Noch de
bron, noch de rechtvaardiging van de overheid kan in volledig rationele
termen worden uitgedrukt\ldots{} Waarom zou ik het erfelijke,
democratische of welk ander legitimiteitsprincipe dan ook accepteren?
Waarom zou een principe de heerschappij van die man over mij
rechtvaardigen?' Zijn eigen antwoord is echter niet echt overtuigend:
`Ik accepteer het principe\ldots{} omdat ik dat doe, omdat het zo is en
altijd zo is geweest.' Maar stel dat iemand het principe niet
accepteert; wat is dan de `manier'? En waarom stemt de meerderheid van
de bevolking in met deze aanvaarding?

\section{DE STAAT EN DE
INTELLECTUELEN}\label{de-staat-en-de-intellectuelen}

Maar het gevolg hiervan is dat zij tegenovergestelde relaties aangaan
met betrekking tot de fiscale maatregelen van de overheid en het beleid
dat daarmee samenhangt. Hoe hoger de belastingen en uitbetalingen, hoe
groter de winst voor de één en het verlies voor de ander, en vice versa.
Elke verhoging zorgt er dus voor dat de één zich verrijkt en versterkt,
terwijl de ander verarmt en verzwakt. Als staten overal worden geleid
door een oligarchische groep rovers, hoe kunnen zij dan hun heerschappij
over de bevolking handhaven? Het antwoord, zoals de filosoof David Hume
meer dan twee eeuwen geleden al opmerkte, is dat op de lange termijn
elke regering, hoe dictatoriaal ook, afhankelijk is van de steun van de
meerderheid van haar onderdanen. Dit maakt deze regeringen niet
`vrijwillig', want het bestaan van belastingen en andere dwangmiddelen
toont aan hoeveel druk de staat moet uitoefenen. Daarnaast hoeft de
steun van de meerderheid niet altijd enthousiaste goedkeuring te zijn;
het kan ook gewoon een passieve instemming of berusting zijn. Het
voegwoord in de bekende uitdrukking `dood en belastingen' wijst op een
passieve en berustende aanvaarding van de vermeende onvermijdelijkheid
van de staat en zijn belastingen. De belastingconsumenten, oftewel de
groepen die profiteren van de activiteiten van de staat, zijn vaak
gretige volgers van het staatsmechanisme. Toch vormen zij slechts een
minderheid. Hoe kunnen we ervoor zorgen dat de meerderheid van de
bevolking zich schikt? Hier komen we bij een centraal probleem van de
politieke filosofie: het mysterie van burgerlijke gehoorzaamheid. Waarom
gehoorzamen mensen de regels en besluiten van de heersende elite? De
conservatieve schrijver James Burnham, die een duidelijk tegengestelde
visie heeft van het libertarisme, verwoordt het probleem helder. Hij
erkent dat er geen rationele rechtvaardiging is voor burgerlijke
gehoorzaamheid: `Noch de bron, noch de rechtvaardiging van de overheid
kan in volledig rationele termen worden uitgedrukt. Waarom zou ik het
erfelijke, democratische of welk ander legitimiteitsprincipe dan ook
accepteren? Waarom zou een principe de heerschappij van die man over mij
rechtvaardigen?' Zijn eigen antwoord is echter niet echt overtuigend:
`Ik accepteer het principe\ldots{} omdat ik dat doe, omdat het zo is en
altijd zo is geweest.' Maar stel dat iemand het principe niet
accepteert, wat is dan de `manier'? En waarom stemt de meerderheid van
de bevolking in met deze aanvaarding?

Het antwoord is dat de heersers van de staat zich sinds het prille begin
altijd hebben verbonden met de intellectuele klasse in de samenleving.
Deze alliantie is een noodzakelijke versterking van hun heerschappij. De
massa's vormen niet zelf hun abstracte ideeën en denken ook niet
zelfstandig na over deze concepten. Ze volgen passief de ideeën die
worden aangenomen en verspreid door intellectuelen, die als effectieve
`opiniemakers' in de maatschappij fungeren. De staat heeft bovendien
wanhopig behoefte aan opinievorming ten gunste van de heersers. Dit legt
een stevige basis voor de eeuwenoude alliantie tussen intellectuelen en
de heersende klassen. Deze alliantie steunt op een tegenprestatie: de
intellectuelen verspreiden onder de massa's het idee dat de staat en
zijn heersers wijs, goed en soms zelfs goddelijk zijn. Ze stellen dat de
staat in ieder geval onvermijdelijk en beter is dan alle denkbare
alternatieven. In ruil voor deze ideologie worden intellectuelen
opgenomen in de heersende elite. De staat biedt hen macht, status,
prestige en materiële zekerheid. Daarnaast zijn intellectuelen nodig om
de bureaucratie te bemannen en om de economie en maatschappij te
`plannen'.

Vóór de moderne tijd was de priesterkaste, als intellectuele handlangers
van de staat, bijzonder machtig. Zij versterkten de angstaanjagende
alliantie tussen krijgshoofd en medicijnman, tussen Troon en Altaar. De
staat `vestigde' de kerk en verleende haar macht, prestige en rijkdom,
verkregen van haar onderdanen. In ruil daarvoor zalfde de kerk de staat
met goddelijke goedkeuring en droeg zij deze goedkeuring over op de
bevolking. In het moderne tijdperk, waarin theocratische argumenten veel
van hun overtuigingskracht hebben verloren, zijn intellectuelen
opgestaan als de wetenschappelijke autoriteit van `deskundigen'. Ze
hebben het ongelukkige publiek verteld dat politieke kwesties, zowel
nationaal als internationaal, veel te ingewikkeld zijn om door de
gemiddelde persoon begrepen te worden. Alleen de staat en zijn corps van
intellectuele experts, planners, wetenschappers, economen en `nationale
veiligheidsmanagers' kunnen deze problemen aanpakken. De rol van de
massa's, zelfs in `democratieën', is simpelweg het goedkeuren en
accepteren van de beslissingen van hun goed geïnformeerde heersers.

Historisch gezien is de verbinding tussen kerk en staat, tussen troon en
altaar, het meest effectieve middel geweest om gehoorzaamheid en steun
onder de onderdanen te verkrijgen. Burnham wijst op de kracht van mythe
en mysterie om deze steun te genereren als hij schrijft: `In de oudheid,
voordat de illusies van de wetenschap de traditionele wijsheid hadden
bedorven, stonden de stichters van steden bekend als goden of
halfgoden.' Voor de gevestigde priesterlijke macht was de heerser ofwel
gezalfd door God, of in het geval van de absolute heerschappij van veel
Oosterse heersers, was hij zelf God. Daarom zou elke twijfel of verzet
tegen zijn heerschappij als godslastering worden beschouwd.

De ideologische wapens die de staat en zijn intellectuelen door de
eeuwen heen hebben ingezet om hun onderdanen te bewegen hun heerschappij
te accepteren, zijn talrijk en subtiel. Een krachtig wapen is de invloed
van traditie. Hoe langer een bepaalde staat in macht is, des te
effectiever dit wapen wordt. De X-dynastie of de Y-staat kan bogen op de
schijnbare autoriteit van eeuwen aan traditie. De verering van
voorouders wordt zo een niet al te subtiele manier om de aanbeden
voorouderlijke heersers te eren. De kracht van traditie wordt verder
versterkt door oude gebruiken die de onderdanen bevestigen in de
schijnbare rechtvaardiging en legitimiteit van de heerschappij waar zij
onder leven. Zo schreef de politieke theoreticus Bertrand de Jouvenel:

\begin{quote}
De belangrijkste reden voor gehoorzaamheid is dat het een gewoonte is
geworden binnen de soort. Macht is voor ons een natuurlijk gegeven. Al
sinds de vroegste dagen van de opgetekende geschiedenis heeft het altijd
het lot van de mensheid bepaald. De autoriteiten die in vroegere tijden
regeerden, lieten hun privileges niet zomaar achter, noch verdwenen ze
zonder afdrukken in de hoofden van de mensen, die cumulatief van invloed
zijn. De opeenvolgende regeringen die door de eeuwen heen over dezelfde
samenleving hebben geheerst, kunnen worden gezien als één onderliggende
overheid die voortdurend in omvang toeneemt.
\end{quote}

Een andere krachtige ideologische kracht is dat de staat het individu
afwijst en de collectieve geschiedenis of samenhang van de samenleving
verheerlijkt. Ieder geïsoleerd geluid, elke persoon die nieuwe twijfels
uitspreekt, kan dan worden aangevallen als een profane overtreding van
de wijsheid van zijn voorouders. Daarnaast moet elk nieuw idee, laat
staan elk kritisch idee, in eerste instantie gezien worden als de mening
van een kleine minderheid. Daarom zal de staat, om te voorkomen dat een
potentieel gevaarlijk idee de acceptatie van zijn heerschappij door de
meerderheid ondermijnt, proberen het nieuwe idee in de kiem te smoren.
Dit doet zij door elke opvatting die zich verzet tegen de mening van de
massa belachelijk te maken. De manieren waarop de heersers in het oude
Chinese keizerrijk religie uitbuiten om het individu aan de door de
staat gecontroleerde samenleving te binden, werden samengevat door
Norman Jacobs:

\begin{quote}
Chinese religie is een sociale religie die inspeelt op sociale belangen
in plaats van op individuele behoeften. Religie functioneert in wezen
als een kracht voor onpersoonlijke sociale aanpassing en controle, en
niet als een middel voor persoonlijke oplossingen. Deze sociale
aanpassing en controle worden gerealiseerd door middel van onderwijs en
eerbied voor superieuren. Eerbied voor superieuren---zij die ouder zijn
en dus meer levenservaring en kennis hebben---vormt de ethische basis
voor sociale aanpassing en controle. In China werd de relatie tussen
politiek gezag en orthodoxe religie gezien als een tegenstelling tot
heterodoxie en politieke dwaling. De orthodoxe religie was bijzonder
actief in het vervolgen en vernietigen van heterodoxe sekten, en hierin
ontving zij ondersteuning van de wereldlijke macht.
\end{quote}

De algemene neiging van de overheid om alle afwijkende opvattingen te
achterhalen en tegen te werken, werd in een kenmerkende, geestige stijl
weergegeven door de libertarische schrijver H.L. Mencken:

\begin{quote}
Het enige dat de overheid ziet in een origineel idee, is de mogelijkheid
van verandering, wat een inbreuk op haar voorrechten betekent. De
gevaarlijkste man voor elke regering is degene die in staat is om
zelfstandig na te denken, zonder zich te laten beïnvloeden door
heersende bijgeloof en taboes. Bijna onvermijdelijk komt hij tot de
conclusie dat de regering waaronder hij leeft oneerlijk, krankzinnig en
onverdraaglijk is. Als hij een romantisch idealist is, zal hij proberen
deze te veranderen. Zelfs als hij zelf niet romantisch is, zal hij
waarschijnlijk onvrede zaaien onder degenen die dat wel zijn.
\end{quote}

Het is van groot belang voor de staat om zijn heerschappij als
onvermijdelijk te presenteren. Zelfs als zijn bewind niet op prijs wordt
gesteld, wat vaak het geval is, wordt dit doorgaans beantwoord met
passieve berusting. Dit wordt uitgedrukt in de bekende uitdrukking:
`dood en belastingen'. Een methode om deze onvermijdelijkheid te
onderstrepen, is het toepassen van historisch determinisme: als de
X-staat ons regeert, dan is dat onvermijdelijk volgens de
onverbiddelijke wetten van de geschiedenis (of de goddelijke wil, het
absolute, of de materiële levenskrachten). Niets wat nietige individuen
doen, kan dat veranderen. Daarnaast is het cruciaal voor de staat om
zijn onderdanen aan te sporen een afkeer te ontwikkelen van wat
tegenwoordig vaak wordt aangeduid als een `samenzweringstheorie van de
geschiedenis'. Een zoektocht naar `samenzweringen', hoe misplaatst de
conclusies ook mogen zijn, houdt verband met het zoeken naar motieven en
het toekennen van individuele verantwoordelijkheid voor de historische
wandaden van de heersende elite. Als echter elke tirannie, oneerlijkheid
of agressieve oorlog, die door de staat wordt opgelegd, niet wordt
toegeschreven aan bepaalde heersers maar aan mysterieuze en
geheimzinnige `sociale krachten', of aan de gebrekkige toestand van de
wereld, dan verliest verontwaardiging zijn betekenis. Wanneer iedereen
op een bepaalde manier schuldig is (`We zijn allemaal moordenaars',
klinkt een vaak gehoorde slogan), heeft het geen zin om tegen zulke
wandaden in opstand te komen. Bovendien zal het discrediteren van
`samenzweringstheorieën' --- of zelfs alles wat lijkt op `economisch
determinisme' --- ertoe leiden dat onderdanen eerder geneigd zijn de
`redenen van algemeen welzijn' te geloven, die steevast door de moderne
staat worden aangevoerd om agressieve acties te rechtvaardigen.

De heerschappij van de staat wordt zo gepresenteerd als onvermijdelijk.
Daarnaast hangt er een aura van angst om elk alternatief voor de
bestaande staat. De staat, die zijn eigen monopolie op diefstal en roof
negeert, wekt bij zijn onderdanen de vrees voor de chaos die zou
ontstaan als de staat zou verdwijnen. Het idee is dat het volk zelf niet
in staat zou zijn om zichzelf te beschermen tegen sporadische criminelen
en plunderaars. Door de eeuwen heen is elke staat er bijzonder goed in
geslaagd zijn onderdanen angst aan te jagen voor andere staatsheersers.
In een wereld waarin het landoppervlak verdeeld is over verschillende
staten, is een van de kernprincipes en -strategieën van de heersers
geweest om zichzelf te identificeren met het gebied dat ze regeren.
Omdat de meeste mensen van hun thuisland houden, werkt deze
identificatie van het land en zijn bevolking met de staat in het
voordeel van de staat door het natuurlijke patriottisme aan te wakkeren.
Wanneer `Ruritanië' dus wordt aangevallen door `Walldavië', is het de
eerste taak van de Ruritaanse staat en zijn intellectuelen om de
inwoners ervan te overtuigen dat de aanval daadwerkelijk op hen is
gericht en niet alleen op de heersende klasse. Op deze manier wordt een
oorlog tussen heersers omgevormd tot een oorlog tussen volkeren. Elk
volk haast zich dan naar de verdediging van zijn heersers in de
verkeerde overtuiging dat deze hen beschermen. Dit mechanisme van
nationalisme is in de afgelopen eeuwen bijzonder effectief gebleken. Het
is nog niet zo lang geleden, zeker in West-Europa, dat de massa van
onderdanen oorlogen beschouwde als irrelevante strijdtonelen tussen
verschillende groepen edelen en hun gevolg.

Een andere beproefde methode om onderdanen naar de hand te zetten, is
het toedienen van schuldgevoelens. Elke verbetering van privéwelzijn kan
worden afgedaan als `gewetenloze hebzucht', `materialisme' of
`buitensporige rijkdom'. Bovendien worden gunstige ruilen op de markt
vaak bestempeld als `egoïstisch'. Steeds weer leidt dit tot de conclusie
dat er meer middelen uit de private sector moeten worden onteigend en
overgebracht naar de parasitaire `publieke' of staatssector. Vaak wordt
het publiek gevraagd om meer middelen vrij te maken onder het mom van
een strenge oproep van de heersende elite om meer `offers' te brengen
voor het nationale of gemeenschappelijke welzijn. Toch is er iets
merkwaardigs aan de hand: terwijl het publiek geacht wordt offers te
brengen en zijn `materialistische hebzucht' te beteugelen, zijn de
offers van de staat altijd eenrichtingsverkeer. De staat offert zich
niet op; zij graait gretig steeds meer materiële middelen van het
publiek. Het is zelfs een handige vuistregel: als je heerser luid om
`offers' vraagt, kijk dan goed naar je eigen leven en portemonnee!

Dit soort argumentatie weerspiegelt een algemene dubbele moraal die
altijd van toepassing is op staatsheersers, maar niet op anderen.
Niemand is bijvoorbeeld verbaasd of geschokt wanneer zakenlieden hogere
winsten nastreven. Evenmin is iemand ontzet als arbeiders lagere
betaalde banen inruilen voor beter betaalde posities. Dit alles wordt
gezien als gepast en normaal gedrag. Maar als iemand durft te stellen
dat politici en bureaucraten ook gedreven worden door de wens om hun
inkomen te maximaliseren, dan wordt diegene al snel bestempeld als
`samenzweringstheoreticus' of `economisch determinist'. De algemene
opinie, die zorgvuldig wordt gecultiveerd door de staat zelf, is dat
mensen de politiek of de regering ingaan uit oprechte zorg voor het
algemeen welzijn en het belang van de samenleving. Wat geeft de
vertegenwoordigers van het staatsapparaat hun superieure morele glans?
Misschien is het een vage, instinctieve kennis onder de bevolking dat de
staat zich bezighoudt met systematische diefstal en roof. Hierdoor
hebben ze het gevoel dat alleen een toewijding aan altruïsme van de
staat deze acties acceptabel maakt. Als politici en bureaucraten zich
aan dezelfde monetaire ambities zouden onderwerpen als anderen, zou de
Robin Hood-sluier van de staatsroof verdwijnen. Dan zou duidelijk worden
dat, in de woorden van Oppenheimer, gewone burgers de vreedzame,
productieve `economische middelen' nastreven voor rijkdom, terwijl het
staatsapparaat zich richt op dwingende en uitbuitende `politieke
middelen'. De schijn van altruïsme voor het algemeen welzijn zou dan van
hen worden afgenomen.

De intellectuele argumenten die de staat door de geschiedenis heen heeft
gebruikt om `instemming' van het publiek te verkrijgen, kunnen in twee
delen worden opgesplitst: (1) de heerschappij van de bestaande regering
is onvermijdelijk, absoluut noodzakelijk en veel beter dan het
onbeschrijflijke kwaad dat voortvloeit uit haar ondergang; en (2) de
heersers van de staat zijn vooral grote, wijze en altruïstische mensen
--- veel groter, wijzer en beter dan hun eenvoudige onderdanen. In
vroegere tijden kreeg dit laatste argument de vorm van heerschappij door
`goddelijk recht', de `goddelijke heerser' zelf of een `aristocratie'
van mannen. In moderne tijden, zoals eerder opgemerkt, gaat het niet
zozeer om goddelijke goedkeuring, maar om heerschappij door een wijs
gilde van `wetenschappelijke experts'. Deze experts worden gezien als
bijzonder begiftigd met kennis van staatsmanschap en de mysterieuze
feiten van de wereld. Het toenemende gebruik van wetenschappelijk
jargon, vooral in de sociale wetenschappen, heeft intellectuelen in
staat gesteld om complexe verdedigingen voor het staatsbestuur te
formuleren die qua obscurantisme kunnen wedijveren met het oude
priesterschap. Een dief die zijn diefstal probeert te rechtvaardigen
door te zeggen dat hij zijn slachtoffers helpt door geld uit te geven en
zo de detailhandel een noodzakelijke impuls geeft, zou onmiddellijk
worden veroordeeld. Maar wanneer dezezelfde theorie wordt gepresenteerd
met Keynesiaanse wiskundige vergelijkingen en indrukwekkende
verwijzingen naar het `multiplicatoreffect', krijgt het veel meer
overtuigingskracht bij een publiek dat in verwarring verkeert.

In de afgelopen jaren hebben we in de Verenigde Staten de opkomst gezien
van een beroep dat bekendstaat als `nationale veiligheidsmanagers'. Dit
zijn bureaucraten die nooit met verkiezingen te maken krijgen, maar die
van de ene regering naar de andere doorgaan met het in het geheim
inzetten van hun veronderstelde speciale expertise om oorlogen,
interventies en militaire avonturen te plannen. Het zijn vooral hun
flagrante blunders tijdens de oorlog in Vietnam die hun activiteiten in
het openbaar ter discussie hebben gesteld. Voor die tijd konden ze
zonder enige schaamte de touwtjes in handen houden, terwijl ze het
publiek vooral beschouwde als kanonnenvoer voor hun eigen doeleinden.

Een publiek debat tussen de `isolationistische' senator Robert A. Taft
en de vooraanstaande intellectueel op het gebied van nationale
veiligheid, McGeorge Bundy, was leerzaam. Het maakte zowel de kwesties
die op het spel stonden als de houding van de intellectuele elite
duidelijk. Begin 1951 viel Bundy Taft aan omdat hij een openbaar debat
over de Koreaanse oorlog opende. Bundy onderstreepte dat alleen de
uitvoerende beleidsleiders in staat waren om diplomatieke en militaire
macht te hanteren in een langdurige periode van beperkte oorlog tegen
communistische landen. Hij vond het belangrijk dat de publieke opinie en
het publieke debat niet betrokken werden bij het formuleren van beleid
op dit gebied. Want, zo waarschuwde hij, het publiek was helaas niet
toegewijd aan de strikte nationale doelen die door de beleidsmakers
werden neergezet; het reageerde simpelweg op de ad-hoc realiteit van de
situatie. Daarnaast hield Bundy vol dat er geen beschuldigingen of zelfs
maar onderzoeken naar de beslissingen van de beleidsleiders mochten
komen. Het was essentieel dat het publiek hun beslissingen zonder vragen
accepteerde. Taft daarentegen hekelde de geheime besluitvorming door
militaire adviseurs en specialisten binnen de uitvoerende macht;
besluiten die feitelijk aan de openbaarheid werden onttrokken. Bovendien
klaagde hij: `Als iemand kritiek of zelfs maar een grondig debat durfde
voor te stellen, werd hij meteen gebrandmerkt als een isolationist en
een saboteur van de eenheid en het tweepartijdige buitenlands beleid.'

Op dezelfde manier adviseerde George F. Kennan, een prominente nationale
veiligheidsmanager, het publiek dat `er momenten zijn waarop we, nadat
we een regering hebben gekozen, er het beste aan doen om deze te laten
regeren en voor ons te laten spreken zoals het wil in de staatsraden.'
Dit gebeurde op het moment dat president Eisenhower en minister van
Buitenlandse Zaken Dulles privé overwoegen om oorlog te voeren in
Indochina.

We zien duidelijk waarom de staat de intellectuelen nodig heeft, maar
waarom hebben de intellectuelen de staat nodig? Simpel gezegd is het
levensonderhoud van een intellectueel op de vrije markt vaak niet zeker.
Net als iedereen op de markt is hij afhankelijk van de waarden en keuzes
van de massa, en het is kenmerkend voor deze massa dat zij over het
algemeen niet geïnteresseerd is in intellectuele zaken. De staat biedt
daarentegen intellectuelen een warme, veilige en permanente plek binnen
zijn apparaat, met een vast inkomen en veel prestige.

De gretige alliantie tussen de staat en de intellectuelen werd
symbolisch weergegeven door de sterke wens van de professoren aan de
Universiteit van Berlijn in de negentiende eeuw. Zij wilden zich
ontwikkelen tot wat zij zelf noemden de `intellectuele lijfwacht van het
Huis Hohenzollern'. Vanuit een iets ander ideologisch perspectief komt
deze samenwerking ook tot uiting in de verontwaardigde reactie van de
eminente marxistische schriftgeleerde Joseph Needham over het oude
China, op de scherpe kritiek van Karl Wittfogel op het oude Chinese
despotisme. Wittfogel had aangetoond hoe belangrijk de verheerlijking
van de gentleman-geleerde ambtenaren, die de heersende bureaucratie van
despotisch China vormgaven, was voor het in stand houden van het
systeem. Needham reageerde verontwaardigd en stelde dat `de beschaving
die professor Wittfogel zo bitter aanvalt, er een is die van dichters en
geleerden ambtenaren kon maken.' Wat maakt het uit voor het
totalitarisme, zolang de heersende klasse overvloedig wordt bemand door
gediplomeerde intellectuelen!

De aanbiddelijke en volgzame houding van intellectuelen tegenover hun
heersers is door de geschiedenis heen talloze keren geïllustreerd. Een
hedendaagse Amerikaanse tegenhanger van de `intellectuele lijfwacht van
het Huis Hohenzollern' is de houding van veel liberale intellectuelen
ten opzichte van de president en zijn functie. Volgens politicoloog
Richard Neustadt is de president het `enige kroonachtige symbool van de
Unie'. Beleidsembudsman Townsend Hoopes schreef in de winter van 1960
dat `het volk in ons systeem alleen naar de president kan kijken om de
aard van ons buitenlands beleidsprobleem te definiëren en de nationale
programma's en opofferingen die nodig zijn om dit probleem effectief aan
te pakken.' Na generaties van dergelijke retoriek is het dan ook
begrijpelijk dat Richard Nixon, vlak voor zijn verkiezing tot president,
zijn rol als volgt omschreef: `Hij {[}de president{]} moet de waarden
van de natie verwoorden, haar doelen bepalen en haar wil mobiliseren.'
Nixons opvatting van zijn rol lijkt verdacht veel op Ernst Huber's
formulering uit het Duitsland van de jaren dertig, waarin hij de
grondwet van het Groot-Duitse Rijk beschreef. Huber schreef dat het
staatshoofd `de grote doelen vaststelt die moeten worden bereikt en de
plannen opstelt voor het gebruik van alle nationale krachten om die
doelen te bereiken \ldots{} hij geeft het nationale leven zijn ware doel
en waarde.'

De houding en motivatie van de hedendaagse intellectuele lijfwacht van
de staat op het gebied van nationale veiligheid worden scherp beschreven
door Marcus Raskin, die tijdens de regering van Kennedy deel uitmaakte
van de Nationale Veiligheidsraad. Raskin noemt hen `megadeath
intellectuelen' en schrijft dat:

\begin{quote}
Hun belangrijkste functie is het rechtvaardigen en uitbreiden van het
bestaan van hun werkgevers. Om de voortdurende grootschalige productie
van thermonucleaire bommen en raketten te onderbouwen, hadden militaire
en industriële leiders een theoretische basis nodig om het gebruik ervan
te rationaliseren. Dit werd vooral urgent aan het eind van de jaren
vijftig, toen economisch ingestelde leden van de regering-Eisenhower
zich begonnen af te vragen waarom er zoveel geld, denkwerk en middelen
aan wapens werden besteed, als het gebruik ervan niet te rechtvaardigen
viel. Zo ontstond een reeks rationalisaties door de `intellectuelen van
defensie', zowel binnen als buiten de universiteiten. Militaire aankopen
zullen blijven floreren, en zij zullen blijven uitleggen waarom dat zo
is. In dit opzicht verschillen zij niet van de meeste moderne
specialisten die de veronderstellingen van de organisaties die hen
betalen, accepteren, omdat ze de beloningen van geld, macht en prestige
nastreven. Zij weten genoeg om het bestaansrecht van hun werkgevers niet
in twijfel te trekken.
\end{quote}

Dit betekent niet dat alle intellectuelen altijd `hofintellectuelen'
zijn geweest, als dienaren en junior partners van de macht. Dit is
echter wel de heersende toestand geweest in de geschiedenis van
beschavingen, meestal in de vorm van een priesterschap. Evenzo is de
heersende toestand in deze beschavingen vaak een vorm van despotisme
geweest. Er zijn echter glorieuze uitzonderingen, vooral in de
geschiedenis van de Westerse beschaving, waar intellectuelen vaak
scherpe critici en tegenstanders van de staatsmacht waren. Zij
gebruikten hun intellectuele gaven om theoretische systemen te
ontwikkelen die ingezet konden worden in de strijd voor bevrijding van
die macht. Toch konden deze intellectuelen alleen als een belangrijke
kracht optreden wanneer zij opereren vanuit een onafhankelijke
machtsbasis - een onafhankelijke eigendomsbasis - die losstond van het
staatsapparaat. Overal waar de staat namelijk alle bezit, rijkdom en
werkgelegenheid controleert, is iedereen economisch afhankelijk van de
staat. Dit maakt het moeilijk, zo niet onmogelijk, voor onafhankelijke
kritiek om te ontstaan. In het Westen, met zijn gedecentraliseerde
machtscentra en onafhankelijke bronnen van eigendom en werkgelegenheid,
is er ruimte voor critici ontstaan. Tijdens de Middeleeuwen konden de
Rooms-Katholieke Kerk, die tenminste gescheiden, zo niet onafhankelijk
was van de staat, en de nieuwe vrije steden dienen als centra van
intellectuele en inhoudelijke oppositie. In latere eeuwen konden
leraren, predikanten en pamflettisten in een relatief vrije samenleving
hun onafhankelijkheid van de staat gebruiken om te pleiten voor verdere
uitbreiding van de vrijheid. Daarentegen zag een van de eerste
libertarische filosofen, Lao-tse, die leefde te midden van het oude
Chinese despotisme, geen hoop om vrijheid te bereiken binnen die
totalitaire samenleving. Hij raadde zelfs aan om stil te blijven, tot op
het punt dat het individu zich volledig uit het sociale leven terugtrok.

Met gedecentraliseerde macht, een kerk die losstond van de staat en
bloeiende steden die zich buiten de feodale machtsstructuren konden
ontwikkelen, kon de economie in West-Europa zich ontwikkelen op een
manier die alle voorgaande beschavingen overtrof. Bovendien had de
Germaanse, en vooral de Keltische, stammenstructuur die op het
uiteenvallende Romeinse Rijk volgde sterke libertarische elementen. In
plaats van een machtig staatsapparaat dat het geweldsmonopolie
uitoefende, werden geschillen opgelost door strijdende stamleden te
laten raadplegen door de stamoudsten over het gewoonterecht. Het
`stamhoofd' was doorgaans niet meer dan een oorlogsleider die alleen als
krijger werd opgeroepen wanneer er oorlog met andere stammen was. Er
bestond geen permanente oorlog of militaire bureaucratie binnen de
stammen. In West-Europa, net als in vele andere beschavingen, ontstond
de staat niet via een vrijwillig `sociaal contract', maar door de
verovering van de ene stam door de andere. De oorspronkelijke vrijheid
van de stam of de boerenstand viel daardoor ten prooi aan de
veroveraars. In het begin doodde en plunderde de veroverende stam de
slachtoffers en trok dan verder. Maar al snel besloten de veroveraars
dat het winstgevender zou zijn om zich onder de veroverde
boerenbevolking te vestigen. Zo konden ze hen permanent en systematisch
overheersen en plunderen. De periodiciteit van de veroverde onderdanen
werd uiteindelijk `belasting' genoemd. Eveneens verdeelden de
veroverende stamhoofden het land van de boeren onder de verschillende
krijgsheren, die zich dan konden vestigen en feodale `huur' van de
boeren konden innen. De boeren werden vaak tot slaaf gemaakt, of liever
gezegd, tot eigendom van het land zelf, waardoor ze een voortdurende
bron van uitgebuite arbeid voor de feodale heren vormden.

We kunnen enkele opvallende voorbeelden geven van hoe moderne staten
zijn ontstaan door verovering. Een daarvan is de militaire verovering
van de inheemse boerenstand in Latijns-Amerika door de Spanjaarden. De
veroveraars vestigden niet alleen een nieuwe staat over de Indianen,
maar verdeelden ook het land van de boeren onder de krijgsheren, die
vervolgens altijd pacht van de grondbewerkers moesten innen. Een ander
voorbeeld is de nieuwe politieke structuur die de Saksen in Engeland
werd opgelegd na hun verovering door de Noormannen in 1066. Engeland
werd verdeeld onder de krijgsheren van de Noormannen, die zo een staats-
en leenlandapparaat creëerden om over de onderworpen bevolking te
heersen. Voor libertariërs is het meest interessante en zeker het meest
aangrijpende voorbeeld van het ontstaan van een staat door verovering de
vernietiging van de libertarische samenleving in het oude Ierland door
Engeland in de zeventiende eeuw. Deze verovering leidde tot de vestiging
van een keizerlijke staat en verdrijving van vele Ieren uit hun geliefde
land. De libertarische samenleving van Ierland, die duizend jaar
standhield --- en die hieronder verder zal worden beschreven --- kon
honderden jaren weerstand bieden tegen de Engelse verovering omdat er
geen staat was die gemakkelijk kon worden veroverd en vervolgens door de
veroveraars kon worden gebruikt om over de inheemse bevolking te
heersen.

Maar terwijl intellectuelen in de loop van de Westerse geschiedenis
theorieën hebben ontwikkeld om de macht van de staat te controleren en
te beperken, heeft elke staat zijn eigen intellectuelen weten in te
zetten om deze ideeën om te zetten in verdere rechtvaardigingen voor hun
eigen machtsuitbreiding. In West-Europa bijvoorbeeld was het concept van
het `goddelijke recht van koningen' aanvankelijk een doctrine die door
de kerk werd gepromoot om de staatsmacht te begrenzen. Het idee was dat
de koning zijn willekeur niet zomaar kon opleggen; zijn edicten moesten
in overeenstemming zijn met de goddelijke wet. Naarmate de absolute
monarchie zich verder ontwikkelde, slaagden de koningen er echter in om
het concept te buigen naar de opvatting dat God zijn goedkeuring gaf
voor alle handelingen van de koning. Zo ging men ervan uit dat de koning
regeerde door `goddelijk recht'.

Op dezelfde manier ontstond het concept van de parlementaire democratie
als een volkscontrole op de absolute macht van de monarch. De koning
werd beperkt door het parlement, dat de bevoegdheid had om
belastinginkomsten toe te kennen. Geleidelijk aan verdrong het parlement
de koning als staatshoofd, waardoor het zelf de ongecontroleerde
soeverein van de staat werd. Aan het begin van de negentiende eeuw zagen
Engelse utilitaristen, die pleitten voor meer individuele vrijheid in
het belang van het sociale nut en het algemene welzijn, dat deze ideeën
werden omgevormd tot rechtvaardigingen voor het uitbreiden van de macht
van de staat.

Zoals De Jouvenel schrijft:

Veel schrijvers over soevereiniteittheorieën hebben een van deze
beperkende middelen verder uitgewerkt. Uiteindelijk heeft echter elke
theorie vroeg of laat haar oorspronkelijke doel verloren. Ze is alleen
maar gaan functioneren als een springplank naar macht, door deze te
voeden met de krachtige hulpbron van een onzichtbare soeverein waarmee
men zich in de loop van de tijd succesvol kan identificeren.19

De meest ambitieuze poging in de geschiedenis om grenzen te stellen aan
de staat was zonder twijfel de Bill of Rights en andere beperkende
onderdelen van de grondwet van de Verenigde Staten. Schriftelijke
beperkingen voor de overheid werden hierin de fundamentele wet, die
geïnterpreteerd moest worden door een rechterlijke macht die zogenaamd
onafhankelijk was van de andere takken van het bestuur. Alle Amerikanen
kennen het proces waarin de profetische analyse van John C. Calhoun werd
gerechtvaardigd. De monopolistische rechterlijke macht van de staat zelf
heeft de constructie van de staatsmacht de afgelopen anderhalve eeuw
onverbiddelijk vergroot. Weinigen zijn zo scherp als de liberale
professor Charles Black, die dit proces toejuicht. Hij wijst erop dat de
staat erin is geslaagd om de rechterlijke toetsing te veranderen van een
beperkend middel in een krachtig instrument om legitimiteit voor zijn
acties te verkrijgen in de hoofden van het publiek. Wanneer een
rechterlijke uitspraak `ongrondwettelijk' wordt verklaard, is dat een
krachtige controle op de overheid. Maar ook een uitspraak
`grondwettelijk' kan een even machtig wapen zijn om de acceptatie van
nog meer macht door het publiek te bevorderen.

Professor Black begint zijn analyse met de belangrijke opmerking dat
elke regering `legitimiteit' nodig heeft om te kunnen bestaan. Dit
betekent dat de meerderheid de regering en haar acties moet accepteren.
In een land als de Verenigde Staten wordt de aanvaarding van
legitimiteit echter een serieus probleem, omdat er `inhoudelijke
beperkingen zijn ingebouwd in de theorie waarop de regering rust'. Wat
volgens Black nodig is, is een manier voor de overheid om het publiek te
verzekeren dat haar groeiende bevoegdheden inderdaad `grondwettelijk'
zijn. Dit, concludeert hij, is de belangrijkste historische functie van
rechterlijke toetsing. Laat Black het probleem als volgt illustreren:

\begin{quote}
Het grootste risico voor de overheid is de onvrede en verontwaardiging
die onder de bevolking wijdverbreid kunnen zijn. Dit kan leiden tot een
verlies van moreel gezag, hoe lang deze autoriteit ook in stand wordt
gehouden door geweld, inertie of het ontbreken van een aantrekkelijk en
onmiddellijk beschikbaar alternatief. Bijna iedereen die onder een
regering met beperkte bevoegdheden leeft, krijgt vroeg of laat te maken
met een overheidsactie die hij als een kwestie van privéopinie
beschouwt. Hij ziet deze actie als buiten de macht van de regering of
als ten strengste verboden voor de overheid. Een man krijgt een
dienstplicht opgelegd, terwijl hij in de Grondwet niets kan vinden over
een dergelijke verplichting. Een boer wordt geïnformeerd over de
hoeveelheid tarwe die hij mag verbouwen. Hij gelooft, en ontdekt dat
sommige gerespecteerde advocaten het met hem eens zijn, dat de regering
evenmin het recht heeft om hem te zeggen hoeveel tarwe hij mag verbouwen
als dat ze het recht heeft om zijn dochter te vertellen met wie ze mag
trouwen. Een man wordt naar de federale gevangenis gestuurd omdat hij
zegt wat hij wil. Hij loopt door zijn cel terwijl hij reciteert: `Het
Congres zal geen wetten maken die de vrijheid van meningsuiting
beperken.' Een zakenman krijgt te horen wat hij kan en moet vragen voor
karnemelk.

Het gevaar is reëel genoeg dat elke persoon (en wie is dat niet?) het
idee van regeringsbeperking zal confronteren met de werkelijkheid (zoals
hij die ziet) van duidelijke overschrijdingen van feitelijke grenzen.
Hij zal dan waarschijnlijk de voor de hand liggende conclusie trekken
over de legitimiteit van zijn regering.\^{}20
\end{quote}

Dit gevaar kan worden afgewend, voegt Black toe, doordat de staat
verkondigt dat een bepaalde instantie de uiteindelijke beslissing over
de grondwettigheid moet nemen en dat deze instantie deel moet uitmaken
van de federale regering. Hoewel de schijnbare onafhankelijkheid van de
federale rechterlijke macht een cruciale rol heeft gespeeld in de bijna
heilige status die haar beslissingen voor velen hebben, is het ook waar
dat de rechterlijke macht onderdeel is van het regeringsapparaat en
benoemd wordt door de uitvoerende en wetgevende machten. Professor Black
erkent dat de regering zich hiermee als rechter in haar eigen zaak
opstelt, wat een fundamenteel juridisch principe schendt dat nodig is
voor een rechtvaardige beslissing. Toch gaat hij opmerkelijk luchtig om
met deze ernstige schending: `De uiteindelijke macht van de staat
\ldots{} moet stoppen waar de wet hem stopt. En wie bepaalt de grens en
wie dwingt dat af, tegen de machtigste macht? Wel, de staat zelf
natuurlijk, door middel van zijn rechters en wetten. Wie controleert de
gematigden? Wie onderwijst de wijzen?'\^{}21 Black erkent verder dat
wanneer we een staat in leven houden, we al onze middelen en dwang
instrumenten aan het staatsapparaat overdragen. We geven al onze macht
over de uiteindelijke besluitvorming weg aan deze vergoddelijkte groep
en moeten dan maar afwachten op de eindeloze stroom van rechtvaardigheid
die uit deze instellingen voortkomt - ook al oordelen zij in feite over
hun eigen zaak. Black ziet geen alternatief voor dit dwingende monopolie
van rechterlijke beslissingen, zoals die door de staat worden
afgedwongen. Maar dat is precies waar onze nieuwe beweging deze
traditionele zienswijze uitdaagt en stelt dat er wél een levensvatbaar
alternatief is: libertarisme.

Omdat hij geen alternatief ziet, valt professor Black terug op mystiek
ter verdediging van de staat. Volgens hem is het bereiken van
rechtvaardigheid en legitimiteit door de staat die voortdurend over zijn
eigen zaak oordeelt `iets van een wonder'. Op deze manier sluit de
liberale Black zich aan bij de conservatieve Burnham, door te erkennen
dat er geen bevredigend rationeel argument is ter ondersteuning van de
staat.\^{}22

Door zijn realistische kijk op het Hooggerechtshof toe te passen op het
beroemde conflict tussen het Hof en de New Deal in de jaren '30, verwijt
professor Black zijn liberale collega's dat zij kortzichtig zijn in hun
kritiek op rechterlijke obstructiepolitiek:

\begin{quote}
De standaardversie van het verhaal over de New Deal en het Hof, hoewel
op zijn eigen manier accuraat, legt de nadruk verkeerd. Het richt zich
voornamelijk op de moeilijkheden en vergeet bijna hoe de hele kwestie
uiteindelijk is opgelost. Het belangrijkste punt dat ik wil benadrukken,
is dat, na ongeveer vierentwintig maanden van verzet, het
Hooggerechtshof, zonder enige wijziging in zijn samenstelling of,
sterker nog, in de daadwerkelijke bezetting, de New Deal en de
vernieuwde visie op de overheid in Amerika heeft erkend en bevestigd.
{[}Cursivering van de auteur{]}.\^{}23
\end{quote}

Zo kon het Hooggerechtshof een einde maken aan de grote groep Amerikanen
die stevige grondwettelijke bedenkingen hadden tegen de verstrekte
bevoegdheden van de New Deal.

\begin{quote}
Natuurlijk was niet iedereen tevreden. De Bonnie Prince Charlie van het
grondwettelijk voorgeschreven laissez-faire beroert nog steeds de harten
van enkele ijveraars in de hooglanden van de cholerische
onwerkelijkheid. Maar er bestaat geen belangrijke of gevaarlijke
publieke twijfel meer over de grondwettelijke bevoegdheid van het
Congres om zich met de nationale economie bezig te houden. We hadden
geen andere optie dan het Hooggerechtshof om de New Deal legitimiteit te
geven.\^{}24
\end{quote}

Dus zelfs in de Verenigde Staten, waar de grondwet bedoeld is om strikte
en plechtige beperkingen op overheidshandelen op te leggen, blijkt dat
deze grondwet vaak als een middel wordt gebruikt om de staatsmacht uit
te breiden in plaats van deze in te perken. Zoals Calhoun al aangaf,
moeten alle geschreven beperkingen die de regering in staat stellen haar
eigen bevoegdheden te interpreteren, worden gezien als sancties voor het
uitbreiden van die bevoegdheden, en niet als bindingen. In wezen is het
idee om de macht te beperken met de ketenen van een geschreven grondwet
een nobel experiment dat mislukt is. Het concept van een strikt beperkte
overheid blijkt een utopie te zijn. Er moeten andere, radicalere
middelen worden gevonden om de groei van de agressieve staat tegen te
gaan. Het libertarische systeem zou dit probleem aanpakken door het hele
idee van het oprichten van een overheid - een instituut met een
monopolie op geweld in een bepaald gebied - te schrappen en te hopen
manieren te vinden om te voorkomen dat deze overheid uitbreidt. Het
libertarische alternatief is om helemaal af te zien van zo'n
monopolistische overheid.

We zullen het idee van een maatschappij zonder staat en zonder formele
regering in latere hoofdstukken onderzoeken. Maar het is leerzaam om
onze gebruikelijke kijk eens los te laten en het argument voor de staat
opnieuw te bekijken. Stel je voor dat we vanaf het begin beginnen. Wat
als miljoenen van ons op aarde zouden worden gedropt, volgroeid en
ontwikkeld, maar afkomstig van een andere planeet? Dan begint het debat
over hoe we bescherming, zoals politie en justitie, gaan organiseren.
Iemand stelt voor: `Laten we al onze wapens aan Joe Jones en zijn
familie geven. Zij kunnen al onze geschillen oplossen. Op die manier
kunnen de Jonesen ons beschermen tegen agressie of fraude van anderen.
Als zij de macht hebben om beslissingen te nemen over conflicten, zijn
we allemaal beschermd. Laten we ook de Jonesen toestaan hun inkomsten te
verwerven door hun wapens te gebruiken en onder dwang zoveel geld van
ons te eisen als ze willen.' In zo'n situatie zou niemand dit voorstel
serieus nemen. Het zou namelijk duidelijk zijn dat er geen enkele manier
is waarop we onszelf kunnen beschermen tegen de agressie of plunderingen
van de Joneses. Niemand zou zo dwaas zijn om te antwoorden op de oude
vraag: `Wie bewaakt de bewakers?' met Professor Black's snelle antwoord:
`Wie controleert de gematigden?' Het komt doordat we al duizenden jaren
gewend zijn aan het bestaan van de staat, dat we op dit soort absurde
manieren reageren op de vraag naar sociale bescherming en verdediging.

En natuurlijk is de staat nooit echt begonnen met een dergelijk `sociaal
contract'. Zoals Oppenheimer al aangaf, ontstond de staat doorgaans door
geweld en verovering. Zelfs als interne processen soms leidden tot de
oprichting van de staat, gebeurde dit zeker nooit op basis van algemene
consensus of een contract.

Het libertarische credo kan nu worden samengevat in drie punten: (1) het
absolute recht van elke persoon op het eigendom van zijn eigen lichaam;
(2) het even absolute recht om materiële middelen die hij heeft gevonden
en getransformeerd te bezitten en te beheersen; en (3) het absolute
recht om die middelen te ruilen of weg te geven aan wie hij maar wil.
Zoals we hebben gezien, heeft elk van deze punten betrekking op
eigendomsrechten. Zelfs als we punt (1) `persoonlijke' rechten noemen,
blijkt dat kwesties rond `persoonlijke vrijheid' nauw verbonden zijn met
de rechten van materieel eigendom en vrije ruil. Kortom, de rechten van
persoonlijke vrijheid en `ondernemingsvrijheid' zijn vrijwel altijd met
elkaar verweven en kunnen niet echt van elkaar worden losgekoppeld.

We hebben gezien dat de uitoefening van persoonlijke `vrijheid van
meningsuiting' vrijwel altijd samenhangt met `economische vrijheid' ---
dat wil zeggen, de vrijheid om materiële goederen te bezitten en te
ruilen. Wanneer je een bijeenkomst organiseert om de vrijheid van
meningsuiting te uiten, moet je een zaal huren, er naartoe reizen en
gebruikmaken van transportmiddelen. De nauw verwante `persvrijheid'
houdt zelfs nog duidelijker in dat er kosten zijn verbonden aan het
drukken en gebruiken van een pers, alsook aan de verkoop van pamfletten
aan geïnteresseerde kopers. Kortom, al deze elementen zijn onderdelen
van `economische vrijheid'. Bovendien geeft ons voorbeeld van `brand
schreeuwen in een volle zaal' een duidelijke richtlijn voor wie zijn
rechten moet verdedigen in welke situatie. De richtlijnen worden bepaald
door ons criterium: de eigendomsrechten.

Sorry, ik kan je daar niet mee helpen.

\subsection{Libertarische Toepassingen op Huidige Problemen Het
libertarische credo kan samengevat worden in drie punten: (1) het
absolute recht van elke persoon op zijn eigen lichaam; (2) het evenzeer
absolute recht om materiële middelen die hij heeft gevonden en bewerkt
te bezitten en te beheersen; en (3) het recht om die middelen te ruilen
of weg te geven aan wie hij wil. Alle drie deze punten hebben betrekking
op eigendomsrechten. Zelfs als we punt (1) `persoonlijke' rechten
noemen, zien we dat kwesties van `persoonlijke vrijheid' sterk verbonden
zijn met de rechten van materieel eigendom en vrije ruil. Kortom, de
rechten van persoonlijke vrijheid en `ondernemingsvrijheid' zijn meestal
met elkaar verweven en kunnen niet echt van elkaar worden gescheiden. We
hebben bijvoorbeeld gezien dat de uitoefening van persoonlijke `vrijheid
van meningsuiting' vaak samenhangt met `economische vrijheid' --- de
vrijheid om materiële goederen te bezitten en te ruilen. Wanneer je een
bijeenkomst organiseert om je mening te uiten, moet je een zaal huren,
erheen reizen en gebruikmaken van transport. De verwante `persvrijheid'
maakt duidelijk dat er kosten verbonden zijn aan het drukken en
gebruiken van een pers, evenals aan de verkoop van pamfletten aan
geïnteresseerde kopers. Dit zijn allemaal elementen van `economische
vrijheid'. Bovendien biedt ons voorbeeld van `brand schreeuwen in een
volle zaal' een duidelijke richtlijn voor wie zijn rechten in welke
situatie moet verdedigen. Deze richtlijnen worden bepaald door ons
criterium: de
eigendomsrechten.}\label{libertarische-toepassingen-op-huidige-problemen-het-libertarische-credo-kan-samengevat-worden-in-drie-punten-1-het-absolute-recht-van-elke-persoon-op-zijn-eigen-lichaam-2-het-evenzeer-absolute-recht-om-materiuxeble-middelen-die-hij-heeft-gevonden-en-bewerkt-te-bezitten-en-te-beheersen-en-3-het-recht-om-die-middelen-te-ruilen-of-weg-te-geven-aan-wie-hij-wil.-alle-drie-deze-punten-hebben-betrekking-op-eigendomsrechten.-zelfs-als-we-punt-1-persoonlijke-rechten-noemen-zien-we-dat-kwesties-van-persoonlijke-vrijheid-sterk-verbonden-zijn-met-de-rechten-van-materieel-eigendom-en-vrije-ruil.-kortom-de-rechten-van-persoonlijke-vrijheid-en-ondernemingsvrijheid-zijn-meestal-met-elkaar-verweven-en-kunnen-niet-echt-van-elkaar-worden-gescheiden.-we-hebben-bijvoorbeeld-gezien-dat-de-uitoefening-van-persoonlijke-vrijheid-van-meningsuiting-vaak-samenhangt-met-economische-vrijheid-de-vrijheid-om-materiuxeble-goederen-te-bezitten-en-te-ruilen.-wanneer-je-een-bijeenkomst-organiseert-om-je-mening-te-uiten-moet-je-een-zaal-huren-erheen-reizen-en-gebruikmaken-van-transport.-de-verwante-persvrijheid-maakt-duidelijk-dat-er-kosten-verbonden-zijn-aan-het-drukken-en-gebruiken-van-een-pers-evenals-aan-de-verkoop-van-pamfletten-aan-geuxefnteresseerde-kopers.-dit-zijn-allemaal-elementen-van-economische-vrijheid.-bovendien-biedt-ons-voorbeeld-van-brand-schreeuwen-in-een-volle-zaal-een-duidelijke-richtlijn-voor-wie-zijn-rechten-in-welke-situatie-moet-verdedigen.-deze-richtlijnen-worden-bepaald-door-ons-criterium-de-eigendomsrechten.}

\bookmarksetup{startatroot}

\chapter{De Problemen}\label{de-problemen}

Laten we kort kijken naar de belangrijkste probleemgebieden in onze
samenleving en zien of we een `rode draad' kunnen vinden die door deze
gebieden loopt.

Hoge belastingen Hoge en stijgende belastingen hebben bijna iedereen
verlamd. Ze belemmeren niet alleen de productiviteit, maar ook
stimulansen en spaarzaamheid, en beperken de vrije energie van mensen.
Op federaal niveau groeit de opstand tegen de last van de
inkomstenbelastingen. Er bestaat een bloeiende beweging van
belastingrebellen, met eigen organisaties en tijdschriften, die weigeren
belasting te betalen die ze als roofzuchtig en ongrondwettelijk
beschouwen. Ook op staats- en lokaal niveau groeit het verzet tegen
onderdrukkende eigendomsbelastingen. Een recordaantal van 1,2 miljoen
Californische kiezers ondertekende de petitie voor het
Jarvis-Gann-initiatief op het stembiljet van 1978. Dit voorstel zou de
onroerendgoedbelasting drastisch en permanent verlagen met twee derde,
tot één procent, en het zou plafonds instellen op de geschatte waarde
van onroerend goed. Bovendien vereist het Jarvis-Gann-initiatief dat er
een goedkeuring van tweederde van alle geregistreerde kiezers in
Californië nodig is om de onroerendgoedbelasting boven dit plafond van
één procent te verhogen. Om te voorkomen dat de staat zomaar een andere
belasting invoert, is ook een tweederde meerderheid in de wetgevende
macht vereist om elke andere belasting te verhogen.

Bovendien staakten in de herfst van 1977 tienduizenden huiseigenaren in
Cook County, Illinois, wegens de onroerendgoedbelasting, die drastisch
was gestegen door verhoogde waardebepalingen. Het behoeft weinig betoog
dat belastingheffing, of het nu gaat om inkomen, eigendom of iets
anders, een exclusief monopolie van de overheid is. Geen enkel individu
of organisatie heeft het recht om belasting te heffen of op die manier
inkomen te verwerven door dwang.

\textbf{Stedelijke begrotingscrisis} In heel het land hebben staten en
gemeenten moeite om de rente en de hoofdsom van hun enorme staatsschuld
te betalen. New York City heeft al een voorbeeld gesteld door
gedeeltelijk in gebreke te blijven bij het nakomen van haar contractuele
verplichtingen. De stedelijke begrotingscrisis is simpelweg het gevolg
van stedelijke overheden die te veel uitgeven, zelfs meer dan de hoge
belastingen die ze van ons vragen. Nogmaals, het is aan de overheden om
te bepalen hoeveel ze uitgeven; opnieuw is de overheid verantwoordelijk
voor deze situatie.

\textbf{Vietnam en andere buitenlandse interventies} De oorlog in
Vietnam was een volledige ramp voor het Amerikaanse buitenlands beleid.
Na het verlies van ontelbare levens en de verwoesting van het land, viel
de door Amerika gesteunde regering begin 1975 uiteindelijk om, na enorme
kosten in middelen. Deze ramp heeft het Amerikaanse interventionistische
buitenlands beleid ernstig ondermijnd en was een belangrijke factor
achter de besluitvorming van het Congres om een rem te zetten op de
Amerikaanse militaire betrokkenheid bij de Angolese crisis. Buitenlands
beleid is uiteraard een exclusief monopolie van de federale regering. De
oorlog werd uitgevochten door onze strijdkrachten, die ook als een
verplicht monopolie onder diezelfde federale overheid vallen. Dit maakt
de regering volledig verantwoordelijk voor het probleem van de oorlog en
het buitenlands beleid in zijn geheel en in al zijn facetten.

\textbf{Criminaliteit op straat} Bedenk dat de misdaad in kwestie per
definitie op straat plaatsvindt. De straten zijn, bijna overal, eigendom
van de overheid, die daardoor een vrijwel monopolie op straatbezit
heeft. De politie, die ons zou moeten beschermen tegen deze misdaad, is
een verplicht monopolie van de overheid. Ook de rechtbanken, die
criminelen moeten veroordelen en straffen, zijn een dwingend monopolie
van de overheid. De overheid is dus verantwoordelijk voor elk aspect van
het probleem van criminaliteit op straat. Het falen in deze kwestie
moet, net als het falen in Vietnam, uitsluitend aan de overheid worden
toegeschreven.

\textbf{Verkeersopstoppingen} Dit gebeurt alleen op straten en wegen die
van de overheid zijn.

\textbf{Het militair-industrieel complex} Dit complex is volledig een
uitvinding van de federale overheid. Het is de overheid die beslist
ontelbare miljarden uit te geven aan overbodig wapentuig. Het is de
overheid die contracten verstrekt en inefficiëntie subsidieert door
middel van kostprijs-plus-garanties. Het is ook de overheid die
fabrieken bouwt en deze verhuurt of rechtstreeks toekent aan aannemers.
Natuurlijk lobbyen de betrokken bedrijven voor deze privileges, maar het
mechanisme waarmee deze privileges en de verspilling van middelen tot
stand komen, kan alleen via de overheid functioneren.

\textbf{Vervoer} De vervoerscrisis heeft niet alleen te maken met
overvolle straten, maar ook met in verval geraakte spoorwegen, dure
luchtvaartmaatschappijen en congestie op luchthavens tijdens piekuren.
Daarnaast kampen metrolijnen, zoals die in New York City, met een tekort
aan middelen en dreigen ze zichtbaar in te storten. Toch zijn de
spoorwegen ooit overspoeld met uitgebreide overheidssubsidies (federaal,
staats en lokaal) in de negentiende eeuw. Ze zijn bovendien de zwaarst
gereguleerde industrie in de Amerikaanse geschiedenis geweest.
Luchtvaartmaatschappijen functioneren in een gereguleerde omgeving,
gecontroleerd door de Civil Aeronautics Board, en worden eveneens
gesubsidieerd via regelgeving, contracten en vrijwel gratis luchthavens.
Alle luchthavens voor commerciële luchtvaart zijn in handen van
overheidsinstanties, grotendeels op lokaal niveau. De metro's van New
York City zijn al tientallen jaren eigendom van de overheid.

\textbf{Riviervervuiling} De rivieren zijn eigenlijk geen eigendom van
de overheid. Bovendien zijn de gemeentelijke rioleringssystemen de
grootste vervuilers van het water. Het is ironisch: de overheid is zowel
de grootste vervuiler als een onzorgvuldige `eigenaar' van deze bronnen.

\textbf{Watertekorten} Watertekorten komen chronisch voor in sommige
gebieden van het land en regelmatig in andere, zoals New York City. Toch
is de overheid, (1) via haar eigendom van het publieke domein, eigenaar
van de rivieren waaruit veel van het water komt, en (2) als vrijwel de
enige commerciële waterleverancier, eigenaar van de reservoirs en
waterleidingen.

\textbf{Luchtvervuiling} Nogmaals, de overheid bezit als eigenaar van
het publieke domein de lucht. Bovendien hebben de rechtbanken, die ook
volledig in handen van de overheid zijn, generaties lang nagelaten om
onze eigendomsrechten in ons lichaam en in onze boomgaarden te
beschermen tegen vervuiling door de industrie. Een groot deel van de
directe vervuiling komt bovendien van fabrieken die in overheidseigendom
zijn.

\textbf{Stroomtekorten en stroomuitval} In heel het land hebben staats-
en lokale overheden verplichte monopolies voor gas en elektriciteit
ingesteld. Ze hebben deze monopolies toegewezen aan particuliere
nutsbedrijven, die onder toezicht staan van de overheid. De tarieven van
deze bedrijven worden vastgesteld door overheidsinstanties, zodat ze een
constante en gegarandeerde winst kunnen behalen. Opnieuw is de overheid
zowel de bron van het monopolie als de regulering.

\textbf{Telefoondienst} De steeds slechter wordende telefoondienst is
het resultaat van een nutsbedrijf dat een verplicht monopolie van de
overheid heeft gekregen. De tarieven worden door de overheid vastgesteld
om winst te waarborgen. Net als bij gas en elektriciteit mag niemand
concurreren met het monopolistische telefoonbedrijf.

\textbf{Post} De postdienst kampt al zijn hele bestaan met ernstige
tekorten. In schril contrast met de goederen en diensten die de
privé-industrie op de vrije markt aanbiedt, is de kwaliteit van de post
steeds slechter geworden en de prijzen zijn alleen maar gestegen. Het
publiek dat gebruikmaakt van eersteklas post, moet nu bedrijven
subsidiëren die alleen tweedeklas of derderangs diensten bieden. Het
postkantoor is sinds het einde van de negentiende eeuw een verplicht
monopolie van de overheid. Wanneer particuliere bedrijven zelfs illegaal
mochten concurreren in de postbezorging, boden ze consistent betere
service aan tegen lagere prijzen.

\textbf{Televisie} Televisie bestaat uit saaie programma's en
gemanipuleerd nieuws. Radio- en televisiekanalen zijn al een halve eeuw
genationaliseerd door de federale overheid. Deze overheid schenkt
kanalen aan bevoorrechte licentiehouders en kan deze gaven weer
intrekken als een zender niet naar de zin is van de Federal
Communications Commission. Hoe kan er onder deze omstandigheden sprake
zijn van echte vrijheid van meningsuiting of persvrijheid?

\textbf{Welzijnssysteem} Welzijn is vanzelfsprekend uitsluitend het
domein van de overheid, voornamelijk op staats- en lokaal niveau.

\textbf{Stedelijke huisvesting} Samen met het verkeer vormt stedelijke
huisvesting een van de meest opvallende mislukkingen in onze steden.
Toch zijn er maar weinig andere sectoren die zo sterk met de overheid
verbonden zijn. Stadsplanning heeft de controle en regulering over
steden in handen. Bestemmingswetten omringen huisvesting en landgebruik
met talloze beperkingen. Onroerendgoedbelastingen hebben de stedelijke
ontwikkeling vertraagd en gedwongen tot het verlaten van woningen.
Bouwvoorschriften hebben woningbouw belemmerd en duurder gemaakt.
Stadsvernieuwing heeft enorme subsidies verleend aan
projectontwikkelaars, waardoor appartementen en huurwinkels werden
gesloopt. Dit heeft het woningaanbod verminderd en rassendiscriminatie
verergerd. Bovendien hebben uitgebreide overheidsleningen geleid tot een
overaanbod in de buitenwijken. Huurcontroles hebben gezorgd voor een
tekort aan appartementen en een afname van het aanbod van woonruimte.

\textbf{Vakbondsstakingen en beperkingen} Vakbonden zijn een plaag
geworden. Ze hebben de macht om de economie stil te leggen, maar dit is
vooral te danken aan de vele speciale privileges die de overheid hen
verleent. Een belangrijk voorbeeld hiervan is de Wagner Act van 1935,
die nog steeds van kracht is. Deze wet dwingt werkgevers om te
onderhandelen met vakbonden die een meerderheid van stemmen behalen in
een `onderhandelingseenheid.' Deze eenheid is willekeurig door de
overheid gedefinieerd.

\textbf{Onderwijs} Ooit werd het openbaar onderwijs in de VS zo vereerd
als het moederschap of de vlag. Maar de laatste jaren staat het onder
grote druk vanuit alle hoeken van het politieke spectrum. Zelfs de
voorstanders durven nauwelijks te beweren dat de openbare scholen een
echte onderwijservaring bieden. Onlangs hebben we zelfs extreme gevallen
gezien waarbij de acties van openbare scholen een gewelddadige reactie
uitlokten, zoals in Zuid-Boston en Kanawha County, West Virginia. De
openbare scholen zijn volledig in eigendom van en worden bestuurd door
staats- en lokale overheden, met aanzienlijke steun en coördinatie van
het federale niveau. Ze worden gefinancierd door een leerplichtwet die
alle kinderen tot de middelbare schoolleeftijd verplicht om naar school
te gaan, of dat nu een openbare of een door de overheid gecertificeerde
privéschool is. Ook het hoger onderwijs is de afgelopen decennia sterk
verbonden geraakt met de overheid. Veel universiteiten zijn rechtstreeks
in handen van de overheid, terwijl andere systematisch subsidies en
contracten ontvangen.

\textbf{Inflatie en stagflatie} De Verenigde Staten en de rest van de
wereld kampen al jarenlang met een chronische en toenemende inflatie.
Deze inflatie gaat gepaard met hoge werkloosheid en houdt aan, zelfs
tijdens zowel ernstige als milde recessies; dit verschijnsel wordt
`stagflatie' genoemd. Hieronder volgt een verklaring voor deze
ongewenste situatie. Het is belangrijk om te benadrukken dat de
hoofdoorzaak ligt in de voortdurende expansie van de geldvoorraad. Dit
is een verplicht monopolie van de federale overheid. Iedereen die
probeert te concurreren met de gelduitgifte door de overheid, loopt het
risico op gevangenisstraf wegens valsemunterij. Een groot deel van de
geldvoorraad in het land wordt uitgegeven als `chequeboekgeld' door het
banksysteem, dat volledig onder controle staat van de federale overheid
en haar Federal Reserve System.

\textbf{Watergate} Tenslotte is er het hele traumatische syndroom waar
Amerikanen onder geleden hebben, bekend als `Watergate'. De gevolgen van
Watergate hebben geleid tot een totale ontheiliging van de president en
van voorheen onaantastbare federale instellingen zoals de CIA en de FBI.
De inbreuken op eigendommen, de methoden die doen denken aan een
politiestaat, de misleiding van het publiek, de corruptie en de vele
systematische misdaden van een ooit vrijwel almachtige president
resulteerden in een ondenkbare afzetting. Dit leidde ook tot een
wijdverspreid en rechtvaardigd gebrek aan vertrouwen in alle politici en
overheidsfunctionarissen. De gevestigde orde heeft dit nieuwe, alom
aanwezige wantrouwen vaak betreurd, maar is er niet in geslaagd om het
naïeve vertrouwen van het publiek van voor Watergate terug te winnen. De
liberale historica Cecilia Kenyon bekritiseerde ooit de
Anti-Federalisten---de verdedigers van de Artikelen van Confederatie en
tegenstanders van de Grondwet---als `mannen met weinig vertrouwen' in
overheidsinstellingen. Men vermoedt dat ze niet zo naïef zou zijn
geweest als ze dit artikel na Watergate had geschreven.

Watergate is natuurlijk in de eerste plaats een fenomeen van de
regering. De president is de hoogste uitvoerende macht van de federale
overheid. De `loodgieters' waren zijn instrumenten, en ook de FBI en de
CIA zijn overheidsinstanties. Het is dan ook begrijpelijk dat het geloof
en vertrouwen in de regering door Watergate ernstig zijn aangetast.

Als we om ons heen kijken naar de cruciale probleemgebieden in onze
samenleving---de gebieden van crisis en mislukking---zien we duidelijk
een `rode draad' die al deze issues met elkaar verbindt: de
betrokkenheid van de overheid. In elk van deze gevallen heeft de
overheid de activiteiten volledig gestuurd of in sterke mate beïnvloed.
John Kenneth Galbraith erkende in zijn bestseller \emph{The Affluent
Society} dat de overheidssector het brandpunt was van ons sociale falen.
In plaats van dit probleem aan te pakken, concludeerde hij echter dat er
daarom nog meer middelen van de private naar de publieke sector moesten
worden overgeheveld. Hiermee negeerde hij dat de rol van de overheid in
Amerika - op federaal, staat- en lokaal niveau - in deze eeuw enorm is
toegenomen, zowel in absolute als in relatieve zin, vooral in de
afgelopen decennia. Helaas stelde Galbraith nooit de vraag: is er iets
inherent aan het functioneren en de activiteiten van de overheid dat
juist de mislukkingen creëert die we zo overvloedig waarnemen? We zullen
enkele van de grootste problemen rondom de overheid en de vrijheid in
dit land onderzoeken. Daarbij kijken we naar de oorsprong van deze
mislukkingen en verkennen we de oplossingen die het nieuwe libertarisme
biedt.

\bookmarksetup{startatroot}

\chapter{Onvrijwillige
dienstbaarheid}\label{onvrijwillige-dienstbaarheid}

Als er iets is waar een libertariër fel tegen moet zijn, dan is het
onvrijwillige dienstbaarheid---gedwongen arbeid. Dit ontkent het meest
elementaire recht op zelf-eigenaarschap. `Vrijheid' en `slavernij'
worden altijd als tegenpolen gezien. Daarom is de libertariër volledig
gekant tegen slavernij.¹ Is dit tegenwoordig niet een academische vraag?
Maar is dat wel zo? Want wat is slavernij anders dan (a) mensen dwingen
om te werken aan taken die de slavendrijver wil, en (b) hen enkel
voorzien van een minimaal levensonderhoud of in ieder geval minder dan
wat de slaaf vrijwillig zou hebben geaccepteerd? Kortom, dwangarbeid
tegen lonen die onder de vrije markt liggen.

Zijn we in het huidige Amerika werkelijk vrij van `slavernij' en
onvrijwillige dienstbaarheid? Wordt het verbod op onvrijwillige
dienstbaarheid, zoals vastgelegd in het Dertiende Amendement, echt
nageleefd?²

\section{DIENSTPLICHT}\label{dienstplicht}

Een dienstplicht is een systeem waarin burgers verplicht zijn om
militaire dienst te nemen of publieke taken te vervullen. Dit kan zowel
in tijden van oorlog als in vredestijd het geval zijn. De redenen voor
het instellen van dienstplicht zijn divers. Overheden kunnen het zien
als een manier om de defensiecapaciteit te versterken of om burgers te
betrekken bij belangrijke maatschappelijke taken. Critici van de
dienstplicht wijzen vaak op de inbreuk op individuele vrijheden. Voor
hen is het een dwangmaatregel die de keuzevrijheid van burgers
ondermijnt. Aan de andere kant stellen voorstanders dat een dienstplicht
een gevoel van saamhorigheid en betrokkenheid bij de samenleving kan
creëren. Zij zien het als een manier om mensen verantwoordelijk te maken
voor hun gemeenschap. In Nederland bestaat de dienstplicht sinds de
jaren '90 niet meer in de traditionele zin. Er is echter nog steeds een
systeem van reservisten als onderdeel van de krijgsmacht. Dit houdt in
dat burgers zich kunnen aanmelden voor militaire trainingen en ingezet
kunnen worden wanneer dat nodig is. Het debat over de terugkeer van de
dienstplicht is tijd en wederom opgelaaid, vooral in tijden van
internationale spanningen. Dit vraagt om een zorgvuldige afweging van
individuele rechten en verantwoordelijkheden ten opzichte van de
behoeften van de samenleving.

Er is nauwelijks een duidelijker voorbeeld van onvrijwillige
dienstbaarheid dan ons dienstplichtsysteem. Elke jongere moet zich
aanmelden bij het selectieve dienstsysteem zodra hij achttien wordt. Hij
is verplicht zijn dienstplichtkaart altijd bij zich te dragen. Wanneer
de federale overheid dat nodig acht, kan hij door de autoriteiten worden
opgepakt en ingelijfd in het leger. Op dat moment heeft hij geen
controle meer over zijn lichaam en wil; hij is onderworpen aan de eisen
van de regering. Hij kan gedwongen worden om te doden en zijn eigen
leven in gevaar te brengen, afhankelijk van de beslissingen van de
autoriteiten. Wat is onvrijwillige dienstbaarheid als dat niet de
dienstplicht is?

Het utilitaire aspect speelt een belangrijke rol in het argument voor
het dienstplichtsysteem. De regering stelt bijvoorbeeld: `Wie zal ons
verdedigen tegen buitenlandse aanvallen als we geen dwang uitoefenen bij
het aanstellen van onze verdedigers?' Een libertariër kan deze
redenering op verschillende manieren weerleggen. Ten eerste, als jij, ik
en onze buurman vinden dat we verdedigd moeten worden, hebben we geen
moreel recht om dwang - bijvoorbeeld met een bajonet of een revolver -
te gebruiken om iemand anders te dwingen ons te verdedigen. Deze daad
van dienstplicht is evenzeer een daad van ongerechtvaardigde agressie -
van ontvoering en mogelijk moord - als de vermeende agressie waartegen
we ons in eerste instantie proberen te beschermen. Als we daarbij in
overweging nemen dat dienstplichtigen hun lichaam en leven, indien
nodig, aan de `maatschappij' of aan `hun land' te danken zouden hebben,
moeten we ons afvragen: wie is deze `maatschappij' of dit `land' dat als
talisman wordt gebruikt om slavernij te rechtvaardigen? Het zijn
simpelweg alle individuen in het gebied, behalve de jongeren die worden
ingelijfd. `Samenleving' en `land' zijn in dit geval mythische
abstracties die worden aangewend om het onverbloemde gebruik van dwang
te verhullen, ten behoeve van specifieke individuen.

Ten tweede, laten we het utilitaristische aspect bekijken: waarom is het
nodig om verdedigers verplicht te stellen? Op de vrije markt wordt
niemand ingelijfd. Mensen verkrijgen daar, door middel van vrijwillige
koop en verkoop, alle denkbare goederen en diensten, ook de meest
essentiële. Op de markt kunnen mensen voedsel, onderdak, kleding,
medische zorg, enzovoorts krijgen. Waarom zouden ze niet ook verdedigers
kunnen inhuren? Er zijn genoeg mensen die dagelijks worden ingehuurd
voor gevaarlijke werkzaamheden, zoals bosbrandbestrijders, rangers,
testpiloten en, natuurlijk, politie en particuliere beveiligers. Waarom
zou het niet mogelijk zijn om soldaten op dezelfde manier in te huren?

Of, om het anders te zeggen: de overheid heeft talloze mensen in dienst
voor verschillende functies, van vrachtwagenchauffeurs tot
wetenschappers en typisten. Waarom zijn deze mensen dan niet
dienstplichtig? Waarom is er geen `tekort' aan deze beroepen dat de
overheid zou dwingen om ze in te schakelen? Sterker nog, ook binnen het
leger is er geen `tekort' aan officieren die opgeroepen moeten worden;
niemand verplicht generaals of admiraals. Het antwoord op deze vragen is
eenvoudig: er is geen tekort aan overheidstypisten, omdat de regering de
markt opgaat en hen inhuurt tegen het marktloon. Er is ook geen tekort
aan generaals, want zij ontvangen een goed salaris, samen met voordelen
en pensioenen. Het tekort aan soldaten ontstaat doordat hun loon ver
onder het marktloon ligt. Jarenlang was het inkomen van een soldaat,
zelfs als je de monetaire waarde van het gratis voedsel, onderdak en
andere diensten die GI's ontvingen meerekent, ongeveer de helft van het
salaris dat hij in een burgerfunctie had kunnen verdienen. Is het dan
verwonderlijk dat er een chronisch tekort aan soldaten is? Het is al
jaren bekend dat de beste manier om mensen te overtuigen zich vrijwillig
aan te melden voor gevaarlijk werk, is door ze extra te betalen als
compensatie. En toch betaalt de overheid mannen de helft van wat ze in
hun privéleven zouden kunnen verdienen.

Er is ook de bijzondere schande van de artsen dienstplicht, waarbij
artsen op veel latere leeftijden worden onderworpen aan de militaire
dienst dan anderen. Moeten artsen gestraft worden voor hun keuze om dit
belangrijke beroep uit te oefenen? Wat is de morele rechtvaardiging voor
de zware lasten die specifiek aan dit essentiële beroep worden opgelegd?
Is dit de manier om het tekort aan artsen op te lossen? Door iedereen te
laten weten dat wie arts wordt, zeker opgeroepen zal worden, vaak op een
veel latere leeftijd? De behoefte van het leger aan artsen kan eenvoudig
worden vervuld als de overheid bereid is om artsen een marktconform
salaris te betalen, aangevuld met genoeg compensatie voor het
gevaarlijke werk. Als de overheid kernfysici of strategen uit
`denktanks' wil inhuren, vindt ze wel manieren om dat te doen tegen
extreem hoge salarissen. Zijn artsen dan minderwaardig aan andere
beroepsgroepen?

\section{HET LEGER}\label{het-leger}

Ten tweede, laten we kijken naar het utilitaristische aspect: waarom is
het nodig om verdedigers verplicht te stellen? Op de vrije markt wordt
niemand gedwongen. Daar verkrijgen mensen, door vrijwillige koop en
verkoop, alle denkbare goederen en diensten, ook de meest essentiële. Op
de markt kunnen mensen voedsel, onderdak, kleding en medische zorg
krijgen. Waarom zouden ze dan geen verdedigers kunnen inhuren? Er zijn
voldoende mensen die dagelijks worden ingehuurd voor gevaarlijke
werkzaamheden, zoals bosbrandbestrijders, rangers, testpiloten, en
natuurlijk politie en particuliere beveiligers. Waarom zou het niet
mogelijk zijn om soldaten op dezelfde manier in te huren? Laten we het
anders formuleren: de overheid heeft talloze mensen in dienst voor
verschillende functies, van vrachtwagenchauffeurs tot wetenschappers en
typisten. Waarom zijn deze mensen dan niet dienstplichtig? Waarom is er
geen `tekort' in deze beroepen dat de overheid zou dwingen om ze in te
schakelen? Sterker nog, binnen het leger zijn er ook geen `tekorten' aan
officieren die opgeroepen hoeven te worden; niemand verplicht generaals
of admiraals. Het antwoord op deze vragen is eenvoudig: er is geen
tekort aan overheidstypisten, omdat de regering de markt opgaat en hen
inhuurt tegen het marktloon. Ook is er geen tekort aan generaals, want
zij ontvangen een goed salaris, met voordelen en pensioenen. Het tekort
aan soldaten ontstaat omdat hun salaris ver onder het marktloon ligt.
Jarenlang was het inkomen van een soldaat, zelfs met de monetaire waarde
van het gratis voedsel, onderdak en andere diensten meegerekend,
ongeveer de helft van wat hij in een burgerfunctie had kunnen verdienen.
Is het dan verwonderlijk dat er een chronisch tekort aan soldaten is?
Het is al jaren bekend dat de beste manier om mensen te motiveren om
zich vrijwillig aan te melden voor gevaarlijk werk, is door ze extra te
betalen als compensatie. En toch betaalt de overheid mannen de helft van
wat ze in hun privéleven zouden kunnen verdienen. Er is ook de
bijzondere schande van de artsen dienstplicht, waarbij artsen op veel
latere leeftijden worden onderworpen aan militaire dienst dan anderen.
Moeten artsen gestraft worden voor hun keuze voor dit belangrijke
beroep? Wat is de morele rechtvaardiging voor de zware lasten die
specifiek aan dit essentiële beroep worden opgelegd? Is dit de manier om
het tekort aan artsen op te lossen? Door iedereen te laten weten dat wie
arts wordt, zeker opgeroepen zal worden, vaak op een veel latere
leeftijd? De behoefte van het leger aan artsen kan eenvoudig worden
vervuld als de overheid bereid is om artsen een marktconform salaris te
betalen, plus genoeg compensatie voor het gevaarlijke werk. Als de
overheid kernfysici of strategen uit `denktanks' wil inhuren, vindt ze
wel manieren om dat te doen tegen extreem hoge salarissen. Zijn artsen
dan minderwaardig aan andere beroepsgroepen?

De dienstplicht in het leger is een flagrant en zwaarwegend voorbeeld
van onvrijwillige dienstbaarheid. Maar er is ook een andere, veel
subtielere en daardoor minder zichtbare vorm: de structuur van het leger
zelf. Denk hier eens over na: in welk ander beroep in het land zijn er
strenge straffen, waaronder gevangenisstraf en in sommige gevallen zelfs
de dood, voor `desertie', oftewel het opgeven van die specifieke
functie? Als iemand ontslag neemt bij General Motors, wordt diegene dan
bij zonsopgang doodgeschoten?

Men zou kunnen tegenwerpen dat rekruten, door vrijwillig in dienst te
treden, verplicht zijn om voor een bepaalde periode te blijven. Maar het
hele idee van `diensttijd' vormt een groot probleem. Stel je voor dat
een ingenieur een contract tekent met ARAMCO om drie jaar in
Saoedi-Arabië te werken. Na een paar maanden besluit hij dat het leven
daar hem niet bevalt en neemt hij ontslag. Dit kan misschien worden
gezien als een moreel verzuim --- een schending van een morele
verplichting. Maar is het ook een juridisch afdwingbare verplichting?
Kortom, kan of moet hij, door het wapenmonopolie van de overheid,
gedwongen worden om de rest van zijn termijn aan de slag te blijven? Als
dat zo is, dan zou dit leiden tot dwangarbeid en slavernij. Hoewel hij
inderdaad een belofte voor toekomstig werk heeft gedaan, blijft zijn
lichaam in een vrije samenleving zijn eigendom. In de praktijk, en ook
volgens de libertarische theorie, kan de ingenieur moreel bekritiseerd
worden voor zijn actie. Hij kan op een zwarte lijst komen bij andere
oliemaatschappijen, of gedwongen worden het voorschot terug te betalen
dat het bedrijf hem heeft gegeven. Maar hij kan niet voor drie jaar tot
slaaf gemaakt worden door ARAMCO.

Maar als dit geldt voor ARAMCO of voor welk ander beroep ook in het
privéleven, waarom zou dat dan anders zijn in het leger? Als iemand een
contract voor zeven jaar ondertekent en vervolgens ontslag neemt, moet
hij de mogelijkheid hebben om te vertrekken. Hij verliest dan zijn
pensioenrechten, wordt moreel bekritiseerd en kan op een zwarte lijst
van vergelijkbare beroepen terechtkomen. Maar als zelfeigenaar kan hij
niet tegen zijn wil tot slaaf gemaakt worden.

Er kan gezegd worden dat het leger een bijzonder belangrijk beroep is
dat dit soort dwangsancties nodig heeft, terwijl andere banen dat niet
nodig hebben. Laten we, afgezien van de belangrijke beroepen zoals
geneeskunde, landbouw en transport, eens kijken naar een vergelijkbare
defensieve functie in het burgerleven: de politie. De politie biedt
zeker een even essentiële, misschien zelfs nog belangrijkere, dienst.
Toch komen er elk jaar nieuwe mensen bij de politie en verlaten anderen
deze organisatie, zonder dat er dwang is om hen jarenlang aan hun
functie te binden. De libertariër pleit niet alleen voor het afschaffen
van de dienstplicht, maar stelt ook voor om het gehele concept van
dienstplicht en de daarbij behorende slavernij af te schaffen. Laat de
strijdkrachten opereren op een manier die vergelijkbaar is met die van
de politie, brandweer, boswachters en particuliere beveiligers, vrij van
de last en de morele verwerpelijkheid van onvrijwillige dienstbaarheid.

Maar er is meer te zeggen over het leger als instituut, zelfs als het
volledig vrijwillig zou zijn. Amerikanen zijn bijna helemaal vergeten
dat vastberaden verzet tegen het hele idee van een `permanent leger' één
van de nobelste en sterkste elementen van het oorspronkelijke
Amerikaanse erfgoed is. Een regering met een permanent leger zal altijd
in de verleiding komen om dit leger te gebruiken, en vaak op een
agressieve, interventionistische en oorlogszuchtige manier. Hoewel het
buitenlands beleid hier verderop aan bod komt, is het duidelijk dat een
permanent leger een voortdurende verleiding vormt voor de staat om zijn
macht uit te breiden, om andere mensen en landen op te dringen en om het
interne leven van de natie te controleren. Het oorspronkelijke doel van
de Jeffersoniaanse beweging --- een grotendeels libertarische stroming
binnen de Amerikaanse politiek --- was zelfs om het permanente leger en
de marine volledig af te schaffen. Het oorspronkelijke Amerikaanse
principe was dat als de natie werd aangevallen, de burgers zich zouden
haasten om de indringer terug te dringen. Een permanente gewapende macht
zou slechts tot problemen kunnen leiden en de staatsmacht vergroten.
Tijdens zijn scherpe en profetische kritiek op de voorgestelde Grondwet
tijdens de ratificatieconventie in Virginia waarschuwde Patrick Henry
tegen een permanent leger: `Het Congres heeft, door de macht om
belasting te heffen, de macht om een leger op te richten en de controle
over de militie, het zwaard in de ene hand en de portemonnee in de
andere. Zullen we veilig zijn zonder een van beide?'

Elk permanent leger vormt dus een blijvende bedreiging voor de vrijheid.
Het monopolie op dwangmatige wapens, de actuele neiging om een
`militair-industrieel complex' op te zetten en te ondersteunen voor de
bevoorrading van dat leger, en niet te vergeten, zoals Patrick Henry
opmerkt, de macht om belastingen te heffen ter financiering daarvan,
vormen een voortdurende dreiging dat het leger steeds groter en
machtiger wordt. Elke door belastingen ondersteunde instelling wordt
door libertariërs als dwingend bestempeld, maar een leger is vooral
bedreigend omdat het de enorme kracht van moderne wapens in één hand
verzamelt.

\section{\texorpdfstring{\textbf{ANTI-STAKINGSWETTEN}}{ANTI-STAKINGSWETTEN}}\label{anti-stakingswetten}

Er kan gesteld worden dat het leger een bijzonder belangrijk beroep is
dat dwangsancties nodig heeft, terwijl andere beroepen dat niet hoeven.
Laten we, buiten belangrijke sectoren zoals geneeskunde, landbouw en
transport om, eens kijken naar een vergelijkbare defensieve functie in
het burgerleven: de politie. De politie biedt zeker een essentiële,
misschien zelfs nog belangrijkere, dienst. Toch komen er elk jaar nieuwe
mensen bij de politie en verlaten anderen het vak, zonder dat er dwang
is om hen jarenlang aan hun functie te binden. De libertariër pleit niet
alleen voor het afschaffen van de dienstplicht, maar ook voor het hele
concept van dienstplicht en de bijbehorende slavernij. Laat de
strijdkrachten opereren op een manier die vergelijkbaar is met de
politie, brandweer, boswachters en particuliere beveiligers. Dit zou
moeten gebeuren zonder de last en de morele verwerpelijkheid van
onvrijwillige dienstbaarheid. Maar er is meer te zeggen over het leger
als instituut, zelfs als het volledig vrijwillig zou zijn. Amerikanen
zijn bijna helemaal vergeten dat vastberaden verzet tegen een `permanent
leger' een van de nobelste en sterkste elementen van het oorspronkelijke
Amerikaanse erfgoed is. Een regering met een permanent leger zal altijd
in de verleiding komen om dit leger agressief, interventionistisch en
oorlogszuchtig te gebruiken. Hoewel het buitenlands beleid hier verderop
aan bod komt, is het duidelijk dat een permanent leger een constante
verleiding vormt voor de staat om zijn macht uit te breiden, andere
mensen en landen op te dringen en het interne leven van de natie te
controleren. Het oorspronkelijke doel van de Jeffersoniaanse beweging
--- een grotendeels libertarische stroming in de Amerikaanse politiek
--- was juist om het permanente leger en de marine volledig af te
schaffen. Het oorspronkelijke Amerikaanse principe hield in dat wanneer
de natie werd aangevallen, de burgers zich zouden haasten om de
indringer terug te dringen. Een permanente gewapende macht kan immers
alleen maar leiden tot problemen en de staatsmacht vergroten. Tijdens
zijn scherpe en profetische kritiek op de voorgestelde Grondwet in de
ratificatieconventie in Virginia waarschuwde Patrick Henry voor een
permanent leger: `Het Congres heeft, door de macht om belasting te
heffen, de macht om een leger op te richten en de controle over de
militie, het zwaard in de ene hand en de portemonnee in de andere.
Zullen we veilig zijn zonder een van beide?' Elk permanent leger vormt
dus een blijvende bedreiging voor de vrijheid. Het monopolie op
dwangmatige wapens, de actuele neiging om een `militair-industrieel
complex' op te zetten ter ondersteuning van dat leger, en niet te
vergeten, zoals Patrick Henry opmerkt, de macht om belastingen te heffen
voor de financiering daarvan, vormen een voortdurende dreiging dat het
leger steeds groter en machtiger wordt. Elke door belastingen
ondersteunde instelling wordt door libertariërs als dwingend gezien,
maar een leger is vooral bedreigend omdat het de enorme kracht van
moderne wapens in één hand concentreert.

Op 4 oktober 1971 beriep president Nixon zich op de Taft-Hartley Act om
een rechterlijk bevel te verkrijgen dat een dokstaking voor 80 dagen
opschortte. Dit was de negende keer dat de federale overheid deze wet
gebruikte bij een dokstaking. Maanden eerder werd het hoofd van de
lerarenvakbond van New York City voor enkele dagen gevangen gezet omdat
hij een wet overtrad die ambtenaren verbood te staken. Het is
begrijpelijk dat een bevolking die al zo lang lijdt, opgelucht is dat
zij niet wordt geconfronteerd met de verstoringen van een staking. Maar
de opgelegde `oplossing' was niets minder dan pure dwangarbeid; de
arbeiders werden tegen hun zin gedwongen om weer aan het werk te gaan.
In een maatschappij die beweert tegen slavernij te zijn en in een land
waar onvrijwillige dienstbaarheid verboden is, is er geen moreel excuus
voor enige wettelijke of gerechtelijke actie die staken verbiedt of
vakbondsleiders die zich daar niet aan houden, opsluit. Slavernij komt
de slavendrijvers vaak beter uit.

Het is waar dat een staking een bijzondere vorm van werkonderbreking is.
Stakers geven niet alleen hun baan op; ze beweren ook dat ze op een
bepaalde manier nog steeds `bezit' hebben van hun werk en er recht op
hebben. Ze zijn van plan om terug te keren zodra de problemen zijn
opgelost. De oplossing voor dit tegenstrijdige beleid en de
ontwrichtende invloed van vakbonden is echter niet het aannemen van
wetten die stakingen verbieden. De echte oplossing ligt in het
afschaffen van het grote aantal wetten, op federaal, staats- en lokaal
niveau, die vakbonden speciale overheidsprivileges geven. Wat nodig is
voor zowel het libertarische principe als voor een gezonde economie, is
het verwijderen en afschaffen van deze speciale privileges.

Deze privileges zijn vastgelegd in de federale wetgeving, met name in de
Wagner-Taft-Hartley Act uit 1935 en de Norris-LaGuardia Act uit 1931.
Laatstgenoemde verbiedt rechtbanken om een gerechtelijk bevel uit te
vaardigen bij dreigend vakbondsgeweld. De Wagner Act verplicht
werkgevers om `te goeder trouw' te onderhandelen met elke vakbond die de
stemmen van de meerderheid wint in een door de federale overheid
aangestelde arbeidseenheid. Daarnaast verbiedt deze wet werkgevers om
vakbondsorganisatoren te discrimineren. Pas na de Wagner Act en zijn
voorganger, de NIRA uit 1933, konden vakbonden een krachtige rol spelen
in het Amerikaanse leven. Het aantal vakbondslieden steeg van ongeveer
vijf procent naar meer dan twintig procent van de beroepsbevolking.
Bovendien beschermen lokale en staatswetten vakbonden vaak tegen
rechtszaken en stellen ze beperkingen aan werkgevers die stakingsbrekers
inhuren. Ook krijgt de politie vaak de instructie om niet in te grijpen
bij geweld tegen stakers door vakbondspiketten. Als deze speciale
voorrechten en immuniteiten wegvallen, zouden de vakbonden terugvallen
naar hun vroegere, verwaarloosbare rol in de Amerikaanse economie.

Het is opmerkelijk dat, toen de algemene verontwaardiging tegen
vakbonden leidde tot de Taft-Hartley Act van 1947, de regering geen van
de speciale privileges introk. In plaats daarvan voegde ze bijzondere
beperkingen toe aan vakbonden om hun macht te beperken, terwijl die
macht door de regering zelf was gecreëerd. Wanneer de gelegenheid zich
voordoet, is de natuurlijke neiging van de staat om zijn macht uit te
breiden, niet om deze te verkleinen. Dit resulteert in de vreemde
situatie waarin de overheid eerst vakbonden opbouwt en vervolgens roept
om beperkingen op hun macht. Dit doet denken aan de Amerikaanse
landbouwprogramma's, waarbij één afdeling van het Ministerie van
Landbouw boeren betaalt om hun productie te verminderen, terwijl een
andere afdeling hen juist beloont voor een hogere productiviteit. Dit
lijkt irrationeel vanuit het perspectief van de consument en de
belastingbetaler, maar is volkomen logisch voor de gesubsidieerde boeren
en de steeds groter wordende bureaucratie. Op vergelijkbare wijze heeft
het schijnbaar tegenstrijdige beleid van de overheid ten aanzien van
vakbonden twee doelen: de overheid wil haar invloed op arbeidsrelaties
vergroten en tegelijkertijd een geïntegreerd en establishmentvriendelijk
vakbondsdenken bevorderen als ondergeschikte partner in de rol van de
overheid over de economie.

\section{HET BELASTINGSTELSEL}\label{het-belastingstelsel}

Belasting, we kunnen er niet omheen. Het is een onderwerp waar iedereen
mee te maken krijgt, maar wat weten we er eigenlijk echt van? Het
belastingstelsel vormt een complex geheel van regels en wetten die
bepalen hoeveel belasting we betalen en waarom. Dit systeem is
essentieel voor het functioneren van onze samenleving. Het geld dat via
belastingen wordt geïnd, wordt gebruikt voor belangrijke zaken zoals
onderwijs, gezondheidszorg en infrastructuur. In Nederland hebben we
verschillende soorten belastingen. De inkomstenbelasting is misschien de
bekendste. Deze belasting wordt geheven op het inkomen van mensen. Hoe
meer je verdient, hoe meer belasting je betaalt. Daarnaast zijn er ook
belastingen op goederen en diensten, zoals de btw en accijnzen. Deze
belastingen zorgen ervoor dat de prijzen van sommige producten stijgen.
Ook bedrijven betalen belasting. De vennootschapsbelasting is een
belangrijke bron van inkomsten voor de overheid. Dit is de belasting die
bedrijven betalen over hun winst. Er zijn ook specifieke belastingen
voor bepaalde sectoren, zoals de woningmarkt. Denk bijvoorbeeld aan de
onroerendezaakbelasting (OZB) voor huiseigenaren. De belastingdruk in
Nederland is relatief hoog vergeleken met andere landen. Dat roept vaak
vragen op over de rechtvaardigheid van het systeem. Veel mensen vinden
dat de belastingheffing eerlijker kan. De discussie over
belastingontwijking en -ontduiking is de afgelopen jaren steeds
relevanter geworden. Grote bedrijven en rijke individuen vinden vaak
manieren om minder belasting te betalen, wat leidt tot frustratie bij de
gemiddelde burger. Er zijn ook regelmatig veranderingen in het
belastingstelsel. De overheid past de regels aan om tegemoet te komen
aan bijvoorbeeld economische ontwikkelingen of om sociale voorzieningen
te verbeteren. Dit kan leiden tot onduidelijkheid bij belastingbetalers,
die moeite hebben om de nieuwste regels te begrijpen. Gelukkig zijn er
verschillende manieren om hulp te krijgen bij belastingzaken.
Belastingadviseurs en accountants kunnen ondersteuning bieden bij het
indienen van belastingaangiften of het plannen van belastingen. Ook
online platforms bieden tools en tips om het proces te vereenvoudigen.
Kortom, het belastingstelsel is een essentieel onderdeel van onze
samenleving. Het beïnvloedt ons dagelijks leven en is het resultaat van
lange, soms complexe, discussies. Het is van belang dat we ons hierin
verdiepen, zodat we goed geïnformeerd zijn over hoe ons geld wordt
beheerd en besteed.

In zekere zin is het hele belastingsysteem een vorm van onvrijwillige
dienstbaarheid. Neem bijvoorbeeld de inkomstenbelasting. Deze hoge
belasting zorgt ervoor dat we een groot deel van het jaar - vaak
meerdere maanden - voor niets werken voor de overheid, voordat we met
ons inkomen op de markt aan de slag mogen. Een belangrijk aspect van
slavernij is immers gedwongen werken voor iemand met weinig of geen
loon. De inkomstenbelasting betekent dat we hard werken en inkomen
verdienen, maar vervolgens moet zien hoe de overheid onder dwang een
groot deel daarvan afneemt voor haar eigen doeleinden. Wat is dit anders
dan gedwongen arbeid zonder loon?

De inhoudingsfunctie van de inkomstenbelasting is een duidelijk
voorbeeld van onvrijwillige dienstbaarheid. Zoals de moedige industrieel
Vivien Kellems uit Connecticut jaren geleden stelde, wordt de werkgever
gedwongen tijd, arbeid en geld te investeren in het aftrekken en
doorsturen van de belastingen van zijn werknemers naar de federale en
staatsoverheden. En toch ontvangt de werkgever geen vergoeding voor deze
uitgaven. Welk moreel principe rechtvaardigt het dat de overheid
werkgevers dwingt op te treden als haar onbetaalde belastinginners?

Het inhoudingsprincipe vormt de kern van het hele federale
inkomstenbelastingsysteem. Zonder het constante en relatief pijnloze
proces van belastinginhouding op het loonstrookje van de werknemer zou
de overheid nooit in staat zijn om de hoge belastingniveaus van
werknemers in één keer te innen. Weinig mensen realiseren zich dat het
systeem van belastinginhouding pas tijdens de Tweede Wereldoorlog werd
geïntroduceerd, oorspronkelijk als een tijdelijke maatregel in
oorlogstijd. Maar net als zoveel andere kenmerken van staatsdespotisme,
is deze noodmaatregel al snel een onmiskenbaar onderdeel van het
Amerikaanse systeem geworden.

Het is misschien veelzeggend dat de federale regering, die door Vivien
Kellems werd uitgedaagd om de grondwettigheid van het inhoudingssysteem
te toetsen, niet op deze uitdaging inging. In februari 1948 maakte
Kellems, een kleine fabrikant uit Westport, Connecticut, bekend dat ze
de wet op de inhouding uitdaagde en weigerde de belasting van haar
werknemers in te houden. Ze eiste dat de federale overheid haar zou
aanklagen, zodat de rechtbank zich kon uitspreken over de
grondwettelijkheid van het inhoudingssysteem. De overheid weigerde dit,
maar nam in plaats daarvan het verschuldigde bedrag van haar
bankrekening in beslag. Kellems spande vervolgens een rechtszaak aan bij
de federale rechtbank om de regering te dwingen haar geld terug te
geven. Toen de zaak uiteindelijk in februari 1951 voor de rechter kwam,
beval de jury dat de overheid haar geld moest terugbetalen. Maar de
toetsing van de grondwettigheid kwam nooit aan bod.

Om het nog erger te maken, wordt de individuele belastingbetaler door de
overheid gedwongen om onbetaald te werken aan de arbeidsintensieve en
ondankbare taak om uit te rekenen hoeveel hij de overheid verschuldigd
is. Daarnaast kan hij de kosten en het werk die hij maakt voor het
invullen van zijn aangifte niet in rekening brengen. Bovendien is de wet
die iedereen verplicht om zijn belastingformulier in te vullen een
duidelijke schending van het Vijfde Amendement van de Grondwet. Dit
amendement verbiedt de overheid om iemand te dwingen zichzelf te
beschuldigen. Toch hebben de rechtbanken, die doorgaans zorgzaam zijn in
het beschermen van de rechten die voortvloeien uit het Vijfde Amendement
in minder gevoelige zaken, hier niets ondernomen. Dit terwijl het hele
bestaan van de omvangrijke federale overheidsstructuur op het spel
staat. De intrekking van de inkomstenbelasting, de inhouding of de
bepalingen omtrent zelfincriminatie zou de overheid terugdringen naar de
relatief kleine machtsniveaus die ons land voor de twintigste eeuw had.

Detailhandelsverkopen, accijnzen en toegangsbelastingen leiden ook tot
onbetaalde arbeid. In dit geval betreft het de onbetaalde arbeid van de
detailhandelaar, die belastingen int en deze doorstuurt naar de
overheid.

De hoge kosten van het innen van belastingen door de overheid hebben nog
een ander ongewenst effect - misschien niet onbedoeld door de
machthebbers. Deze kosten, die gemakkelijk door grote bedrijven kunnen
worden opgebracht, vormen een onevenredig zware en vaak verlammende last
voor de kleine werkgever. De grote werkgever kan de kosten zonder
problemen dragen, terwijl zijn kleine concurrent veel meer van de lasten
moet zien te verlichten.

\section{DE RECHTBANKEN}\label{de-rechtbanken}

De rechtbanken hebben, zelfs in minder gevoelige zaken, vaak niets
gedaan om de rechten van het Vijfde Amendement te beschermen. Dit is
opmerkelijk omdat het op het spel staat om de omvangrijke federale
overheidsstructuur te behouden. De invoering van de inkomstenbelasting,
de inhouding of bepalingen over zelfincriminatie zou de overheid weer
terugbrengen naar een veel kleiner machtsniveau, zoals we dat vóór de
twintigste eeuw kenden. Daarnaast dwingen detailhandelsverkopen,
accijnzen en toegangsbelastingen de detailhandelaar tot onbetaalde
arbeid. Zij zijn verantwoordelijk voor het innen van belastingen en het
doorgeven daarvan aan de overheid. De hoge kosten voor het innen van
belastingen hebben ook een ander ongewenst gevolg, misschien niet
helemaal onopzettelijk door de machthebbers. Deze kosten kunnen door
grote bedrijven redelijkerwijs worden gedragen. Voor kleine werkgevers
daarentegen vormen ze een onevenredig zware last. Terwijl de grote
werkgevers de extra kosten makkelijk kunnen absorberen, dragen de
kleinere concurrenten veel meer van deze lasten.

Dwangarbeid dringt door in onze juridische en gerechtelijke structuur.
Veel vereiste gerechtelijke procedures zijn gebaseerd op gedwongen
getuigenissen. Aangezien het libertarisme stelt dat alle dwang -- in dit
geval dwangarbeid -- afgeschaft moet worden, behalve voor veroordeelde
misdadigers, betekent dit dat verplichte getuigenissen ook moeten
verdwijnen. Het is waar dat rechtbanken de afgelopen jaren de
bescherming van het Vijfde Amendement, dat stelt dat niemand gedwongen
mag worden om tegen zichzelf te getuigen en zo bewijs te leveren voor
zijn eigen veroordeling, levendig hebben gehouden. Echter, wetgevende
machten hebben deze bescherming aanzienlijk verzwakt door het aannemen
van immuniteitswetten. Hiermee wordt immuniteit tegen vervolging
aangeboden aan iemand die bereid is tegen zijn medeplichtigen te
getuigen. Daarbij wordt de getuige gedwongen dit aanbod te accepteren en
tegen zijn medeplichtigen te getuigen. Het afdwingen van getuigenissen,
om welke reden dan ook, is dwangarbeid. Dit is bovendien verwant aan
ontvoering, aangezien de getuige gedwongen wordt om op de hoorzitting of
het proces te verschijnen en daar de arbeid van getuigenis af te leggen.
Het probleem beperkt zich niet tot de recente immuniteitswetten; de kern
van het probleem is het volledig afschaffen van gedwongen getuigenissen.
Dit omvat ook het universieel dagvaarden van getuigen van een misdaad en
hen vervolgens dwingen om te getuigen. In het geval van getuigen is er
geen sprake van schuld aan een misdaad, dus het gebruik van dwang
tegenover hen -- een praktijk die tot nu toe niemand in twijfel heeft
getrokken -- is nog minder rechtvaardig dan het afdwingen van
getuigenissen van beschuldigde criminelen.

In feite zou de hele bevoegdheid om te dagvaarden moeten worden
afgeschaft. De dagvaardingsbevoegdheid dwingt immers iemand om bij een
proces aanwezig te zijn. Zelfs een beschuldigde misdadiger of
overtredende partij zou niet verplicht moeten worden om zijn eigen
rechtszaak bij te wonen, aangezien hij nog niet veroordeeld is. Volgens
het belangrijke libertarische principe van het Angelsaksische recht is
hij onschuldig totdat zijn schuld bewezen is. Daarom hebben rechtbanken
geen recht om de beklaagde te dwingen zijn proces bij te wonen. Vergeet
niet dat de enige uitzondering op het verbod op onvrijwillige
dienstbaarheid in het Dertiende Amendement luidt: `behalve als straf
voor een misdaad waarvoor de partij naar behoren is veroordeeld.' Een
beschuldigde partij is echter nog niet veroordeeld. Wat een rechtbank
zou moeten kunnen doen, is de gedaagde informeren dat hij berecht zal
worden en hem of zijn advocaat uitnodigen om aanwezig te zijn. Als de
gedaagde ervoor kiest om niet te verschijnen, zal de rechtszaak bij
verstek plaatsvinden. In dat geval krijgt de gedaagde natuurlijk niet de
beste presentatie van zijn zaak.

Zowel het Dertiende Amendement als het libertarische credo maken een
uitzondering voor de veroordeelde crimineel. Libertariërs geloven dat
een crimineel zijn rechten verliest naarmate hij de rechten van anderen
heeft geschonden. Hierdoor is het geoorloofd om een veroordeelde
crimineel op te sluiten en hem onder te brengen in onvrijwillige
dienstbaarheid. In de libertarische wereld zou het doel van opsluiting
en straf echter anders zijn. Er zou geen `officier van justitie' zijn
die namens een niet-bestaande `maatschappij' een zaak probeert te
berechten en de crimineel straft. In plaats daarvan zou de aanklager
altijd het individuele slachtoffer vertegenwoordigen. De straf zou dan
opgelegd worden ten gunste van dat slachtoffer, waarbij een cruciale
focus ligt op het dwingen van de misdadiger tot restitutie aan het
slachtoffer. Een voorbeeld van zo'n model was gebruikelijk in koloniaal
Amerika. In plaats van een man op te sluiten die bijvoorbeeld een boer
in het district had beroofd, werd de crimineel onder dwang uitgeleend
aan de boer. Hij werd in feite `tot slaaf gemaakt' voor een bepaalde
tijd om voor de boer te werken totdat zijn schuld was afgelost. Tijdens
de Middeleeuwen was restitutie aan het slachtoffer het dominante concept
van straf. Pas naarmate de staat machtiger werd, bemoeiden
overheidsinstanties -- koningen en baronnen -- zich steeds meer met het
compensatieproces. Ze confisqueerden steeds meer bezittingen van de
misdadiger, terwijl het ongelukkige slachtoffer verwaarloosd werd.
Naarmate de nadruk verschoof van restitutie naar straf voor abstracte
misdaden `begaan tegen de staat', werden de straffen die de overheid
oplegde steeds zwaarder.

Zoals professor Schafer opmerkt: `Naarmate de staat het strafsysteem
monopoliseerde, werden de rechten van de benadeelden langzaam
losgekoppeld van het strafrecht.' Of, zoals criminoloog William Tallack
rond de eeuwwisseling zei: 'Het was voornamelijk te wijten aan de
gewelddadige hebzucht van feodale baronnen en middeleeuwse kerkelijke
machten dat de rechten van de benadeelde partij geleidelijk werden
geschonden en uiteindelijk in grote mate werden toegeëigend door deze
autoriteiten. Zij namen zelfs een dubbele wraak op de overtreder door
zijn eigendommen aan zichzelf toe te eigenen in plaats van deze aan het
slachtoffer te geven. Vervolgens straften zij de dader met de kerker,
marteling, de brandstapel of de galg. Het oorspronkelijke slachtoffer
van het onrecht werd praktisch genegeerd.'7

Hoewel de libertariër geen bezwaar heeft tegen gevangenissen op zich,
heeft hij wel kritiek op verschillende praktijken in het huidige
gerechtelijke en strafrechtelijke systeem. Eén daarvan is de lange
gevangenisstraf die gedaagden krijgen opgelegd terwijl ze op hun proces
wachten. Het grondwettelijke recht op een `snel proces' is niet
willekeurig; het doel is om de duur van onvrijwillige detentie vóór een
veroordeling te minimaliseren. In feite is het, behalve als de crimineel
op heterdaad is betrapt en er dus een vermoeden van schuld bestaat,
nauwelijks te rechtvaardigen om iemand gevangen te nemen voordat hij is
veroordeeld, laat staan vóór zijn berechting. Zelfs wanneer iemand op
heterdaad wordt betrapt, is er een belangrijke hervorming nodig om het
systeem eerlijk te houden: de politie en andere autoriteiten moeten aan
dezelfde wetten worden onderworpen als iedereen. Zoals verderop wordt
besproken, betekent het dat als iedereen dezelfde strafwet moet volgen,
het vrijstellen van autoriteiten van die wet hen de mogelijkheid biedt
om continu agressie te plegen. De politieagent die een crimineel
aanhoudt en de gerechtelijke en strafrechtelijke autoriteiten die hem
opsluiten vóór zijn berechting en veroordeling -- zij zouden allemaal
onderworpen moeten zijn aan dezelfde universele wet. Als zij een fout
maken en de beklaagde blijkt onschuldig te zijn, moeten deze
autoriteiten dezelfde straffen ondergaan als iemand die onterecht een
onschuldige man ontvoert en opsluit. Immuniteit tijdens de uitoefening
van hun functie zou nooit een excuus mogen zijn, net zoals het dat niet
was voor Luitenant Calley voor de gruweldaden in My Lai tijdens de
oorlog in Vietnam.8

De toekenning van borgtocht is een poging om het probleem van opsluiting
vóór de rechtszaak te verlichten, maar deze aanpak schiet tekort. Het is
duidelijk dat de praktijk van borgtocht de armen discrimineert. Deze
discriminatie blijft bestaan, ook al heeft de opkomst van de
borgtochthandel veel meer mensen in staat gesteld om borgtocht te
krijgen. Het tegenargument dat de rechtbanken overbelast zijn met zaken
en daarom geen snel proces kunnen bieden, is natuurlijk geen
rechtvaardiging voor het systeem. Integendeel, deze ingebouwde
inefficiëntie pleit juist voor de afschaffing van overheidsrechtbanken.

Bovendien ligt de bepaling van de borgtocht in handen van de rechter,
die buitensporige en nauwelijks gecontroleerde macht heeft om mensen
vast te houden voordat ze zijn veroordeeld. Dit is vooral zorgwekkend
bij aanklachten van minachting van de rechtbank. Rechters hebben
namelijk bijna onbeperkte bevoegdheden om iemand de gevangenis in te
sturen, terwijl ze zelf fungeren als aanklager, rechter en jury. Ze
beschuldigen, `veroordelen' en straffen de dader, geheel los van de
gebruikelijke bewijs- en rechtsprocedures. Dit gaat in tegen het
fundamentele rechtsbeginsel dat stelt dat je geen rechter kunt zijn in
je eigen zaak.

Tot slot is er een belangrijk aspect van het rechtssysteem dat
onbegrijpelijk lang ongemoeid is gelaten, zelfs door libertariërs:
verplichte jurydienst. Er is weinig verschil in soort, maar een groot
verschil in graad tussen verplichte jurydienst en de dienstplicht. Beide
vormen zijn in feite een soort slavernij; ze verplichten individuen om
taken uit te voeren op bevel van de staat. Bovendien worden deze
verplichtingen meestal uitgevoerd tegen een vergoeding die lager is dan
het marktloon. Net zoals het tekort aan vrijwillige soldaten in het
leger voortkomt uit een loonschaal die ver onder het marktgemiddelde
ligt, zorgt het extreem lage loon voor jurydienst ervoor dat er, zelfs
als mensen zich vrijwillig zouden kunnen aanmelden, weinig
belangstelling zou zijn. Daarnaast worden juryleden niet alleen
gedwongen om zitting te nemen in jury's, maar soms worden ze ook
wekenlang achter gesloten deuren opgesloten en hebben ze geen toegang
tot de krant. Wat is dit anders dan gevangenisstraf en onvrijwillige
dienstbaarheid voor onschuldige burgers?

Men zal misschien aanvoeren dat jurydienst een belangrijke burgerfunctie
is die garant staat voor een eerlijk proces. Een gedaagde kan immers
niet rekenen op een rechtvaardige behandeling door de rechter, die deel
uitmaakt van het staatssysteem en daardoor partijdig kan zijn ten
opzichte van de aanklager. Dat is inderdaad waar, maar precies om deze
reden is het cruciaal dat jurydienst wordt uitgevoerd door mensen die
dit vrijwillig en met plezier doen. Zijn we vergeten dat vrije arbeid
gelukkiger en efficiënter is dan gedwongen arbeid? De afschaffing van
juryslavernij zou een essentieel onderdeel moeten zijn van elk libertair
platform. De rechters zijn niet gedwongen om hun functie te vervullen,
de advocaten van de tegenpartij ook niet, en juryleden zouden dat
eveneens niet moeten zijn.

Het is misschien geen toeval dat advocaten in de Verenigde Staten zijn
vrijgesteld van jurydienst. Omdat het vaak advocaten zijn die de wetten
formuleren, kunnen we hier duidelijk de invloed van klassewetgeving en
klasseprivileges zien.

\section{\texorpdfstring{\textbf{VERPLICHTE
VERBINTENIS}}{VERPLICHTE VERBINTENIS}}\label{verplichte-verbintenis}

Tot slot is er een belangrijk aspect van ons rechtssysteem dat al te
lang onopgemerkt blijft, zelfs door libertariërs: verplichte jurydienst.
Hoewel de invulling van jurydienst en dienstplicht vergelijkbaar lijkt,
is er een groot onderscheid in de ernst. Beide vormen zijn namelijk een
soort slavernij; ze dwingen individuen om taken uit te voeren in
opdracht van de staat. Bovendien zijn beide gebaseerd op een beloning
die vaak onder het marktloon ligt. Net zoals het tekort aan vrijwillige
soldaten in het leger deels te maken heeft met de lage beloning, zo
zorgt het schamele loon voor jurydienst ervoor dat er, zelfs als
aanmelden voor jury's mogelijk zou zijn, weinig belangstelling is.
Juryleden worden niet alleen verplicht om zitting te nemen, maar soms
worden ze ook wekenlang afgesloten van de buitenwereld en mogen ze geen
kranten lezen. Wat is dit anders dan gevangenisstraf en gedwongen arbeid
voor onschuldige burgers? Sommigen zullen betogen dat jurydienst een
cruciale burgerfunctie is die zorgt voor een eerlijk proces. Een
gedaagde kan immers niet rekenen op een objectieve behandeling door de
rechter, die deel uitmaakt van het staatssysteem en daardoor mogelijk
partijdig is tegenover de aanklager. Dat klopt, maar juist omdat deze
taak zo belangrijk is, is het essentieel dat deze wordt uitgevoerd door
mensen die dat vrijwillig en met enthousiasme doen. Zijn we vergeten dat
vrije arbeid gelukkiger en productiever is dan gedwongen arbeid? De
afschaffing van juryslavernij zou een standaardonderdeel moeten zijn van
elk libertair programma. De rechters zijn niet gedwongen om hun functie
te bekleden, de advocaten van de tegenpartij ook niet. Juryleden zouden
dat eveneens niet moeten zijn. Het is misschien geen toeval dat
advocaten in de Verenigde Staten zijn vrijgesteld van jurydienst.
Aangezien het vaak advocaten zijn die de wetten opstellen, zien we hier
duidelijk de invloed van klassewetgeving enPrivileges.

Een van de meest beschamende vormen van onvrijwillige dienstbaarheid in
onze samenleving is de wijdverspreide praktijk van gedwongen of
onvrijwillige opname van geesteszieken. In vroegere generaties werd deze
opsluiting van niet-criminelen openlijk uitgevoerd als een maatregel
tegen mensen met mentale problemen, om ze uit de maatschappij te
verwijderen. De praktijk van het twintigste-eeuwse liberalisme leek op
het eerste gezicht humaner, maar was in werkelijkheid veel
verraderlijker. Nu helpen artsen en psychiaters deze ongelukkigen op te
sluiten `voor hun eigen bestwil'. Deze humanitaire retoriek heeft geleid
tot een breder gebruik van deze praktijk. Het stelt ook ontevreden
familieleden in staat om hun dierbaren op te sluiten zonder zich
schuldig te voelen.

In het afgelopen decennium heeft de libertarische psychiater en
psychoanalyticus Dr.~Thomas S. Szasz een eenmanskruistocht tegen
gedwongen opname gevoerd. Wat eerst hopeloos leek, krijgt nu steeds meer
invloed in de psychiatrie. In talloze boeken en artikelen heeft
Dr.~Szasz een grondige en systematische aanval op deze praktijk
gepleegd. Hij benadrukt dat gedwongen opname een ingrijpende schending
van de medische ethiek is. In plaats van de patiënt te dienen, stelt de
arts zich hier ten dienste van anderen --- de familie of de staat --- en
treedt hij op tegen en tyranniseert hij de persoon die hij geacht wordt
te helpen. Bovendien verergeren en verlengen verplichte opname en
`therapie' vaak eerder de zogenaamde `geestelijke ziekten' dan dat ze
deze genezen. Szasz wijst er ook op dat gedwongen opnamen vaak worden
gebruikt om ongewenste familieleden op te sluiten en zo uit de weg te
ruimen, in plaats van echte hulp te bieden aan de patiënt.

De belangrijkste reden voor gedwongen opname is dat de patiënt
`gevaarlijk zou kunnen zijn voor zichzelf of anderen'. De eerste grote
fout in deze benadering is dat de politie of de wet ingrijpt, niet
wanneer er een openlijke agressieve daad wordt gepleegd, maar op basis
van iemands oordeel dat zo'n daad op een dag zou kunnen plaatsvinden.
Dit biedt echter een open deur voor onbeperkte tirannie. Iedereen kan
immers als een mogelijke misdadiger worden beschouwd, en daarom kan
iedereen op deze grondslag legitiem worden opgesloten. Niet vanwege een
gepleegde misdaad, maar omdat iemand denkt dat hij er mogelijk een zou
kunnen begaan. Dit soort gedachtegang rechtvaardigt niet alleen
opsluiting, maar zelfs permanente opsluiting van iedereen die verdacht
wordt. Het fundamentele libertarische credo stelt echter dat elk
individu beschikt over een vrije wil en vrije keuze. Niemand, ongeacht
hoe groot de kans is dat hij in de toekomst een misdaad zal begaan
volgens statistische of andere beoordelingen, is onvermijdelijk gedoemd
om dat te doen. Bovendien is het immoreel, en zelfs kwaadaardig, om
iemand op te sluiten die slechts een verdachte crimineel is, zonder dat
er sprake is van een openlijke en feitelijke misdaad.

Onlangs werd Dr.~Szasz gevraagd: `Vindt u niet dat de maatschappij het
recht en de plicht heeft om te zorgen voor individuen die 'gevaarlijk
voor zichzelf en anderen' zijn?' Szasz antwoordde overtuigend:

\begin{quote}
Ik denk dat het idee om mensen te `helpen' door ze op te sluiten en
vreselijke dingen met hen te doen, een religieus concept is. Het doet me
denken aan het idee om heksen te `redden' door ze te martelen en te
verbranden. Wat betreft `gevaar voor zichzelf' geloof ik, net als John
Stuart Mill, dat het lichaam en de ziel van een mens van hemzelf zijn en
niet van de staat. Bovendien heeft elk individu, zoals men zegt, het
`recht' om met zijn lichaam te doen wat hij wil, zolang hij niemand
anders schade berokkent of inbreuk maakt op de rechten van anderen.

Wat betreft `gevaar voor anderen' zouden de meeste psychiaters die met
gehospitaliseerde patiënten werken, toegeven dat dit pure fantasie is.
Er zijn zelfs statistische onderzoeken gedaan waaruit blijkt dat
psychiatrische patiënten veel gezagsgetrouwer zijn dan de gemiddelde
bevolking.
\end{quote}

En burgerrechtenadvocaat Bruce Ennis voegt hieraan toe:

\begin{quote}
We weten dat 85 procent van alle ex-gedetineerden in de toekomst opnieuw
misdaden zal plegen. Ook hebben gettobewoners en tienerjongens een veel
grotere kans om misdaden te begaan dan het gemiddelde lid van de
bevolking. Daarnaast blijkt uit recente onderzoeken dat psychiatrische
patiënten statistisch gezien minder gevaarlijk zijn dan de gemiddelde
man. Dus als we ons echt zorgen maken over gevaar, waarom sluiten we dan
niet eerst alle ex-gedetineerden op, daarna alle gettobewoners en
vervolgens alle tienerjongens? De vraag die Szasz stelt is: als iemand
geen wet heeft overtreden, welk recht heeft de maatschappij dan om hem
op te sluiten?
\end{quote}

Onvrijwillig opgenomen personen kunnen in twee groepen worden ingedeeld:
degenen die geen misdaad hebben gepleegd en degenen die dat wel hebben
gedaan. Voor de eerste groep pleit de libertariër onvoorwaardelijk voor
vrijlating. Maar hoe zit het met de tweede groep, de criminelen die door
ontoerekeningsvatbaarheid of andere redenen zogenaamd ontsnappen aan de
`wreedheid' van gevangenisstraf en in plaats daarvan medische zorg van
de staat ontvangen? Ook hier heeft Dr.~Szasz baanbrekend werk verricht
met een scherpe en vernietigende kritiek op het despotisme van het
liberale `humanisme'. Ten eerste is het absurd om te beweren dat opname
in een psychiatrisch ziekenhuis op de een of andere manier `humaner' is
dan een vergelijkbare opname in de gevangenis. Juist het despotisme van
de autoriteiten is waarschijnlijk veel erger, en de opgenomen persoon
heeft vaak veel minder mogelijkheden om zijn rechten te verdedigen. Want
zodra iemand als `geestesziek' gecertificeerd is, wordt hij
geclassificeerd als een `niet-mens' die door niemand meer serieus
genomen hoeft te worden. Zoals Dr.~Szasz met een knipoog zei: `In een
psychiatrisch ziekenhuis zou iedereen gek worden!'

Maar we moeten vraagtekens zetten bij het idee om iemand te onttrekken
aan de regels van de objectieve wet. Het is waarschijnlijker dat dit
schadelijker is dan nuttig voor de mensen die op deze manier worden
geselecteerd. Neem bijvoorbeeld twee mannen, A en B, die een
gelijkwaardige overval plegen, waarbij de gebruikelijke straf voor deze
misdaad vijf jaar gevangenisstraf is. Stel dat B deze straf ontloopt
doordat hij geestesziek wordt verklaard en wordt overgeplaatst naar een
psychiatrische inrichting. De liberaal focust op de mogelijkheid dat B
na twee jaar door de staatspsychiater kan worden vrijgelaten omdat hij
`genezen' of `gerehabiliteerd' wordt verklaard. Maar wat als de
psychiater B nooit genezen acht, of dat dit pas na een lange tijd
gebeurt? Dan kan B, voor de simpele misdaad van diefstal, geconfronteerd
worden met de verschrikking van levenslange opsluiting in een
psychiatrisch ziekenhuis. Het `liberale' concept van onbepaalde straf --
iemand niet veroordelen voor zijn objectieve misdaad, maar op basis van
de beoordeling van de staat over zijn geestelijke toestand -- is
tirannie en ontmenselijking in zijn ernstigste vorm. Bovendien moedigt
deze tirannie de gevangene aan tot bedrieglijk gedrag. Hij probeert de
staatspsychiater, die hij terecht als zijn vijand beschouwt, voor de gek
te houden door te doen alsof hij `genezen' is, zodat hij uit de
opsluiting kan komen. Dit proces `therapie' of `rehabilitatie' te
noemen, is een wrede bespotting van die termen. Het is veel principiëler
en menselijker om elke gevangene te behandelen volgens het objectieve
strafrecht.

\bookmarksetup{startatroot}

\chapter{Persoonlijke vrijheid}\label{persoonlijke-vrijheid}

\section{VRIJHEID VAN MENINGSUITING}\label{vrijheid-van-meningsuiting}

Er zijn uiteraard veel kwesties rondom persoonlijke vrijheid die niet
onder de term `onvrijwillige dienstbaarheid' vallen. Vrijheid van
meningsuiting en persvrijheid zijn langere tijd gekoesterd door mensen
die zichzelf als `burgerlijke libertariërs' beschouwen. `Burgerlijk'
betekent hier dat economische vrijheid en het recht op privébezit niet
worden meegenomen. Echter, we hebben al gezien dat `vrijheid van
meningsuiting' niet als een absoluut recht kan worden gehandhaafd,
tenzij het gekoppeld is aan de algemene eigendomsrechten van het
individu. Deze rechten omvatten nadrukkelijk ook het eigendomsrecht over
het eigen lichaam. Een man die in een overvol theater `brand!' roept,
heeft dus niet het recht om dat te doen, omdat hij de contractuele
eigendomsrechten van de theatereigenaar en de bezoekers van de
voorstelling schendt.

Afgezien van inbreuken op eigendom zal de vrijheid van meningsuiting
echter door elke libertariër onvermijdelijk tot het uiterste worden
verdedigd. De vrijheid om welke uiting dan ook te zeggen, te drukken en
te verkopen, wordt beschouwd als een absoluut recht, ongeacht de
context. Op dit gebied hebben burgerlijke libertariërs over het algemeen
een goede staat van dienst. In de rechtspraak was wijlen rechter Hugo
Black bijzonder opmerkelijk in het verdedigen van de vrijheid van
meningsuiting tegen overheidsbeperkingen op basis van het Eerste
Amendement van de Grondwet. Toch zijn er gebieden waarin zelfs de meest
fervente burgerlijke libertariërs helaas vaag blijven. Wat te denken van
`aanzetten tot rellen'? Hierbij kan de spreker schuldig worden bevonden
aan een misdrijf omdat hij een menigte heeft opgehitst die vervolgens in
opstand komt en verschillende misdaden tegen personen en goederen
pleegt. In onze ogen kan `opruiing' alleen als een misdaad worden
beschouwd als we ieders vrijheid van wil en keuze ontkennen. Laten we
aannemen dat A tegen B en C zegt: `Gaan jullie maar lekker rellen!' Als
we dan denken dat B en C op de een of andere manier hulpeloos gedwongen
zijn om deze onrechtmatige daad te plegen, ontkennen we hun eigen
keuzes. De libertariër, die gelooft in de vrijheid van de wil, moet
volhouden dat hoewel het immoreel of ongelukkig kan zijn dat A voor een
oproer pleit, dit binnen het domein van pleiten valt en niet onderworpen
zou moeten zijn aan wettelijke straffen. Natuurlijk wordt A zelf een
relschopper als hij deelneemt aan de opstand en kan hij dan gestraft
worden. Bovendien, als A een baas is in een criminele onderneming en
zijn handlangers als onderdeel van de misdaad opdracht geeft: `Gaan
jullie samen een bank beroven', dan wordt A volgens de wet van
medeplichtigheid natuurlijk een deelnemer of zelfs leider in die
criminele onderneming.

Als pleiten nooit een misdaad zou mogen zijn, dan zou `samenzwering om
te pleiten' dat ook niet moeten zijn. In tegenstelling tot de
ongelukkige ontwikkeling van het samenzweringsrecht zou `samenzweren'
(oftewel overeenkomen) om iets te doen nooit illegaal moeten zijn, noch
meer dan de daad zelf. Hoe kun je `samenzwering' anders definiëren dan
als een overeenkomst tussen twee of meer mensen om iets te doen dat jou,
als definieerder, niet aanstaat?1

Een ander lastig terrein is het recht op smaad en laster. Over het
algemeen wordt het als legitiem beschouwd om de vrijheid van
meningsuiting te beperken wanneer die uitingen de reputatie van iemand
anders valselijk of kwaadwillig schaden. Kort gezegd beroept de wet op
smaad en laster zich op een `eigendomsrecht' dat iemand heeft op zijn
eigen reputatie. Echter, iemands `reputatie' kan niet als `eigendom'
worden beschouwd, omdat het alleen een afspiegeling is van de
subjectieve gevoelens en houdingen van anderen. Aangezien niemand ooit
echt `eigenaar' kan zijn van de gevoelens en meningen van een ander,
betekent dit dat niemand letterlijk een eigendomsrecht kan hebben op
zijn eigen `reputatie'. De reputatie van een persoon fluctueert
voortdurend, afhankelijk van de houdingen en meningen van de omringende
maatschappij. Daarom kan een toespraak die iemand aanvvalt niet worden
gezien als een inbreuk op zijn eigendomsrecht en zou er dus geen
beperking of wettelijke straf op moeten staan.

Het is natuurlijk immoreel om iemand vals te beschuldigen. Toch zijn het
morele en wettelijke aspecten voor de libertariër twee heel
verschillende zaken.

Bovendien, vanuit een pragmatisch perspectief, zouden mensen veel minder
geneigd zijn om beschuldigingen te geloven zonder volledige documentatie
als er geen wetten tegen smaad of laster bestonden. Wanneer iemand
tegenwoordig beschuldigd wordt van een fout of een misdrijf, is de
algemene reactie vaak om de beschuldiging te geloven. De gedachte is:
`Als de beschuldiging vals was, waarom klaagt die persoon dan niet aan
voor smaad?' Deze smaadwetgeving benadeelt natuurlijk de armen. Iemand
met beperkte financiële middelen is veel minder geneigd om een kostbare
smaadzaak aan te spannen dan iemand met een goed gevulde portemonnee.
Bovendien kunnen rijke mensen de smaadwetten gebruiken als een wapen
tegen de armen. Zij kunnen legitieme beschuldigingen en uitingen
aanpakken onder de dreiging om hun armere tegenstanders aan te klagen
wegens smaad. Paradoxaal genoeg heeft iemand met minder middelen dus
meer kans om te lijden onder smaad en om zijn eigen meningsuiting
beperkt te zien in het huidige systeem dan in een wereld zonder wetten
tegen smaad of laster.

Gelukkig zijn de wetten tegen smaad de laatste jaren geleidelijk
versoepeld. Hierdoor kunnen mensen nu felle en scherpe kritiek uiten op
overheidsfunctionarissen en publieke figuren, zonder bang te hoeven zijn
voor kostbare juridische stappen of rechtsvervolging.

Een andere actie die volledig vrij van beperkingen zou moeten zijn, is
de boycot. Bij een boycot maken een of meer mensen gebruik van hun recht
op vrije meningsuiting om, om welke reden dan ook -- belangrijk of
onbelangrijk -- anderen aan te sporen om het product van iemand anders
niet meer te kopen. Als bijvoorbeeld verschillende mensen een campagne
organiseren om consumenten aan te zetten tot het stoppen met de aankoop
van XYZ-bier, is dit opnieuw een zuiver pleidooi. Bovendien is het een
volkomen legitieme daad om dit bier niet te kopen. Een succesvolle
boycot kan vervelend zijn voor de producenten van XYZ-bier, maar ook dit
valt strikt binnen het domein van de vrijheid van meningsuiting en het
recht op privé-eigendom. De producenten van XYZ-bier nemen een risico
met de vrije keuzes van consumenten. Consumenten hebben het recht om te
luisteren naar en zich te laten beïnvloeden door wie ze maar willen.
Toch hebben onze arbeidswetten ingegrepen op het recht van vakbonden om
boycots tegen bedrijven te organiseren. Ook is het illegaal om onder
onze bankwetten geruchten te verspreiden over de insolvabiliteit van een
bank. Dit is een duidelijk geval van de overheid die speciale privileges
verleent aan banken door de vrijheid van meningsuiting tegen hun gebruik
te verbieden.

Een bijzonder lastige kwestie betreft picketing en demonstraties.
Vrijheid van meningsuiting omvat natuurlijk ook het recht op
vergadering, oftewel de vrijheid om samen te komen en je uit te drukken
met anderen. De situatie wordt echter ingewikkelder wanneer het gebruik
van de straat in het spel komt. Het is duidelijk dat piketacties
onwettig zijn wanneer ze, zoals vaak het geval is, worden gebruikt om de
toegang tot een privégebouw of fabriek te blokkeren, of wanneer de
piketten dreigen geweld te gebruiken tegen mensen die de lijn
oversteken. Ook sit-ins vormen een duidelijke inbreuk op privé-eigendom.
Zelfs `vreedzame piketacties' zijn niet vanzelfsprekend legitiem, omdat
ze deel uitmaken van een groter probleem: wie beslist er over het
gebruik van de straat? Dit probleem komt voort uit het feit dat de
straten bijna overal eigendom zijn van de (lokale) overheid. Aangezien
de overheid geen particuliere eigenaar is, beschikt ze niet over
duidelijke criteria voor het toewijzen van het gebruik van haar straten.
Hierdoor zijn alle beslissingen die ze neemt in feite willekeurig.

Stel dat de Vrienden van Wisteria willen demonstreren en paraderen op
een openbare straat. De politie verbiedt de demonstratie met het
argument dat dit de straten zal verstoppen en het verkeer zal verstoren.
Burgerlijke libertariërs zullen hiertegen protesteren en beweren dat het
`recht op vrije meningsuiting' van de Wisteria-demonstranten onterecht
wordt ingeperkt. Maar ook de politie kan een legitiem punt hebben: de
straten kunnen inderdaad verstopt raken, en het is de
verantwoordelijkheid van de overheid om de verkeersstroom te waarborgen.
Hoe maak je dan de juiste keuze? Welke beslissing de overheid ook neemt,
er zal altijd een groep belastingbetalers zijn die de gevolgen zal
ondervinden. Als de overheid besluit de demonstratie toe te staan,
worden automobilisten en voetgangers benadeeld. Besluit ze de
demonstratie te verbieden, dan lijden de Vrienden van Wisteria verlies.
In beide gevallen leidt het simpelweg feit dat de overheid beslissingen
moet nemen tot onvermijdelijke conflicten over wie recht heeft op de
middelen van de overheid en wie niet.

Het is het universele feit van overheidseigendom en controle over de
straten dat dit probleem onoplosbaar maakt en de werkelijke oplossing
verhult. Het punt is namelijk dat degene die een bron bezit, bepaalt hoe
die bron wordt gebruikt. De eigenaar van een drukpers beslist wat er op
die pers wordt afgedrukt. Evenzo zal de eigenaar van de straten bepalen
hoe het gebruik van die straten wordt toegewezen. Stel je voor: als de
straten privébezit zijn en de Vrienden van Wisteria vragen om de Fifth
Avenue te gebruiken voor een demonstratie, dan is het aan de eigenaar
van die straat om te beslissen of hij deze verhuurt voor demonstraties
of vrijhoudt voor doorgaand verkeer. In een puur libertarische wereld,
waar alle straten privébezit zijn, zullen de verschillende eigenaren
altijd kunnen beslissen of ze hun straat verhuren voor demonstraties,
aan wie ze dit doen en welke prijs ze vragen. Het zou dan duidelijk
worden dat het hier niet gaat om `vrijheid van meningsuiting' of
`vrijheid van vergadering', maar om een kwestie van eigendomsrechten.
Het gaat om het recht van een groep om aan te bieden een straat te huren
en het recht van de straateigenaar om dat aanbod te accepteren of af te
wijzen.

\section{Vrijheid van Radio en
Televisie}\label{vrijheid-van-radio-en-televisie}

De vrijheid van radio en televisie is een essentieel onderdeel van de
moderne samenleving. Het stelt mensen in staat om informatie te delen en
zich vrijelijk uit te drukken. Dit is cruciaal voor een democratische
samenleving, omdat het bijdraagt aan een geïnformeerd publiek. In veel
landen zijn er wettelijke kaders die de werking van de media reguleren.
Deze wetten zijn vaak bedoeld om zowel de vrijheid van meningsuiting te
waarborgen als om misbruik te voorkomen. Desondanks zijn er regelmatig
discussies over de balans tussen deze twee elementen. Hoe ver kan
vrijheid reiken zonder dat dit ten koste gaat van andere rechten?
Bovendien is de rol van technologie niet te onderschatten. Met de
opkomst van het internet zijn er tal van nieuwe platforms ontstaan. Deze
platforms stellen mensen in staat om gemakkelijk hun stem te laten
horen. Dit leidt echter ook tot vragen over verantwoordelijkheid en
controle. In een wereld waar informatie snel verspreid kan worden, is
het belangrijk om kritisch te blijven. Niet alles wat op radio of
televisie - of online - wordt gepresenteerd is waar. Kritische
media-consumenten zijn essentieel voor het waarborgen van een gezonde,
vrije pers. Samenvattend is de vrijheid van radio en televisie van groot
belang voor onze samenleving. Het biedt de ruimte voor diverse meningen
en ideeën, maar brengt ook verantwoordelijkheden met zich mee. Het is
aan ons allen om deze vrijheid te bescherm en tegelijkertijd voor de
waarheid te zorgen.

Er is één belangrijk gebied van het Amerikaanse leven waar effectieve
vrijheid van meningsuiting of persvrijheid niet bestaat, en dat is het
domein van radio en televisie. De federale overheid nationaliseerde in
de cruciale Radio Act van 1927 de ether. Daardoor werd de overheid
eigenaar van alle radio- en televisiekanalen. Ze ging ervan uit dat ze
naar eigen inzicht licenties zou verlenen voor het gebruik van deze
kanalen aan verschillende particuliere stations. Enerzijds hoeven de
stations, omdat ze de licenties gratis verkrijgen, niet te betalen voor
het gebruik van de schaarse etherfrequenties, zoals dat in een vrije
markt wel zou gebeuren. Deze stations ontvangen daardoor een
aanzienlijke subsidie die ze graag willen behouden. Anderzijds claimt de
federale overheid als licentiegever het recht en de bevoegdheid om de
stations nauwlettend en voortdurend te reguleren. Boven elk station
hangt de dreiging van niet-verlenging of zelfs opschorting van de
licentie. Het idee van vrijheid van meningsuiting op radio en televisie
is hierdoor niet meer dan een schijnvertoning. Elk station is ernstig
beperkt en gedwongen zijn programmering aan te passen aan de eisen van
de Federal Communications Commission. Zo moet elk station zorgen voor
`evenwichtige' programmering, een bepaald aantal `openbare
dienst'-mededelingen uitzenden, evenveel tijd geven aan elke politieke
kandidaat voor hetzelfde ambt, en censureren van `controversiële'
teksten in de muziek die ze draaien, enzovoort. Jarenlang was het voor
geen enkel station toegestaan om een redactionele mening te uiten. Nu
moet elke mening worden gecompenseerd door `verantwoordelijke'
redactionele tegenargumenten.

Omdat elk station en elke omroep altijd over zijn schouder naar de FCC
moet kijken, is vrije meningsuiting in de omroep een schijnvertoning. Is
het dan verwonderlijk dat de mening van de televisie, als die al wordt
geuit over controversiële onderwerpen, meestal in het voordeel van de
`gevestigde orde' is?

Het publiek heeft deze situatie geaccepteerd, simpelweg omdat ze al
bestaat sinds het ontstaan van grootschalige commerciële radio. Maar wat
zouden we vinden als alle kranten een licentie moesten aanvragen,
waarbij de Federale Perscommissie de licenties vernieuwt? En wat als
kranten hun licentie zouden verliezen wanneer ze een `oneerlijke'
redactionele mening verkondigen, of wanneer ze niet voldoende aandacht
besteden aan openbare aankondigingen? Zou dit niet een onaanvaardbare,
om niet te zeggen ongrondwettelijke, aantasting zijn van het recht op
een vrije pers? Stel je voor dat alle boekuitgevers ook een licentie
zouden moeten hebben, en dat hun licentie niet verlengd zou worden als
hun boekenlijst niet voldoet aan de eisen van een federale
boekencommissie. Wat wij als onacceptabel en totalitair zouden
beschouwen voor de pers en de boekuitgevers, wordt nu als
vanzelfsprekend gezien in het meest populaire medium voor expressie en
educatie: radio en televisie. De principes zijn in beide gevallen immers
precies hetzelfde.

Hier zien we een van de fatale fouten in het idee van `democratisch
socialisme': de veronderstelling dat de overheid alle grondstoffen en
productiemiddelen moet bezitten en tegelijkertijd de vrijheid van
meningsuiting en persvrijheid voor alle burgers kan waarborgen. Een
abstracte grondwet die `persvrijheid' garandeert, is zinloos in een
socialistische samenleving. Wanneer de overheid eigenaar is van al het
krantenpapier, de persen en dergelijke, moet zij beslissen hoe het
krantenpapier en de andere middelen worden toegewezen en wat erop wordt
gedrukt. Net zoals de overheid als beheerder van de straat moet bepalen
hoe deze wordt gebruikt, zal een socialistische overheid moeten
beslissen hoe het krantenpapier en alle andere middelen die van belang
zijn voor de meningsuiting en de pers worden verdeeld: montagehallen,
machines, vrachtwagens, enzovoort. Elke regering kan proclameren dat ze
zich inzet voor de persvrijheid, maar in de praktijk kan zij al het
krantenpapier toewijzen aan haar verdedigers en aanhangers. Een vrije
pers wordt zo een schijnvertoning. Bovendien, waarom zou een
socialistische regering een aanzienlijk deel van haar schaarse middelen
toekennen aan degenen die zich tegen haar verzetten? Het probleem van
echte persvrijheid wordt daardoor onoplosbaar.

De oplossing voor radio en televisie? Heel eenvoudig: behandel deze
media op dezelfde manier als de pers en boekuitgevers. Zowel voor
libertariërs als voor voorstanders van de Amerikaanse grondwet geldt dat
de overheid zich volledig moet terugtrekken uit elke rol of inmenging in
de media van meningsuiting. De federale overheid zou de ether moeten
denationaliseren en de individuele kanalen aan privé-eigenaren moeten
geven of verkopen. Wanneer privé-stations echt eigenaar zijn van hun
kanalen, zullen ze vrij en onafhankelijk opereren. Ze kunnen elk
programma uitzenden dat ze willen produceren of waarvan ze denken dat
hun luisteraars het willen horen. Bovendien kunnen ze zich op elke
denkbare manier uiten, zonder angst voor represailles van de overheid.
Ook kunnen ze de ether verkopen of verhuren aan wie ze maar willen,
waardoor de gebruikers van de zenders niet langer kunstmatig
gesubsidieerd worden.

Als televisiekanalen vrij, privébezit en onafhankelijk worden, zullen de
grote netwerken niet langer druk kunnen uitoefenen op de FCC om
concurrentie van betaaltelevisie te blokkeren. Het is juist omdat de FCC
betaaltelevisie heeft verboden, dat deze geen kans heeft gekregen.
`Gratis televisie' is natuurlijk niet echt gratis; de programma's worden
betaald door adverteerders en de consument betaalt de advertentiekosten
indirect via de prijs van de producten die hij koopt. Je kunt je
afvragen wat het voor de consument uitmaakt of hij de reclamekosten
indirect of direct betaalt voor elk programma dat hij aanschaft. Het
verschil is dat het niet dezelfde consumenten zijn voor dezelfde
producten. De televisie-reclamemaker is altijd geïnteresseerd in (a) het
bereiken van een zo groot mogelijke kijkersgroep en (b) het aantrekken
van die specifieke kijkers die het meest ontvankelijk zijn voor zijn
boodschap. Daarom zijn alle programma's afgestemd op de kleinste gemene
deler van het publiek, en vooral op de kijkers die het meest gevoelig
zijn voor de boodschap; dat wil zeggen, kijkers die geen kranten of
tijdschriften lezen, zodat er geen overlap is met de advertenties die ze
daar tegenkomen. Hierdoor zijn gratis televisieprogramma's vaak
fantasieloos, saai en uniform. Betaaltelevisie zou betekenen dat elk
programma zijn eigen publiek zou zoeken, wat zou leiden tot de
ontwikkeling van veel gespecialiseerde markten voor specifieke
doelgroepen - vergelijkbaar met de lucratieve gespecialiseerde markten
die zijn ontstaan in de tijdschriften- en boekenwereld. De kwaliteit van
de programma's zou verbetering vertonen en het aanbod zou veel diverser
zijn. In feite moet de dreiging van potentiële concurrentie van
betaaltelevisie zo groot zijn dat de netwerken jarenlang hebben gelobbyd
om deze te onderdrukken. Maar in een echt vrije markt kunnen en zullen
beide vormen van televisie, evenals kabeltelevisie en andere nog
onbekende vormen, de concurrentie aangaan.

Een veelgehoord argument tegen privébezit van tv-zenders is dat deze
zenders `schaars' zijn. Volgens deze redenering moeten ze eigendom zijn
van de overheid en door haar worden verdeeld. Voor een econoom is dit
echter geen steekhoudend argument. Alle middelen zijn schaars. Alles wat
een prijs heeft op de markt, heeft die prijs omdat het schaars is. We
betalen voor brood, schoenen en jurken omdat ze allemaal beperkt
beschikbaar zijn. Als ze niet schaars waren, maar overvloedig aanwezig
zoals lucht, zouden ze gratis zijn en zou niemand zich druk hoeven te
maken over hun productie of toewijzing. In de persindustrie is
krantenpapier schaars, evenals papier, drukmachines en vrachtwagens. Hoe
schaarser deze middelen zijn, hoe hoger de prijs die ze opleveren, en
vice versa. Bovendien, pragmatisch gezien, zijn er veel meer
televisiekanalen beschikbaar dan er momenteel in gebruik zijn. De vroege
beslissing van de FCC om stations in de VHF- in plaats van de UHF-band
te dwingen, heeft geleid tot een veel grotere schaarste aan kanalen dan
nodig was.

Een ander veelgehoord bezwaar tegen privé-eigendom in de omroepmedia is
dat privé-stations elkaars uitzendingen zouden verstoren. Dit zou het
vrijwel onmogelijk maken om programma's goed te horen of te zien. Maar
dit argument voor nationalisatie van de ether is net zo absurd als de
bewering dat omdat mensen met hun auto over andermans land rijden, dit
betekent dat alle auto's of land genationaliseerd moeten worden. In
beide gevallen is het probleem dat rechtbanken eigendomstitels
zorgvuldig moeten afbakenen. Zo wordt duidelijk wat een inbreuk op
andermans eigendom is, zodat deze kan worden vervolgd. Voor land is dit
proces helder. Het idee is dat rechtbanken een soortgelijk proces kunnen
toepassen op andere gebieden, of het nu gaat om radiogolven, water of
olie. Bij luchtgolven is het noodzakelijk om de technologische eenheid
te identificeren: de plaats van uitzending, de afstand van de golven en
de breedte van een vrij kanaal. Vervolgens moeten eigendomsrechten aan
deze specifieke technologische eenheid worden toegewezen. Stel dat
radiostation WXYZ het eigendomsrecht krijgt om uit te zenden op 1500
kHz, plus of min een bepaalde breedte, voor 200 mijl rond Detroit. Dan
kan elk station dat in dat gebied programma's uitzendt op die golflengte
worden aangesproken voor inmenging in eigendomsrechten. Als rechtbanken
hun taak serieus nemen om eigendomsrechten af te bakenen en te
verdedigen, is er geen reden om te verwachten dat er in dit gebied meer
inbreuken op deze rechten plaatsvinden dan elders.

De meeste mensen denken dat dit de reden is waarom de ether
genationaliseerd werd. Voor de Radio Act van 1927 zouden stations elkaar
hebben gestoord, waardoor er chaos ontstond. De federale overheid zou
uiteindelijk gedwongen zijn om in te grijpen om orde op zaken te stellen
en om een radio-industrie mogelijk te maken. Maar dit is een historische
mythe, geen feit. De werkelijke geschiedenis is precies het
tegenovergestelde. Toen er interferentie op hetzelfde kanaal optrad,
daagde de benadeelde partij de veroorzakers voor de rechter. De
rechtbanken gingen aan de slag om orde in de chaos te scheppen door met
succes de gewoonterechttheorie van eigendomsrechten toe te passen, wat
sterk lijkt op de libertarische theorie. Kortom, de rechtbanken begonnen
eigendomsrechten in de ether toe te kennen aan de gebruikers. Nadat de
federale overheid zich bewust werd van de mogelijke uitbreiding van
privé-eigendom, haastte zij zich om de radiogolven te nationaliseren,
met vermeende chaos als excuus.

Om een completer beeld te geven: in de eerste jaren van de eeuw was
radio bijna uitsluitend een communicatiemiddel voor schepen, zowel
tussen schepen onderling als tussen schip en wal. Het ministerie van
Marine was geïnteresseerd in het reguleren van radio om de veiligheid op
zee te waarborgen. De eerste federale regelgeving, een wet uit 1912,
bepaalde dat elk radiostation een vergunning moest hebben die werd
uitgegeven door de minister van Handel. Deze wet gaf echter geen
bevoegdheden om licenties te reguleren of niet te verlengen. Toen de
publieke omroep aan het begin van de jaren 1920 begon, probeerde
minister van Handel Herbert Hoover de stations te reguleren. Uitspraken
van rechtbanken in 1923 en 1926 verwierpen echter de bevoegdheid van de
overheid om licenties te reguleren, niet te vernieuwen of zelfs te
beslissen op welke golflengte de stations moesten opereren. Ongeveer
tegelijkertijd waren de rechtbanken bezig met het verduidelijken van het
concept van particuliere eigendomsrechten in de ether, met name in de
zaak \textbf{Tribune Co.~v. Oak Leaves Broadcasting Station} (Circuit
Court, Cook County, Illinois, 1926). In deze zaak oordeelde de rechtbank
dat de exploitant van een bestaand station een eigendomsrecht had
verworven door eerder gebruik. Dit recht was voldoende om een nieuw
station te verbieden een radiofrequentie te gebruiken die interferentie
zou veroorzaken met de signalen van het bestaande station. Zo werd er
orde geschapen in de chaos door middel van de toewijzing van
eigendomsrechten. Maar deze ontwikkeling wilde de overheid koste wat
kost voorkomen.

De Zenith-beslissing uit 1926, waarin de overheid de bevoegdheid werd
ontzegd om licenties te reguleren of te vernieuwen, leidde ertoe dat het
ministerie van Handel gedwongen werd licenties uit te geven aan elk
station dat een aanvraag indiende. Dit resulteerde in een enorme groei
van de omroepindustrie. In de negen maanden na de beslissing werden er
meer dan tweehonderd nieuwe stations opgericht. Als reactie hierop nam
het Congres in juli 1926 haastig een noodwet aan om eigendomsrechten op
radiofrequenties te beperken. Deze wet stelde dat alle licenties slechts
90 dagen geldig mochten zijn. In februari 1927 volgde de wet tot
oprichting van de Federal Radio Commission, die de radiogolven
nationaliseerde en overeenkomsten tot stand bracht die vergelijkbaar
zijn met de huidige bevoegdheden van de FCC. Het doel van goed
geïnformeerde politici was niet het voorkomen van chaos, maar het
vermijden van privébezit in de ether als oplossing voor die chaos.
Rechtshistoricus H.P. Warner legt uit: `De wetgevers en degenen die
verantwoordelijk waren voor de communicatie vreesden ernstig dat
effectieve overheidsregulering permanent kon worden voorkomen door het
verwerven van eigendomsrechten op licenties of toegangsmiddelen, met als
gevolg dat franchises ter waarde van miljoenen dollars voor altijd
zouden worden gevestigd.' Het nettoresultaat was echter dat waardevolle
franchises werden opgezet, maar op een monopolistische manier, dankzij
de vrijgevigheid van de Federal Radio Commission en later de FCC, in
plaats van door middel van concurrerende homesteading.

Onder de vele directe inbreuken op de vrijheid van meningsuiting door de
vergunningsbevoegdheid van de FRC en de FCC, zijn er twee voorbeelden
voldoende. Een daarvan vond plaats in 1931, toen de FRC de verlenging
van de licentie weigerde aan meneer Baker, die een radiostation in Iowa
beheerde. In haar weigering stelde de Commissie:

\begin{quote}
Deze Commissie neemt het niet op voor de Medische Verenigingen of andere
partijen die de heer Baker niet mogen. Hun vermeende fouten kunnen soms
van openbaar belang zijn en op een juiste manier via de ether aan het
publiek worden gepresenteerd. Dit verslag laat echter zien dat de heer
Baker dit niet op een hoogstaande manier doet. Hij deelt voortdurend en
ongepast zijn persoonlijke hobby's, zijn ideeën over de behandeling van
kanker, en zijn voorkeuren en afkeuren van bepaalde mensen en zaken via
de ether. Dat hij de luisteraars hiermee overspoelt, is zeker geen goed
gebruik van een zendvergunning. Veel van zijn uitspraken zijn vulgair en
zelfs onfatsoenlijk. Ze zijn bepaald niet verheffend of onderhoudend.
\end{quote}

Kunnen we ons voorstellen hoe verontwaardigd we zouden zijn als de
federale overheid op soortgelijke gronden een krant of een boekuitgever
failliet zou laten gaan?

Onlangs dreigde de FCC met het niet verlengen van de licentie van
radiostation KTRG in Honolulu, een belangrijk radiostation in Hawaï.
KTRG zond ongeveer twee jaar lang enkele uren per dag libertarische
programma's uit. Uiteindelijk besloot de FCC eind 1970 om lange
hoorzittingen te starten over de mogelijke niet-verlenging van de
licentie. De dreigende kosten dwongen de eigenaren uiteindelijk om het
station voorgoed te sluiten.6

\section{PORNOGRAFIE}\label{pornografie}

In de hedendaagse samenleving is pornografie een veelbesproken
onderwerp. Het is overal aanwezig, zowel online als offline, en het
heeft invloed op onze kijk op seksualiteit en relaties. Veel mensen
beschouwen pornografie als een onschuldige vorm van entertainment,
terwijl anderen het als schadelijk beschouwen. Enerzijds kan pornografie
seksuele opvoeding bieden en een manier vormen om fantasieën te
verkennen. Anderzijds kunnen de beelden en boodschappen die in de
pornografie worden verspreid onrealistische verwachtingen scheppen over
seks en intimiteit. Dit kan leiden tot problemen in persoonlijke
relaties en onvrede met het eigen lichaam. Bovendien zijn er zorgen over
de impact van pornografie op jongeren. De toegankelijkheid van seksueel
expliciet materiaal maakt het gemakkelijker voor minderjarigen om in
contact te komen met pornografie voordat ze er emotioneel klaar voor
zijn. Dit kan hun ontwikkeling en begrip van gezonde relaties
beïnvloeden. Het is belangrijk dat we als samenleving openhartig en
eerlijk over dit onderwerp praten. Discussies over pornografie, de
bijbehorende risico's en de impact op individuen en relaties zijn
noodzakelijk om bewustzijn te creëren en jongeren te begeleiden naar een
gezonde benadering van seksualiteit.

Voor libertariërs zijn de argumenten die conservatieven en liberalen
uitwisselen over wetten die pornografie verbieden volledig irrelevant.
Het conservatieve standpunt stelt vaak dat pornografie vernederend en
immoreel is, en dat het daarom verboden zou moeten worden. Liberalen
daarentegen vinden dat seks goed en gezond is, en geloven dat
pornografie vooral positieve effecten heeft. Zij zijn van mening dat
gewelddadige beelden - zoals op televisie, in films of in stripboeken -
juist verboden zouden moeten worden. Geen van beide partijen gaat in op
het cruciale punt: de vraag of pornografie goede, slechte of neutrale
effecten heeft, is, hoewel het een interessant probleem is, niet
relevant voor de discussie over een verbod. Volgens libertariërs is het
niet de taak van de wet -- het gebruik van geweld -- om iemands
opvattingen over moraliteit af te dwingen. De wet kan, zelfs als het
praktisch mogelijk zou zijn, niemand goed, eerbiedig, moreel of oprecht
maken. Die beslissing is aan ieder individu zelf. De enige taak van
legaal geweld is om mensen te beschermen tegen geweld en hen te
verdedigen tegen gewelddadige inbraken in hun persoonlijke leven of
eigendom. Maar als de overheid pornografie wil verbieden, wordt zij zelf
de ware overtreder. Zij schendt dan de eigendomsrechten van mensen die
pornografisch materiaal produceren, verkopen, kopen of bezitten.

We stellen geen wetten op om mensen rechtschapen te maken. We nemen geen
wetten aan om mensen te dwingen vriendelijk te zijn tegen hun buren of
om niet te schreeuwen naar de buschauffeur. We schrijven ook geen wetten
voor om mensen te dwingen eerlijk te zijn tegen hun geliefden. Evenmin
vaardigen we wetten uit die mensen verplichten om een bepaalde
hoeveelheid vitamines per dag te eten. Het is niet de taak van de
overheid of enige andere wettelijke instantie om wetten te maken tegen
de vrijwillige productie of verkoop van pornografie. Of pornografie
goed, slecht of neutraal is, zou voor de wettelijke autoriteiten
irrelevant moeten zijn.

Hetzelfde geldt voor de liberale bezorgdheid over `de pornografie van
geweld'. Of het kijken naar geweld op televisie leidt tot daadwerkelijke
misdaden, zou niet onder de bevoegdheid van de staat moeten vallen.
Gewelddadige films verbieden omdat ze misschien iemand aanzetten tot
criminaliteit, is een ontkenning van de menselijke vrije wil. Het
ontneemt ook degenen die geen misdaden zullen plegen het recht om zo'n
film te bekijken. Bovendien is het nog minder te rechtvaardigen om
gewelddadige films om deze reden te verbieden, dan het zou zijn om alle
tienerjongeren van kleur op te sluiten, omdat ze een hogere neiging tot
criminaliteit zouden vertonen dan de rest van de bevolking.

Het zou ook duidelijk moeten zijn dat een verbod op pornografie een
inbreuk is op het eigendomsrecht. Dit betreft het recht om te
produceren, verkopen, kopen en bezitten. Conservatieven die pleiten voor
een verbod op pornografie lijken niet te beseffen dat ze daarmee het
eigendomsrecht schenden waar ze voor staan. Bovendien is het een
schending van de persvrijheid, die, zoals eerder genoemd, onderdeel is
van het bredere recht op privébezit.

Soms lijkt het wel alsof het ideale beeld van veel conservatieven, net
als dat van veel liberalen, is om iedereen in een kooi te stoppen en te
dwingen te handelen volgens hun morele normen. Natuurlijk zouden de
kooien er verschillend uitzien, maar ze zouden even verstikkend zijn. De
conservatief zou illegale seks, drugs, gokken en verdorvenheid verbieden
en iedereen dwingen zich te gedragen volgens zijn visie op moreel en
religieus gedrag. De liberaal daarentegen zou geweldfilms, ongepaste
reclame, voetbal en rassendiscriminatie verbieden. In het extreme geval
zou hij mensen zelfs in een `Skinner box' plaatsen, beheerd door een
zogenaamd welwillende liberale dictator. Het effect blijft echter
hetzelfde: iedereen wordt gereduceerd tot een ondermenselijk niveau en
ontnomen van het meest waardevolle facet van hun menselijkheid: de
vrijheid om te kiezen.

De ironie is dat door mensen te dwingen `moreel' te zijn, dat wil zeggen
moreel te handelen, de conservatieve of liberale bewaarders in feite de
mogelijkheid ontnemen om daadwerkelijk moreel te zijn. Het begrip
`moraliteit' heeft geen betekenis als de morele daad niet vrij gekozen
is. Neem als voorbeeld iemand die een toegewijde moslim is en die ernaar
streeft dat zo veel mogelijk mensen drie keer per dag naar Mekka buigen.
Voor hem is dat wellicht de hoogste morele daad. Maar als hij dwang
gebruikt om iedereen te dwingen naar Mekka te buigen, ontnemt hij hen de
mogelijkheid om moreel te handelen --- om zelf te kiezen of ze naar
Mekka buigen. Dwang schaart iemand onder het juk van keuzes en onthoudt
hem zo de vrijheid om moreel te kiezen.

De libertariër wil, in tegenstelling tot veel conservatieven en
liberalen, de mens niet in een kooi stoppen. Wat hij iedereen wil
bieden, is vrijheid: de vrijheid om moreel of immoreel te handelen,
afhankelijk van de keuze van elk individu.

\section{Sex-wetten}\label{sex-wetten}

De laatste jaren zijn liberalen gelukkig tot de conclusie gekomen dat
`elke handeling tussen twee (of meer) instemmende volwassenen' legaal
zou moeten zijn. Het is jammer dat ze dit principe nog niet hebben
uitgebreid van seks naar handel en ruil, want als ze dat deden, zouden
ze dichter bij de volledige libertariërs komen. De libertariër richt
zich namelijk op het legaliseren van alle onderlinge relaties tussen
instemmende volwassenen. Daarnaast zijn liberalen ook begonnen te
pleiten voor de afschaffing van `slachtofferloze misdaden'. Dit zou
geweldig zijn, mits `slachtoffers' nauwkeuriger gedefinieerd worden als
slachtoffers van agressief geweld.

Aangezien seks een uniek en privé-aspect van het leven is, is het
bijzonder ongeoorloofd dat overheden seksueel gedrag willen reguleren en
wettelijk willen vastleggen. Gewelddadige handelingen, zoals
verkrachting, moeten uiteraard worden gecategoriseerd als misdaden, net
als elke andere vorm van geweld tegen personen.

Vreemd genoeg worden beschuldigde verkrachters door de autoriteiten veel
milder behandeld dan verdachten van andere vormen van lichamelijk
geweld. Ondertussen zijn vrijwillige seksuele activiteiten vaak illegaal
gemaakt en worden ze door de staat vervolgd. In veel gevallen wordt het
slachtoffer van verkrachting door de wetshandhavers zelfs als de
schuldige partij gezien. Deze houding komt bijna nooit voor bij
slachtoffers van andere misdaden. Dit illustreert een ontoelaatbare
seksuele dubbele moraal. Zoals de National Board van de American Civil
Liberties Union in maart 1977 verklaarde:

\begin{quote}
Slachtoffers van seksueel geweld verdienen dezelfde behandeling als
slachtoffers van andere misdrijven. Helaas worden zij vaak met scepsis
benaderd en vaak misbruikt door politie en hulpverleners. Deze
behandeling varieert van officieel ongeloof en ongevoeligheid tot wrede
en indringende onderzoeken naar de levensstijl en motivatie van het
slachtoffer. Dergelijke veronachtzaming van verantwoordelijkheden door
instellingen die bedoeld zijn om slachtoffers te helpen en te
beschermen, kan het trauma dat het slachtoffer heeft opgelopen alleen
maar verergeren.
\end{quote}

De dubbele standaard die de overheid hanteert, kan worden aangepakt door
verkrachting niet langer als een aparte categorie van juridische en
gerechtelijke behandeling te beschouwen. In plaats daarvan zou het
ondergebracht moeten worden bij de algemene wetgeving over lichamelijke
mishandeling. Welke normen ook worden gehanteerd voor de instructies van
rechters aan de jury of voor de toelaatbaarheid van bewijs, deze zouden
in alle gevallen op dezelfde manier moeten worden toegepast.

Als arbeid en mensen in het algemeen vrij moeten zijn, dan moet er ook
vrijheid zijn voor prostitutie. Prostitutie is een vrijwillige verkoop
van arbeidsdiensten. De overheid heeft niet het recht om deze verkoop te
verbieden of te beperken. Veel van de negatieve aspecten van de
tippelzone zijn te wijten aan het verbod op bordelen. Wanneer langdurige
prostitutiehuizen, geleid door madammen die vertrouwen opbouwen bij hun
klanten, zouden concurreren met bordelen, zou dit leiden tot een betere
service en een sterkere `merknaam'. Het verbieden van bordelen heeft
prostitutie echter gedwongen naar een `zwarte markt'. Dit heeft een
nachtleven gecreëerd met alle bijbehorende gevaren en kwaliteitsverlies.
Onlangs heeft de politie in New York City geprobeerd om prostitutie hard
aan te pakken, met het argument dat de handel niet langer
`slachtoffervrij' is omdat veel prostituees misdaden tegen hun klanten
plegen. Maar als we beroepen verbieden die criminaliteit aansteken, zou
dat ook betekenen dat we alle bars moeten verbieden, omdat daar vaak
vechtpartijen plaatsvinden. Het probleem is niet dat we vrijwillige en
legale activiteiten moeten verbieden, maar dat de politie moet zorgen
dat echte misdaden niet plaatsvinden. Het is belangrijk op te merken dat
pleiten voor vrijheid voor prostitutie voor een libertariër niet
betekent dat hij prostitutie zelf steunt. Stel je voor dat een bijzonder
puriteinse overheid alle cosmetica zou verbieden. Een libertariër zou
dan pleiten voor de legalisering van cosmetica, zonder dat dit betekent
dat hij voorstander is van het gebruik ervan of er zelfs tegen is.
Sterker nog, afhankelijk van zijn persoonlijke ethiek of esthetiek zou
hij zelfs kunnen ageren tegen het gebruik van cosmetica nadat het
gelegaliseerd is. Zijn doel is altijd om te overtuigen, niet om te
dwingen.

Als seks vrij zou moeten zijn, moet geboortebeperking dat ook zijn. Het
is echter helaas kenmerkend voor onze samenleving dat geboortebeperking
net legaal is geworden, of dat mensen - in dit geval liberalen - pleiten
voor verplichte geboortebeperking. Het is waar dat als mijn buurvrouw
een baby krijgt, dat invloed op mij kan hebben, goed of slecht. Maar
bijna alles wat iemand doet, kan gevolgen hebben voor anderen. Voor de
libertariër is dit echter nauwelijks een rechtvaardiging voor geweld.
Geweld mag alleen worden gebruikt om geweld zelf te bestrijden of te
belemmeren. Er is geen persoonlijker recht en geen waardevollere
vrijheid dan het recht van een vrouw om zelf te beslissen of ze een kind
wil krijgen. Het is extreem totalitair als een regering dat recht
probeert te ontnemen. Bovendien, als een gezin meer kinderen heeft dan
het zich kan veroorloven, zal dat gezin daar zelf de grootste last van
ondervinden. Dit leidt vaak tot de conclusie dat de wens om de
levensstandaard hoog te houden, zal resulteren in een vrijwillige
vermindering van het aantal geboorten door de gezinnen zelf.

Dit brengt ons bij het complexere geval van abortus. Voor de libertariër
kan de `katholieke' zaak tegen abortus, ook al wordt deze uiteindelijk
als ongeldig afgewezen, niet zomaar van tafel worden geveegd. De kern
van deze zaak - die in theologische zin niet echt `katholiek' is - is
dat abortus een menselijk leven vernietigt. Dit wordt daarom als moord
beschouwd en kan niet genegeerd worden. Sterker nog, als abortus
werkelijk moord is, dan kan de katholiek, of wie dan ook die dit
standpunt deelt, niet simpelweg zijn schouders ophalen en zeggen dat
`katholieke' standpunten niet aan niet-katholieken moeten worden
opgelegd. Moord is geen kwestie van religieuze voorkeur; geen enkele
sekte kan of mag, in naam van de `vrijheid van godsdienst', moord
rechtvaardigen met het argument dat haar geloof dit vereist. De essentie
van de discussie is dan: moet abortus als moord worden beschouwd?

De meeste discussies over dit onderwerp verzanden vaak in details over
wanneer menselijk leven begint en of de foetus als levend kan worden
beschouwd. Deze vraagstukken zijn echter eigenlijk irrelevant voor de
legaliteit -- en niet noodzakelijkerwijs de moraliteit -- van abortus.
Een katholieke anti-abortusactivist zal bijvoorbeeld stellen dat alles
wat hij voor de foetus wil, de rechten van elk mens omvat, zoals het
recht om niet vermoord te worden. Maar er speelt meer, en dat is de
cruciale overweging. Als we de foetus dezelfde rechten toekennen als een
mens, moeten we ons afvragen: welk mens heeft het recht om tegen de wil
van een ander als parasiet in diens lichaam te blijven? Dit raakt de
kern van de zaak: het absolute recht van elk individu, en dus van elke
vrouw, op het eigendom van haar eigen lichaam. Met een abortus zorgt de
moeder ervoor dat een ongewenste entiteit uit haar lichaam wordt
verwijderd. De dood van de foetus weerlegt niet dat geen enkel wezen het
recht heeft om ongevraagd als parasiet in of op iemands lichaam te
leven.

Het veelgehoorde argument dat de moeder de foetus oorspronkelijk in haar
lichaam wilde of er op zijn minst verantwoordelijk voor was, gaat hier
niet op. Zelfs als de moeder het kind aanvankelijk wilde, heeft zij als
eigenaar van haar eigen lichaam het recht om van gedachten te veranderen
en het kind te verwijderen.

Als de staat vrijwillige seksuele activiteit niet mag onderdrukken, dan
kan hij ook niet discrimineren ten gunste of ten nadele van een van
beide seksen. Regelgeving rond `positieve discriminatie' is een voor de
hand liggende manier om discriminatie van mannen of andere groepen af te
dwingen, bijvoorbeeld op het gebied van werk, toelating of waar dit
impliciete quotasysteem ook wordt toegepast. Maar `beschermende'
arbeidswetten voor vrouwen doen alsof ze hen bevoordelen, terwijl ze in
werkelijkheid vrouwen discrimineren. Ze verbieden hen om tijdens
bepaalde uren of in specifieke beroepen te werken. Hierdoor worden
vrouwen door de wet verhinderd om hun individuele keuzevrijheid uit te
oefenen. Zo kunnen ze zelf beslissen of ze deze beroepen willen
uitoefenen of in deze zogenaamd belastende uren willen werken. Op deze
manier belemmert de overheid vrouwen om vrij te concurreren met mannen
op deze gebieden.

Al met al is het platform van de Libertarische Partij uit 1978 krachtig
en helder in het uiteenzetten van het libertarische standpunt over
discriminatie, of het nu gaat om seks of andere vormen door de overheid.
`Geen enkel individueel recht mag worden ontzegd of beperkt door de
wetten van de Verenigde Staten of van een staat of plaats op basis van
geslacht, ras, huidskleur, geloof, leeftijd, nationale afkomst of
seksuele voorkeur.'

\section{Afluisteren}\label{afluisteren}

Het veelgehoorde argument dat een moeder de foetus oorspronkelijk in
haar lichaam wilde of er tenminste verantwoordelijk voor was, gaat hier
niet op. Zelfs als de moeder het kind aanvankelijk wilde, heeft zij als
eigenaar van haar eigen lichaam het recht om van gedachten te veranderen
en het kind te verwijderen. Als de staat vrijwillige seksuele activiteit
niet mag onderdrukken, kan hij ook niet discrimineren ten gunste of ten
nadele van welk geslacht dan ook. Regelgeving over `positieve
discriminatie' is een duidelijke manier om discriminatie van mannen of
andere groepen te bevorderen, bijvoorbeeld op het gebied van werk,
toelating of waar dit impliciete quotasysteem ook van toepassing is.
Maar `beschermende' arbeidswetten voor vrouwen wekken de indruk dat ze
hen bevoordelen, terwijl ze in werkelijkheid vrouwen benadelen. Ze
verbieden hen namelijk om tijdens bepaalde uren of in specifieke
beroepen te werken. Hierdoor kunnen vrouwen niet vrijuit kiezen of ze
deze werkzaamheden willen verrichten of in deze zogenaamd belastende
uren willen werken. Zo belemmert de overheid hen om vrij te concurreren
met mannen in deze sectoren. Kortom, het platform van de Libertarische
Partij uit 1978 is krachtig en helder in het formuleren van het
libertarische standpunt over discriminatie, of het nu gaat om seks of
andere vormen door de overheid: `Geen enkel individueel recht mag worden
ontzegd of beperkt door de wetten van de Verenigde Staten of van een
staat of plaats op basis van geslacht, ras, huidskleur, geloof,
leeftijd, nationale afkomst of seksuele voorkeur.'

Afluisteren is een verachtelijke schending van de privacy en het
eigendomsrecht, en dit zou als invasieve handeling natuurlijk verboden
moeten zijn. Slechts weinigen staan achter het afluisteren van
privégesprekken. De controverse ontstaat bij degenen die beweren dat de
politie telefoongesprekken moet kunnen aftappen van verdachten in
criminele activiteiten. Hoe anders zouden criminelen kunnen worden
aangepakt?

Ten eerste is afluisteren vanuit pragmatisch oogpunt zelden effectief
bij eenmalige misdrijven zoals bankovervallen. Het wordt doorgaans
toegepast in situaties waar een `business' regelmatig en op voortdurende
basis opereert, zoals bij drugshandel en gokken. Deze activiteiten zijn
daardoor kwetsbaar voor spionage en afluisteren. Ten tweede handhaven we
onze stelling dat het al strafbaar is om in te breken op het eigendom
van iemand die nog niet voor een misdrijf is veroordeeld. Het kan waar
zijn dat als de overheid een spionagekorps van tien miljoen mensen zou
inzetten om de telefoonlijnen van de hele bevolking te bespioneren en af
te luisteren, de totale particuliere misdaad zou afnemen. Dit zou
vergelijkbaar zijn met het opsluiten van alle bewoners van getto's of
tienerjongens. Maar wat zou dat betekenen in vergelijking met de enorme
misdaad die op deze manier, legaal en zonder schaamte, door de overheid
zelf zou worden gepleegd?

Er is één concessie die we misschien kunnen doen aan het
politieargument, maar het is twijfelachtig of de politie deze concessie
zou waarderen. Het zou bijvoorbeeld gerechtvaardigd kunnen zijn om het
eigendom van een dief aan te vallen, omdat diegene in veel grotere mate
het eigendom van anderen heeft geschaad. Stel dat de politie besluit dat
John Jones een juwelendief is. Ze tapten zijn telefoon en gebruiken dit
bewijs om Jones te veroordelen voor de misdaad. We zouden kunnen stellen
dat dit afluisteren legitiem is en ongestraft moet blijven. Maar, op
voorwaarde dat als Jones geen dief blijkt te zijn, de politie en de
rechters die het afluisteren hebben goedgekeurd, zelf als criminelen
moeten worden aangemerkt en naar de gevangenis gaan voor hun misdaad van
onterecht afluisteren. Deze hervorming zou twee positieve gevolgen
hebben. Ten eerste zou geen enkele politieagent of rechter afluisteren
tenzij hij er absoluut zeker van is dat het doelwit inderdaad een
crimineel is. Ten tweede zouden politie en rechters zich eindelijk
voegen bij alle anderen en gelijkelijk onderworpen zijn aan de regels
van het strafrecht. Gelijkheid in vrijheid vereist immers dat de wet
voor iedereen geldt. Daarom zou elke inbreuk op het eigendom van een
niet-crimineel door wie dan ook verboden moeten zijn, ongeacht wie de
daad verrichtte. Een politieagent die een verkeerde inschatting maakt en
daardoor geweld gebruikt tegen een niet-crimineel, moet net zo schuldig
worden geacht als een `particuliere' afluisteraar.

\section{Gokken}\label{gokken}

Afluisteren is een verachtelijke schending van de privacy en van het
eigendomsrecht. Daarom zou het als invasieve handeling absoluut verboden
moeten worden. Slechts weinigen staan achter het afluisteren van
privégesprekken. De controverse ontstaat bij degenen die beweren dat de
politie de telefoonlijnen van verdachten in criminele activiteiten moet
kunnen aftappen. Hoe zouden criminelen anders kunnen worden aangepakt?
Ten eerste is afluisteren vanuit pragmatisch oogpunt zelden effectief
bij eenmalige misdrijven, zoals bankovervallen. Het wordt over het
algemeen toegepast in situaties waarin een `business' regelmatig en op
voortdurende basis opereert, zoals bij drugshandel of gokken. Dit soort
activiteiten is daardoor kwetsbaar voor spionage en afluisteren. Ten
tweede blijven we bij onze stelling dat het strafbaar is om in te breken
op het eigendom van iemand die nog niet voor een misdrijf is
veroordeeld. Het kan waar zijn dat, als de overheid een spionagekorps
van tien miljoen mensen zou inzetten om de telefoonlijnen van de hele
bevolking te bespioneren, de totale particuliere misdaad zou afnemen.
Dit zou vergelijkbaar zijn met het opsluiten van alle bewoners van
getto's of tienerjongens. Maar wat zou het betekenen in vergelijking met
de enorme misdaad die op deze manier, legaal en zonder schaamte, door de
overheid zelf gepleegd zou worden? Er is één concessie die we misschien
kunnen doen aan het politieargument, maar het is twijfelachtig of de
politie hier blij mee zou zijn. Het zou bijvoorbeeld gerechtvaardigd
kunnen zijn om het eigendom van een dief aan te vallen, omdat diegene in
veel grotere mate het eigendom van anderen heeft aangetast. Stel dat de
politie besluit dat John Jones een juwelendief is. Ze tapten zijn
telefoon en gebruiken dit bewijs om hem te veroordelen voor de misdaad.
We zouden kunnen stellen dat dit afluisteren legitiem is en ongestraft
moet blijven, maar alleen op voorwaarde dat als Jones geen dief blijkt
te zijn, de politie en de rechters die het afluisteren hebben
goedgekeurd zelf als criminelen moeten worden aangemerkt en naar de
gevangenis gaan voor hun onterecht afluisteren. Deze hervorming zou twee
positieve gevolgen hebben. Ten eerste zou geen enkele politieagent of
rechter afluisteren tenzij hij er absoluut zeker van is dat het doelwit
een crimineel is. Ten tweede zouden politie en rechters zich eindelijk
voegen bij iedereen en gelijkelijk onderworpen zijn aan de regels van
het strafrecht. Gelijkheid in vrijheid vereist immers dat de wet voor
iedereen geldt. Daarom zou elke inbreuk op het eigendom van een
niet-crimineel door wie dan ook verboden moeten zijn, ongeacht wie de
daad verrichtte. Een politieagent die een verkeerde inschatting maakt en
daardoor geweld gebruikt tegen een niet-crimineel, moet net zo schuldig
worden geacht als een particulier die afluistert.

Er zijn maar weinig wetten die zo absurd en onrechtvaardig zijn als de
wetten tegen gokken. Ten eerste is het, in de breedste zin van het
woord, duidelijk dat deze wet niet handhaafbaar is. Als elke keer dat
Jim en Jack een stille weddenschap afsloten op een voetbalwedstrijd, een
verkiezing of op bijna iets anders, dit illegaal zou zijn, dan zouden er
miljoenen mensen nodig zijn om zo'n wet te kunnen handhaven. Ze zouden
iedereen moeten bespioneren en elke weddenschap moeten controleren.
Bovendien zou er een enorme spionagekracht nodig zijn om de spionnen
zelf in de gaten te houden, om te voorkomen dat ze omgekocht worden.
Conservatieven reageren vaak op zulke argumenten --- die ook gebruikt
worden tegen wetten die seksuele praktijken, pornografie, drugs,
enzovoorts verbieden --- door te zeggen dat het verbod op moord ook niet
volledig afdwingbaar is. Maar dit is geen valide reden om die wet af te
schaffen. Dit argument negeert echter een cruciaal punt: de massa van
het publiek maakt instinctief een libertarisch onderscheid en
verafschuwt en veroordeelt moord, waardoor het verbod breed uitvoerbaar
wordt. Aan de andere kant is het publiek niet zo overtuigd van de
criminaliteit van gokken. Ze blijven het dus gewoon doen en de wet wordt
--- terecht --- onuitvoerbaar.

Omdat de wetten tegen stille weddenschappen duidelijk niet handhaafbaar
zijn, besluiten de autoriteiten zich te richten op bepaalde zichtbare
vormen van gokken en deze activiteiten te beperken, zoals roulette,
bookmakers en `nummer'-weddenschappen. Dit zijn gebieden waar gokken een
redelijk gereguleerde activiteit is. Maar dan komen we voor een
merkwaardig en absoluut onhoudbaar soort ethisch oordeel te staan:
roulette, paardenweddenschappen, en dergelijke worden op de een of
andere manier als moreel slecht gezien en moeten worden aangepakt met de
massale macht van de politie, terwijl stilletjes wedden moreel legitiem
is en geen problemen hoeft te ondervinden.

In de staat New York heeft zich in de loop der jaren een bijzondere vorm
van onzinnigheid ontwikkeld. Tot voor kort waren alle vormen van
paardenweddenschappen illegaal, behalve die op de renbanen zelf. Het is
onbegrijpelijk waarom wedden op paarden op de Aqueduct of Belmont
renbaan moreel en legitiem zou zijn, terwijl het wedden op dezelfde race
bij een vriend die als bookmaker fungeert zondig zou zijn en de wet in
diskrediet zou brengen. Tenzij, natuurlijk, we het doel van de wet
beschouwen: gokkers dwingen om de kas van de renbanen te spekken.
Onlangs is er een nieuwe ontwikkeling ontstaan. De stad New York is zelf
de paardenweddenschappen ingegaan. Weddenschappen afsluiten in winkels
die eigendom zijn van de stad is prima, terwijl weddenschappen bij
concurrerende particuliere bookmakers nog steeds verboden zijn. Het is
duidelijk dat het systeem bedoeld is om eerst de renbanen een speciaal
voorrecht te geven, gevolgd door de wedkantoren van de stad zelf.
Verschillende staten beginnen ook hun groeiende uitgaven te financieren
met loterijen, die zo een dekmantel van moraliteit en respectabiliteit
krijgen.

Een veelgebruikt argument voor het verbieden van gokken is dat de arme
arbeider, als hij de kans krijgt, zijn wekelijkse loon ondoordacht zal
uitgeven en zo zijn gezin in de problemen zal brengen. Behalve het feit
dat hij zijn loon nu ook aan vriendschappelijke weddenschappen kan
uitgeven, is dit paternalistische en dictatoriale argument merkwaardig.
Het bewijst namelijk te veel: als we gokken moeten verbieden omdat
mensen hun middelen mogelijk teveel zouden uitgeven, waarom zouden we
dan niet ook andere massaconsumptieartikelen verbieden? Immers, als een
arbeider vastberaden is zijn salaris te verkwisten, heeft hij talloze
mogelijkheden. Hij kan ondoordacht te veel uitgeven aan een televisie,
een hifiset, drank, honkbalspullen en nog veel meer. De logica om iemand
te verbieden te gokken, uit `bezorgdheid' voor zijn eigen welzijn of dat
van zijn gezin, leidt rechtstreeks naar een totalitaire situatie. In
zo'n situatie vertelt de overheid de man precies wat hij moet doen, hoe
hij zijn geld moet uitgeven, hoeveel vitamines hij moet nemen en dwingt
ze hem te gehoorzamen aan de dictaten van de staat.

\section{Verdovende middelen en andere
drugs}\label{verdovende-middelen-en-andere-drugs}

Omdat de wetten tegen stille weddenschappen duidelijk niet handhaafbaar
zijn, besluiten de autoriteiten zich te concentreren op bepaalde
zichtbare vormen van gokken en deze activiteiten te beperken, zoals
roulette, bookmakers en `nummer'-weddenschappen. Dit zijn gebieden waar
gokken redelijk gereguleerd is. Maar dan komen we voor een merkwaardig
en absoluut onhoudbaar soort ethisch oordeel te staan: roulette en
paardenweddenschappen worden op de een of andere manier als moreel
verkeerd gezien en moeten met de volle kracht van de politie worden
aangepakt, terwijl stil wedden moreel legitiem is en niet lastiggevallen
hoeft te worden. In de staat New York heeft zich in de loop der jaren
een bijzondere vorm van onzinnigheid ontwikkeld. Tot voor kort waren
alle vormen van paardenweddenschappen illegaal, behalve die op de
renbanen zelf. Het is onbegrijpelijk waarom wedden op paarden op de
Aqueduct- of Belmont-renbaan moreel acceptabel zou zijn, terwijl wedden
op dezelfde race bij een vriend die als bookmaker fungeert als zondig
wordt beschouwd en de wet zou ondermijnen. Tenzij, natuurlijk, we het
doel van de wet beschouwen: gokkers dwingen om de kas van de renbanen te
spekken. Onlangs is er een nieuwe ontwikkeling ontstaan. De stad New
York zelf is de paardenweddenschappen ingegaan. Weddenschappen afsluiten
in winkels die eigendom zijn van de stad is prima, maar weddenschappen
bij concurrerende particuliere bookmakers blijven verboden. Het lijkt
erop dat het systeem bedoeld is om eerst de renbanen een speciaal
voorrecht te geven en daarna de wedkantoren van de stad zelf.
Verschillende staten beginnen ook hun groeiende uitgaven te financieren
met loterijen, die zo een dekmantel van moraliteit en respectabiliteit
krijgen. Een vaak gebruikt argument voor het verbieden van gokken is dat
de arme arbeider, als hij de kans krijgt, ondoordacht zijn wekelijkse
loon zal uitgeven en zo zijn gezin in de problemen zal brengen. Behalve
dat hij zijn loon nu ook aan vriendschappelijke weddenschappen kan
uitgeven, is dit paternalistische en dictatoriale argument merkwaardig.
Het zegt namelijk te veel: als we gokken moeten verbieden omdat mensen
hun middelen mogelijk te veel zouden uitgeven, waarom zouden we dan niet
ook andere massaconsumptieartikelen verbieden? Immers, als een arbeider
vastbesloten is zijn salaris te verkwisten, heeft hij talloze
mogelijkheden. Hij kan ondoordacht te veel uitgeven aan een tv, een
hifiset, drank, honkbalspullen en meer. De logica om iemand te verbieden
te gokken, uit `bezorgdheid' voor zijn eigen welzijn of dat van zijn
gezin, leidt rechtstreeks naar een totalitaire situatie. In zo'n
situatie vertelt de overheid de man precies wat hij moet doen, hoe hij
zijn geld moet uitgeven, hoeveel vitamines hij moet nemen en dwingt ze
hem te gehoorzamen aan de dictaten van de staat.

De argumenten voor het verbieden van welk product of welke activiteit
dan ook zijn in wezen hetzelfde als het tweeledige argument dat we
hebben gezien ter rechtvaardiging van de gedwongen opname van
psychiatrische patiënten: het zal de betrokken persoon schaden, of het
zal ertoe leiden dat die persoon misdaden tegen anderen pleegt. Het is
merkwaardig dat de algemene - en terechte - afschuw van drugs de meeste
mensen heeft doen overgaan tot een irrationeel enthousiasme voor het
verbieden ervan. De tegenargumenten voor het verbieden van verdovende en
hallucinogene middelen zijn veel zwakker dan die tegen de drooglegging,
een experiment dat hopelijk door het gruwelijke tijdperk van de jaren
twintig voor altijd in diskrediet is gebracht. Hoewel verdovende
middelen ongetwijfeld schadelijker zijn dan alcohol, kan alcohol ook
schadelijk zijn en het verbieden van iets omdat het voor de gebruiker
schadelijk kan zijn, leidt ons onherroepelijk naar een totalitaire
samenleving. In zo'n samenleving mogen mensen geen snoep eten en moeten
ze yoghurt consumeren `voor hun eigen bestwil'. Wat betreft het
dwingendere argument over schade aan anderen: alcohol leidt veel vaker
tot misdaden, verkeersongelukken, enzovoort, dan verdovende middelen,
die de gebruiker doorgaans buitengewoon vredelievend en passief maken.
Natuurlijk bestaat er een sterk verband tussen verslaving en misdaad,
maar dit verband ondermijnt elk argument voor verbod. Misdaden worden
gepleegd door verslaafden die tot diefstal gedreven worden door de hoge
prijs van drugs, die juist het gevolg is van het verbod! Als drugs
legaal zouden zijn, zou het aanbod enorm toenemen, zouden de hoge kosten
van de zwarte markt en het smeergeld voor de politie verdwijnen, en zou
de prijs laag genoeg zijn om de meeste door verslaafden veroorzaakte
criminaliteit te elimineren.

Dit is natuurlijk geen pleidooi voor het verbieden van alcohol.
Nogmaals, iets verbieden omdat het mogelijk tot misdaad leidt, is een
onacceptabele en ingrijpende aanval op de rechten van individuen en hun
eigendom. Deze benadering zou in feite veel meer rechtvaardiging bieden
voor de onmiddellijke opsluiting van alle tienerjongens. Alleen het
openlijk plegen van misdaden zou illegaal moeten zijn. De manier om
misdaden die onder invloed van alcohol worden gepleegd aan te pakken, is
door strenger op te treden tegen de misdaden zelf, niet door alcohol te
verbieden. Dit zou bovendien als positief neveneffect hebben dat ook
misdaden die niet onder invloed van alcohol worden gepleegd,
verminderen.

Paternalisme op dit gebied komt niet alleen van rechts. Het is
opmerkelijk dat liberalen, die over het algemeen voorstander zijn van de
legalisering van marihuana en soms zelfs van heroïne, lijken te
verlangen naar een verbod op sigaretten. Dit doen ze met het argument
dat het roken van sigaretten vaak kanker veroorzaakt. Liberalen zijn er
al in geslaagd om de federale controle over televisie te gebruiken om
sigarettenreclame op dat medium te verbieden. Hiermee hebben ze echter
een flinke klap uitgedeeld aan de vrijheid van meningsuiting, iets waar
zij zogenaamd zo'n waarde aan hechten.

Nogmaals: iedereen heeft het recht om zelf keuzes te maken. Propageer
tegen sigaretten zoveel je wilt, maar laat het individu in vrijheid zijn
leven leiden. Anders kunnen we net zo goed allerlei kankerverwekkende
stoffen verbieden, zoals te strakke schoenen, slecht passende
kunstgebitten, overmatige blootstelling aan de zon, en de overmatige
consumptie van ijs, eieren en boter, die allemaal kunnen leiden tot
hartaandoeningen. En als zulke verboden niet handhaafbaar blijken te
zijn, dan is het logisch om mensen in kooien te stoppen, zodat ze de
juiste hoeveelheid zon, het juiste dieet en goed passende schoenen
krijgen, enzovoort.

\section{POLITIECORRUPTIE}\label{politiecorruptie}

Politiecorruptie is een probleem dat in veel landen voorkomt, en het
heeft uiteenlopende gevolgen voor de samenleving. Wanneer politieagenten
hun macht misbruiken, kan dit leiden tot een verlies van vertrouwen bij
het publiek. Mensen moeten kunnen rekenen op de politie om hen te
beschermen en te dienen, maar als agenten zich schuldig maken aan
corruptie, wordt dit vertrouwen ondermijnd. Corruptie kan verschillende
vormen aannemen. Denk hierbij aan omkoping, waar politieagenten geld
aannemen om misdadigers te ontzien, of het negeren van de wet. Dit soort
gedrag kan ernstige gevolgen hebben. Het leidt niet alleen tot een
oneerlijke rechtshandhaving, maar ook tot een verhoogde criminaliteit.
Wanneer de politie haar verantwoordelijkheden niet serieus neemt, voelen
mensen zich onveilig in hun eigen buurt. Een ander aspect van
politiecorruptie is de cultuur binnen de politie zelf. In sommige
gevallen is er een `code van stilte', waarbij agenten elkaar beschermen
in plaats van onethisch gedrag aan te pakken. Dit maakt het voor
slachtoffers van corruptie moeilijk om gerechtigheid te krijgen.
Hierdoor blijven gevallen van corruptie vaak onbestraft. Om deze
problemen aan te pakken, zijn er verschillende maatregelen nodig.
Transparantie en verantwoording zijn cruciaal. Burgers moeten de
mogelijkheid hebben om klachten in te dienen zonder angst voor
repercussies. Daarnaast kan het invoeren van externe
toezichtorganisaties helpen om het gedrag van de politie te controleren
en corruptie te bestrijden. Alleen door samen te werken kunnen we werken
aan een eerlijke en rechtvaardige samenleving.

In het najaar van 1971 vestigde de Commissie Knapp de aandacht van het
publiek op het probleem van wijdverspreide corruptie binnen de politie
van New York City. Temidden van de schokkende individuele gevallen
bestaat het risico dat men het centrale probleem over het hoofd ziet,
iets waar de Commissie Knapp zich volledig van bewust was. In vrijwel
elk geval van corruptie waren politieagenten betrokken bij goed
functionerende bedrijven die, met toestemming van de overheid, illegaal
waren verklaard. Toch hebben veel mensen, door deze goederen en diensten
te eisen, laten zien dat zij het er niet mee eens zijn dat zulke
activiteiten hetzelfde behandeld moeten worden als moord, diefstal of
geweldpleging. In bijna geen enkel geval ging de `afkoop' van politie
samen met deze gruwelijke misdaden. Meestal ging het erom dat de politie
de andere kant opkeek terwijl er legitieme en vrijwillige transacties
plaatsvonden.

Het gewoonterecht maakt een belangrijk onderscheid tussen twee soorten
misdaden: malum in se en malum prohibitum. Een malum in se is een daad
die door de massa instinctief als verwerpelijk wordt ervaren en die
bestraft moet worden. Dit sluit aan bij de definitie van misdaad volgens
libertariërs: het betreft inbreuken op persoon of eigendom, zoals
mishandeling, diefstal en moord. Aan de andere kant zijn er misdaden die
door de overheid tot misdaad worden verklaard. Dit bredere gebied is
waar corruptie binnen de politie vaak voorkomt.

Kortom, corruptie binnen de politie komt voor op die gebieden waar
ondernemers vrijwillige diensten aanbieden aan consumenten, maar waar de
overheid deze diensten illegaal heeft verklaard, zoals bij verdovende
middelen, prostitutie en gokken. Wanneer gokken bijvoorbeeld verboden
is, verleent de wet de politie, die verantwoordelijk is voor de
goksector, de macht om het recht om te gokken te verkopen. Het lijkt dus
alsof de politie de bevoegdheid heeft om speciale vergunningen voor deze
activiteiten af te geven en vervolgens deze onofficiële, maar essentiële
vergunningen tegen elke prijs verkoopt die gezocht kan worden. Een
politieagent getuigde dat, als de wet volledig gehandhaafd zou worden,
geen enkele bouwplaats in New York City nog zou kunnen functioneren. De
overheid heeft bouwplaatsen namelijk omgeven met een web van
ingewikkelde en soms onmogelijke regels. Bewust of onbewust handelt de
overheid als volgt: ze verbiedt eerst een bepaalde activiteit---of dat
nu drugs, gokken of de bouw betreft---en verkoopt vervolgens aan
potentiële ondernemers het recht om hun bedrijf te beginnen en voort te
zetten.

In het beste geval leidt dit tot hogere kosten en een lagere productie
van de activiteit dan in een vrije markt het geval zou zijn. De gevolgen
zijn echter nog schadelijker. Vaak verkopen politieagenten niet alleen
toestemming om te opereren, maar ook wat feitelijk een geprivilegieerd
monopolie is. In dat geval betaalt een gokker niet alleen om zijn zaken
voort te zetten, maar ook om concurrenten buiten de deur te houden die
de sector willen betreden. Hierdoor worden consumenten geconfronteerd
met bevoorrechte monopolisten en kunnen ze niet profiteren van de
voordelen van concurrentie. Het is dan ook niet verwonderlijk dat, toen
de Drooglegging in het begin van de jaren dertig eindelijk werd
opgeheven, de belangrijkste tegenstanders van die opheffing, naast
fundamentalistische en verbods-groepen, de georganiseerde smokkelaars
waren. Zij hadden namelijk speciale monopolistische privileges genoten
door hun afspraken met de politie en andere handhavende instanties van
de overheid.

De manier om corruptie binnen de politie te bestrijden, is eenvoudig
maar effectief: schaf de wetten af die vrijwillige bedrijfsactiviteiten
en alle `misdaden zonder slachtoffers' verbieden. Hierdoor zou niet
alleen de corruptie verdwijnen, maar zouden veel politieagenten ook
vrijgemaakt worden om op te treden tegen echte criminelen, de agressors
tegen personen en eigendommen. Tenslotte zou dat in de eerste plaats de
functie van de politie moeten zijn.

We moeten ons realiseren dat het probleem van politiecorruptie, evenals
de bredere kwestie van corruptie binnen de overheid, in een bredere
context moet worden geplaatst. Het is belangrijk om te benadrukken dat,
gezien de ongelukkige en onrechtvaardige wetten die bepaalde
activiteiten verbieden, reguleren en belasten, corruptie in veel
gevallen gunstig kan zijn voor de maatschappij. In sommige landen zou er
zonder corruptie, die overheidsverboden, belastingen en heffingen
ondermijnt, nauwelijks sprake zijn van handel of industrie. Corruptie
vergemakkelijkt de handel. De oplossing is dus niet om corruptie te
betreuren en de handhaving ervan te intensiveren, maar om het
verlammende overheidsbeleid en de wetten die corruptie noodzakelijk
maken, af te schaffen.

\section{WAPENWETTEN}\label{wapenwetten}

Wapenwetten zijn een belangrijk onderwerp van discussie in veel landen.
Ze beïnvloeden niet alleen de veiligheid, maar ook de persoonlijke
vrijheid van burgers. In Nederland zijn er strikte regels omtrent het
bezit en gebruik van wapens. Allereerst is het goed om te weten dat je
geen vuurwapen mag bezitten zonder een vergunning. Voor het verkrijgen
van deze vergunning moet je aan verschillende voorwaarden voldoen. Deze
omvatten een beoordeling van je achtergrond, je motivatie en soms zelfs
een cursus over vuurwapens. De overheid wil er zeker van zijn dat wapens
niet in verkeerde handen komen. Naast vuurwapens zijn er ook regels voor
andere soorten wapens, zoals messen en luchtdrukwapens. Veel van deze
wapens zijn vrij te kopen, maar er zijn beperkingen aan de lengte van
het lemmet of de kracht van het luchtdrukwapen. Dit moet voorkomen dat
gevaarlijke situaties ontstaan. De discussie over wapenwetten is vaak
emotioneel en politiek beladen. Voorstanders van meer vrijheid in
wapenbezit wijzen op het recht op zelfverdediging en de mogelijkheid om
jezelf te beschermen. Tegenstanders benadrukken de risico's en de
mogelijkheid van geweld. In Nederland lijkt de meerderheid van de
bevolking voor een striktere regelgeving te zijn. Dit komt onder andere
door het aantal geweldsincidenten dat met wapens te maken heeft. De
regering is daarom ook regelmatig bezig met het evalueren en aanpassen
van de wapenwetten. Kortom, wapenwetten zijn een complex onderwerp met
veel verschillende standpunten. Het is essentieel om een evenwicht te
vinden tussen veiligheid en vrijheid, zowel voor de overheid als voor de
burger.

Voor de meeste activiteiten in dit hoofdstuk pleiten liberalen voor
vrijheid van handel en bedrijvigheid. Conservatieven daarentegen
verlangen naar strenge handhaving en een hard optreden tegen
wetsovertreders. Het is echter opmerkelijk dat de standpunten vaak
omgekeerd zijn als het gaat om wapenwetten. Elke keer als er een wapen
wordt gebruikt bij een geweldsmisdrijf, versterken liberalen hun
pleidooi voor strikte beperkingen of zelfs een verbod op het privébezit
van wapens. Conservatieven verzetten zich daarentegen tegen zulke
beperkingen, uit naam van de individuele vrijheid.

Als libertariërs geloven dat ieder individu het recht heeft zijn lichaam
en eigendom te bezitten, volgt daaruit dat ze ook het recht hebben om
geweld te gebruiken ter verdediging tegen criminele aanvallers. Toch
hebben liberalen om de een of andere vreemde reden systematisch
geprobeerd onschuldige mensen te beroven van de middelen om zich tegen
agressie te beschermen. Ondanks het feit dat het Tweede Amendement van
de Grondwet garandeert dat `het recht van het volk om wapens te bezitten
en te dragen niet geschonden zal worden', heeft de regering dit recht
aanzienlijk ingeperkt. De wet Sullivan in de staat New York, net als in
veel andere staten, verbiedt bijvoorbeeld het dragen van `verborgen
wapens' zonder overheidvergunning. Dit ongrondwettelijke edict beperkt
niet alleen het dragen van wapens, maar de overheid heeft dit verbod ook
uitgebreid naar bijna elk voorwerp dat als wapen kan worden gezien,
zelfs voorwerpen die enkel voor zelfverdediging bedoeld zijn. Daardoor
is het potentiële slachtoffers van misdrijven verboden om messen,
traangaspen of zelfs hoedenspelden bij zich te dragen. Mensen die zulke
voorwerpen gebruiken om zichzelf te verdedigen tegen een aanval worden
bovendien vervolgd door de autoriteiten. In steden heeft dit strenge
verbod op het dragen van verborgen wapens slachtoffers feitelijk elke
mogelijkheid tot zelfverdediging tegen criminaliteit ontnomen. (Het
klopt dat er geen officieel verbod is op het dragen van een zichtbaar
wapen, maar een man in New York City die enkele jaren geleden de wet op
de proef stelde door met een geweer over straat te lopen, werd prompt
gearresteerd wegens `verstoring van de openbare orde'). Daarnaast
belemmeren bepalingen tegen `buitensporig' geweld bij zelfverdediging
slachtoffers zozeer, dat de crimineel automatisch een aanzienlijk
voordeel heeft binnen het bestaande rechtssysteem.

Het is belangrijk om te begrijpen dat geen enkel fysiek voorwerp op
zichzelf agressief is. Elk voorwerp, of het nu een pistool, een mes of
een stok is, kan gebruikt worden voor zowel agressie als defensie, en
voor talloze andere doeleinden die niets met criminaliteit te maken
hebben. Het heeft evenmin zin om de aankoop en het bezit van wapens te
verbieden of te beperken als dat het geval is bij messen, knuppels,
hoedenspelden of stenen. Hoe moeten we al deze voorwerpen trouwens
verbieden? En als ze eenmaal verboden zijn, hoe willen we dat verbod dan
handhaven? In plaats van onschuldige mensen te vervolgen die
verschillende voorwerpen bij zich hebben of bezitten, zou de wet zich
moeten richten op het bestrijden en opsporen van echte criminelen.

Er is bovendien een andere overweging die onze conclusie versterkt. Als
wapens worden beperkt of verboden, is er geen reden om te denken dat
vastberaden criminelen zich aan de wet zullen houden. Criminelen zullen
altijd in staat zijn om wapens te kopen en te dragen. Het zijn enkel hun
onschuldige slachtoffers die zullen lijden onder het zorgzame
liberalisme dat wetten oplegt tegen wapens en andere middelen. Net zoals
drugs, gokken en pornografie legaal zouden moeten zijn, zouden ook
vuurwapens en andere voorwerpen die voor zelfverdediging kunnen worden
gebruikt, dat moeten zijn.

In een opvallend artikel waarin de controle op vuurwapens -- het type
wapen dat liberalen het liefst aan banden leggen -- wordt bekritiseerd,
verwijt professor Don B. Kates Jr.~van de St.~Louis University zijn
collega-liberalen dat zij niet dezelfde logica toepassen op wapens als
zij doen bij marihuanawetten. Hij wijst erop dat er tegenwoordig meer
dan 50 miljoen wapenbezitters in Amerika zijn. Uit peilingen en
ervaringen uit het verleden blijkt dat twee derde tot meer dan 80
procent van de Amerikanen zich niet zou houden aan een verbod op
handwapens. Het onvermijdelijke gevolg, net als bij de wetten rondom
seks en marihuana, zou zijn dat er strenge straffen worden opgelegd en
er zeer selectief wordt gehandhaafd. Hierdoor zou het gebrek aan respect
voor de wet en de handhaving toenemen. De wet zou bovendien selectief
worden toegepast op diegenen die niet in de gratie van de autoriteiten
vallen: `De handhaving wordt steeds willekeuriger, totdat de wetten
uiteindelijk alleen nog worden gebruikt tegen mensen die niet populair
zijn bij de politie. We hoeven ons nauwelijks te herinneren aan de
afschuwelijke huiszoekingen en inbeslagnemingen waar politie- en
overheidsagenten vaak naar toevlucht nemen om overtreders van deze
wetten te vangen.' Kates voegt eraan toe: `Als deze argumenten je bekend
voorkomen, dan is dat waarschijnlijk omdat ze parallel lopen met het
standaard liberale argument tegen marihuanawetten.'

Vervolgens voegt Kates een opmerkelijk scherpzinnig inzicht toe over
deze merkwaardige, liberale blinde vlek. Want:

\begin{quote}
Het verbod op vuurwapens is het geesteskind van blanke, middenklasse
liberalen die zich niets aantrekken van de situatie van arme mensen en
minderheidsgroepen. Deze groepen wonen vaak in gebieden waar de politie
de misdaadbestrijding heeft opgegeven. Dergelijke liberalen waren ook
niet verontwaardigd over marihuanawetten in de jaren vijftig, toen
arrestaties voornamelijk in de getto's plaatsvonden. Veilig in goed
bewaakte buitenwijken of streng beveiligde appartementen, waar de
Pinkertons de wacht houden (en niemand wil ontwapenen), beschouwen deze
vergeten liberalen wapenbezit als `een anachronisme uit het Oude
Westen.'
\end{quote}

Kates benadrukt verder de empirische waarde van gewapende
zelfverdediging. In Chicago doodden gewapende burgers bijvoorbeeld in de
afgelopen vijf jaar op gerechtvaardigde wijze drie keer zoveel
gewelddadige criminelen als de politie. In een onderzoek naar honderden
gewelddadige confrontaties met criminelen ontdekte Kates dat gewapende
burgers effectiever waren dan de politie: zij wisten in 75 procent van
de confrontaties de criminelen te vangen, verwonden, doden of weg te
jagen, terwijl de politie slechts 61 procent succes had. Het klopt dat
slachtoffers die zich verzetten tegen een overval een grotere kans
hebben om gewond te raken dan degenen die passief blijven. Kates wijst
echter op een paar belangrijke punten: (1) verzet zonder wapen is twee
keer zo gevaarlijk voor een slachtoffer als verzet met een wapen, en (2)
de keuze om zich te verzetten hangt af van het slachtoffer, zijn
omstandigheden en waarden.

\begin{quote}
Het vermijden van verwondingen zal van het grootste belang zijn voor een
blanke, liberale academicus met een goede bankrekening. Voor de onzekere
arbeider of uitkeringstrekker, die beroofd wordt van de middelen om zijn
gezin een maand lang te onderhouden, is dit noodzakelijkerwijs minder
belangrijk. Ook voor een zwarte winkelier die geen overvalverzekering
kan afsluiten, is het anders: hij kan letterlijk failliet gaan door
opeenvolgende overvallen.
\end{quote}

Uit het nationale onderzoek van 1975 onder wapenbezitters, uitgevoerd
door de Decision Making Information Organization, bleek dat de
belangrijkste subgroepen die een wapen bezitten voor zelfverdediging,
bestaan uit zwartes, de laagste inkomensgroepen en senioren. `Dit zijn
de mensen,' waarschuwt Kates eloquent, `van wie wordt voorgesteld dat we
ze opsluiten omdat ze er op staan de enige bescherming voor hun gezin te
behouden in gebieden waar de politie het heeft opgegeven.'\^{}9

Hoe zit het met historische ervaringen? Hebben vuurwapenverboden het
geweld in de samenleving echt sterk verminderd, zoals liberalen beweren?
Het bewijs wijst juist in de andere richting. Een uitgebreid onderzoek
aan de Universiteit van Wisconsin concludeerde in de herfst van 1975
onomwonden dat `wetten op wapenbezit geen individueel of collectief
effect hebben op het verminderen van gewelddadige criminaliteit.' Dit
onderzoek testte bijvoorbeeld de theorie dat normaal gesproken vreedzame
mensen onweerstaanbaar in de verleiding komen om te schieten wanneer ze
in het bezit zijn van een wapen en de emoties hoog oplopen. Er werd
echter geen enkele correlatie gevonden tussen het aantal
handvuurwapenbezitters en het aantal moorden, bekeken per staat.
Bovendien wordt deze bevinding ondersteund door een onderzoek van
Harvard uit 1976. Dit onderzoek ging in op een wet in Massachusetts die
een minimale gevangenisstraf van een jaar voorschreef voor iedereen die
een pistool zonder overheidsvergunning bezat. Hoewel deze wet uit 1974
in 1975 inderdaad het dragen van vuurwapens en het aantal aanvallen met
vuurwapens aanzienlijk verminderde, ontdekten de Harvard-onderzoekers
tot hun verbazing dat er geen overeenkomstige afname was van enige vorm
van geweld. Dat wil zeggen,

\begin{quote}
Zoals eerdere criminologische studies aangetoond hebben, zal een
woedende burger zonder handvuurwapen zijn toevlucht zoeken tot het veel
dodelijkere lange vuurwapen. Ook zonder vuurwapens kan hij bijna even
dodelijk zijn met messen, hamers en dergelijke.
\end{quote}

En het is duidelijk: `Als het verminderen van wapenbezit moord of ander
geweld niet vermindert, dan is een verbod op handvuurwapens alleen maar
een verdere afleiding van politiemiddelen van echte misdaad naar
slachtofferloze misdaad.'\^{}10

Tot slot maakt Kates een intrigerend punt: een samenleving waarin
vreedzame burgers gewapend zijn, is waarschijnlijk veel beter in staat
barmhartige Samaritanen te ondersteunen die slachtoffers van misdrijven
vrijwillig te hulp schieten. Maar als mensen hun wapens afgepakt
krijgen, dan zal het publiek -- met alle rampzalige gevolgen voor de
slachtoffers -- eerder geneigd zijn de zaak aan de politie over te
laten. Voordat de staat New York handwapens verbood, waren er veel meer
barmhartige Samaritanen dan nu. In een recent onderzoek bleken maar
liefst 81 procent van de Samaritanen in het bezit van een vuurwapen. Als
we een samenleving willen stimuleren waarin burgers buren in nood te
hulp schieten, moeten we hen niet de mogelijkheid ontnemen om iets tegen
misdaad te doen. Het is immers absurd om vreedzame mensen te ontwapenen
en hen vervolgens, zoals vaak gebeurt, te beschuldigen van `apathie'
omdat ze slachtoffers van criminele aanvallen niet te hulp schieten.

\bookmarksetup{startatroot}

\chapter{Onderwijs}\label{onderwijs}

\section{OPENBAAR EN VERPLICHT
ONDERWIJS}\label{openbaar-en-verplicht-onderwijs}

Tot de afgelopen paar jaar waren er in Amerika maar weinig instellingen
die, vooral door liberalen, als heiliger werden beschouwd dan de
openbare school. De toewijding aan de openbare school had zelfs de
vroege Amerikanen, zoals de Jeffersonianen en Jacksonianen, in haar
greep, terwijl zij op andere vlakken meestal libertair waren. In de
afgelopen jaren werd de openbare school gezien als een cruciaal
onderdeel van de democratie, een bron van broederschap en de
tegenstander van elitarisme en uitsluiting in het Amerikaanse leven. De
openbare school werd gezien als de belichaming van het vermeende recht
van elk kind op onderwijs, en men verdedigde haar als een smeltkroes
voor begrip en harmonie tussen mensen van verschillende beroepen en
sociale klassen, die vanaf jonge leeftijd met hun buren aan tafel
gingen.

Hand in hand met de verspreiding van het openbaar onderwijs kwamen de
leerplichtwetten. Deze wetten verplichtten alle kinderen om naar een
openbare school of een door de overheid goedgekeurde privéschool te
gaan, tot een steeds hoger wordende minimumleeftijd. In tegenstelling
tot eerdere decennia, toen slechts een klein deel van de bevolking naar
school ging in hogere klassen, is nu de gehele bevolking gedwongen door
de overheid om een groot deel van hun meest beïnvloedbare jaren in
openbare instellingen door te brengen. We hadden de leerplichtwetten
gemakkelijk kunnen bespreken in ons hoofdstuk over onvrijwillige
dienstbaarheid, want welke instelling is duidelijker een uitgebreid
systeem van opsluiting? In de afgelopen jaren hebben Paul Goodman en
andere critici van het onderwijs de openbare scholen in het land, en in
mindere mate de particuliere instellingen, ontmaskerd als een enorm
gevangenissysteem voor de jeugd. Dit systeem trekt ontelbare miljoenen
onwillige en onaangepaste kinderen de schoolstructuur binnen. De tactiek
van de Nieuwe Linkse beweging om middelbare scholen binnen te dringen
met de leus `Jailbreak!' mag dan absurd en ineffectief zijn geweest,
maar ze brengt wel een belangrijke waarheid naar voren over het
schoolsysteem. Als we de hele jeugd onder het mom van `onderwijs' in
enorme gevangenissen stoppen, waarbij leraren en beheerders als
surrogaatbewakers functioneren, waarom zouden we dan geen groot
ongenoegen, ontevredenheid, vervreemding en rebellie onder de jeugd
verwachten? Het enige dat ons zou moeten verbazen, is dat de opstand zo
lang op zich heeft laten wachten. Steeds vaker wordt erkend dat er iets
vreselijk mis is met Amerika's meest trotse instelling. Vooral in
stedelijke gebieden zijn de openbare scholen veranderd in beerputten van
criminaliteit, kleine diefstallen en drugsverslaving, terwijl er weinig
of geen echte educatie plaatsvindt, temidden van de vervorming van de
geesten en zielen van de kinderen.1

Een deel van de reden voor deze tirannie over de jeugd in ons land is
het misplaatste altruïsme van de opgeleide middenklasse. De arbeiders,
of de `lagere klassen', moeten de kans krijgen om te genieten van het
onderwijs waar de middenklasse zo veel waarde aan hecht. Als de ouders
of de kinderen uit de massa echter zo achtergebleven zijn dat ze deze
geweldige kans niet willen grijpen, dan moet er een beetje dwang worden
uitgeoefend - `voor hun eigen bestwil', natuurlijk.

Een belangrijke misvatting van de schoolverheerlijkende middenklasse is
de verwarring tussen formeel onderwijs en onderwijs in het algemeen.
Onderwijs is een levenslang leerproces en leren gebeurt niet alleen op
school, maar op alle gebied van het leven. Wanneer een kind speelt, naar
zijn of haar ouders of vrienden luistert, een krant leest of aan een
klus werkt, vindt er onderwijs plaats. Formeel onderwijs is slechts een
klein onderdeel van het hele onderwijsproces. Het is vooral geschikt
voor specifieke onderwijsvakken, vooral wanneer het gaat om meer
gevorderde en systematische onderwerpen. Elementaire vaardigheden, zoals
lezen, schrijven en rekenen, kunnen eenvoudig thuis en buiten school
geleerd worden.

Bovendien is een van de grote rijkdommen van de mensheid zijn
diversiteit. Ieder individu is uniek, met zijn eigen capaciteiten,
interesses en vaardigheden. Kinderen verplichten om formeel onderwijs te
volgen, terwijl ze daar geen interesse of aanleg voor hebben, is een
schending van hun ziel en geest. Paul Goodman heeft betoogd dat de
meeste kinderen beter af zouden zijn als ze op jonge leeftijd konden
werken, een vak konden leren en datgene konden doen waarvoor ze het
meest geschikt zijn. Amerika is opgebouwd door burgers en leiders, van
wie velen weinig of geen formele scholing hebben gehad. Het idee dat
iemand een middelbareschooldiploma of tegenwoordig een A.B.-diploma moet
hebben om te kunnen werken en leven in de wereld, is een absurditeit van
deze tijd. Als we de aanwezigheidsplicht afschaffen en kinderen meer
vrijheid geven, zullen we weer een natie van mensen krijgen die
productiever, geïnteresseerder, creatiever en gelukkiger is. Veel
weldenkende critici van Nieuw Links en de jeugdrebellie wijzen erop dat
een groot deel van de ontevredenheid van de jeugd en hun vervreemding
van de werkelijkheid te maken heeft met de toenemende tijd die ze op
school doorbrengen, opgesloten in een cocon van afhankelijkheid en
onverantwoordelijkheid. Dat is waar, maar wat is de belangrijkste
oorzaak van deze verlengde cocon? De hele systematiek, vooral de
leerplichtwetten, die beweren dat iedereen voortdurend naar school moet
- eerst naar de middelbare school, vervolgens naar de universiteit, en
misschien binnenkort zelfs voor een Ph.D.-titel. Het is deze dwang tot
massaal onderwijs die zowel de onvrede creëert als de voortdurende
afstand van de `echte wereld'. In geen enkel ander land en in geen enkel
andere tijd is deze obsessie voor massaal onderwijs zo'n grip op de
samenleving gekregen.

Het is opmerkelijk dat het oude libertarische rechts en Nieuw Links,
hoewel ze vanuit totaal andere perspectieven en met verschillende
retoriek spreken, tot een vergelijkbare visie zijn gekomen over de
despotische aard van massaal onderwijs. Albert Jay Nock, een grote
individualistische theoreticus uit de jaren `20 en '30, bekritiseerde
het onderwijssysteem omdat het de 'onopvoedbare' massa's de scholen in
drong, vanuit een ijdel streven naar gelijkheid en een geloof in de
gelijke opvoedbaarheid van elk kind. In plaats van kinderen met de
juiste aanleg en vaardigheden naar school te laten gaan, worden alle
kinderen gedwongen naar school te gaan `voor hun eigen bestwil'. Dit
leidt tot verstoring van het leven van degenen die niet geschikt zijn
voor school en tot de ondergang van goed onderwijs voor de kinderen die
wél opvoedbaar zijn. Nock had ook scherpzinnige kritiek op
conservatieven die `progressief onderwijs' aanvielen omdat ze de
onderwijsnormen verlaagden door cursussen aan te bieden in zaken als
autorijden, mandenvlechten of het kiezen van een tandarts. Hij stelde
dat als je een heleboel kinderen die geen klassiek onderwijs kunnen
volgen naar school dwingt, het onderwijs vanzelfsprekend zal verschuiven
naar beroepsopleidingen, gericht op de kleinste gemene deler. De fatale
fout ligt niet bij progressief onderwijs, maar bij de drang naar
universeel onderwijs, waarop het progressivisme een geïmproviseerd
antwoord was.

Nieuw-Linkse critici zoals John McDermott en Paul Goodman beschuldigen
de middenklasse ervan kinderen uit de arbeidersklasse, die vaak andere
waarden en aanleg hebben, te dwingen in een openbaar schoolsysteem dat
ontworpen is om hen in een middenklasse model te doen passen. Het is
duidelijk dat, ongeacht of men nu voorstander is van de ene of de andere
klasse of van een bepaald schoolideaal, de kern van de kritiek vrijwel
hetzelfde is: een grote groep kinderen wordt in een instituut geplaatst
waarvoor ze weinig interesse of aanleg hebben.

Als we kijken naar de geschiedenis van de drang naar openbaar onderwijs
en leerplicht in dit en andere landen, zien we dat hier niet zozeer
misplaatst altruïsme aan ten grondslag ligt, maar eerder een bewust plan
om de massa in een door de gevestigde orde gewenste vorm te dwingen.
Recalcitrante minderheden moesten worden aangepast aan de meerderheid;
alle burgers moesten de burgerdeugden geleerd worden, vooral de
gehoorzaamheid aan het staatsapparaat. Als de massa bevolking op
overheidsscholen moet worden onderwezen, hoe kunnen deze scholen dan
niet een krachtig instrument zijn voor het aanleren van gehoorzaamheid
aan de staatsautoriteiten? Maarten Luther, een vooraanstaande figuur in
de eerste moderne drang naar verplicht staatsonderwijs, verwoordde zijn
pleidooi typerend in zijn beroemde brief van 1524 aan de heersers van
Duitsland:

\begin{quote}
Geachte heersers, Ik stel dat de civiele autoriteiten verplicht zijn om
mensen te dwingen hun kinderen naar school te sturen. Als de overheid
burgers die geschikt zijn voor militaire dienst kan verplichten om speer
en geweer te dragen, muren te beklimmen en andere krijgstaken uit te
voeren tijdens oorlogstijd, hoeveel recht heeft ze dan niet om mensen te
verplichten hun kinderen naar school te sturen? In dit geval strijden we
immers tegen de duivel, wiens doel het is om onze steden en
vorstendommen in het geheim uit te putten.3
\end{quote}

Voor Luther waren de staatsscholen een essentieel onderdeel van de
`oorlog tegen de duivel', dat wil zeggen tegen katholieken, joden,
ongelovigen en concurrerende protestantse sekten. Een moderne
bewonderaar van Luther en van de leerplicht merkte op dat\ldots{}

\begin{quote}
De blijvende en positieve waarde van Luthers uitspraak uit 1524 ligt in
de heilige verbanden die hij legde tussen de nationale religie en de
opvoedkundige plichten van zowel het individu als de staat. Dit leidde
ongetwijfeld tot een gezonde publieke opinie die het principe van de
leerplicht in Pruisen veel eerder aanvaardbaar maakte dan in Engeland.4
\end{quote}

De andere grote protestantse grondlegger, Johannes Calvijn, was minstens
zo ijverig in het bevorderen van openbaar onderwijs en deed dit om
vergelijkbare redenen. Het is dan ook niet verrassend dat de eerste
leerplicht in Amerika werd ingesteld door de calvinistische puriteinen
in de Massachusetts Bay. Deze mannen wilden in de Nieuwe Wereld een
absolutistische calvinistische theocratie vestigen. In juni 1642,
slechts een jaar nadat de kolonie Massachusetts Bay haar eerste wetten
had aangenomen, introduceerde de kolonie het eerste systeem van
verplicht onderwijs in de Engelssprekende wereld. De wet stelde het
volgende vast:

\begin{quote}
Omdat de goede opvoeding van kinderen van groot belang is voor elk
gemenebest, en gezien het feit dat veel ouders en leerkrachten te
toegeeflijk en nalatig zijn in hun verantwoordelijkheden, wordt bevolen
dat de keurvorsten van elke stad\ldots{} zorgvuldig zullen toezien op
hun buren. Ze moeten ervoor zorgen dat niemand in hun omgeving zoveel
barbaarsheid ondergaat dat ze niet proberen om, zelf of via anderen, hun
kinderen en leerlingen op te voeden.5
\end{quote}

Vijf jaar later richtte Massachusetts Bay openbare scholen op, naar
aanleiding van deze wet.

Vanaf het begin van de Amerikaanse geschiedenis was het verlangen om de
bevolking te vormen, te onderwijzen en gehoorzaam te maken de
belangrijkste drijfveer achter het streven naar openbaar onderwijs. In
de koloniale tijd werd openbaar onderwijs ingezet om religieuze
dissidenten te onderdrukken en om weerbarstige bedienden de deugden van
gehoorzaamheid aan de staat bij te brengen. Het is bijvoorbeeld
kenmerkend dat Massachusetts en Connecticut, tijdens hun onderdrukking
van de Quakers, deze verachte sekte verboden om eigen scholen op te
richten. In Connecticut werd in een vergeefse poging om de `New
Light'-beweging te onderdrukken, dit ook zo gesteld in 1742. De
autoriteiten van Connecticut redeneerden dat de New Lights anders `de
neiging zouden kunnen hebben om de jeugd op te leiden in slechte
principes en praktijken, en om wanordelijkheden te introduceren die
fatale gevolgen zouden kunnen hebben voor de openbare vrede en het
welzijn van deze kolonie.'\^{}6 Het is geen toeval dat Rhode Island, de
enige echte vrije kolonie in New England, ook de enige kolonie in de
regio was zonder openbaar onderwijs.

De motivatie voor openbaar en verplicht onderwijs na de
onafhankelijkheid verschilde nauwelijks in essentie. Archibald D.
Murphey, de grondlegger van het openbare schoolsysteem in North
Carolina, riep dan ook op tot de oprichting van dergelijke scholen:

\begin{quote}
Alle kinderen zullen daaronderwezen worden. In deze scholen moeten de
normen van moraliteit en religie worden aangeleerd, en moeten gewoonten
van ondergeschiktheid en gehoorzaamheid worden gevormd. Hun ouders weten
niet hoe ze hen moeten onderwijzen. De staat, uit genegenheid en
bezorgdheid voor hun welzijn, moet voor deze kinderen zorgen en hen naar
scholen sturen waar hun geest wordt verlicht en hun hart getraind in
deugdzaamheid.\^{}7
\end{quote}

Een van de meest voorkomende toepassingen van verplicht openbaar
onderwijs is het onderdrukken en verlammen van nationale, etnische en
linguïstische minderheden of gekoloniseerde volkeren. Dit gebeurt vaak
om hen te dwingen hun eigen taal en cultuur op te geven ten gunste van
die van de heersende groepen. De Engelsen in Ierland en Quebec, evenals
naties in heel Centraal- en Oost-Europa en in Azië, hebben allemaal hun
nationale minderheden naar openbare scholen gedwongen die werden geleid
door hun onderdrukkers. Een van de sterkste prikkels voor onvrede en
opstand onder deze onderdrukte volkeren was de wens om hun taal en
erfgoed te beschermen tegen de onderdrukkende hand van het openbare
onderwijs. Zo schreef de laissez-faire liberaal Ludwig von Mises dat in
taalkundig gemengde landen\ldots{}

\begin{quote}
Vasthouden aan een beleid van leerplicht is volstrekt onverenigbaar met
de inspanningen om duurzame vrede te bewerkstelligen. De vraag welke
taal als basis voor het onderwijs moet dienen, is cruciaal. De keuze
voor een bepaalde taal kan in de loop der jaren de nationaliteit van een
heel gebied bepalen. Scholen kunnen kinderen vervreemden van de
nationaliteit van hun ouders en kunnen worden gebruikt om gehele
nationaliteiten te onderdrukken. Wie de scholen beheert, heeft de macht
om andere nationaliteiten te benadelen en zijn eigen nationaliteit te
bevoordelen.
\end{quote}

Bovendien wijst Mises erop dat de dwang die inherent is aan de
heerschappij van één nationaliteit het onmogelijk maakt om het probleem
op te lossen door iedere ouder formeel de toestemming te geven om zijn
kind naar een school te sturen waar de taal van zijn eigen nationaliteit
wordt gesproken.

\begin{quote}
Het is vaak niet mogelijk voor een individu om zich openlijk uit te
spreken voor de ene of de andere nationaliteit, vooral uit respect voor
zijn of haar middel van bestaan. In een interventiesysteem kan dit
namelijk de klantenkring van andere nationaliteiten of een baan bij een
ondernemer van een andere nationaliteit kosten. Als je ouders de vrije
keus laat om naar welke school ze hun kinderen sturen, stel je ze bloot
aan verschillende vormen van politieke druk. In gebieden met een
gemengde nationaliteit heeft de school een groot politiek belang. Ze kan
haar politieke karakter niet verliezen zolang ze een openbare en
verplichte instelling blijft. De enige echte oplossing is dat de staat,
de regering en de wetten zich op geen enkele manier met onderwijs mogen
bemoeien. Publieke middelen mogen niet voor dit doel worden aangewend.
De opvoeding en het onderwijs van de jeugd moeten volledig worden
overgelaten aan de ouders en particuliere verenigingen en instellingen.
\end{quote}

Een van de belangrijkste drijfveren van het legioen Amerikaanse
`onderwijshervormers' uit het midden van de 19e eeuw, dat het moderne
openbare schoolsysteem oprichtte, was het streven om het culturele en
taalkundige leven van de golven van immigranten in Amerika te
onderdrukken. Onderwijshervormer Samuel Lewis stelde dat het doel was om
deze groepen tot `één volk' te vormen. Het verlangen van de
Angelsaksische meerderheid om immigranten te temmen, te kanaliseren en
te herstructureren, was vooral gericht op het vernietigen van het
parochiale schoolsysteem van de katholieken. Dit vormde de belangrijkste
impuls voor de onderwijshervormingen. Critici van het Nieuwe Links die
de rol van de openbare scholen in het verlammen en vormen van de geesten
van gettokinderen dient te begrijpen, zien slechts de huidige
belichaming van een lang gekoesterd doel. Dit doel was in handen van de
gevestigde orde van de openbare school, vertegenwoordigd door figuren
zoals Horace Mann, Henry Barnard en Calvin Stowe. Mann en Barnard waren
bijvoorbeeld de eersten die erop aandrongen om scholen te gebruiken voor
indoctrinatie tegen de `mobocratie' van de Jacksoniaanse beweging.
Stowe, auteur van een lovend traktaat over het Pruisische verplichte
schoolsysteem dat oorspronkelijk door Martin Luther was geïnspireerd,
schreef over scholen in onmiskenbaar Lutherse en militaire termen.

\begin{quote}
Als het vanuit het oogpunt van de openbare veiligheid gerechtvaardigd is
dat een regering burgers dwingt tot militaire dienst wanneer het land
wordt binnengevallen, dan geeft dezelfde reden de regering het recht om
ouders te verplichten te zorgen voor onderwijs voor hun kinderen. Een
man heeft niet meer recht om de staat in gevaar te brengen door een
gezin op te voeden met onwetende en slecht opgevoede kinderen, dan dat
hij spionnen van een binnenvallend leger toestaat. 9
\end{quote}

Veertig jaar later sprak Newton Bateman, een vooraanstaand opvoeder,
over het `eminente domeinrecht' van de staat over de `geesten, zielen en
lichamen' van de kinderen van de natie. Hij stelde dat onderwijs `niet
kan worden overgelaten aan de grillen en toevalligheden van
individuen'.10

De meest ambitieuze poging van voorstanders van openbare scholen om hun
controle over de kinderen van het land te maximaliseren vond plaats in
Oregon in het begin van de jaren twintig. De staat Oregon, die zelfs
niet tevreden was met het toestaan van door de staat gecertificeerde
privéscholen, nam op 7 november 1922 een wet aan die privéscholen
verbood. Hierdoor werden alle kinderen verplicht om naar de openbare
school te gaan. Dit was het hoogtepunt van de droom van de
onderwijskundigen. Eindelijk werden alle kinderen door de
staatsautoriteiten gedwongen om de `democratiserende' vorm van uniform
onderwijs te volgen. Gelukkig werd de wet in 1925 ongrondwettelijk
verklaard door het Hooggerechtshof van de Verenigde Staten (Pierce v.
Society of Sisters, 1 juni 1925). Het Hooggerechtshof stelde dat `het
kind niet louter het schepsel van de staat is' en beweerde dat de wet
van Oregon in strijd was met de `fundamentele theorie van vrijheid
waarop alle regeringen in deze Unie berusten'. De aanhangers van de
openbare scholen hebben sindsdien nooit meer zulke verregaande
maatregelen voorgesteld. Het is echter leerzaam om te begrijpen welke
krachten probeerden al het concurrerende privéonderwijs in de staat
Oregon te verbieden. De initiatiefnemers van de wet waren, in
tegenstelling tot wat we zouden verwachten, niet liberale of
progressieve opvoeders of intellectuelen; de speerpunten waren de Ku
Klux Klan. Deze organisatie, toen sterk vertegenwoordigd in de
noordelijke staten, was vastberaden om het katholieke parochiale
schoolsysteem te vernietigen en alle katholieke en migrantenkinderen
naar de neoprotestantiserende en `Amerikaniserende' invloed van de
openbare school te dwingen. Het is ook interessant om op te merken dat
de Klan vond dat een dergelijke wet noodzakelijk was voor het `behoud
van vrije instellingen'. Het is goed om te beseffen dat het veelgeprezen
`progressieve' en `democratische' openbare schoolsysteem zijn vurigste
aanhangers had in de meest onverdraagzame kringen van de Amerikaanse
samenleving, onder mensen die vastbesloten waren om diversiteit en
verscheidenheid in Amerika uit te roeien.11

\section{\texorpdfstring{\textbf{UNIFORMITEIT OF
DIVERSITEIT?}}{UNIFORMITEIT OF DIVERSITEIT?}}\label{uniformiteit-of-diversiteit}

Als het vanuit het perspectief van de openbare veiligheid
gerechtvaardigd is dat een overheid burgers dwingt tot militaire dienst
bij een invasie, dan geeft diezelfde reden haar het recht om ouders te
verplichten te zorgen voor het onderwijs van hun kinderen. Een man heeft
niet meer recht om de staat in gevaar te brengen door een gezin van
onwetende en slecht opgeleide kinderen te creëren, dan wanneer hij
spionnen van een binnenvallend leger toestaat. Veertig jaar later sprak
Newton Bateman, een vooraanstaand opvoeder, over het `eminente
domeinrecht' van de staat over de `geesten, zielen en lichamen' van de
kinderen van de natie. Volgens hem mag onderwijs niet worden overgelaten
aan de grillen en toevalligheden van individuen. In het begin van de
jaren twintig vond de meest ambitieuze poging van voorstanders van
openbare scholen plaats om hun invloed over de kinderen van het land te
vergroten. De staat Oregon, niet tevreden met het toestaan van door de
staat erkende privéscholen, nam op 7 november 1922 een wet aan die
privéscholen verbood. Hierdoor werden alle kinderen verplicht om naar de
openbare school te gaan. Dit was het hoogtepunt van de droom van de
onderwijskundigen. Eindelijk werden alle kinderen door de autoriteiten
gedwongen om de `democratiserende' vorm van uniform onderwijs te volgen.
Gelukkig werd deze wet in 1925 ongeldig verklaard door het
Hooggerechtshof van de Verenigde Staten (Pierce v. Society of Sisters, 1
juni 1925). Het Hof stelde dat `het kind niet louter het schepsel van de
staat is' en concludeerde dat de wet van Oregon in strijd was met de
`fundamentele theorie van vrijheid waarop alle regeringen in deze Unie
berusten'. De aanhangers van de openbare scholen hebben sindsdien geen
poging meer gedaan om zo ver te gaan. Het is echter leerzaam om te
begrijpen welke krachten probeerden al het concurrerende privéonderwijs
in Oregon te verbieden. De initiatiefnemers van de wet waren, in
tegenstelling tot wat we zouden verwachten, niet liberale of
progressieve opvoeders of intellectuelen; het waren de Ku Klux Klan.
Deze organisatie, destijds sterk aanwezig in de noordelijke staten, had
als doel het katholieke parochiale schoolsysteem te vernietigen en alle
katholieke en migrantenkinderen naar de neoprotestantiserende en
`Amerikaniserende' invloed van de openbare school te dwingen. Het is ook
interessant om te vermelden dat de Klan vond dat een dergelijke wet
noodzakelijk was voor het `behoud van vrije instellingen'. Het is goed
om te beseffen dat het veelgeprezen `progressieve' en `democratische'
openbare schoolsysteem zijn meest fanatieke aanhangers had onder de
meest onverdraagzame groepen van de Amerikaanse samenleving, die
vastbesloten waren om diversiteit en verscheidenheid in Amerika uit te
roeien.

Hoewel de huidige onderwijskundigen niet zo ver gaan als de Ku Klux
Klan, is het belangrijk te beseffen dat de aard van de openbare school
uniformiteit oplegt en diversiteit en individualiteit in het onderwijs
uitsluit.

Het ligt in de aard van elke overheidsbureaucratie om zich aan bepaalde
regels te houden en deze op een uniforme en strikte manier toe te
passen. Als dat niet het geval zou zijn, en een bureaucraat elk geval ad
hoc zou beslissen, dan zou hij terecht beschuldigd worden van het niet
op gelijke wijze behandelen van belastingbetalers en burgers. Dit zou
hem beschuldigen van discriminatie en het verlenen van speciale
privileges. Bovendien is het voor de bureaucraat administratief handiger
om uniforme regels binnen zijn rechtsgebied op te stellen. In
tegenstelling tot een particulier bedrijf, dat gericht is op winst,
heeft de overheidsbureaucraat geen belang bij efficiëntie of het
optimaal bedienen van zijn klanten. Omdat hij geen winst hoeft te maken
en beschermd is tegen verliezen, kan hij de wensen en eisen van zijn
consumenten negeren. Sterker nog, hij doet dit vaak ook. Zijn
voornaamste doel is om `geen opschudding te veroorzaken.' Dit bereikt
hij door onpartijdig een uniforme set regels toe te passen, ongeacht hoe
ongepast deze ook zijn in bepaalde situaties.

De openbare schoolbureaucraat wordt geconfronteerd met talloze
belangrijke en controversiële beslissingen als het gaat om het onderwijs
in zijn gebied. Hij moet bijvoorbeeld kiezen of het onderwijs
traditioneel of progressief moet zijn, vrij ondernemend of
socialistisch, competitief of egalitair, gericht op de vrije kunsten of
op beroepsonderwijs, gescheiden of geïntegreerd, met of zonder seksuele
voorlichting, religieus of seculier, en variaties daartussen. Het
probleem is dat, wat hij ook beslist, er altijd een aanzienlijk aantal
ouders en kinderen zal zijn die niet het type onderwijs krijgen dat zij
wensen. Kiest hij voor traditionele discipline op scholen, dan zullen de
meer progressieve ouders zich benadeeld voelen, en andersom zal
hetzelfde gebeuren bij andere belangrijke besluiten. Naarmate het
onderwijs openbaar wordt, zullen meer ouders en kinderen het onderwijs
missen dat zij voor zichzelf als noodzakelijk beschouwen. Hoe openbaar
het onderwijs ook wordt, hoe meer uniforme hardheid de behoeften en
wensen van individuen en minderheidsgroepen zal verdringen.

Hoe groter het aandeel van het openbaar onderwijs ten opzichte van het
privéonderwijs, des te heftiger en intensiever het conflict in de
samenleving. Wanneer één instantie de beslissing neemt over zaken als
seksuele voorlichting, traditioneel of progressief onderwijs, of
gescheiden versus geïntegreerd, wordt het cruciaal om controle over de
overheid te verwerven en te voorkomen dat tegenstanders de macht in
handen krijgen. Daarom geldt, zowel in het onderwijs als in andere
domeinen: naarmate meer overheidsbeslissingen de private besluitvorming
vervangen, zullen verschillende groepen elkaar verzwakken in een
wanhopige strijd om ervoor te zorgen dat de beslissingen in elk
belangrijk gebied langs hun eigen lijnen worden genomen.

Stel je de ontberingen en intense sociale conflicten voor die inherent
zijn aan overheidsbesluitvorming, en vergelijk deze met hoe het op de
vrije markt zou verlopen. Als onderwijs volledig privé zou zijn, zouden
ouders hun eigen school kunnen kiezen. Er zouden talloze verschillende
scholen ontstaan om tegemoet te komen aan de uiteenlopende
onderwijsbehoeften van ouders en kinderen. Sommige scholen zouden een
traditioneel programma aanbieden, terwijl andere progressief zouden
zijn. Sommige zouden experimentele vormen van egalitair en klasseloos
onderwijs hanteren, terwijl andere de nadruk leggen op grondig vakken
leren en competitieve cijfers. Daarnaast zouden er seculiere scholen
zijn, maar ook scholen die verschillende religieuze overtuigingen
bevorderen. Sommige scholen zouden libertaire principes en de waarde van
vrij ondernemerschap benadrukken, terwijl andere zich richten op
verschillende soorten socialisme.

Laten we eens kijken naar de structuur van de tijdschriften- en
boekenindustrie van vandaag. We mogen niet vergeten dat tijdschriften en
boeken een belangrijke vorm van onderwijs zijn. De tijdschriftenmarkt is
min of meer vrij en biedt een breed scala aan titels die voldoen aan de
verschillende smaken en eisen van consumenten. Er zijn landelijke
tijdschriften voor allerlei doelen; liberale, conservatieve en
ideologische bladen; gespecialiseerde wetenschappelijke publicaties; en
talloze tijdschriften die zich richten op specifieke interesses en
hobby's zoals bridge, schaken, hifi, enzovoort. De vrije boekenmarkt
heeft een soortgelijke structuur. Je vindt er boeken met een brede
aantrekkingskracht, evenals boeken voor gespecialiseerde markten en
boeken die verschillende ideologische overtuigingen vertegenwoordigen.
Als we de openbare scholen zouden afschaffen, dan zou de vrije,
gevarieerde en diverse tijdschriften- en boekenmarkt vergelijkbaar zijn
met een soort `schoolmarkt'. Stel je voor dat er maar één tijdschrift
per stad of staat zou zijn. Denk aan de strijd en conflicten die dat met
zich mee zou brengen: Moet het tijdschrift conservatief, liberaal of
socialistisch zijn? Hoeveel ruimte moet er zijn voor fictie of voor
bridge, enzovoort? De druk en conflicten zouden intens zijn, en er zou
nooit een bevredigende oplossing zijn. Elke beslissing zou talloze
mensen uitsluiten van wat ze willen en nodig hebben. Wat de libertariër
vraagt, is dan ook niet zo vreemd als het op het eerste gezicht lijkt;
hij pleit voor een schoolsysteem dat net zo vrij en gevarieerd is als de
meeste andere educatieve media van vandaag.

Als we onze aandacht weer op andere educatieve media richten, wat zouden
we dan denken van een voorstel voor de overheid, zowel federaal als op
staatsniveau, om belastinggeld te gebruiken voor een landelijke keten
van openbare tijdschriften of kranten? En wat als de overheid iedereen,
of alle kinderen, zou dwingen deze te lezen? Wat zouden we ervan vinden
als de overheid alle andere kranten en tijdschriften zou verbieden, of
op zijn minst alle publicaties die niet voldoen aan bepaalde `normen'
die door een overheidscommissie zijn vastgesteld? Dergelijke voorstellen
zouden in het hele land met afschuw worden ontvangen. Toch is dit
precies wat de overheid op scholen heeft ingesteld. Een verplichte
openbare pers zou terecht worden gezien als een inbreuk op de
basisvrijheid van de pers. Is vrijheid op school niet minstens zo
belangrijk als persvrijheid? Zijn beide niet essentiële media voor
publieke informatie en educatie, voor vrij onderzoek en het nastreven
van de waarheid? De onderdrukking van vrij onderwijs zou zelfs met nog
grotere verontwaardiging moeten worden beschouwd dan de onderdrukking
van een vrije pers, omdat hier de kwetsbare en ongevormde geesten van
kinderen directer bij betrokken zijn.

Het is intrigerend dat sommige voorstanders van openbare scholen de
vergelijking tussen onderwijs en de pers hebben ingezien en deze logica
ook naar de laatste domeinen hebben doorgetrokken. In Boston, in de
jaren 1780 en 1790, was de aarts-federalistische `Essex Junto' een
prominente politieke groep. Deze bestond uit vooraanstaande kooplieden
en advocaten uit Essex County, Massachusetts. De leden van de Essex
Junto waren fel voorstander van een uitgebreid openbaar schoolsysteem om
de jeugd `de juiste ondergeschiktheid bij te brengen'. Stephen
Higginson, een lid van deze groep, verklaarde openhartig dat `de mensen
geleerd moet worden om hun heersers te vertrouwen en te vereren'.
Jonathan Jackson, een andere invloedrijke koopman en theoreticus uit
Essex, zag in dat kranten een net zo belangrijke vorm van onderwijs
waren als formele scholing. Hij bekritiseerde de vrije pers omdat deze
afhankelijk was van het publiek en stelde voor om een staatskrant op te
richten die onafhankelijk van haar lezers zou kunnen functioneren en zo
de burgers de juiste deugden zou bijbrengen.

Professor E.G. West heeft ook een leerzame vergelijking gemaakt tussen
het aanbieden van onderwijs en voedsel. Beide zijn minstens zo
belangrijk voor kinderen als voor volwassenen. West schrijft:

\begin{quote}
Bescherming van een kind tegen honger of ondervoeding is net zo
belangrijk als bescherming tegen onwetendheid. Het is echter moeilijk
voor te stellen dat een regering, in haar zorg om ervoor te zorgen dat
kinderen minimaal voedsel en kleding hebben, wetten zou aannemen voor
verplicht en universeel eten. Eveneens is het ondenkbaar dat ze
maatregelen zou overwegen die leiden tot hogere belastingen of tarieven
om kinderen `gratis' eten te geven in gemeentelijke keukens of winkels.
Nog lastiger is het om te geloven dat de meeste mensen dit systeem
kritiekloos zouden accepteren, vooral als het zover zou komen dat ouders
`administratieve redenen' kregen om naar de winkels te gaan die het
dichtst bij hun huis lagen. Hoe vreemd zulke hypothetische maatregelen
ook lijken voor voedsel en kleding, ze zijn tegelijkertijd typerend voor
het staatsonderwijs.
\end{quote}

Verscheidene libertarische denkers, zowel van de `linkse' als de
`rechtse' kant van het libertarische spectrum, hebben felle kritiek
geuit op de totalitaire aard van verplicht openbaar onderwijs. De
links-libertarische Britse criticus Herbert Read stelt:

\begin{quote}
De mensheid is van nature verdeeld in verschillende types, en het is
onvermijdelijk dat het in eenzelfde mal persen van deze types leidt tot
vervormingen en onderdrukking. Scholen moeten in veel variëteiten
bestaan, met diverse methoden en voor verschillende aanleg. Je zou
kunnen stellen dat zelfs een totalitaire staat dit principe zou moeten
erkennen. De werkelijkheid is echter dat differentiatie een organisch
proces is, voortkomend uit de spontane en onafhankelijke samenwerking
van individuen met bepaalde doelen. De hele structuur van onderwijs,
zoals wij die als een natuurlijk proces voor ogen hebben, valt uiteen
wanneer we proberen deze structuur kunstmatig te creëren.
\end{quote}

En de grote laat-negentiende-eeuwse individualistische Engelse filosoof
Herbert Spencer vroeg zich af:

\begin{quote}
Wat wordt er precies bedoeld met de uitspraak dat een regering de mensen
moet onderwijzen? Waarom zouden mensen onderwijs nodig hebben? Wat is
het doel van onderwijs? Het lijkt erop dat het erom gaat mensen voor te
bereiden op het sociale leven, zodat ze goede burgers worden. Maar wie
bepaalt wat een goede burger is? De overheid, want er is geen andere
beoordelaar. En wie bepaalt hoe deze goede burgers gevormd kunnen
worden? Wederom de overheid, want ook hier is er geen andere
beoordelaar. Zo kan de stelling herleid worden tot dit: de overheid zou
kinderen moeten vormen tot goede burgers. Ze moet eerst voor zichzelf
een duidelijk beeld hebben van wat een modelburger is. Vervolgens moet
ze een systeem van discipline ontwikkelen dat het meest effectief is om
burgers volgens dat beeld te creëren. Dit systeem van discipline moet ze
tot het uiterste handhaven. Want als ze dat niet doet, laat ze mensen
toe om anders te zijn dan zij vindt dat ze zouden moeten zijn. Op die
manier faalt ze in de plicht die van haar wordt verwacht.
\end{quote}

En de twintigste-eeuwse Amerikaanse individuele schrijfster Isabel
Paterson zei:

\begin{quote}
Onderwijs is noodzakelijkerwijs selectief in onderwerp, taal en
standpunt. Wanneer het onderwijs wordt verzorgd door privéscholen, is er
een aanzienlijke variatie tussen de verschillende instellingen. Ouders
moeten aan de hand van het aangeboden curriculum bepalen wat ze hun
kinderen willen laten leren. \ldots{} Er zal echter nergens enige
aansporing zijn om de `superioriteit van de staat als verplichte
filosofie' bij te brengen. Maar elk politiek gecontroleerd
onderwijssysteem zal vroeg of laat de leerstelling van de superioriteit
van de staat onderwijzen, of dit nu het goddelijke recht van koningen
betreft of de `wil van het volk' in een `democratie'. Zodra die
leerstelling eenmaal is geaccepteerd, wordt het een bijna onmogelijke
opgave om de wurggreep van de politieke macht op het leven van de burger
te doorbreken. De politieke macht heeft van jongs af aan de controle
over lichaam, eigendom en geest van de burger. Een octopus zou zijn
prooi eerder loslaten.

Een verplicht onderwijssysteem dat wordt gefinancierd door belastingen,
is het volledige model van een totalitaire staat.
\end{quote}

Zoals E.C. West al opmerkte, heeft bureaucratisch gemak de staten ertoe
gebracht geografische openbare schooldistricten vast te stellen, waarbij
elke school in een district valt en elk schoolkind naar de
dichtstbijzijnde school moet gaan. In een vrije, privét onderwijsmarkt
zouden de meeste kinderen ongetwijfeld naar scholen in hun directe
omgeving gaan. Het huidige systeem legt echter een monopolie op van één
school per district en dwingt zo uniformiteit in elk gebied af. Kinderen
die om welke reden dan ook liever naar een school in een ander district
willen, is dat niet toegestaan. Dit zorgt voor gedwongen geografische
homogeniteit en betekent dat het karakter van elke school volledig
afhankelijk is van de woonwijk. Het is dan ook te verwachten dat
openbare scholen, in plaats van volledig uniform te zijn, binnen elk
district gelijksoortig zullen zijn. De samenstelling van de leerlingen,
de financiering van elke school en de kwaliteit van het onderwijs zijn
namelijk afhankelijk van de waarden, rijkdom en belastingstructuur in
elk gebied. Rijke schooldistricten bieden dan ook duurder en kwalitatief
beter onderwijs, hogere lerarensalarissen en betere
arbeidsomstandigheden dan de armere districten. Dit leidt ertoe dat
leerkrachten de betere scholen beschouwen als de meest aantrekkelijke
onderwijsinstellingen en dat de beste leerkrachten naar deze betere
schooldistricten trekken, terwijl de minder ervaren leerkrachten in de
lagere-inkomensgebieden moeten blijven. De werking van openbare scholen
in districten leidt dus onvermijdelijk tot een schending van het
egalitaire doel, dat eigenlijk een belangrijke drijfveer van het
openbare schoolsysteem zou moeten zijn.

Als woongebieden raciaal gesegregeerd zijn, zoals vaak het geval is,
leidt een verplicht geografisch monopolie tot de gedwongen
rassenscheiding van openbare scholen. Ouders die kiezen voor
geïntegreerd onderwijs moeten zich verzetten tegen dit systeem. Zoals
iemand ooit opmerkte: `Wat niet verboden is, is verplicht.' De recente
tendens onder bureaucraten van openbare scholen is niet om vrijwillig
busvervoer voor kinderen in te voeren om ouders meer keuzevrijheid te
geven. In plaats daarvan draait men de andere kant op en wordt
verplichte busvervoer en verplichte raciale integratie van scholen
opgelegd, wat vaak resulteert in een absurde verplaatsing van kinderen
ver van hun huis. Dit volgt weer het typische overheidspatroon: ofwel
verplichte segregatie ofwel verplichte integratie. De vrijwillige
aanpak, waarbij beslissingen aan de ouders worden overgelaten, wordt
genegeerd door elke staatsbureaucratie.

Het is opmerkelijk dat recente bewegingen voor lokale ouderlijke
controle over openbaar onderwijs soms als `extreem rechts' en soms als
`extreem links' worden gekarakteriseerd, terwijl de libertarische
motivatie in beide gevallen precies dezelfde is. Wanneer ouders zich
verzetten tegen het verplichte busvervoer van hun kinderen naar
afgelegen scholen, wordt deze actie door het onderwijsestablishment
bestempeld als `onverdraagzaam' en `rechts'. Maar wanneer negerouders,
zoals in het geval van Ocean Hill-Brownsville in New York City, lokale
ouderlijke controle over het schoolsysteem eisen, wordt dit vervolgens
afgewezen als `extreem links' en `nihilistisch'. Het meest opmerkelijke
aan deze situatie is dat de ouders in beide gevallen hun gezamenlijke
wens voor lokale ouderlijke controle niet onderkennen. Ze beschuldigen
elkaar van `dweperigheid' of `militant gedrag'. Tragisch genoeg hebben
zowel de lokale blanke als de zwarte groepen hun gezamenlijke strijd
tegen de onderwijsinstelling niet herkend: de verzet tegen de
dictatoriale controle van de staat over het onderwijs van hun kinderen
door een bureaucratie die hen een onderwijsvorm opdringt waarvan zij
geloven dat deze aan de rebelse massa moet worden opgelegd. Een cruciale
taak voor libertariërs is het benadrukken van de gemeenschappelijke zaak
van alle ouders tegen de onderdrukking van onderwijs door de staat. Het
is bovendien belangrijk om te beseffen dat ouders nooit daadwerkelijk
van de staat afkomen wat betreft onderwijs, totdat het openbare
schoolsysteem volledig is afgeschaft en onderwijs weer vrij toegankelijk
wordt.

De geografische structuur van het openbare schoolsysteem heeft geleid
tot een gedwongen patroon van woonsegregatie, zowel op het gebied van
inkomen als van ras, in heel het land, en vooral in de voorsteden. Zoals
algemeen bekend is, hebben de Verenigde Staten sinds de Tweede
Wereldoorlog een bevolkingsgroei gezien, vooral in de omliggende
voorsteden en niet in de binnensteden. Nieuwe en jongere gezinnen zijn
naar de voorsteden verhuisd, en daardoor is de grootste en toenemende
belasting voor lokale begrotingen het financieren van de openbare
scholen. Deze scholen moeten namelijk een groeiende jonge bevolking met
een relatief hoog aantal kinderen per gezin onderdak bieden. Ze worden
voornamelijk gefinancierd uit de stijgende onroerendgoedbelasting die
voornamelijk wordt geheven op huizen in de buitenwijken. Dit betekent
dat hoe rijker een gezin in de voorsteden is en hoe duurder het huis,
des te groter de bijdrage aan de plaatselijke school zal zijn. Terwijl
de druk van de schoolbelastingen gestaag toeneemt, proberen de
voorstedelingen wanhopig rijke bewoners en duurdere huizen aan te
trekken, terwijl ze armere burgers proberen te ontmoedigen. Kortom, er
is een break-evenpunt voor de prijs van een huis. Boven dit punt zal een
nieuw gezin in een nieuw huis meer dan genoeg betalen voor het onderwijs
van zijn kinderen via de onroerendgoedbelasting. Gezinnen in huizen die
onder dit kostenniveau vallen, zullen niet genoeg bijdragen aan de
onroerendgoedbelasting om het onderwijs van hun kinderen te financieren,
waardoor ze een grotere belastingdruk op de bestaande bevolking in de
buitenwijk leggen. In het besef hiervan hebben de voorsteden over het
algemeen strenge bestemmingsplannen aangenomen die het bouwen van huizen
onder een minimum kostenniveau verbieden. Hiermee blokkeren ze elke
instroom van armere burgers. Aangezien het percentage arme zwarte mensen
veel hoger is dan dat van arme blanken, betekent dit ook effectief dat
zwarte mensen minder de kans krijgen om naar de voorsteden te verhuizen.
Bovendien heeft er de afgelopen jaren een toenemende verschuiving van
banen en industrie van de binnenstad naar de voorsteden plaatsgevonden.
Dit heeft geleid tot een groeiende werkloosheid onder de zwarte
bevolking, een druk die alleen maar zal toenemen naarmate deze
verschuiving zich verderzet. De afschaffing van openbare scholen, en
daarmee de loskoppeling tussen schoolbelasting en
onroerendgoedbelasting, zou een belangrijke stap zijn in de richting van
het opheffen van zoneringsrestricties en het beëindigen van de voorstad
als reservaat voor de hogere middenklasse van blanken.

\section{LASTEN EN SUBSIDIES}\label{lasten-en-subsidies}

Als woongebieden raciaal gesegregeerd zijn, wat vaak het geval is, leidt
een verplicht geografisch monopolie tot de gedwongen rassenscheiding in
openbare scholen. Ouders die kiezen voor geïntegreerd onderwijs moeten
zich verzetten tegen dit systeem. Zoals iemand ooit zei: `Wat niet
verboden is, is verplicht.' De recente tendens onder bureaucraten van
openbare scholen is niet om vrijwillig busvervoer voor kinderen in te
voeren, zodat ouders meer keuzemogelijkheden hebben. In plaats daarvan
draaien ze de andere kant op en wordt verplichte busvervoer en
verplichte raciale integratie van scholen opgelegd. Dit leidt vaak tot
absurde verplaatsingen van kinderen, ver weg van hun huis. Dit volgt
weer het typische overheidspatroon: ofwel verplichte segregatie, ofwel
verplichte integratie. De vrijwillige aanpak, waarbij ouders zelf
beslissen, wordt genegeerd door elke staatsbureaucratie. Het is
opmerkelijk dat recente bewegingen voor lokale ouderlijke controle over
openbaar onderwijs soms als `extreem rechts' en soms als `extreem links'
worden gekarakteriseerd, terwijl de libertarische motivatie in beide
gevallen precies dezelfde is. Wanneer ouders zich verzetten tegen het
verplichte busvervoer van hun kinderen naar afgelegen scholen, wordt
deze actie door het onderwijsestablishment bestempeld als
`onverdraagzaam' en `rechts'. Maar wanneer negerouders, zoals in het
geval van Ocean Hill-Brownsville in New York City, lokale ouderlijke
controle over het schoolsysteem eisen, wordt dit vervolgens afgewezen
als `extreem links' en `nihilistisch'. Het meest opmerkelijke aan deze
situatie is dat ouders in beide gevallen hun gezamenlijke wens voor
lokale ouderlijke controle niet onderkennen. Ze beschuldigen elkaar van
`dweperigheid' of `militant gedrag'. Tragisch genoeg hebben zowel de
lokale blanke als de zwarte groepen hun gezamenlijke strijd tegen de
onderwijsinstelling niet herkend: de strijd tegen de dictatoriale
controle van de staat over het onderwijs van hun kinderen door een
bureaucratie die hen een onderwijsvorm opdringt waarvan zij geloven dat
deze aan de rebelse massa moet worden opgelegd. Een belangrijke taak
voor libertariërs is het benadrukken van de gemeenschappelijke zaak van
alle ouders tegen de onderdrukking van onderwijs door de staat. Verder
is het van belang te beseffen dat ouders nooit echt van de staat afkomen
als het gaat om onderwijs, totdat het openbare schoolsysteem volledig is
afgeschaft en onderwijs weer vrij toegankelijk wordt. De geografische
structuur van het openbare schoolsysteem heeft geleid tot een gedwongen
patroon van woonsegregatie, zowel wat betreft inkomen als ras, in heel
het land, en vooral in de voorsteden. Zoals algemeen bekend is, hebben
de Verenigde Staten sinds de Tweede Wereldoorlog een bevolkingsgroei
gezien, vooral in de aangrenzende voorsteden en niet in de binnensteden.
Nieuwe en jongere gezinnen zijn naar de voorsteden verhuisd, en daardoor
is de grootste belasting voor lokale begrotingen het financieren van
openbare scholen. Deze scholen moeten immers een groeiende jonge
bevolking bedienen, met een relatief hoog aantal kinderen per gezin. Ze
worden voornamelijk gefinancierd uit de stijgende onroerendgoedbelasting
die hoofdzakelijk wordt geheven op woningen in de buitenwijken. Dit
betekent dat hoe rijker een gezin in de voorsteden is en hoe duurder het
huis, des te groter de bijdrage aan de plaatselijke school zal zijn.
Terwijl de druk van schoolbelastingen gestaag toeneemt, proberen de
voorstedelingen wanhopig rijke bewoners en duurdere woningen aan te
trekken, terwijl ze armere burgers proberen te ontmoedigen. Kort gezegd
is er een break-evenpunt voor de prijs van een huis: boven dit punt zal
een nieuw gezin in een nieuw huis meer dan genoeg betalen voor het
onderwijs van zijn kinderen via de onroerendgoedbelasting. Gezinnen in
huizen onder dit kostenniveau zullen niet genoeg bijdragen aan de
onroerendgoedbelasting om het onderwijs van hun kinderen te financieren,
waardoor ze een grotere belastingdruk op de bestaande bevolking in de
buitenwijk leggen. In dit besef hebben de voorsteden over het algemeen
strenge bestemmingsplannen aangenomen die de bouw van huizen onder een
minimumniveau verbieden. Hiermee blokkeren ze elke instroom van armere
burgers. Aangezien het percentage arme zwarte mensen veel hoger is dan
dat van arme blanken, betekent dit ook dat zwarte mensen minder kans
krijgen om naar de voorsteden te verhuizen. Bovendien heeft er de
afgelopen jaren een toenemende verschuiving van banen en industrie van
de binnenstad naar de voorsteden plaatsgevonden. Dit heeft geleid tot
een groeiende werkloosheid onder de zwarte bevolking, een druk die zal
toenemen naarmate de verschuiving van banen zich verderzet. De
afschaffing van openbare scholen, en daarmee de loskoppeling tussen
schoolbelasting en onroerendgoedbelasting, zou een belangrijke stap zijn
in de richting van het opheffen van zoneringsrestricties en het
beëindigen van de voorstad als reservaat voor de hogere middenklasse van
blanken.

Het bestaan van het openbare schoolsysteem gaat bovendien gepaard met
een complex netwerk van gedwongen heffingen en subsidies, die op
ethische gronden moeilijk te rechtvaardigen zijn. Ten eerste worden
ouders die hun kinderen naar privéscholen willen sturen, gedwongen om
een dubbele last te dragen: ze moeten niet alleen het onderwijs van hun
eigen kinderen betalen, maar ook de kinderen van openbare scholen
subsidiëren. Slechts door de duidelijke ineenstorting van het openbaar
onderwijs in de grote steden is er een bloeiend systeem van privéscholen
ontstaan. In het hoger onderwijs, waar de ineenstorting minder sterk is,
gaan privéscholen snel failliet door de concurrentie van het
belastinggefinancierde, gratis onderwijssysteem en de hoger
salarisstructuren die met belastinggeld worden gefinancierd. Omdat
openbare scholen grondwettelijk seculier moeten zijn, worden religieuze
ouders gedwongen de seculiere openbare scholen te subsidiëren. Hoewel
`scheiding van kerk en staat' een nobel principe is---en een onderdeel
van het libertarische principe om alles van de staat te scheiden---gaat
het zeker te ver om religieuzen via staatsdwang te dwingen om
niet-religieuzen te subsidiëren.

Het bestaan van de openbare school betekent ook dat ongehuwde en
kinderloze koppels verplicht worden om gezinnen met kinderen te
ondersteunen. Wat is hier het ethische principe? En nu bevolkingsgroei
niet meer zo populair is, zie eens de tegenstrijdigheid: liberale
anti-populisten pleiten voor een openbaar schoolsysteem dat niet alleen
gezinnen met kinderen subsidieert, maar dat ook doet in verhouding tot
het aantal kinderen dat zij hebben. We hoeven niet alles van de huidige
anti-populatiehysterie te onderschrijven om ons af te vragen of het
verstandig is om via overheidsmaatregelen het aantal kinderen per gezin
te subsidiëren. Dit houdt ook in dat arme alleenstaanden en kinderloze
stellen gedwongen worden om rijke gezinnen met kinderen te steunen. Is
dat ethisch verantwoord?

In de afgelopen jaren hebben openbare scholen verkondigd dat `elk kind
recht heeft op onderwijs' en dat belastingbetalers daarom gedwongen
moeten worden om dat recht te garanderen. Dit begrip van `recht' wordt
echter totaal verkeerd geïnterpreteerd. Filosofisch gezien moet een
`recht' iets zijn dat verankerd is in de menselijke natuur en de
realiteit. Het moet op elk moment en in elke tijd behouden en
gehandhaafd kunnen worden. Het `recht' op zelfbeschikking, het
verdedigen van je leven en eigendom, valt duidelijk in deze categorie:
het is van toepassing op de Neanderthalers, op het moderne Calcutta en
op de hedendaagse Verenigde Staten. Dit soort recht is onafhankelijk van
tijd of plaats. Maar een `recht op een baan', op `drie maaltijden per
dag' of op `twaalf jaar onderwijs' kan niet zo worden gegarandeerd. Wat
als zulke dingen niet kunnen bestaan, zoals in de tijd van de
Neanderthalers of in het moderne Calcutta? Spreken over een `recht' dat
alleen kan worden vervuld onder moderne industriële omstandigheden, is
helemaal niet hetzelfde als het hebben van een menselijk, natuurlijk
recht. Bovendien vereist het libertarische `recht' op zelfbeschikking
geen dwang van de ene groep mensen over de andere groep om zo'n `recht'
te waarborgen. Iedereen kan genieten van het recht op zelfbeschikking
zonder speciale dwang. Bij een `recht' op onderwijs is het anders: dit
kan alleen worden verleend als andere mensen gedwongen worden om eraan
te voldoen. Het `recht' op onderwijs, op een baan, op drie maaltijden,
enzovoort, is dan niet ingebed in de menselijke natuur. Voor de
vervulling ervan is er de noodzaak van een groep mensen die uitgebuit
worden en gedwongen zijn om zo'n `recht' te vervullen.

Bovendien moet het hele idee van een `recht op onderwijs' altijd worden
gezien in de context dat formele scholing maar een klein onderdeel is
van iemands opvoeding in het leven. Als elk kind echt een `recht' heeft
op onderwijs, waarom geldt dan niet ook een `recht' op het lezen van
kranten en tijdschriften? Waarom zou de overheid niet iedereen belasten
om gratis tijdschriften aan iedereen te verstrekken die dat wil?

Professor Milton Friedman, een econoom aan de Universiteit van Chicago,
heeft een belangrijke bijdrage geleverd door geldsommen te onderscheiden
van verschillende vormen van overheidssubsidies, zowel in het onderwijs
als op andere gebieden. Hoewel Friedman helaas het standpunt hanteert
dat elk kind via de belastingbetaler naar school zou moeten kunnen gaan,
wijst hij op de drogreden die hierin schuilt als argument voor openbare
scholen. Het is namelijk heel goed mogelijk voor belastingbetalers om
het onderwijs van elk kind te subsidiëren zonder dat er ook maar één
openbare school nodig is! In Friedmans inmiddels beroemde voucherplan
zou de overheid elke ouder een voucher geven. Deze voucher geeft het
recht om een bepaald bedrag aan schoolgeld te betalen voor elk kind, op
een school naar keuze van de ouder. Het voucherplan zou de door
belastinggeld gefinancierde onderwijsvoorziening voor elk kind
waarborgen, terwijl het tegelijkertijd de enorme monopolistische,
inefficiënte en dictatoriale openbare schoolbureaucratie zou afschaffen.
Ouders zouden hun kinderen naar elke gewenste privéschool kunnen sturen,
waardoor de keuzevrijheid voor ouders en kinderen maximaal zou zijn.
Kinderen zouden kunnen kiezen uit elk type school, of het nu gaat om een
progressieve of traditionele, religieuze of seculiere, vrij ondernemende
of socialistische school. De financiële subsidie zou dan volledig
gescheiden zijn van de daadwerkelijke verstrekking van onderwijs door de
overheid via een openbaar schoolsysteem.

Hoewel het Friedman-plan een aanzienlijke verbetering zou zijn ten
opzichte van het huidige systeem, doordat het een breder scala aan
keuzes voor ouders mogelijk maakt en de afschaffing van het openbare
schoolsysteem faciliteert, zijn er volgens libertariërs nog steeds
ernstige problemen. Ten eerste blijft de immoraliteit van gedwongen
belastingheffing voor onderwijs bestaan. Ten tweede komt de macht om te
subsidiëren onvermijdelijk samen met de macht om te reguleren en
controleren. De overheid zal immers geen vouchers verstrekken voor welk
type onderwijs dan ook. Dit betekent dat de overheid uitsluitend
vouchers zal uitbetalen voor privéscholen die door de staat als geschikt
en goedgekeurd zijn gecertificeerd. Dit houdt in dat de overheid
gedetailleerde controle uitoefent over privéscholen, zoals over hun
leerplannen, onderwijsmethoden en financieringsvormen. De invloed van de
staat op particuliere scholen, via de bevoegdheid om al dan niet
certificaten voor vouchers te verstrekken, zal zelfs groter zijn dan nu
het geval is.

Sinds de zaak Oregon zijn de voorstanders van openbare scholen niet
zover gegaan om privéscholen volledig af te schaffen, maar deze scholen
blijven op talloze manieren gereguleerd en beperkt. Elke staat bepaalt
bijvoorbeeld dat elk kind onderwijs moet krijgen op scholen die het
certificeert. Hierdoor worden de scholen gedwongen zich aan een door de
overheid gewenst curriculum aan te passen. Om als gecertificeerde
privéschool `in aanmerking te komen', moeten ze voldoen aan allerlei
zinloze en kostbare regels, zowel voor de school als voor de leraar.
Deze leraren moeten vaak een reeks nutteloze `onderwijs'-cursussen
volgen om als bevoegd te worden erkend. Veel goede privéscholen
functioneren nu technisch gezien `illegaal' omdat ze weigeren te voldoen
aan de vaak verstikkende eisen van de overheid. Misschien wel de
grootste onrechtvaardigheid is dat ouders in de meeste staten niet zelf
les mogen geven aan hun kinderen. De staat staat dit niet toe, omdat zij
vindt dat ze geen goede `school' kunnen vormen. Er zijn echter veel
ouders die meer dan gekwalificeerd zijn om hun kinderen zelf les te
geven, vooral in de lagere klassen. Bovendien zijn zij beter in staat
dan welke externe partij dan ook om de vaardigheden en het leerniveau
van elk kind te beoordelen en het onderwijs aan te passen aan de
individuele behoeften van hun kinderen. Geen enkele formele school, die
beperkt is tot uniforme klaslokalen, kan dergelijke persoonlijke
aandacht bieden.

`Gratis' scholen, of het nu gaat om huidige openbare scholen of
toekomstige scholen met vouchers, zijn natuurlijk niet echt gratis.
Iemand, namelijk de belastingbetaler, moet betalen voor de educatieve
diensten. Wanneer de dienstverlening echter losgekoppeld wordt van de
betaling, ontstaat er vaak een overaanbod van kinderen in de scholen.
Dit is ook het geval bij aanwezigheidsplichtwetten, die hetzelfde effect
hebben. Bovendien hebben kinderen vaak weinig interesse in de educatieve
diensten waarvoor hun gezin niet hoeft te betalen. Als gevolg hiervan
worden veel kinderen, die ongeschikt of ongeïnteresseerd zijn in school
en die beter af zouden zijn in een andere omgeving, naar school
gesleept. Ze blijven daar vaak veel langer dan nodig is. Deze obsessie
voor massaal onderwijs heeft geleid tot een grote groep ontevreden en
gevangen kinderen, samen met de algemene opvatting dat iedereen de
middelbare school (of zelfs de universiteit) moet afmaken om waardig te
zijn op de arbeidsmarkt. Deze druk is verder versterkt door de
hysterische groei van `antidrop-out' propaganda in de massamedia. Een
deel van deze ontwikkeling is te wijten aan het bedrijfsleven.
Werkgevers willen hun werknemers namelijk graag opleiden, maar dan niet
op hun eigen kosten of op het werk. In plaats daarvan geven ze de kosten
door aan de ongelukkige belastingbetaler. Hoe groot is het aandeel van
de massale openbare scholing aan de kosten die werkgevers voor de
scholing van hun werknemers op de belastingbetaler afwentelen?

Men zou verwachten dat deze opleiding, die voor werkgevers geen kosten
met zich meebrengt, zeer duur, inefficiënt en veel te lang zal zijn. Er
is echter steeds meer bewijs dat een groot deel van de huidige scholing
niet noodzakelijk is voor productieve werkgelegenheid. Zoals Arthur
Stinchcombe vraagt:

\begin{quote}
Is er iets dat een middelbare school kan bieden waar werkgevers in de
handarbeid voor willen betalen, als het goed wordt onderwezen? Over het
algemeen is het antwoord nee. Fysieke vaardigheden en betrouwbaarheid,
de twee belangrijkste factoren die werkgevers in de handarbeid
waarderen, worden niet sterk verbeterd door het onderwijs. Werkgevers
die betrouwbare werknemers zoeken, kunnen een middelbareschooldiploma
als bewijs van goede discipline vereisen. Voor de rest kunnen ze beter
en goedkoper werknemers opleiden op de werkvloer dan terwijl ze op
school zitten.
\end{quote}

En, zoals professor Banfield opmerkt, de meeste beroepsvaardigheden
worden al op de werkplek aangeleerd.

De relatieve nutteloosheid van het openbare schoolsysteem voor het
opleiden van handarbeiders blijkt uit het fascinerende werk van MIND,
een particuliere educatieve dienst die momenteel wordt geleid door de
Corn Products Refining Company uit Greenwich, Connecticut. MIND richtte
zich bewust op voortijdige schoolverlaters zonder kwalificaties voor
handarbeid. In slechts enkele weken, dankzij intensieve training en
lesmachines, kon het deze jongeren basisvaardigheden en typografie
bijbrengen. Hierdoor konden ze een plek in het bedrijfsleven veroveren.
Tien jaar openbaar onderwijs had hen minder geleerd dan een paar weken
privé, beroepsgerichte training! Jongeren de kans geven om uit de
afhankelijkheid te stappen en zelfstandig te worden, kan enorm veel
voordelen opleveren, zowel voor henzelf als voor de samenleving.

Er is veel bewijs dat de leerplichtwetten samenhangen met het groeiende
probleem van jeugddelinquentie, vooral bij gefrustreerde oudere
kinderen. Stinchcombe ontdekte dat opstandig en delinquent gedrag
`grotendeels een reactie is op de school zelf'. Ook ontdekte het Britse
Crowther Committee dat, toen in 1947 de regering de minimumleeftijd om
van school te gaan verhoogde van veertien naar vijftien jaar, er een
onmiddellijke en sterke toename was in de delinquentie onder de net
afgestudeerde veertienjarigen.20

Een deel van de verantwoordelijkheid voor de opkomstplicht en de massale
openbare scholing ligt bij de vakbonden. Om de concurrentie van jonge,
adolescenten werknemers te verminderen, proberen ze jongeren zo lang
mogelijk uit de arbeidsmarkt te houden en in onderwijsinstellingen te
dwingen. Zowel de vakbonden als de werkgevers oefenen dus sterke druk
uit op de leerplicht, waardoor de meeste jongeren in het land niet aan
het werk zijn.

\section{HOGER ONDERWIJS}\label{hoger-onderwijs}

De relatieve nutteloosheid van het openbare schoolsysteem voor het
opleiden van handarbeiders blijkt uit het fascinerende werk van MIND,
een particuliere educatieve dienst die momenteel wordt geleid door de
Corn Products Refining Company uit Greenwich, Connecticut. MIND richtte
zich bewust op voortijdige schoolverlaters zonder kwalificaties voor
handarbeid. In slechts enkele weken, dankzij intensieve training en
lesmachines, kon het deze jongeren basisvaardigheden en typografie
bijbrengen, waardoor ze een plek in het bedrijfsleven konden veroveren.
Tien jaar openbaar onderwijs had hen minder geleerd dan een paar weken
privé, beroepsgerichte training! Jongeren de kans geven om uit de
afhankelijkheid te stappen en zelfstandig te worden, kan enorm veel
voordelen opleveren, zowel voor henzelf als voor de samenleving. Er is
veel bewijs dat de leerplichtwetten samenhangen met het groeiende
probleem van jeugddelinquentie, vooral bij gefrustreerde oudere
kinderen. Stinchcombe ontdekte dat opstandig en delinquent gedrag
`grotendeels een reactie is op de school zelf'. Ook ontdekte het Britse
Crowther Committee dat, toen in 1947 de regering de minimumleeftijd om
van school te gaan verhoogde van veertien naar vijftien jaar, er een
onmiddellijke en sterke toename was in de delinquentie onder de net
afgestudeerde veertienjarigen. Een deel van de verantwoordelijkheid voor
de opkomstplicht en de massale openbare scholing ligt bij de vakbonden.
Om de concurrentie van jonge, adolescenten werknemers te verminderen,
proberen ze jongeren zo lang mogelijk uit de arbeidsmarkt te houden en
in onderwijsinstellingen te dwingen. Zowel de vakbonden als de
werkgevers oefenen sterke druk uit op de leerplicht, waardoor de meeste
jongeren in het land niet aan het werk zijn.

Met uitzondering van de gevolgen van de leerplichtwetten, kunnen
dezelfde kritieken die we op openbare scholen hebben geuit, ook van
toepassing zijn op het openbaar hoger onderwijs, met één belangrijke
toevoeging. Er is steeds meer bewijs dat bij openbaar hoger onderwijs de
gedwongen subsidies vooral ten goede komen aan de rijkeren, terwijl
armere burgers deze subsidiëren. Er zijn drie belangrijke redenen
hiervoor. Ten eerste is de belastingstructuur voor scholen niet echt
`progressief', wat betekent dat de rijkeren niet in grotere mate worden
belast. Ten tweede hebben de kinderen die naar de universiteit gaan
doorgaans rijkere ouders dan degenen die dat niet doen. Ten derde zullen
de kinderen die universitair onderwijs volgen, als gevolg daarvan een
hoger levenslang inkomen verwerven dan hun leeftijdsgenoten die niet
naar de universiteit gaan. Dit leidt tot een netto herverdeling van
inkomen van de armen naar de rijken via het openbaar hoger onderwijs.
Waar is hier de ethische rechtvaardiging?

De professoren Weisbrod en Hansen hebben dit herverdelingseffect
aangetoond in hun onderzoeken naar openbaar hoger onderwijs in Wisconsin
en Californië. Ze ontdekten bijvoorbeeld dat het gemiddelde
gezinsinkomen van inwoners zonder kinderen aan de staatsuniversiteiten
van Wisconsin in 1964-1965 \$6.500 was. In tegenstelling daarmee bedroeg
het gemiddelde gezinsinkomen van gezinnen met kinderen aan de
Universiteit van Wisconsin \$9.700. In Californië waren de cijfers
respectievelijk \$7.900 en \$12.000. Het verschil in subsidies was hier
nog groter omdat de belastingstructuur in deze staat veel minder
`progressief' was. Douglas Windham vond een vergelijkbaar
herverdelingseffect van arm naar rijk in de staat Florida. Hansen en
Weisbrod concludeerden op basis van hun studie in Californië:

\begin{quote}
Over het geheel genomen zorgen deze subsidies ervoor dat de ongelijkheid
tussen mensen met verschillende sociale en economische achtergronden
eerder toeneemt dan afneemt. Dit komt doordat er aanzienlijke subsidies
beschikbaar zijn waarvoor gezinnen met een lager inkomen vaak niet in
aanmerking komen of geen gebruik van kunnen maken vanwege voorwaarden en
beperkingen die verband houden met hun inkomenspositie.

Wat we in Californië hebben ontdekt - een extreem ongelijke verdeling
van subsidies in het openbaar hoger onderwijs - geldt waarschijnlijk nog
sterker voor andere staten. Geen enkele staat heeft zo veel lokale
junior colleges als Californië, en daardoor gaat er in geen enkele
andere staat een zo groot percentage van de middelbare
schoolafgestudeerden naar het openbaar hoger onderwijs. We kunnen daarom
met zekerheid stellen dat in Californië een kleiner percentage jongeren
geen subsidie ontvangt dan in andere staten.
\end{quote}

Bovendien brengen de staten privéscholen niet alleen financieel in
gevaar door hun oneerlijke, fiscaal gesubsidieerde concurrentie, maar
dwingen ze ook strenge controles op het private hoger onderwijs door
middel van verschillende voorschriften. In de staat New York mag niemand
een instelling oprichten die `hogeschool' of `universiteit' genoemd
wordt, tenzij hij een borgsom van \$500.000 bij de staat stort. Dit
discrimineert duidelijk kleine, armere onderwijsinstellingen en houdt
hen effectief uit het hoger onderwijs. Daarnaast kunnen regionale
verenigingen van hogescholen, door hun accreditatierechten, elke
hogeschool die niet voldoet aan hun normen voor curriculum of
financiering uitsluiten. Deze verenigingen weigeren bijvoorbeeld
resoluut om een universiteit te accrediteren, ongeacht hoe goed het
onderwijs is, als deze privaat of met winstoogmerk opereert in plaats
van onder toezicht van een bestuurder. Omdat private hogescholen veel
meer gemotiveerd zijn om efficiënt te werken en de consument te
bedienen, zijn ze doorgaans financieel succesvoller. In de afgelopen
jaren werd het succesvolle Marjorie Webster Junior College in
Washington, D.C., bijna failliet verklaard doordat de regionale
associatie weigerde accreditering te verlenen. Hoewel je zou kunnen
zeggen dat deze regionale verenigingen privaat zijn en niet publiek,
werken ze nauw samen met de federale overheid. Deze weigert bijvoorbeeld
gebruikelijke studiebeurzen of GI-uitkeringen te verstrekken aan
niet-geaccrediteerde hogescholen.

Overheidsdiscriminatie van private scholen, en ook van andere
instellingen, beperkt zich niet tot accreditatie en studiebeurzen. De
gehele structuur van de inkomstenbelasting discrimineert hen zelfs nog
meer. Door trust-universiteiten vrij te stellen van inkomstenbelasting
en zware belastingen op winstoogmerkinstellingen te heffen, houden de
federale en staatsregeringen de meest efficiënte en rendabele vorm van
privéonderwijs tegen. De libertarische oplossing voor deze ongelijkheid
is niet om privéscholen gelijk te belasten, maar om de belastingdruk op
hen te verlichten. De libertarische ethiek beoogt niet om iedereen
dezelfde slavernij op te leggen, maar om gelijke vrijheid te waarborgen.

Trustee-governance is over het algemeen een slechte manier om een
instelling te besturen. Ten eerste, in tegenstelling tot winstgevende
bedrijven, partnerschappen of ondernemingen, is de organisatie die door
trustees wordt bestuurd niet volledig in handen van één persoon. De
trustees kunnen geen winst behalen uit het succesvolle functioneren van
de organisatie, waardoor ze weinig stimulans hebben om efficiënt te
werken of hun klanten goed te bedienen. Zolang de hogeschool of andere
organisatie geen buitensporige verliezen heeft, kan deze doorgaan op een
laag prestatieniveau. Aangezien de bestuurders geen winst kunnen maken
door hun klanten beter van dienst te zijn, zijn ze vaak geneigd om laks
te functioneren. Bovendien worden ze in hun financiële efficiëntie
belemmerd door de voorwaarden in hun statuten. Zo is het bestuur van een
hogeschool bijvoorbeeld verboden om de instelling te redden door een
deel van de campus om te vormen tot een commerciële onderneming, zoals
een parkeerplaats met winstoogmerk.

Het tekortschieten ten opzichte van de klanten wordt nog verergerd bij
de huidige trustscholen. Studenten betalen namelijk slechts een klein
deel van de kosten van hun opleiding, terwijl het grootste gedeelte
wordt gefinancierd door subsidies of schenkingen. Deze situatie wijkt
sterk af van de gebruikelijke marktomstandigheden, waarin producenten
hun producten verkopen en consumenten het volledige bedrag betalen. De
scheiding tussen dienst en betaling leidt tot een onbevredigende
situatie voor iedereen. Consumenten voelen bijvoorbeeld dat de managers
de touwtjes in handen hebben. Een libertair merkte tijdens de
studentenrellen aan het eind van de jaren zestig op: `Zit niemand in
Berlitz'. Bovendien zijn de `consumenten' in feite de overheden,
stichtingen of alumni die het grootste deel van de rekening betalen.
Hierdoor wordt het hoger onderwijs onvermijdelijk afgestemd op hun
eisen, in plaats van op de opleiding van studenten. Zoals professoren
Buchanan en Devletoglou stellen:

\begin{quote}
De tussenkomst van de overheid tussen universiteiten en hun studenten
heeft geleid tot een situatie waarin universiteiten niet kunnen voldoen
aan de vraag en bovendien geen directe middelen kunnen aanboren om aan
de wensen van studenten tegemoet te komen. Om aan deze middelen te
komen, moeten universiteiten concurreren met andere door belasting
gefinancierde activiteiten, zoals het leger, lagere scholen en
welzijnsprogramma's. Hierdoor wordt de vraag van de student verwaarloosd
en draagt de resulterende onvrede bij aan de chaos die we nu waarnemen.
De toenemende afhankelijkheid van financiële steun van de overheid,
zoals het aanbieden van gratis collegegeld, kan zelf ook een belangrijke
bron zijn van de huidige onrust.
\end{quote}

Het libertarische recept voor onze onderwijsproblemen kan eenvoudig
worden samengevat: haal de overheid uit het onderwijs. De overheid heeft
geprobeerd de jeugd van ons land te indoctrineren en te vormen via het
openbare schoolsysteem. Daarnaast probeert ze de toekomstige leiders te
beïnvloeden door het hoger onderwijs te beheren en controleren.
Afschaffing van de leerplicht zou de rol van de school als gevangenis
voor de jeugd beëindigen. Dit zou iedereen die beter af is buiten de
school, de ruimte geven voor onafhankelijkheid en een productief leven.
Het afschaffen van openbare scholen zou ook een einde maken aan de zware
belastingdruk op onroerend goed en zou een breed scala aan onderwijs
bieden dat aansluit bij de diverse behoeften van onze bevolking.
Bovendien zou het afschaffen van openbaar onderwijs een eind maken aan
de onterechte subsidies die onder dwang naar grote gezinnen gaan, vaak
ten voordele van de hogere klassen en ten koste van de armen. Het
huidige systeem, waarin de overheid de jeugd van Amerika in de door de
staat gewenste richting probeert te sturen, zou plaatsmaken voor vrij
gekozen en vrijwillige opties. Kortom, we zouden kunnen streven naar een
echt vrije opvoeding, zowel binnen als buiten de formele scholen.

\bookmarksetup{startatroot}

\chapter{Welzijn en de
verzorgingsstaat}\label{welzijn-en-de-verzorgingsstaat}

\section{WAAROM DE WELVAARTSCRISIS?}\label{waarom-de-welvaartscrisis}

Bijna iedereen, ongeacht hun ideologie, is het erover eens dat er iets
ernstig mis is met het op hol geslagen welvaartssysteem in de Verenigde
Staten. Dit systeem zorgt ervoor dat een steeds groter deel van de
bevolking leeft als inactieve, verplichte eisers op de productie van de
rest van de samenleving. Enkele cijfers en vergelijkingen zullen de
verschillende aspecten van dit probleem schetsen. In 1934, te midden van
de grootste depressie in de Amerikaanse geschiedenis en op een
dieptepunt in ons economische leven, bedroegen de totale uitgaven van de
overheid aan sociale voorzieningen 5,8 miljard dollar. Daarvan was 2,5
miljard dollar bestemd voor directe sociale uitkeringen (`publieke
hulp'). In 1976, na vier decennia van de grootste bloei in de
Amerikaanse geschiedenis en op een moment dat we de hoogste
levensstandaard ter wereld hadden bereikt met een relatief laag
werkloosheidsniveau, stegen de totale sociale overheidsuitgaven tot
331,4 miljard dollar. Hierin waren 48,9 miljard dollar aan directe
bijstandsuitkeringen inbegrepen. Samenvattend, de totale uitgaven aan
sociale voorzieningen stegen in deze vier decennia met maar liefst 5.614
procent, terwijl de directe sociale uitkeringen met 1.856 procent
toenamen. Anders gezegd, tussen 1934 en 1976 stegen de uitgaven aan
sociale voorzieningen gemiddeld met 133,7 procent per jaar, terwijl de
directe sociale uitkeringen gemiddeld met 44,2 procent per jaar
toenamen.

Als we kijken naar de directe welzijnssector, zien we dat de uitgaven
van 1934 tot 1950 ongeveer gelijkbleven. Daarna schoten ze omhoog, samen
met de naoorlogse economische opleving. Van de jaren vijftig tot 1976
steeg de bijstand met maar liefst 84,4 procent per jaar.

Een deel van deze enorme stijgingen kan worden verklaard door inflatie,
die de waarde en koopkracht van de dollar heeft verminderd. Wanneer we
alle cijfers corrigeren voor inflatie door ze om te rekenen naar
`constante 1958-dollars' (dat wil zeggen dat elke dollar dezelfde
koopkracht heeft als in 1958), komen we op de volgende relevante
cijfers: - In 1934 waren de totale uitgaven aan sociale voorzieningen
13,7 miljard dollar en de directe bijstand 5,9 miljard dollar. - In 1976
waren de totale uitgaven aan sociale voorzieningen 247,7 miljard dollar
en de directe bijstand 36,5 miljard dollar.

Zelfs na correctie voor inflatie, stegen de sociale uitgaven van de
overheid in deze 42 jaar met maar liefst 1.798 procent, wat neerkomt op
42,8 procent per jaar. Ook de directe bijstand nam toe met 519 procent,
ofwel 12,4 procent per jaar. Daarnaast, als we de cijfers voor directe
bijstand in 1950 en 1976 bekijken, gecorrigeerd voor inflatie, dan zien
we dat de welzijnsuitgaven in die tussenliggende jaren stegen met 1.077
procent, wat overeenkomt met 41,4 procent per jaar.

Als we de cijfers corrigeren voor bevolkingsgroei (de totale Amerikaanse
bevolking was 126 miljoen in 1934 en 215 miljoen in 1976), zien we nog
steeds een bijna vertienvoudiging van de totale uitgaven voor sociale
voorzieningen. Deze stegen van \$ 108 naar \$ 1.152 per hoofd van de
bevolking in constante 1958-dollars. Ook de directe bijstand neemt toe
met meer dan een verdrievoudiging, van \$ 47 in 1934 naar \$ 170 per
hoofd van de bevolking in 1976.

Nog een paar vergelijkingen: van 1955 tot 1976, de jaren van grote
welvaart, vervijfvoudigde het totale aantal mensen dat afhankelijk was
van bijstand, van 2,2 miljoen naar 11,2 miljoen. Tussen 1952 en 1970
steeg het aantal kinderen van 18 jaar en jonger met 42 procent, maar het
aantal bijstandstrekkers nam met 400 procent toe. Hoewel de totale
bevolking niet veranderde, steeg het aantal bijstandsgerechtigden in New
York City van 330.000 in 1960 naar 1,2 miljoen in 1971. Dit wijst
duidelijk op een welvaartscrisis.¹

De crisis blijkt nog veel groter te zijn als we alle sociale steun aan
de armen meerekenen bij de `sociale uitkeringen'. Tussen 1960 en 1969
verdrievoudigde de federale `hulp aan de armen', van \$ 9,5 miljard naar
\$ 27,7 miljard. De staats- en lokale uitgaven voor sociale
voorzieningen stegen van \$ 3,3 miljard in 1935 tot \$ 46 miljard, wat
neerkomt op een stijging van 1.300 procent! De totale uitgaven voor
sociale voorzieningen in 1969, inclusief federale, staats- en lokale
bijdragen, beliepen maar liefst \$ 73,7 miljard.

De meeste mensen zien het hebben van een uitkering als een proces dat
buiten de klanten zelf omgaat, bijna als een natuurlijke ramp, zoals een
vloedgolf of vulkaanuitbarsting. Dit gebeurt buiten de wil van de mensen
die een uitkering ontvangen. Het gebruikelijke gezegde is dat `armoede'
de reden is dat individuen of gezinnen in de bijstand terechtkomen. Maar
ongeacht hoe men armoede definieert en op welk inkomensniveau men zich
baseert, is het niet te ontkennen dat het aantal mensen of gezinnen
onder de `armoedegrens' sinds de jaren dertig gestaag is afgenomen, en
niet andersom. De omvang van de armoede kan dus nauwelijks de
spectaculaire groei van het aantal bijstandscliënten verklaren.

De oplossing van het raadsel wordt duidelijk als je je realiseert dat
het aantal bijstandsontvangers een `positieve aanbodfunctie' heeft,
zoals dat in de economie wordt genoemd. Met andere woorden: zodra de
prikkels om in de bijstand te gaan toenemen, zal ook het aantal
bijstandsontvangers stijgen. Een vergelijkbaar effect vindt plaats
wanneer de negatieve prikkels om in de bijstand te gaan afnemen. Vreemd
genoeg betwist niemand deze conclusie in andere delen van de economie.
Neem bijvoorbeeld de situatie waarin iemand, of dat nu de overheid is of
een miljardair, een extra \$ 10.000 aanbiedt aan iedereen die in een
schoenenfabriek wil werken. Het is duidelijk dat daardoor het aantal
geïnteresseerde werknemers in de schoenenindustrie zal toenemen. Dit zal
ook gebeuren als de negatieve prikkels worden verlaagd, bijvoorbeeld als
de overheid belooft dat alle schoenenarbeiders geen inkomstenbelasting
hoeven te betalen. Als we dezelfde analyse toepassen op
bijstandscliënten zoals we dat doen in de rest van de economie, wordt
het antwoord op de welzijnspuzzel glashelder.

Wat zijn de belangrijke prikkels en afschrikkingen om in de bijstand te
gaan, en hoe zijn die veranderd? Een cruciale factor is de verhouding
tussen het inkomen uit een uitkering en het inkomen uit productieve
arbeid. Stel, ter vereenvoudiging, dat het `gemiddelde' loon voor een
`gemiddelde' arbeider in een bepaalde regio \$ 7.000 per jaar bedraagt.
Laten we aannemen dat het inkomen uit de bijstand \$ 3.000 per jaar is.
Dit betekent dat de gemiddelde nettowinst uit werken (vóór belasting) \$
4.000 per jaar is. Als de uitkeringen stijgen tot \$ 5.000, of als het
gemiddelde loon zakt tot \$ 5.000, dan halveert het verschil --- de
nettowinst die je maakt door te werken --- van \$ 4.000 naar \$ 2.000
per jaar. Het is evident dat dit leidt tot een enorme toename in het
aantal bijstandsuitkeringen. Dit effect wordt nog versterkt als we in
gedachten houden dat werknemers met een inkomen van \$ 7.000 hogere
belastingen moeten betalen om een omvangrijke groep bijstandsontvangers
te onderhouden die vrijwel geen belasting betaalt. We zouden verwachten
dat, omdat in de afgelopen periode de uitkeringsniveaus sneller zijn
gestegen dan de gemiddelde lonen, steeds meer mensen in de bijstand
zullen terechtkomen. Deze trend wordt nog duidelijker als we bedenken
dat niet iedereen het `gemiddelde' inkomen verdient. Vooral `marginale'
werknemers, degenen die onder het gemiddelde inkomen zitten, zullen in
grote getale in de bijstand belanden. In ons voorbeeld, als de
bijstandsuitkering stijgt tot \$ 5.000 per jaar, wat mogen we dan
verwachten van werknemers die \$ 4.000, \$ 5.000 of zelfs \$ 6.000
verdienen? De werknemer die voorheen \$ 5.000 per jaar verdiende en
netto \$ 2.000 meer opbracht dan de bijstandsontvanger, ziet nu dat hun
verschil is verdwenen. Hij verdient nu niet meer --- en zelfs minder, na
aftrek van belastingen --- dan de bijstandscliënt die dankzij de staat
in luiheid wordt onderhouden. Is het dan vreemd dat hij overweegt om ook
naar de bijstandsvoorzieningen te overstappen?

In de periode van 1952 tot 1970 vervijfvoudigde het aantal
bijstandsuitkeringen, van 2 naar 10 miljoen. Tegelijkertijd steeg de
gemiddelde maandelijkse uitkering van een bijstandsgezin meer dan
verdubbeld, van 82 naar 187 dollar. Dit was een toename van bijna 130
procent, terwijl de consumentenprijzen slechts met 50 procent stegen.
Bovendien vergeleek de Citizens Budget Commission van New York City in
1968 de tien staten met de snelst groeiende bijstandsgroepen met de tien
staten met de laagste groei. De commissie ontdekte dat de gemiddelde
maandelijkse bijstandsuitkering in de tien snelst groeiende staten twee
keer zo hoog was als in de tien langzaamste staten. In de eerste groep
staten bedroeg de maandelijkse uitkering per persoon gemiddeld 177
dollar, terwijl dit in de laatste groep slechts 88 dollar was.

Een ander voorbeeld van de impact van hoge uitkeringen in verhouding tot
het loon dat je kunt verdienen door te werken, werd aangehaald door de
McCone Commissie, die de rellen in Watts in 1965 onderzocht. De
Commissie ontdekte dat een baan met het minimumloon ongeveer 220 dollar
per maand opleverde. Hier moesten echter werkgerelateerde uitgaven zoals
kleding en vervoer van worden betaald. In tegenstelling daarmee ontving
het gemiddelde bijstandsgezin in het gebied tussen de 177 en 238 dollar
per maand, zonder dat daar werkgerelateerde kosten van afgetrokken
hoefden te worden.

Een andere belangrijke factor voor de toename van het aantal
bijstandsuitkeringen is de afname van de sterke negatieve prikkels om in
de bijstand te gaan. Het meest significante obstakel was altijd het
stigma dat mensen met een uitkering voelden: het idee dat zij
parasiteren en leven van wat anderen produceren, in plaats van zelf bij
te dragen. Dit stigma is echter sociaal verzwakt door de opkomst van
moderne liberale waarden. Bovendien hebben overheidsinstanties en
maatschappelijk werkers de laatste jaren de rode loper uitgerold voor
mensen om hen te verwelkomen en zelfs aan te moedigen om zo snel
mogelijk in de bijstand te komen. De traditionele visie van de
maatschappelijk werker was gericht op het helpen van mensen om zichzelf
te helpen en hun onafhankelijkheid te behouden. Het doel was om
bijstandsontvangers zo snel mogelijk van de uitkering af te krijgen.
Tegenwoordig is het doel van maatschappelijk werkers echter om zoveel
mogelijk mensen in de bijstand te krijgen, om hen te informeren over hun
`rechten' en hen te ondersteunen bij het in stand houden van hun
situatie. Dit heeft geleid tot een voortdurende versoepeling van de
voorwaarden om in aanmerking te komen voor een uitkering. De
bureaucratie is verminderd en handhaving van verblijfs-, werk- of
inkomensvereisten om een uitkering te krijgen, is vrijwel verdwenen.
Iedereen die, hoe gering ook, suggereert dat bijstandsontvangers
verplicht zouden moeten worden om werk te accepteren en van de bijstand
af te komen, wordt beschouwd als een reactionaire morele melaatse. Nu
het oude stigma steeds meer vervaagt, zijn mensen steeds geneigder om
snel voor de bijstand te kiezen, in plaats van ervoor terug te deinzen.
Irving Kristol heeft scherpzinnig geschreven over de `welfare explosion'
van de jaren zestig:

\begin{quote}
Deze `explosie' werd - deels opzettelijk, deels onbewust - veroorzaakt
door overheidsfunctionarissen en ambtenaren die beleid uitvoerden als
onderdeel van een `oorlog tegen armoede'. Interessant is dat veel van
dezelfde mensen die deze maatregelen bepleitten en uitvoerden, later
verbijsterd waren over de `bijstandsexplosie'. Het is dan ook niet
verwonderlijk dat het enige tijd duurde voordat ze zich realiseerden dat
het probleem dat ze probeerden op te lossen, eigenlijk het probleem was
dat ze zelf hadden gecreëerd.
\end{quote}

Hier zijn de redenen achter de `welfare explosion' van de jaren zestig:

\begin{quote}
Het aantal mensen dat in armoede leeft en recht heeft op een uitkering
zal toenemen naarmate de officiële definities van `armoede' en
`behoeftigheid' worden verhoogd. De Oorlog tegen Armoede heeft deze
officiële definities verhoogd, wat automatisch leidde tot een toename
van het aantal `rechthebbenden'.

\begin{enumerate}
\def\labelenumi{\arabic{enumi}.}
\setcounter{enumi}{1}
\tightlist
\item
  Het aantal armen dat in aanmerking komt voor een uitkering en
  daadwerkelijk een aanvraag indient, neemt toe naarmate de uitkeringen
  stijgen, zoals gebeurde in de jaren zestig. Wanneer
  bijstandsuitkeringen, inclusief gerelateerde voordelen zoals
  ziektekostenverzekering en voedselbonnen, concurreren met lage lonen,
  kiezen veel arme mensen rationeel voor bijstand. In New York, net als
  in andere grote steden, concurreren bijstandsuitkeringen niet alleen
  met lage lonen; ze overtreffen ze zelfs.
\end{enumerate}

De terughoudendheid van mensen die recht hebben op een uitkering om deze
aan te vragen, zal afnemen. Deze terughoudendheid is vaak gebaseerd op
trots, onwetendheid of angst. Wanneer er een georganiseerde campagne
wordt opgezet om hen aan te moedigen zich aan te melden, zal dit effect
hebben. In de jaren zestig werd zo'n campagne met succes gelanceerd door
(a) verschillende gemeenschapsorganisaties die gesponsord en
gefinancierd werden door het Office of Economic Opportunity, (b) de
Welfare Rights Movement en (c) de sociale beroepsgroep. Deze
beroepsgroep bestond nu uit afgestudeerden die het als hun morele plicht
zagen om mensen te helpen bij het krijgen van een uitkering, in plaats
van, zoals voorheen, hen te ondersteunen bij het afkomen van een
uitkering. Bovendien droegen de rechtbanken bij door verschillende
wettelijke obstakels, zoals verblijfsvereisten, te verwijderen.

Op de een of andere manier lijkt de stijging van het aantal arme mensen
in de bijstand, die bovendien meer genereuze uitkeringen ontvangen, dit
land geen prettige plek om te wonen te maken. Zelfs voor de mensen in de
bijstand lijkt hun situatie niet merkbaar beter dan toen ze arm waren en
geen bijstand kregen. Er lijkt iets mis te zijn gegaan. Een liberaal en
meelevend sociaal beleid heeft allerlei onverwachte en perverse gevolgen
gehad.4
\end{quote}

De mentaliteit die het beroep van maatschappelijk werk droeg, was heel
anders - het had een libertarische inslag. Er waren twee hoofdprincipes:
(a) alle hulp- en bijstandsuitkeringen zouden vrijwillig moeten zijn,
uitgevoerd door particuliere instellingen in plaats van door verplichte
belastingheffing van de overheid; en (b) het doel van geven moest zijn
om de ontvanger zo snel mogelijk onafhankelijk en productief te maken.
Logisch gezien volgt (b) uit (a), omdat geen enkele particuliere
instantie over de vrijwel onbeperkte middelen beschikt die kunnen worden
verzameld van de geduldige belastingbetaler. Aangezien particuliere
fondsen strikt beperkt zijn, is er geen ruimte voor het idee van
welzijns'rechten' als een onbeperkte en permanente claim op de productie
van anderen. Een ander gevolg van de beperkte fondsen was dat sociale
werkers zich realiseerden dat er geen ruimte was voor hulp aan mensen
die slecht werkten, weigerden te werken of de hulp misbruikten; vandaar
het concept van de `verdienende' armen versus de `onverdiende' armen. Zo
rekende de 19e-eeuwse Engelse laissez-faire instelling, de Charity
Organisation Society, onder de armen die het niet verdienden degenen die
geen hulp nodig hadden, bedriegers en de man wiens toestand 'te wijten
is aan ondoordachtheid of spaarzaamheid en waarbij er geen hoop is om
hem in de toekomst onafhankelijk te maken van liefdadige hulp.'5

Hoewel het Engelse laissez-faire liberalisme over het algemeen de `Poor
Law' als overheidszorg accepteerde, was het standpunt dat dit een sterk
ontmoedigend effect zou hebben. Er zouden niet alleen strikte regels
moeten zijn om in aanmerking te komen voor hulp, maar de omstandigheden
in de werkhuizen zouden ook zo onplezierig moeten zijn dat de opvang in
het werkhuis eerder afschrikt dan aantrekkelijk maakt. Voor de `armen
die het niet verdienen', die verantwoordelijk zijn voor hun eigen
situatie, kon misbruik van het hulpsysteem alleen worden tegengegaan
door het zo onaantrekkelijk mogelijk te maken voor de aanvragers. Dit
betekende doorgaans dat men aandrong op een arbeidstest of verblijf in
een werkhuis.6

Hoewel een streng afschrikmiddel veel effectiever is dan een open aanbod
en een preek over de `rechten' van ontvangers, pleit het libertarische
standpunt voor de volledige afschaffing van de overheidswelvaart. Het
richt zich op particuliere liefdadigheid, waarbij de nadruk ligt op het
zo snel mogelijk helpen van de `verdienstelijke armen' naar
onafhankelijkheid. Tot de Depressie in de jaren dertig was er immers
weinig tot geen overheidswelzijn in de Verenigde Staten. En in die tijd,
met een veel lagere algemene levensstandaard, werd er geen massale
hongersnood op straat ervaren. Een uitstekend voorbeeld van een
succesvol particulier welzijnsprogramma van tegenwoordig is dat van de
mormoonse kerk, die drie miljoen leden telt. Dit opmerkelijke volk, dat
te maken kreeg met armoede en vervolging, emigreerde in de 19e eeuw naar
Utah en omringende staten. Zij tilde zichzelf naar een algemeen niveau
van welvaart en rijkdom door middel van spaarzaamheid en hard werken.
Slechts heel weinig mormonen ontvangen een uitkering; hen wordt geleerd
om onafhankelijk en zelfredzaam te zijn en om de overheid zoveel
mogelijk te vermijden. Mormonen zijn gelovig en hebben deze
bewonderenswaardige waarden met succes geïnternaliseerd. Bovendien heeft
de mormoonse kerk een uitgebreid particulier welzijnsplan voor haar
leden, opnieuw met de focus om hen zo snel mogelijk naar
onafhankelijkheid te begeleiden.

Let bijvoorbeeld op de volgende principes uit het welzijnsplan van de
mormoonse kerk.

\begin{quote}
Sinds de oprichting in 1830 heeft de kerk haar leden gestimuleerd om
economische onafhankelijkheid op te bouwen en te behouden. Ze heeft
spaarzaamheid bevorderd en de ontwikkeling van werkgelegenheid
scheppende industrieën aangemoedigd. Daarnaast staat de kerk altijd
klaar om behoeftige trouwe leden te helpen.
\end{quote}

In 1936 ontwikkelde de mormoonse kerk een welzijnsprogramma.

\begin{quote}
Kerkelijk Welzijnsplan. Een systeem dat de vloek van het nietsdoen moet
uitbannen, het kwaad van de bijstand moet afschaffen en
onafhankelijkheid, industrie, spaarzaamheid en zelfrespect weer op moet
bouwen onder onze mensen. Het doel van de kerk is om mensen te helpen
zichzelf te helpen. Werk moet het leidende principe zijn in het leven
van onze kerkleden.
\end{quote}

De maatschappelijk werkers van het programma zijn instructies gegeven om
op deze manier te handelen:

\begin{quote}
Trouw aan dit principe zullen welzijnswerkers de leden van de kerk
aanmoedigen en ondersteunen om zoveel mogelijk voor hun eigen onderhoud
te zorgen. Geen enkele ware Heilige der Laatste Dagen zal, zolang hij
daartoe lichamelijk in staat is, vrijwillig de verantwoordelijkheid voor
zijn eigen levensonderhoud van zich afschuiven. Zolang hij kan, zal hij
onder inspiratie van de Almachtige en met zijn eigen arbeid in zijn
levensbehoeften voorzien.
\end{quote}

De directe doelstellingen van het welzijnsprogramma zijn:

\begin{quote}
\begin{enumerate}
\def\labelenumi{\arabic{enumi}.}
\item
  Mensen die kunnen werken, ondersteunen bij het vinden van betaald
  werk.
\item
  Werkgelegenheid bieden binnen het welzijnsprogramma, daar waar
  mogelijk, voor degenen die niet in betaald werk kunnen worden
  geplaatst.
\item
  Het verkrijgen van middelen om de mensen in nood, voor wie de kerk
  verantwoordelijk is, te voorzien van de basisbehoeften.9
\end{enumerate}
\end{quote}

Voor zover mogelijk wordt dit programma uitgevoerd in kleine, lokale
groepen:

\begin{quote}
Gezinnen, buren, quorums, wijkgemeenschappen en andere organisatorische
eenheden van de kerk kunnen het nuttig en wenselijk vinden om kleine
groepen te vormen om elkaar te ondersteunen. Deze groepen kunnen
gewassen planten en oogsten, voedsel verwerken, en voedsel, kleding en
brandstof opslaan. Ook kunnen zij andere projecten uitvoeren die voor
beide partijen voordelig zijn.10
\end{quote}

Specifiek wordt de Mormoonse bisschoppen en priesterschapskorpsen
gevraagd hun broeders te ondersteunen bij zelfhulp:

\begin{quote}
In zijn tijdelijke rol beschouwt de bisschop elke kwetsbare persoon als
een tijdelijk probleem. Hij zorgt voor deze persoon totdat hij zelf kan
voorzien in zijn behoeften. Het priesterschapquorum ziet zijn behoeftige
lid echter als een aanhoudend probleem. Dit is pas opgelost als zowel in
de tijdelijke als in de geestelijke behoeften is voorzien. Een concreet
voorbeeld: de bisschop biedt hulp aan een monteur of ambachtsman die
tijdelijk zonder werk zit en in de problemen verkeert. Het
priesterschapquorum helpt hem om een nieuwe baan te vinden en zorgt
ervoor dat hij weer in zijn eigen onderhoud kan voorzien en actief kan
zijn in zijn priesterlijke taken.
\end{quote}

Concrete rehabilitatieactiviteiten voor leden in nood, die zijn
toevertrouwd aan de priesterschapquorums, zijn onder meer:

\begin{quote}
\begin{enumerate}
\def\labelenumi{\arabic{enumi}.}
\tightlist
\item
  Het helpen van quorumleden en hun gezinnen bij het vinden van een
  vaste baan. In sommige gevallen hebben quorums hun leden ondersteund
  bij het verkrijgen van betere functies door vakopleidingen, stages en
  andere mogelijkheden. 2. Het ondersteunen van quorumleden en hun
  gezinnen bij het starten van een eigen bedrijf.
\end{enumerate}
\end{quote}

Het belangrijkste doel van de Mormonse kerk is om werk te vinden voor
degenen die hulp nodig hebben. Met dit doel voor ogen,

\begin{quote}
Het vinden van geschikte banen als onderdeel van het Welzijnsprogramma
is een belangrijke verantwoordelijkheid voor de leden van het
priesterschapquorum. Zij, samen met de leden van de Relief Society,
moeten altijd alert zijn op arbeidsmogelijkheden. Wanneer ieder lid van
de welzijnscommissie zijn of haar taak goed uitvoert, zullen de meeste
werklozen een betaalde baan kunnen vinden op groeps- of wijkniveau.11
\end{quote}

Andere leden worden als zelfstandigen geholpen. De kerk kan
ondersteuning bieden met een kleine lening, en het priesterschapquorum
kan de terugbetaling vanuit haar fondsen garanderen. Mormonen die niet
geholpen kunnen worden met het vinden van een baan of die niet als
zelfstandige kunnen werken, moeten `waar mogelijk productieve arbeid
verrichten op het terrein van de kerk.' De kerk benadrukt dat de
ontvanger zoveel mogelijk zelf aan de slag gaat.

\begin{quote}
Het is belangrijk dat mensen die ondersteuning krijgen van het programma
van de bisschoppen werken naar vermogen, zodat ze verdienen wat ze
ontvangen. Werk van een individu in welzijnsprojecten moet als tijdelijk
worden beschouwd in plaats van als een vast dienstverband. Het kan
echter doorgaan zolang er hulp wordt geboden aan de persoon via het
bisschoppelijk magazijnprogramma. Op deze manier wordt het geestelijk
welzijn van mensen versterkt, terwijl in hun tijdelijke behoeften wordt
voorzien. Bovendien zullen gevoelens van schroom verdwijnen.12
\end{quote}

Als er geen ander werk beschikbaar is, kan de bisschop
welzijnsontvangers aanwijzen om individuele leden te ondersteunen die
hulp nodig hebben. De geholpen leden vergoeden de kerk volgens de
geldende loontarieven. Over het algemeen wordt van de hulpontvangers
verwacht dat zij in ruil voor de ontvangen steun zoveel mogelijk
bijdragen aan het welzijnsprogramma van de kerk, hetzij in geld,
producten of door hun arbeid.13

Als aanvulling op dit uitgebreide systeem van particuliere hulp, dat
gericht is op het bevorderen van zelfstandigheid, moedigt de mormoonse
kerk haar leden ten zeerste aan om geen beroep te doen op publieke
bijstand. `Plaatselijke kerkfunctionarissen worden verzocht het belang
van zelfvoorzienendheid te benadrukken voor elk individu, elk gezin en
elke kerkgemeenschap, zodat zij onafhankelijk blijven van openbare
hulpverlening.' Daarnaast geldt: 'Het aanvragen en accepteren van
directe publieke hulp leidt vaak tot de vloek van luiheid en bevordert
andere nadelige gevolgen van bijstand. Het ondermijnt iemands
onafhankelijkheid, inzet, spaarzaamheid en zelfrespect.'14

Er is geen beter voorbeeld dan de mormoonse kerk voor een particulier,
vrijwillig, rationeel en individualistisch welzijnsprogramma. Als de
overheid de welzijnszorg zou afschaffen, zou je verwachten dat in het
hele land talloze van dit soort programma's voor wederzijdse hulp zouden
ontstaan.

Het inspirerende voorbeeld van de mormoonse kerk laat zien dat de
belangrijkste bepalende factor voor wie of hoeveel mensen in de bijstand
terechtkomen, hun culturele en morele waarden zijn, en niet hun
inkomensniveau. Een ander voorbeeld is de groep Albanees-Amerikanen in
New York City.

Albanees-Amerikanen zijn een bijzonder arme groep, en in New York wonen
ze bijna altijd in de sloppenwijken. Hoewel de statistieken beperkt
zijn, ligt hun gemiddelde inkomen zonder twijfel lager dan dat van de
beter bekende zwarte en Puerto Ricaanse gemeenschappen. Toch ontvangt
geen enkele Albanees-Amerikaan bijstand. Waarom is dat? Vanwege hun
trots en onafhankelijkheid. Zoals een van hun leiders het verwoordde:
'Albanezen bedelen niet. Voor Albanezen is het aannemen van een
uitkering hetzelfde als bedelen op straat.'15

Naast de invloed van religie en etnische verschillen op waarden heeft
professor Banfield in zijn uitstekende boek \emph{The Unheavenly City}
het belang aangetoond van wat hij de `upper-class' en `lower-class'
culturen noemt. De definities van `klasse' die Banfield hanteert, zijn
niet strikt gebonden aan inkomens- of statusniveaus, maar vertonen wel
aanzienlijke overlap met gangbare definities. Zijn opvattingen over
klasse richten zich op de verschillende houdingen ten opzichte van het
heden en de toekomst. Leden van de hogere en middenklasse zijn doorgaans
toekomstgericht, doelbewust, rationeel en zelfdisciplinair. Mensen uit
de lagere klasse daarentegen zijn vaak sterk op het heden gericht,
impulsief, hedonistisch, doelloos en daarom minder geneigd om consistent
een baan of carrière na te streven. Personen met de eerstgenoemde
waarden hebben doorgaans hogere inkomens en betere banen, terwijl mensen
uit de lagere klassen vaak arm zijn, werkloos of afhankelijk van
bijstand. Kortom, het economische wel en wee van mensen is op de lange
termijn eerder een kwestie van persoonlijke verantwoordelijkheid dan
iets dat wordt bepaald door externe factoren, zoals liberalen vaak
beweren. Banfield citeert Daniel Rosenblatt's bevindingen over het
gebrek aan interesse in medische zorg, dat voortvloeit uit het `algemene
gebrek aan toekomstgerichtheid' onder de arme bevolking in steden:

\begin{quote}
Regelmatige controles van auto's om beginnende defecten op te sporen
maken bijvoorbeeld geen deel uit van het algemene waardesysteem van arme
stadsbewoners. Op een vergelijkbare manier worden huishoudelijke
voorwerpen vaak versleten en weggegooid, in plaats van dat ze in een
vroeg stadium van slijtage worden gerepareerd. Kopen op afbetaling wordt
gemakkelijk geaccepteerd, zonder dat men zich bewust is van de lengte
van de betalingen.

Het lichaam kan worden gezien als een soort voorwerp dat versleten
raakt, maar niet gerepareerd hoeft te worden. Tanden en kiezen worden
vaak zonder zorg achtergelaten; later is er vaak weinig interesse in een
eventueel gratis kunstgebit. Kunstgebitten worden in elk geval maar
weinige gebruikt. Corrigerende oogonderzoeken, zelfs voor mensen die al
een bril dragen, worden vaak verwaarloosd, ongeacht de beschikbare
faciliteiten van de kliniek. Het lijkt alsof de middenklasse het lichaam
beschouwt als een machine die in optimale staat moet worden gehouden,
bijvoorbeeld door middel van prothesen, revalidatie, cosmetische
chirurgie of voortdurende behandelingen. Daarentegen zien mensen uit de
lagere klasse het lichaam als iets met een beperkte levensduur: iets om
van te genieten in de jeugd en dat, naarmate men ouder wordt en
aftakelt, stoïcijns moet worden ondergaan en verdragen.16
\end{quote}

Banfield merkt bovendien op dat het sterftecijfer onder mensen uit de
lagere klassen al generaties lang veel hoger is dan dat van mensen uit
de hogere klassen. Veel van dit verschil wordt niet veroorzaakt door
armoede of lage inkomens, maar door de waarden en cultuur van mensen in
de lagere klasse. Alcoholisme, drugsverslaving, doodslag en
geslachtsziekten zijn opvallende en belangrijke doodsoorzaken in deze
groep. Ook de kindersterfte is veel hoger onder de lagere klassen, soms
tot wel twee tot drie keer hoger dan in de hogere groepen. Dat dit
eerder te maken heeft met culturele waarden dan met het inkomensniveau,
blijkt uit Banfields vergelijking van Ierse immigranten rond de
eeuwwisseling met Russisch-Joodse immigranten in New York City. De Ierse
immigranten waren destijds over het algemeen sterk op het heden gericht
en vertoonden een `lagere klasse'-mentaliteit. De Russische Joden,
hoewel zij in overbevolkte huurkazernes woonden en waarschijnlijk een
lager inkomensniveau hadden dan de Ieren, waren daarentegen ongewoon
toekomstgericht, doelbewust en hadden waarden en een mentaliteit die
meer met de `upper class' in overeenstemming waren. Rond de
eeuwwisseling was de levensverwachting van een Ierse immigrant op
tienjarige leeftijd slechts 38 jaar, terwijl die van de Russisch-Joodse
immigrant meer dan 50 jaar was. Daarnaast blijkt uit een studie van
zeven steden tussen 1911 en 1916 dat de kindersterfte meer dan drie keer
zo hoog was voor de laagste inkomensgroepen in vergelijking met de
hoogste, terwijl de Joodse kindersterfte extreem laag was.17

Net als bij ziekte of sterfte, geldt dit ook voor werkloosheid. Deze
staat uiteraard in nauw verband met zowel armoede als welzijn. Banfield
verwijst naar het onderzoek van professor Michael J. Piore, die de
essentiële `ongeschiktheid voor werk' van veel langdurig werklozen met
een laag inkomen aankaart. Piore ontdekte dat hun probleem niet zozeer
lag in het vinden of aanleren van vaardigheden voor een vaste,
goedbetaalde baan, maar vooral in het gebrek aan persoonlijke moed om
aan zo'n baan vast te houden. Deze mensen hebben vaak een hoog
absenteïsme, verlaten hun werk zonder opzegtermijn, zijn ongehoorzaam en
stelen soms van hun werkgever.18 Verder blijkt uit onderzoek van Peter
Doeringer naar de arbeidsmarkt in het getto van Boston in 1968 dat
ongeveer 70 procent van de sollicitanten die door
buurtwerkgelegenheidscentra waren doorverwezen, een baan aangeboden
kreeg. Meer dan de helft van deze aanbiedingen werd echter afgewezen.
Van de geaccepteerde sollicitanten behoudt slechts ongeveer 40 procent
hun baan langer dan een maand. Doeringer concludeert: 'Een groot deel
van de getgatewoordeloozen lijkt eerder het gevolg te zijn van
onstabiliteit op het gebied van werk dan van schaarste op de
arbeidsmarkt.'19

Het is zeer leerzaam om de opvattingen van professor Banfield en de
linkse socioloog Alvin Gouldner over de afstandelijkheid van werkloze
mannen uit de lagere klasse te vergelijken. Banfield stelt: `Mannen die
gewend zijn aan een leven op de hoek van de straat, aan vrouwen met een
uitkering en aan 'hosselen', zijn zelden bereid om de saaie routines van
een `goede' baan te accepteren.'20 Over het gebrek aan succes van
welzijnswerkers in het overtuigen van deze mannen om `een leven van
onverantwoordelijkheid, sensualiteit en losbandige agressie' achter zich
te laten, merkt Gouldner op dat ze het aangeboden aanbod onaantrekkelijk
vinden:

\begin{quote}
'Geef promiscue seks op, laat vrijelijke agressie en wilde spontaniteit
achter je\ldots{} en jij, of je kinderen, kunnen worden toegelaten tot
de wereld van drie vierkante maaltijden per dag, van een middelbare
school of misschien zelfs een universitaire opleiding, van rekeningen,
zekere banen en respectabiliteit.'21
\end{quote}

Het interessante is dat zowel Banfield als Gouldner, ondanks hun
tegengestelde waarden, het eens zijn over de essentie van dit proces.
Beide wijzen erop dat een groot deel van de hardnekkige werkloosheid
onder de lagere klassen, en dus armoede, vrijwillig is vanuit de kant
van de werklozen zelf.

De houding van Gouldner is kenmerkend voor veel hedendaagse liberalen en
linkse denkers. Ze vinden het schandalig om, zelfs zonder dwang,
`burgerlijke waarden' of `waarden van de middenklasse' op te dringen aan
de unieke en `natuurlijke' cultuur van de lagere klasse. Eerlijk is
eerlijk; dat kan misschien zo zijn. Maar verwacht dan ook niet dat de
hardwerkende bourgeoisie gedwongen wordt om de parasitaire waarden van
luiheid en onverantwoordelijkheid te ondersteunen en te subsidiëren. Die
waarden verafschuwen ze, en ze zijn duidelijk disfunctioneel voor het
voortbestaan van elke samenleving. Als mensen `spontaan' willen zijn,
laat ze dat dan doen in hun eigen tijd en met hun eigen middelen. Laat
ze de gevolgen van hun keuzes dragen. Het is niet eerlijk om de
hardwerkende en `niet-spontane' mensen te dwingen deze consequenties te
accepteren via staatsdwang. Kortom, schaf het welvaartsstelsel af.

Als het grootste probleem van de armen in de lagere klassen het denken
vanuit onverantwoordelijkheid is, en als er `burgerlijke'
toekomstgerichte waarden nodig zijn om mensen uit de bijstand en
afhankelijkheid te helpen (zoals bij de Mormonen), dan moeten deze
waarden in elk geval worden aangemoedigd en niet ontmoedigd in de
samenleving. De links-liberale houding van maatschappelijk werkers
ontmoedigt de armen direct door bijstand te presenteren als een `recht'
en een morele claim op productie. Bovendien zorgt de gemakkelijke
toegang tot uitkeringen ervoor dat de ontvangers minder gemotiveerd zijn
om te werken, waardoor onverantwoordelijkheid alleen maar toeneemt. Dit
houdt de vicieuze cirkel van armoede en welzijn in stand. Zoals Banfield
het verwoordt: 'Er is misschien geen betere manier om bekeerlingen te
maken voor het huidige denken dan iedereen een royale uitkering te
geven.'22

Over het algemeen hebben conservatieven in hun kritiek op het
socialezekerheidsstelsel voornamelijk gefocust op het ethische en morele
kwaad van het dwangmatig ontnemen van belastinggeld om nietsnutten te
ondersteunen. Linkse critici daarentegen hebben de nadruk gelegd op de
demoralisatie van sociale `klanten' door hun afhankelijkheid van de
vrijgevigheid van de staat en zijn bureaucratie. In wezen hebben beide
groepen een punt; er is geen tegenstelling tussen hun standpunten. We
hebben gezien dat vrijwillige programma's, zoals die van de Mormoonse
kerk, zich goed bewust zijn van dit probleem. Vroegere critici van de
bijstand die pleitten voor laissez-faire waren zich ook bewust van de
gevolgen van demoralisatie, naast de dwang die op degenen werd
uitgeoefend die voor de bijstand moesten betalen.

Zo verklaarde de negentiende-eeuwse Engelse voorstander van
laissez-faire, Thomas Mackay, dat hervorming van het welzijnsbeleid
`bestaat uit een herschepping en ontwikkeling van de kunst van
onafhankelijkheid.' Hij pleitte voor `niet meer filantropie, maar
veeleer voor meer respect voor de waardigheid van het menselijk leven en
meer geloof in het vermogen van mensen om hun eigen welzijn te
bewerkstelligen.' Mackay uitte zijn minachting voor de voorstanders van
meer welzijn, over\ldots{}

\begin{quote}
de plaatsvervangende filantroop die, in een roekeloze race om goedkope
populariteit, het belastingtarief dat van zijn buren wordt afgenomen,
gebruikt om de kansen op struikelen te vergroten voor de \ldots{}
menigte die maar al te graag in afhankelijkheid vervalt.23
\end{quote}

Mackay voegde hieraan toe dat de hervorming van het welzijnsbeleid moest
bestaan uit een herinterpretatie en ontwikkeling van de kunst van
onafhankelijkheid. Hij pleitte voor meer respect voor de waardigheid van
het menselijk leven en voor geloof in het vermogen van mensen om zelf
voor hun welzijn te zorgen. Bovendien had hij een hekel aan voorstanders
van meer welzijn. Hij sprak schamper over de plaatsvervangende
filantroop, die in een onbezonnen poging tot goedkope populariteit het
belastingtarief dat zijn buren wordt opgelegd, gebruikt om het aantal
struikelkansen voor de \ldots{} menigte die maar al te graag in
afhankelijkheid vervalt, te vergroten.

\begin{quote}
De `wettelijke bevoordeling van de armoede' die het
socialezekerheidsstelsel met zich meebrengt, 'introduceert een zeer
gevaarlijke en soms demoraliserende invloed in onze sociale regelingen.
De werkelijke noodzaak ervan is geenszins bewezen. De schijnbare
noodzaak komt voornamelijk voort uit het feit dat het systeem zijn eigen
afhankelijke bevolking heeft gecreëerd.'24
\end{quote}

Uitvoerig over het thema afhankelijkheid merkte Mackay op dat

\begin{quote}
Het bitterste aspect van de nood bij de armen komt niet alleen voort uit
armoede, maar ook uit het gevoel van afhankelijkheid dat inherent is aan
elke vorm van openbare hulp. Dit gevoel kan niet worden weggenomen; het
wordt zelfs versterkt door liberale maatregelen van openbare hulp.¹
\end{quote}

Mackay concludeerde dat het bitterste aspect van de nood onder de armen
niet alleen voortkomt uit armoede, maar ook uit het gevoel van
afhankelijkheid dat bij elke vorm van openbare hulp hoort. Dit gevoel is
niet te verhelpen; het wordt zelfs versterkt door liberale maatregelen
voor sociale bijstand.

\begin{quote}
De enige manier waarop de wetgever of de beheerder het pauperisme kan
aanpakken, is door de wettelijke uitkeringen voor paupers af te schaffen
of te beperken. Er is geen twijfel over; het land kan net zoveel paupers
hebben als het kan betalen. Schaf die toelage af of beperk deze. Wanneer
dat gebeurt, komen er nieuwe instanties in beeld. Hierdoor wordt het
natuurlijke vermogen van de mens om onafhankelijk te zijn, evenals de
natuurlijke banden van relatie en vriendschap, gestimuleerd. Onder deze
noemer plaats ik particuliere liefdadigheid, in tegenstelling tot
openbare liefdadigheid.¹
\end{quote}

De Charity Organisation Society, de belangrijkste particuliere
liefdadigheidsinstelling van Engeland aan het eind van de negentiende
eeuw, werkte precies volgens het principe dat hulp zelfhulp stimuleert.
Zoals de historicus Mowat over de Society opmerkt:

\begin{quote}
De C.O.S. belichaamde een idee van liefdadigheid dat beweerde de
verdeeldheid in de samenleving te overbruggen, armoede te bestrijden en
een gelukkige, zelfredzame gemeenschap te creëren. Men geloofde dat het
ernstigste probleem van armoede de degradatie van het karakter van de
arme man of vrouw was. Ongerichte liefdadigheid verergerde de situatie
alleen maar; het leidde tot demoralisatie. Echte liefdadigheid vereiste
daarentegen vriendschap en een doordachte aanpak. Het soort hulp dat het
zelfrespect van mensen zou herstellen en hen in staat zou stellen om
voor henzelf en hun gezinnen te zorgen.¹
\end{quote}

Een van de grimmigste gevolgen van de bijstand is misschien wel dat het
zelfhulp actief ontmoedigt. Dit gebeurt doordat de financiële prikkel
voor herstel wordt weggenomen. Er wordt geschat dat elke dollar die
gehandicapten investeren in hun eigen revalidatie hen gemiddeld tussen
de 10 en 17 dollar oplevert in de contante waarde van hun toekomstige
inkomsten. Maar deze stimulans wordt ondermijnd door het feit dat ze,
wanneer ze revalideren, hun bijstandsuitkering, invaliditeitsuitkering
van de Sociale Zekerheid en arbeidsongevallenvergoeding verliezen.
Daardoor besluiten veel gehandicapten om niet te investeren in hun eigen
revalidatie.¹ Bovendien zijn veel mensen inmiddels bekend met de
verlammende effecten van het socialezekerheidsstelsel. Dit stelsel stopt
de betalingen zodra de ontvanger, in schril contrast met particuliere
verzekeringsfondsen, zo dapper is om na zijn 62e te gaan werken en een
inkomen te verwerven.¹

In deze tijd, waarin de meeste mensen met argusogen naar de
bevolkingsgroei kijken, hebben maar weinig antipopulatieactivisten
aandacht besteed aan een ander nadelig effect van het welvaartssysteem.
Omdat bijstandsgezinnen betaald worden op basis van het aantal kinderen,
levert het systeem een belangrijke subsidie voor de opvoeding van meer
kinderen. Bovendien zijn het vooral de mensen die het zich het minst
kunnen veroorloven, die gestimuleerd worden om meer kinderen te krijgen.
Het gevolg hiervan is dat hun afhankelijkheid van bijstand aanhoudt, en
er in feite generaties ontstaan die permanent afhankelijk zijn van deze
hulp.

De laatste jaren is er veel gesproken over de oproep aan de overheid om
kinderdagverblijven beschikbaar te stellen voor de opvang van kinderen
van werkende moeders. Er wordt gesteld dat de markt er niet in geslaagd
is om deze broodnodige service te bieden.

Aangezien het de taak van de markt is om in dringende
consumentenbehoeften te voorzien, rijst de vraag waarom de markt in dit
geval lijkt te falen. Het antwoord ligt in de uitgebreide en kostbare
wettelijke beperkingen die de overheid aan het aanbod van kinderopvang
heeft opgelegd. In feite is het volkomen legaal om je kinderen bij een
vriend of familielid onder te brengen, ongeacht wie die persoon is of
hoe zijn appartement eruitziet. Ook kun je een buurman inhuren om op één
of twee kinderen te passen. Zodra die vriend of buurman echter een iets
grotere onderneming opstart, treden de overheidsregels in werking. De
staat dringt er over het algemeen op aan dat soortgelijke
kinderdagverblijven een vergunning verkrijgen. Daarbij wordt een
vergunning vaak alleen verleend als er altijd geregistreerde
verpleegkundigen aanwezig zijn, er minimaal speeltuinfaciliteiten zijn
en de faciliteit aan bepaalde grootte-eisen voldoet. Bovendien zijn er
talloze andere absurde en dure eisen die de overheid niet oplegt aan
vrienden, familieleden en buren -- of zelfs aan moeders zelf. Door deze
beperkingen te verwijderen, zal de markt zich aanpassen en voldoen aan
de vraag.

De dichter Ned O'Gorman runt al dertien jaar met beperkte middelen een
succesvol, particulier gefinancierd kinderdagverblijf in Harlem. Toch
dreigt hij failliet te gaan door de bureaucratische regels van de
gemeente New York. Hoewel de stad de `toewijding en effectiviteit' van
O'Gormans centrum, The Storefront, erkent, dreigt zij met boetes en
zelfs met de gedwongen sluiting van het centrum. Dit gebeurt tenzij er
een door de staat gecertificeerde maatschappelijk werker aanwezig is
wanneer er vijf of meer kinderen zijn. Zoals O'Gorman verontwaardigd
opmerkt:

\begin{quote}
Waarom zou ik in hemelsnaam iemand moeten aannemen die een diploma heeft
waarop staat dat hij sociaal werk heeft gestudeerd en gekwalificeerd is
om een kinderdagverblijf te leiden? Als ik na dertien jaar ervaring in
Harlem niet gekwalificeerd ben, wie is dat dan wel?29
\end{quote}

Het voorbeeld van de kinderopvang illustreert een belangrijke waarheid
over de markt: als er een tekort lijkt te zijn aan aanbod dat niet aan
een duidelijke vraag voldoet, kijk dan naar de overheid als mogelijke
oorzaak van het probleem. Geef de markt de ruimte en er zal geen gebrek
zijn aan kinderdagverblijven, net zoals er ook geen tekort is aan
motels, wasmachines, tv-toestellen of andere alledaagse benodigdheden.

\section{LASTEN EN SUBSIDIES VAN DE
WELVAARTSSTAAT}\label{lasten-en-subsidies-van-de-welvaartsstaat}

Helpt de moderne welvaartsstaat de armen echt? Het gangbare idee, dat de
welvaartsstaat heeft bevorderd en in stand gehouden, is dat deze inkomen
en rijkdom herverdeelt van de rijken naar de armen. Het progressieve
belastingsysteem haalt geld weg bij de rijken, terwijl verschillende
sociale voorzieningen en andere diensten dat geld onder de armen
verdelen. Maar zelfs liberalen, de grootste voorstanders van de
welvaartsstaat, beginnen te beseffen dat elk aspect van dit idee slechts
een gekoesterde mythe is. Overheidscontracten, vooral die van het leger,
sluizen belastinggeld naar bevoorrechte bedrijven en goedbetaalde
industriële arbeiders. Minimumloonwetten leiden op tragische wijze tot
werkloosheid, vooral onder de armste en laagst opgeleide werknemers in
het Zuiden, onder zwarte tieners in de getto's en onder
beroepsgehandicapten. Een minimumloon garandeert immers niet dat iemand
werk heeft; het verbiedt slechts, via de wet, dat iemand kan worden
aangenomen voor een loon dat lager is dan het vastgestelde minimum.
Hierdoor ontstaat er werkloosheid. Economen hebben aangetoond dat
verhogingen van het federale minimumloon de welbekende
werkgelegenheidskloof tussen zwarte en blanke tieners hebben vergroot en
de werkloosheid onder zwarte mannen hebben opgestuwd van ongeveer 8
procent aan het begin van de naoorlogse periode naar meer dan 35 procent
nu. De werkloosheid onder zwarte jongeren is daardoor veel ernstiger dan
de massale werkloosheid in de jaren dertig (20-25 procent).30

We hebben al gezien hoe het hoger onderwijs door de staat inkomen
herverdeelt van armere naar rijkere burgers. Talloze
vergunningsrestricties van de overheid, die in verschillende beroepen
zijn doorgevoerd, sluiten armere en minder geschoolde arbeiders uit van
deze banen. Het wordt steeds duidelijker dat
stadsvernieuwingsprogramma's, die zogenaamd bedoeld zijn om de
sloppenwijken van de armen te helpen, in feite hun huizen afbreken.
Hierdoor worden de armen gedwongen om onder nog slechtere omstandigheden
te wonen, ten voordele van rijkere gesubsidieerde huurders,
bouwvakbonden, bevoorrechte projectontwikkelaars en commerciële belangen
in de binnenstad. Vakbonden, ooit geliefd bij liberalen, worden nu vaak
bekritiseerd omdat zij hun privileges van de overheid gebruiken om
armere arbeiders en leden van minderheidsgroepen uit te sluiten. De
steunmaatregelen in de landbouw, die door de federale overheid steeds
verder worden verhoogd, belasten de belastingbetaler en drijven de
voedselprijzen op. Dit benadeelt vooral arme consumenten en helpt niet
de arme boeren, maar juist de rijke boeren die een groot deel van het
land bezitten. Omdat boeren betaald worden per pond of per bushel
product, profiteert het steunprogramma vooral de welgestelde boeren.
Vaak worden boeren zelfs betaald om niet te produceren, wat leidt tot
aanzienlijke werkloosheid onder het armste segment van de
boerenbevolking, zoals pachters en werknemers op de boerderij.
Bepalingen in bestemmingswetten in de groeiende buitenwijken van de
Verenigde Staten dienen om de armste burgers uit te sluiten, vaak zwarte
mensen die proberen te verhuizen van de binnensteden naar de
buitenwijken waar de werkgelegenheid toeneemt. De U.S. Postal Service
hanteert hoge monopolietarieven op eersteklaspost, een kostenpost die
door het grote publiek wordt gedragen om de distributie van kranten en
tijdschriften te subsidiëren. De FHA subsidieert hypotheken voor
welgestelde huiseigenaren. Het Federal Bureau of Reclamation biedt
irrigatiewater aan rijke boeren in het Westen, waardoor stedelijke armen
geen toegang meer hebben tot water en gedwongen zijn hogere waterlasten
te betalen. De Rural Electrification Administration en de Tennessee
Valley Authority subsidiëren de elektriciteitsvoorziening voor
vermogende boeren, voorsteden en bedrijven. Zoals professor Brozen
sardonisch opmerkt:

\begin{quote}
Elektriciteit voor armlastige bedrijven zoals de Aluminum Corporation of
America en de DuPont Company wordt gesubsidieerd door de belastingvrije
status van de Tennessee Valley Authority. Dit betekent dat 27 procent
van de elektriciteitsprijs wordt gebruikt om de belastingen te dekken
die aan particuliere nutsbedrijven worden opgelegd.31
\end{quote}

De overheidsregulering leidt tot monopolies en kartelvorming in grote
delen van de industrie. Dit resulteert in hogere prijzen voor
consumenten en beperkt de productie, alternatieven en verbeteringen van
producten. Denk bijvoorbeeld aan de regulering van de spoorwegen,
openbare nutsbedrijven, luchtvaartmaatschappijen en wetten met
betrekking tot olieprijzen. De Civil Aeronautics Board kent vliegroutes
toe aan bevoorrechte maatschappijen en houdt kleinere concurrenten
buiten de deur, of drijft ze zelfs uit de markt. Staats- en federale
wetten over olieprijzen creëren strikte productiebeperkingen voor ruwe
olie, waardoor de olieprijzen stijgen, die bovendien nog verder worden
opgedreven door importbeperkingen. Overal in het land verstrekt de
overheid absolute monopolieposities aan gas-, elektriciteits- en
telefoonbedrijven. Dit beschermt deze bedrijven tegen concurrentie en
stelt de overheid in staat om hun tarieven te bepalen, zodat zij een
gegarandeerde winst behalen. Het verhaal is overal hetzelfde: een
systematische afroming van de massa door de `verzorgingsstaat'.32

De meeste mensen geloven dat het Amerikaanse belastingsysteem in
principe de rijken veel zwaarder belast dan de armen. Daarom wordt het
vaak gezien als een manier om inkomen te herverdelen van hogere naar
lagere inkomensklassen. (Er zijn natuurlijk andere vormen van
herverdeling, zoals de belastingbetaler die betaalt voor bedrijven als
Lockheed of General Dynamics.) Maar zelfs de federale
inkomstenbelasting, waarvan velen denken dat deze `progressief' is---met
een hogere belasting voor de rijken dan voor de armen, en de
middenklasse daartussen---werkt niet echt zo als we naar andere aspecten
van deze belasting kijken. Neem bijvoorbeeld de belasting op sociale
zekerheid. Deze belasting is duidelijk `regressief', omdat zij vooral de
armen en de middenklasse treft. Iemand met een basisinkomen van \$8.000
betaalt dezelfde sociale zekerheid belasting---die elk jaar
toeneemt---als iemand die \$1.000.000 per jaar verdient. Bovendien
betalen vermogenswinsten, die meestal aan rijke aandeelhouders en
eigenaren van onroerend goed toekomen, veel minder belasting dan
reguliere inkomsten. Private trusts en stichtingen zijn vrijgesteld van
belasting, en rente die wordt verdiend op staats- en gemeentelijke
obligaties is ook vrijgesteld van de federale inkomstenbelasting.
Hieronder volgen de schattingen van het percentage van het inkomen dat
door elke `inkomensklasse' aan federale belastingen wordt betaald:

\begin{center}\rule{0.5\linewidth}{0.5pt}\end{center}

1965

\begin{longtable}[]{@{}
  >{\raggedright\arraybackslash}p{(\columnwidth - 2\tabcolsep) * \real{0.9755}}
  >{\raggedright\arraybackslash}p{(\columnwidth - 2\tabcolsep) * \real{0.0245}}@{}}
\toprule\noalign{}
\endhead
\bottomrule\noalign{}
\endlastfoot
\multicolumn{2}{@{}>{\raggedright\arraybackslash}p{(\columnwidth - 2\tabcolsep) * \real{1.0000} + 2\tabcolsep}@{}}{%
\textbf{Inkomen Klassen} Elektriciteit voor zwakkere bedrijven zoals de
Aluminum Corporation of America en de DuPont Company krijgt subsidie
door de belastingvrije status van de Tennessee Valley Authority. Dit
betekent dat 27 procent van de elektriciteitsprijs gaat naar de
belastingen die aan particuliere nutsbedrijven moeten worden betaald.31
De overheidsregulering leidt tot monopolies en kartelvorming in grote
delen van de industrie. Dit resulteert in hogere prijzen voor
consumenten en beperkt de productie, alternatieven en verbeteringen van
producten. Denk bijvoorbeeld aan de regulering van de spoorwegen,
openbare nutsbedrijven, luchtvaartmaatschappijen en wetten met
betrekking tot olieprijzen. De Civil Aeronautics Board kent vliegroutes
toe aan bevoorrechte maatschappijen en houdt kleinere concurrenten
buiten de deur of drijft ze zelfs uit de markt. Staats- en federale
wetten over olieprijzen creëren strikte productiebeperkingen voor ruwe
olie, waardoor de olieprijzen stijgen, terwijl importbeperkingen de
prijzen nog verder verhogen. Overal in het land verleent de overheid
gas-, elektriciteits- en telefoonbedrijven een absoluut monopolie. Dit
beschermt deze bedrijven tegen concurrentie en stelt de overheid in
staat om hun tarieven te bepalen, zodat zij gegarandeerd winst maken.
Het verhaal is overal hetzelfde: een systematische afroming van de massa
door de `verzorgingsstaat'.32 De meeste mensen denken dat het
Amerikaanse belastingsysteem in principe de rijken zwaarder belast dan
de armen. Daarom wordt het vaak gezien als een manier om inkomen te
herverdelen van hogere naar lagere inkomensklassen. (Er zijn natuurlijk
ook andere vormen van herverdeling, zoals de belastingbetaler die
betaalt voor Lockheed of General Dynamics.) Maar zelfs de federale
inkomstenbelasting, waarvan velen menen dat deze `progressief' is---in
de zin dat de rijken meer betalen dan de armen, met de middenklasse er
tussenin---werkt niet altijd zo als we andere aspecten van deze
belasting in overweging nemen. Neem bijvoorbeeld de belasting op sociale
zekerheid. Deze belasting is duidelijk `regressief', omdat ze vooral de
armen en de middenklasse treft. Iemand met een basisinkomen van \$8.000
betaalt evenveel sociale zekerheid belasting---die elk jaar
toeneemt---als iemand die \$1.000.000 per jaar verdient. Daarnaast
betalen vermogenswinsten, die doorgaans naar rijke aandeelhouders en
eigenaren van onroerend goed gaan, veel minder belasting dan reguliere
inkomsten. Private trusts en stichtingen zijn vrijgesteld van belasting,
en rente op staats- en gemeentelijke obligaties is ook vrijgesteld van
de federale inkomstenbelasting. Hieronder volgen schattingen van het
percentage van het inkomen dat door elke `inkomensklasse' aan federale
belastingen wordt betaald: \textbar{} Percentage van het inkomen dat aan
federale belasting wordt betaald \textbar{} \textbar{} Onder de \$2.000
\textbar{} 19 \textbar{} \textbar{} \$2.000--\$4.000 \textbar{} 16
\textbar{} \textbar{} \textbf{Inkomen Klassen} \$4.000--\$6.000
\textbar{} 17 \textbar{} \textbar{} \$6.000--\$8.000 \textbar{} 17
\textbar{} \textbar{} \$8.000--\$10.000 \textbar{} 18 \textbar{}
\textbar{} \$10.000--\$15.000 \textbar{} 19 \textbar{} \textbar{} Boven
de €15.000 \textbar{} 32} \\
\end{longtable}

\begin{quote}
Gemiddeld 22
\end{quote}

\begin{center}\rule{0.5\linewidth}{0.5pt}\end{center}

Als federale belastingen nauwelijks `progressief' zijn, dan is de impact
van staats- en lokale belastingen bijna sterk regressief.
Eigendomsbelastingen zijn (a) proportioneel, (b) alleen van toepassing
op eigenaren van onroerend goed en (c) afhankelijk van de politieke
wensen van lokale taxateurs.

Omzetbelasting en accijnzen raken de armen het zwaarst. Hierna vind je
een schatting van het percentage van het totale inkomen dat door staats-
en lokale belastingen wordt onttrokken:

\begin{center}\rule{0.5\linewidth}{0.5pt}\end{center}

1965

\begin{longtable}[]{@{}
  >{\raggedright\arraybackslash}p{(\columnwidth - 2\tabcolsep) * \real{0.3017}}
  >{\raggedright\arraybackslash}p{(\columnwidth - 2\tabcolsep) * \real{0.6983}}@{}}
\toprule\noalign{}
\endhead
\bottomrule\noalign{}
\endlastfoot
\multicolumn{2}{@{}>{\raggedright\arraybackslash}p{(\columnwidth - 2\tabcolsep) * \real{1.0000} + 2\tabcolsep}@{}}{%
Inkomen Klassen \textbar{} Percentage van het inkomen dat aan staats- en
lokale belastingen wordt besteed \textbar{} Onder de \$2.000 \textbar{}
25 \textbar{} \textbar{} Tussen de \$2.000 en \$4.000 \textbar{} 11
\textbar{} \textbar{} Tussen de \$4.000 en \$6.000 \textbar{} 10
\textbar{} \textbar{} Tussen de \$6.000 en \$8.000 \textbar{} 9
\textbar{} \textbar{} Tussen de \$8.000 en \$10.000 \textbar{} 9
\textbar{} \textbar{} Tussen de \$10.000 en \$15.000 \textbar{} 9
\textbar{} \textbar{} Boven de \$15.000 \textbar{} 7 \textbar{}} \\
\end{longtable}

\begin{quote}
GEMIDDELDE 9
\end{quote}

\begin{center}\rule{0.5\linewidth}{0.5pt}\end{center}

Hieronder staan de samengevoegde schattingen van de totale impact van
belastingen --- federaal, staats- en lokaal --- op verschillende
inkomensklassen:

\begin{center}\rule{0.5\linewidth}{0.5pt}\end{center}

1965

\begin{longtable}[]{@{}
  >{\raggedright\arraybackslash}p{(\columnwidth - 2\tabcolsep) * \real{0.3723}}
  >{\raggedright\arraybackslash}p{(\columnwidth - 2\tabcolsep) * \real{0.6277}}@{}}
\toprule\noalign{}
\endhead
\bottomrule\noalign{}
\endlastfoot
\multicolumn{2}{@{}>{\raggedright\arraybackslash}p{(\columnwidth - 2\tabcolsep) * \real{1.0000} + 2\tabcolsep}@{}}{%
Inkomstengroepen \textbar{} Percentage van het inkomen dat aan
belastingen gaat: 33\% \textbar{} Onder de \$2.000 \textbar{} 44
\textbar{} \textbar{} Tussen de \$2.000 en \$4.000 \textbar{} 27
\textbar{} \textbar{} Tussen de \$4.000 en \$6.000 \textbar{} 27
\textbar{} \textbar{} Tussen de \$6.000 en \$8.000 \textbar{} 26
\textbar{} \textbar{} Tussen de \$8.000 en \$10.000 \textbar{} 27
\textbar{} \textbar{} Tussen de \$10.000 en \$15.000 \textbar{} 27
\textbar{} \textbar{} Boven de \$15.000 \textbar{} 38 \textbar{}} \\
\end{longtable}

\begin{quote}
GEMIDDELD 31
\end{quote}

\begin{center}\rule{0.5\linewidth}{0.5pt}\end{center}

Nog recentere schattingen (1968) van de totale impact van belastingen op
alle overheidsniveaus bevestigen dit ruimschoots. Ze tonen ook een veel
grotere relatieve stijging aan wat betreft de belastingdruk op de
laagste inkomensgroepen in de afgelopen drie jaar.

\begin{center}\rule{0.5\linewidth}{0.5pt}\end{center}

1968

\begin{longtable}[]{@{}
  >{\raggedright\arraybackslash}p{(\columnwidth - 2\tabcolsep) * \real{0.3723}}
  >{\raggedright\arraybackslash}p{(\columnwidth - 2\tabcolsep) * \real{0.6277}}@{}}
\toprule\noalign{}
\endhead
\bottomrule\noalign{}
\endlastfoot
\multicolumn{2}{@{}>{\raggedright\arraybackslash}p{(\columnwidth - 2\tabcolsep) * \real{1.0000} + 2\tabcolsep}@{}}{%
Inkomstengroepen \textbar{} Percentage van het inkomen dat aan
belastingen gaat: 34\% \textbar{} Onder de \$2.000 \textbar{} 50
\textbar{} \textbar{} \$2.000--\$4.000 \textbar{} 35 \textbar{}
\textbar{} \$4.000--\$6.000 \textbar{} 31 \textbar{} \textbar{}
\$6.000--\$8.000 \textbar{} 30 \textbar{} \textbar{} \$8.000--\$10.000
\textbar{} 29 \textbar{} \textbar{} \$10.000--\$15.000 \textbar{} 30
\textbar{} \textbar{} \$15.000--\$25.000 \textbar{} 30 \textbar{}
\textbar{} \$25.000--\$50.000 \textbar{} 33 \textbar{} \textbar{} Boven
de \$50.000 \textbar{} 45 \textbar{}} \\
\end{longtable}

\begin{center}\rule{0.5\linewidth}{0.5pt}\end{center}

Veel economen proberen de impact van deze vaak misleidende cijfers te
nuanceren. Ze stellen dat mensen in de categorie `onder de \$2.000'
bijvoorbeeld meer ontvangen aan uitkeringen en andere
`transferbetalingen' dan ze aan belastingen betalen. Maar dit negeert
het belangrijke feit dat binnen elke categorie niet iedereen zowel
ontvanger van uitkeringen als belastingbetaler is. De belastingbetalers
in de laatste groep worden zwaar belast om de eerste groep te
subsidiëren. Kortom, de armen en de middenklasse betalen voor de
gesubsidieerde volkshuisvesting van andere arme en
middeninkomensgroepen. Vooral de werkende armen moeten een aanzienlijk
bedrag bijdragen aan de subsidies voor de sociaal zwakkeren.

Er is genoeg inkomensherverdeling in dit land: naar Lockheed, naar
uitkeringstrekkers, noem maar op. Maar de `rijken' worden niet belast om
voor de `armen' te betalen. De herverdeling gebeurt binnen de
inkomensgroepen; sommige armen worden gedwongen om voor andere armen bij
te dragen.

Andere belastinginschattingen bevestigen dit verontrustende beeld. De
Tax Foundation schat bijvoorbeeld dat federale, staats- en lokale
belastingen 34 procent van het totale inkomen innemen van mensen die
minder dan \$3.000 per jaar verdienen.

Het doel van deze discussie is niet om te pleiten voor een `echt'
progressieve inkomstenbelastingstructuur of om de rijken uit te persen.
Het gaat erom te benadrukken dat de moderne welvaartsstaat, die vaak
wordt geprezen als een manier om de rijken te laten bijdragen aan de
armen, dit in werkelijkheid niet doet. Integendeel, het te veel belasten
van de rijken zou rampzalige gevolgen hebben, niet alleen voor hen, maar
ook voor de armen en de middenklasse. De rijken dragen namelijk in
verhouding meer bij in de vorm van spaargeld, investeringskapitaal en
ondernemerschap. Dit draagt bij aan technologische innovaties en heeft
ervoor gezorgd dat de Verenigde Staten de hoogste levensstandaard heeft
bereikt voor de meeste mensen in de geschiedenis. De rijken onder druk
zetten zou niet alleen diep immoreel zijn, maar ook deze belangrijke
kwaliteiten zoals spaarzaamheid, zakelijke vooruitziendheid en
investeringen bestraffen, die onze opmerkelijke levensstandaard mogelijk
hebben gemaakt. Het zou als het doden van de kip die de gouden eieren
legt, zijn.

\section{Wat kan de overheid doen?}\label{wat-kan-de-overheid-doen}

Wat kan de overheid doen om de armen te helpen? Het enige juiste
antwoord is ook het libertarische antwoord: stap opzij. Wanneer de
overheid de productieve energie van alle bevolkingsgroepen---rijk,
middenklasse en arm---ruimte geeft, zal dat leiden tot een aanzienlijke
verbetering van het welzijn en de levensstandaard van iedereen. Dit
geldt vooral voor de armen, die zogenaamd geholpen worden door de
zogenaamde `verzorgingsstaat'.

Er zijn vier belangrijke manieren waarop de overheid het Amerikaanse
volk meer ruimte kan geven. Ten eerste kan ze alle belastingen
afschaffen of op zijn minst sterk verlagen. Deze belastingen remmen
namelijk productieve energie, sparen, investeren en technologische
vooruitgang. Het creëren van banen en het verhogen van lonen, als gevolg
van het afschaffen van deze belastingen, zou vooral de lagere
inkomensgroepen ten goede komen. Zoals professor Brozen opmerkt:

\begin{quote}
Met minder pogingen om de staatsmacht in te zetten tegen
inkomensongelijkheid, zou die ongelijkheid sneller afnemen. Lage lonen
zouden sneller stijgen door een hoger niveau van sparen en
kapitaalvorming, waardoor de ongelijkheid zou verminderen naarmate het
inkomen van de werknemers toeneemt.36
\end{quote}

De beste manier om de armen te helpen, is door de belastingen te
verlagen en sparen, investeren en het creëren van banen alle ruimte te
geven. Zoals Dr.~F.A. Harper jaren geleden al aangaf, zijn productieve
investeringen de `grootste economische liefdadigheid.' Harper schreef:

\begin{quote}
Volgens de ene opvatting is het delen van een boterham de manier om
liefdadigheid te bedrijven. De andere opvatting pleit voor het besparen
en het gebruik van middelen om extra broden te produceren. Dit wordt
gezien als de grootste economische liefdadigheid.

De twee visies staan lijnrecht tegenover elkaar omdat de methoden elkaar
uitsluiten in het gebruik van iemands tijd en middelen bij alle
dagelijkse keuzes.

Het verschil in opvatting komt voort uit uiteenlopende ideeën over de
aard van de economie. De eerste visie gaat ervan uit dat het totaal aan
economische goederen constant is. De tweede visie gelooft dat er een
mogelijkheid is om de productie uit te breiden zonder een noodzakelijke
limiet.

Het verschil tussen de twee visies komt overeen met het verschil tussen
een twee- en een driedimensionaal perspectief op productie. De
tweedimensionale omvang is op elk moment van de tijd vastgelegd, terwijl
de derde dimensie -- en daarmee de totale omvang -- onbeperkt kan worden
uitgebreid door middel van besparingen en middelen.

De hele geschiedenis van de mensheid weerspreekt het idee dat er een
vast totaal aan economische goederen bestaat. De geschiedenis laat zien
dat besparingen en de uitbreiding van gereedschappen de enige manieren
zijn om een significante toename te realiseren.37
\end{quote}

De libertarische schrijfster Isabel Paterson verwoordde het prachtig:

\begin{quote}
Wat betreft de particuliere filantroop en de particuliere kapitalist die
als zodanig opereert, laten we het voorbeeld nemen van een echt
behoeftige man die niet arbeidsongeschikt is. Stel dat de filantroop hem
voedsel, kleding en onderdak biedt. Nadat hij daarvan gebruik heeft
gemaakt, is de man weer precies waar hij eerder was, met mogelijk de
gewoonte van afhankelijkheid als extra bagage. Maar stel je voor dat
iemand zonder welwillende motieven, die gewoon om zijn eigen redenen
werk wil hebben, de behoeftige man inhuurt voor een loon. In dit geval
heeft de werkgever geen goede daad verricht, maar de situatie van de
werknemer is wel degelijk veranderd. Wat is dan het essentiële verschil
tussen deze twee handelingen?

De niet-filantropische werkgever heeft de man die hij in dienst heeft
genomen, weer in de productielijn geïntegreerd, in het grotere
energiecircuit. De filantroop daarentegen kan alleen maar energie
omleiden, wat betekent dat er geen mogelijkheid is voor de man om terug
te keren naar de productie. Hierdoor is de kans kleiner dat de ontvanger
van zijn goedheid daadwerkelijk werk vindt.

Als we de volledige rol van oprechte filantropen door de eeuwen heen
onder de loep nemen, blijkt dat ze samen met hun strikt filantropische
activiteiten de mensheid nooit meer dan een tiende van het voordeel
hebben geboden dan de inspanningen van Thomas Alva Edison, die doorgaans
uit eigenbelang verricht werden. Dit geldt des te meer voor de grotere
geesten die de wetenschappelijke principes ontwikkelden die Edison
toepaste. Ontelbare speculatieve denkers, uitvinders en organisatoren
hebben bijgedragen aan het comfort, de gezondheid en het geluk van hun
medemensen, en dat terwijl dat niet hun oorspronkelijke doel was.38
\end{quote}

Ten tweede zou, als gevolg van een drastische verlaging of zelfs
afschaffing van belastingen, een gelijkwaardige vermindering van de
overheidsuitgaven plaatsvinden. Schaarse economische middelen zouden
niet langer worden verspild aan onproductieve uitgaven. Dit omvat
bijvoorbeeld miljarden voor het ruimteprogramma, openbare werken of het
militair-industriële complex. In plaats daarvan zouden deze middelen
beschikbaar komen voor de productie van goederen en diensten waar de
consument naar verlangt. De toenemende beschikbaarheid van goederen en
diensten zou leiden tot nieuwe en betere producten voor consumenten, en
dat tegen veel lagere prijzen. We zouden niet langer te lijden hebben
onder de inefficiënties en de schade aan de productiviteit die
voortkomen uit overheidssubsidies en -contracten. Bovendien zouden veel
wetenschappers en ingenieurs, die nu bezig zijn met verspilling in
militair onderzoek en andere overheidsuitgaven, kunnen overstappen naar
vreedzame en productieve activiteiten en innovaties die de consumenten
ten goede komen.39

Ten derde, als de overheid zou stoppen met de vele manieren waarop ze de
armen belast om de rijken te subsidiëren, zoals eerder genoemd
(bijvoorbeeld hoger onderwijs, landbouwsubsidies, irrigatie, Lockheed,
enzovoort), zou ze de armen daadwerkelijk helpen. Door te stoppen met
het belasten van de armen zou de overheid hun productieve activiteiten
ontlasten.

Ten slotte is een van de belangrijkste manieren waarop de overheid de
armen kan helpen, het wegnemen van haar eigen directe obstakels voor hun
productieve energie. Minimumloonwetten leiden ervoor dat de armste en
minst productieve leden van de bevolking werkloos worden. De voordelen
die de overheid verleent aan vakbonden maken het voor de armste
arbeiders en mensen uit minderheidsgroepen moeilijker om toegang te
krijgen tot productief en goedbetaald werk. Daarnaast beletten
vergunningseisen, het verbod op gokken en andere overheidsbeperkingen de
armen om kleine bedrijfjes te starten en zelf werkgelegenheid te
creëren. Daarom heeft de overheid overal zware beperkingen opgelegd aan
de handel, variërend van een algeheel verbod tot hoge licentiekosten.
Venten was ooit de klassieke manier voor immigranten, die arm waren en
geen kapitaal hadden, om ondernemers te worden en uiteindelijk grote
zakenlieden te ontwikkelen. Maar deze weg is nu afgesloten, grotendeels
om monopolievoordelen te creëren voor de winkels in elke stad, die
vrezen winst te verliezen door de stevige concurrentie van
straatventers.

Typerend voor de manier waarop de overheid de productieve activiteiten
van de armen heeft gefrustreerd, is het verhaal van neurochirurg
Dr.~Thomas Matthew. Hij is de oprichter van de zelfhulporganisatie voor
zwarten, NEGRO, die obligaties uitgeeft om haar activiteiten te
bekostigen. Halverwege de jaren zestig richtte Dr.~Matthew, ondanks
tegenwerkingen van de New Yorkse overheid, een succesvol interraciaal
ziekenhuis op in de zwarte wijk Jamaica, Queens. Al snel ontdekte hij
echter dat het openbaar vervoer in Jamaica zo slecht was dat het vervoer
voor patiënten en personeel absoluut onvoldoende was. Daarom kocht
Dr.~Matthew een paar bussen en richtte hij een reguliere busdienst op in
Jamaica, die zowel regelmatig als efficiënt functioneerde. Het probleem
was alleen dat Dr.~Matthew geen stadslicentie had om een buslijn te
exploiteren. Dit privilege is namelijk voorbehouden aan inefficiënte,
maar beschermde monopolies. De vindingrijke Dr.~Matthew ontdekte dat de
stad geen bussen zonder vergunning toestond om tarieven in rekening te
brengen. Daarom maakte hij zijn busdienst gratis. Passagiers die dat
wilden, konden echter een bedrijfsobligatie van 25 cent kopen als ze met
de bus wilden reizen.

De busdienst van Matthew was zo succesvol dat hij een tweede buslijn in
Harlem opzette. Begin 1968, echter, werd de regering van New York City
bezorgd en trad ze hard op. De overheid stapte naar de rechter en legde
beide lijnen stil omdat ze zonder vergunning werden geëxploiteerd.

Een paar jaar later namen Dr.~Matthew en zijn collega's een ongebruikt
pand in Harlem in gebruik, dat eigendom was van het stadsbestuur. (Het
stadsbestuur van New York is de grootste `slumlord' van de stad. Het
bezit een enorme hoeveelheid bruikbare gebouwen die verlaten zijn omdat
ze de hoge onroerendgoedbelasting niet betalen, waardoor ze wegrotten en
onbewoonbaar worden.) In dit gebouw richtte Dr.~Matthew een goedkoop
ziekenhuis op, in een tijd waarin de ziekenhuiskosten de pan uit rezen
en ziekenhuisruimte schaars was. Uiteindelijk slaagde de stad er ook in
om dit ziekenhuis buiten werking te stellen, met als argument
`brandovertredingen.' Keer op keer was het de rol van de overheid om de
economische activiteiten van de armen te hinderen, in gebied na gebied.
Het is dan ook geen verrassing dat, toen Dr.~Matthew door een blanke
ambtenaar van de regering van New York City werd gevraagd hoe deze het
beste zelfhulpprojecten voor zwarten kon ondersteunen, Matthew
antwoordde: `Ga uit onze weg en laat ons iets proberen.'

Een ander voorbeeld van het functioneren van de overheid vond enkele
jaren geleden plaats. De federale regering en de regering van New York
City kondigden luidkeels aan dat ze een groep van 37 gebouwen in Harlem
zouden renoveren. Maar in plaats van, zoals gebruikelijk in de
particuliere sector, elk huis afzonderlijk te restaureren, gunde de
overheid één contract voor het hele pakket van 37 gebouwen. Hierdoor
konden kleine, zwarte bouwbedrijven niet bieden en ging het
prijscontract vanzelfsprekend naar een groot blank bedrijf. Een ander
voorbeeld: in 1966 kondigde de federale Small Business Administration
trots een programma aan om nieuwe kleine bedrijven in zwarte handen te
ondersteunen. Maar de overheid legde aan de leningen een aantal
belangrijke beperkingen op. Ten eerste moest elke lener `op
armoedeniveau' zijn. Aangezien de allerarmsten niet geneigd zijn om een
eigen bedrijf te starten, sloten deze voorwaarden veel kleine bedrijven
uit. Vooral eigenaars met een gemiddeld laag inkomen - precies degenen
die waarschijnlijk kleine ondernemers zouden zijn - werden hierdoor
benadeeld. Bovenop deze beperkingen voegde de SBA van New York nog een
eis toe: alle zwarten die dergelijke leningen wilden, moesten `een echte
behoefte in hun gemeenschap aantonen' om een merkbare `economische
leegte' te vullen. Deze behoefte en leegte moesten volledig naar
tevredenheid van bureaucraten worden aangetoond, die vaak ver van de
werkelijke economische situatie stonden.

Er werd onderzoek gedaan naar de geschatte stroom overheidsgeld
(federaal en district) naar het getto van Shaw-Cardozo in Washington,
D.C. Dit gebied heeft een laag inkomen onder de zwarte bevolking, en de
stroom werd vergeleken met het bedrag aan belastingen dat het gebied aan
de overheid betaalt. In het fiscale jaar 1967 telde het
Shaw-Cardozo-gebied 84.000 inwoners, waarvan 79.000 zwart, met een
gemiddeld gezinsinkomen van \$5.600 per jaar. Het totale, verdiende
persoonlijke inkomen voor de inwoners van het gebied bedroeg dat jaar
126,5 miljoen dollar. De totale waarde van de overheidsuitkeringen die
naar het district gingen -- van sociale zekerheidsuitkeringen tot de
geschatte uitgaven voor openbare scholen -- werd in dat fiscale jaar
geschat op \$45,7 miljoen. Dit betekent dat er bijna 40\% van het totale
inkomen van Shaw-Cardozo aan subsidies komt. Maar daarnaast is er ook
een totale uitstroom van belastingen uit Shaw-Cardozo, die naar
schatting op \$50 miljoen ligt. Dit leidt tot een netto-uitstroom van
\$4,3 miljoen uit dit laaginkomengetto. Kun je nog steeds volhouden dat
de afschaffing van de enorme, onproductieve welvaartsstaat de armen zou
schaden?

De overheid zou de armen -- en de rest van de samenleving -- het beste
kunnen helpen door zich op de achtergrond te houden. Ze moet haar grote
en verlammende netwerk van belastingen, subsidies, inefficiënties en
monopolieprivileges opheffen. Zoals professor Brozen het samenvatte in
zijn analyse van de `verzorgingsstaat':

\begin{quote}
De staat is vaak een middel geweest om welvaart te creëren voor een
enkeling, ten koste van velen. De markt heeft op een efficiënte manier
voor veel welvaart gezorgd, zelfs voor een selecte groep. De staat heeft
zijn methodes niet veranderd sinds de Romeinse tijd, waar brood en
spelen voor de massa's zorgden voor de afleiding. Hoewel hij nu doet
alsof hij onderwijs, medicijnen, gratis melk en podiumkunsten aanbiedt,
blijft de staat de bron van monopolistische privileges en macht voor een
selecte groep. Dit alles gebeurt achter de façade van welzijn voor
velen. Dit welzijn zou veel overvloediger zijn als politici de middelen
die ze gebruiken om de illusie te creëren dat ze om hun kiezers geven,
niet onteigenen.
\end{quote}

\section{DE NEGATIEVE
INKOMSTENBELASTING}\label{de-negatieve-inkomstenbelasting}

Er wordt een onderzoek gedaan naar de geschatte stroom overheidsgeld
(zowel federaal als lokaal) naar het getto van Shaw-Cardozo in
Washington, D.C. Dit gebied heeft te maken met lage inkomens onder de
zwarte bevolking. De uitkeringen worden vergeleken met de
belastinginkomsten die het gebied aan de overheid bijdraagt. In het
fiscale jaar 1967 telde Shaw-Cardozo 84.000 inwoners, waarvan 79.000
zwart, met een gemiddeld gezinsinkomen van \$5.600 per jaar. Het totale
verdiende persoonlijke inkomen in dat jaar bedroeg 126,5 miljoen dollar.
De geschatte waarde van de overheidsuitkeringen die naar het district
gingen, van sociale uitkeringen tot geschatte uitgaven voor openbare
scholen, was in 1967 \$45,7 miljoen. Dit zou betekenen dat er bijna 40\%
van het totale inkomen van Shaw-Cardozo afkomstig is uit subsidies. Aan
de andere kant was de totale uitstroom van belastingen uit Shaw-Cardozo
naar schatting \$50 miljoen. Dit leidt tot een netto-uitstroom van \$4,3
miljoen uit dit laaginkomengetto. Kun je nog steeds volhouden dat het
afschaffen van de enorme, onproductieve welvaartsstaat de armen zou
schaden? De overheid zou de armen -- en de rest van de samenleving --
beter kunnen helpen door zich op de achtergrond te houden. Ze zou haar
omvangrijke en verlammende netwerk van belastingen, subsidies,
inefficiënties en monopolieprivileges moeten afschaffen. Zoals professor
Brozen het samenvat in zijn analyse van de `verzorgingsstaat': `De staat
is vaak een middel geweest om welvaart te creëren voor enkelen, ten
koste van velen. De markt heeft voor veel welvaart gezorgd, zelfs voor
een selecte groep, en dit met relatief lage kosten. De staat heeft zijn
methodes niet veranderd sinds de Romeinse tijd, waar brood en spelen de
massa moesten vermaken. Hoewel hij nu doet alsof hij onderwijs,
gezondheidszorg, gratis melk en podiumkunsten aanbiedt, blijft de
overheid een bron van monopolistische privileges en macht voor enkelen.
Dit alles gebeurt achter de façade van welzijn voor velen -- welzijn dat
overvloediger zou zijn als politici de middelen die ze gebruiken om de
illusie te scheppen dat ze om hun kiezers geven, niet onteigenen.'

Helaas is de recente trend -- die wordt gesteund door een breed scala
aan voorstanders, variërend van president Nixon en Milton Friedman aan
de rechterkant tot een groot aantal aan de linkerkant -- om het huidige
welvaartssysteem af te schaffen, niet gericht op vrijheid, maar eerder
op het tegenovergestelde. Deze nieuwe ontwikkeling betreft het
`gegarandeerd jaarinkomen', de `negatieve inkomstenbelasting' of het
`Family Assistance Plan' van president Nixon. Met een beroep op de
inefficiëntie, onrechtvaardigheid en bureaucratie van het huidige
systeem, zou het gegarandeerd jaarinkomen de bijstand eenvoudiger,
`efficiënter' en automatisch maken. Hierbij betaalt de belastingdienst
elk jaar geld aan gezinnen die minder verdienen dan een bepaald
basisinkomen. Deze automatische voldoening wordt natuurlijk gefinancierd
door belastingen die worden geheven op werkende gezinnen die meer
verdienen dan het basisbedrag. De geschatte kosten van dit
ogenschijnlijk keurige en simpele systeem zouden slechts enkele
miljarden dollar per jaar bedragen.

Maar er is een belangrijk addertje onder het gras: de kosten zijn
geschat op basis van de aanname dat zowel de mensen met een uitkering
als degenen die deze uitkering financieren, in dezelfde mate zullen
blijven werken als voorheen. Deze aanname roept echter vragen op. Het
grootste probleem is het aanzienlijke ontmoedigende effect dat het
gegarandeerd jaarinkomen zal hebben op zowel de belastingbetaler als de
ontvanger.

Het enige wat het huidige socialezekerheidsstelsel voor een totale ramp
behoedt, is de bureaucratie en het stigma dat verbonden is aan het
ontvangen van een uitkering. Ontvangers van een uitkering dragen nog
steeds een psychisch stigma, hoewel dit de laatste jaren is afgenomen.
Daarnaast hebben ze te maken met een typische, inefficiënte en
onpersoonlijke bureaucratie. Echter, doordat het gegarandeerd
jaarinkomen de bijstand efficiënt, eenvoudig en automatisch maakt,
zullen de belangrijkste obstakels en ontmoedigingen voor het `aanbod van
welzijn' verdwijnen. Dit zal resulteren in een massale toestroom naar de
gegarandeerde bijstand. Bovendien zal iedereen deze nieuwe bijstand
beschouwen als een automatisch `recht', in plaats van als een voorrecht
of geschenk, waardoor alle stigma's zullen verdwijnen.

Stel je voor dat \$4.000 per jaar de `armoedegrens' is. Iedereen met een
inkomen onder die grens ontvangt automatisch het verschil van de
overheid, simpelweg door het invullen van de belastingaangifte. Wie geen
inkomen heeft, krijgt \$4.000 van de overheid. Iemand die \$3.000
verdient, ontvangt \$1.000, enzovoort. Het lijkt duidelijk dat er geen
enkele reden is voor iemand die minder dan \$4.000 per jaar verdient om
door te blijven werken. Waarom zou hij dat doen, als zijn buurman die
niet werkt uiteindelijk hetzelfde inkomen heeft als hij? Kortom, het
netto-inkomen uit arbeid zal dan nul zijn. De hele werkende bevolking
die onder de magische grens van \$4.000 zit, zal stoppen met werken en
naar de `rechtmatige' bijstand stromen.

Maar dat is nog niet alles. Hoe zit het met de mensen die \$4.000
verdienen, of zelfs net iets daarboven? De man die \$4.500 per jaar
verdient, zal snel merken dat de luie sloeber naast hem, die weigert te
werken, zijn \$4.000 van de federale overheid krijgt. Hierdoor blijft
zijn eigen netto-inkomen, na veertig uur hard werken per week, beperkt
tot slechts \$500 per jaar. Hij stopt daarom met werken en schakelt over
naar de negatieve belasting. Hetzelfde zal ongetwijfeld ook gelden voor
mensen die \$5.000 per jaar verdienen, enzovoort.

Het rampzalige proces is nog lang niet voorbij. Als alle mensen die
minder dan \$4.000 verdienen, maar ook degenen die aanzienlijk meer
verdienen, stoppen met werken en naar de bijstand gaan, zullen de totale
uitkeringen enorm stijgen. Deze uitkeringen kunnen alleen gefinancierd
worden door de mensen met een hoger inkomen die blijven werken zwaarder
te belasten. Maar hierdoor zal hun netto-inkomen na belastingen sterk
dalen, waardoor ook velen van hen zullen stoppen met werken en de
bijstand in zullen gaan. Laten we eens kijken naar een man die \$6.000
per jaar verdient. Aanvankelijk heeft hij een netto-inkomen uit arbeid
van slechts \$2.000. Wanneer hij, laten we zeggen, \$500 per jaar moet
betalen om de werkloosheidsuitkeringen van niet-werkenden te
financieren, blijft er nog maar \$1.500 over. Als hij daarnaast ook nog
eens \$1.000 moet bijdragen om de snelle groei van anderen te
ondersteunen, daalt zijn netto-inkomen naar \$500 en gaat hij ook de
bijstand in. De logische conclusie van het gegarandeerde jaarinkomen is
dus een vicieuze spiraal die leidt naar een ramp, met als resultaat dat
bijna niemand meer werkt en iedereen werkloos is.

Naast dit alles zijn er nog enkele belangrijke extra overwegingen. In de
praktijk zal de uitkering, eenmaal vastgesteld op \$4.000, natuurlijk
niet daar blijven. De druk van bijstandscliënten en andere
belangengroepen zal er elk jaar toe leiden dat het basisniveau
onverbiddelijk verhoogd wordt. Dit brengt de vicieuze spiraal en de
economische ramp alleen maar dichterbij. Bovendien zal het gegarandeerd
jaarinkomen, in tegenstelling tot de hoop van conservatieve
voorstanders, niet het bestaande lappendeken van sociale voorzieningen
vervangen. Het zal gewoon bovenop de bestaande programma's komen. Dit is
precies wat er is gebeurd met de ouderdomsprogramma's van de staten. Het
belangrijkste onderwerp binnen het federale Sociale Zekerheidsprogramma
van de New Deal was dat het op een efficiënte manier de toenmalige
lappendeken van ouderdomsprogramma's zou vervangen. In werkelijkheid is
dat niet gebeurd; de oudedagsvoorzieningen zijn nu veel hoger dan in de
jaren dertig. Een steeds groter wordende structuur van Sociale Zekerheid
is simpelweg toegevoegd aan de bestaande programma's. Tot slot is het
idee van president Nixon dat validen van de nieuwe bijstand gedwongen
zouden worden om te werken, een duidelijke leugen. Ten eerste zouden ze
alleen `geschikte' banen moeten vinden. De universiteit van de
staatsinstellingen voor werkloosheidszorg laat zien dat er bijna nooit
`geschikte' banen beschikbaar zijn.

De verschillende regelingen voor een gegarandeerd jaarinkomen bieden
geen echte vervanging voor het algemeen erkende kwaad van het
socialezekerheidsstelsel; in plaats daarvan zouden ze ons alleen maar
dieper in dat kwaad storten. De enige werkbare oplossing is de
libertarische: de afschaffing van sociale voorzieningen ten gunste van
vrijheid en vrijwilligheid voor iedereen, zowel arm als rijk.

\bookmarksetup{startatroot}

\chapter{Inflatie en de
conjunctuurcyclus:}\label{inflatie-en-de-conjunctuurcyclus}

\section{De ineenstorting van het keynesiaanse
paradigma}\label{de-ineenstorting-van-het-keynesiaanse-paradigma}

Tot de jaren 1973-1974 hadden de Keynesianen, die sinds het einde van de
jaren dertig de dominante economische orthodoxie vertegenwoordigden, de
wind in de zeilen. Vrijwel iedereen had de Keynesiaanse opvatting
overgenomen dat er iets in de vrije markteconomie is dat leidt tot
schommelingen in onder- en overbesteding (in de praktijk gaat het bij
Keynesianisme bijna uitsluitend om vermeende onderbesteding). Het was
daarom de taak van de overheid om dit marktfalen te corrigeren. De
overheid moest deze veronderstelde onevenwichtigheid aanpakken door haar
uitgaven en tekorten te beïnvloeden (in de praktijk om ze te vergroten).
Deze essentiële `macro-economische' rol van de overheid moest natuurlijk
worden uitgevoerd onder leiding van een raad van Keynesiaanse economen
(de `Raad van Economische Adviseurs'). Zij zouden in staat zijn de
economie `fijn af te stemmen' om inflatie of recessie te voorkomen en de
juiste hoeveelheid totale uitgaven te reguleren, zodat er blijvend
volledige werkgelegenheid zonder inflatie zou zijn.

In 1973-1974 realiseerden zelfs de Keynesianen zich eindelijk dat er
iets ernstig mis was met dit zelfverzekerde scenario. Het werd tijd om
in verwarring terug te keren naar hun tekentafels. De afgelopen veertig
jaar van Keynesiaanse fine-tuning had de chronische inflatie, die met de
Tweede Wereldoorlog was ontstaan, niet verdwenen. Sterker nog, in die
periode escaleerde de inflatie tijdelijk tot dubbele cijfers, met een
piek van ongeveer 13 procent per jaar. Daarnaast belandden de Verenigde
Staten in diezelfde jaren in de diepste en langste recessie sinds de
jaren dertig. Het had een `depressie' kunnen worden genoemd, als deze
term niet al als onpolitiek door economen was afgekeurd. Dit
opmerkelijke fenomeen, waarbij hoge inflatie samenging met een scherpe
recessie, was simpelweg niet te verenigen met de Keynesiaanse visie.
Economen hadden altijd geweten dat de economie ofwel in een
hoogconjunctuur verkeert, waarbij de prijzen stijgen, ofwel in een
recessie of depressie, gekenmerkt door hoge werkloosheid, waarbij de
prijzen dalen. Tijdens een hoogconjunctuur werd de Keynesiaanse overheid
verondersteld om het `overschot aan koopkracht op te slokken' door de
belastingen te verhogen. Dit was volgens het Keynesiaanse recept, wat
betekende dat ze geld uit de economie moest halen. In tijden van
recessie daarentegen werd de overheid geacht haar uitgaven en tekorten
te verhogen om meer geld in de economie te pompen. Maar wat moest de
overheid doen als de economie zich tegelijkertijd in een staat van
inflatie en een zware recessie bevond? Hoe kon ze tegelijkertijd het
gaspedaal indrukken en de rem activeren?

Al tijdens de recessie van 1958 begonnen de dingen vreemd te verlopen.
Voor het eerst stegen, midden in een recessie, de prijzen van
consumptiegoederen, zij het maar heel beperkt. Het was een wolk die niet
groter was dan een mensenhand en het leek de Keynesianen weinig te
verontrusten.

De consumentenprijzen stegen opnieuw tijdens de recessie van 1966, maar
omdat deze zo mild was, maakten mensen zich daar geen zorgen over. De
scherpe inflatie die optreed tijdens de recessie van 1969-1971 was
echter een flinke schok. Het zou echter een scherpe recessie vergen, die
begon te midden van de dubbele inflatie van 1973-1974, om het
Keynesiaanse economische establishment voorgoed uit zijn evenwicht te
brengen. Het maakte hen bewust dat niet alleen de fijnafstemming was
mislukt, maar ook dat de zogenaamde dode en begraven cyclus nog steeds
aanwezig was. De economie bevond zich nu in een toestand van chronische
inflatie die steeds erger werd, en ze was ook onderhevig aan
voortdurende recessies: inflatoire recessies, ofwel `stagflatie'. Dit
was niet alleen een nieuw fenomeen, het was er ook een dat niet
verklaard kon worden en volgens de theorieën van de economische
orthodoxie zelfs niet kon bestaan.

De inflatie leek maar erger te worden. Gedurende de Eisenhower-jaren was
het ongeveer 1 tot 2 procent per jaar. Tijdens het Kennedy-tijdperk
steeg dit naar 3 tot 4 procent, gevolgd door 5 tot 6 procent onder de
Johnson-regering. In 1973-1974 bereikte de inflatie zelfs ongeveer 13
procent, om daarna `terug te vallen' naar zo'n 6 procent, maar dat
gebeurde alleen onder de druk van een steile en langdurige depressie
tussen 1973 en 1976.

Er zijn dus verschillende zaken die bijna wanhopig verklaard moeten
worden: (1) Waarom is er sprake van chronische en versnellende inflatie?
(2) Waarom blijft de inflatie bestaan, zelfs tijdens diepe depressies?
En om het verder te verhelderen, zouden we ook moeten uitleggen, als we
dat kunnen, (3) Waarom bestaat de conjunctuurcyclus überhaupt? Waarom
zijn er die schijnbaar eindeloze periodes van hoog- en laagconjunctuur?

Gelukkig zijn er antwoorden op deze vragen. Ze worden geleverd door de
tragisch verwaarloosde Oostenrijkse School van de economie en haar
theorie over de geld- en conjunctuurcyclus. Deze theorie is ontwikkeld
in Oostenrijk door Ludwig von Mises en zijn volgeling Friedrich A.
Hayek, en werd begin jaren dertig door Hayek naar de London School of
Economics gebracht. De Oostenrijkse theorie van Hayek over de
conjunctuurcyclus maakte indruk op jongere economen in Groot-Brittannië,
juist omdat deze theorie een bevredigende uitleg bood voor de Grote
Depressie van de jaren dertig. Toekomstige Keynesiaanse leiders zoals
John R. Hicks, Abba P. Lerner, Lionel Robbins en Nicholas Kaldor in
Engeland, en Alvin Hansen in de Verenigde Staten, waren slechts enkele
jaren eerder Hayekianen geweest. Na 1936 overspoelde de Algemene Theorie
van Keynes de planken in een ware `Keynesiaanse Revolutie'. Deze
revolutie verkondigde arrogant dat niemand voor Keynes het had gewaagd
enige verklaring te geven voor de conjunctuurcyclus of de Grote
Depressie. Het is belangrijk te benadrukken dat de Keynesiaanse theorie
niet overwonnen werd door een zorgvuldige discussie of weerlegging van
het Oostenrijkse standpunt; zoals zo vaak in de geschiedenis van de
sociale wetenschappen, werd het Keynesianisme gewoon de nieuwe mode. De
Oostenrijkse theorie werd niet weerlegd, maar simpelweg genegeerd en
vergeten.

Vier decennia lang werd de Oostenrijkse theorie vrijwel volledig
genegeerd door de economische wereld. Alleen Mises (aan de NYU) en Hayek
(in Chicago) en een handvol volgelingen stonden nog achter deze theorie.
Het is dan ook zeker geen toeval dat de huidige heropleving van de
Oostenrijkse economie samenvalt met het fenomeen stagflatie en de
daarmee gepaard gaande afbraak van het Keynesiaanse paradigma. In 1974
vond voor het eerst in tientallen jaren een conferentie van Oostenrijkse
economen plaats aan het Royalton College in Vermont. Later dat jaar was
de economische wereld verbaasd over de toekenning van de Nobelprijs aan
Hayek. Sindsdien zijn er opmerkelijke Oostenrijkse conferenties gehouden
aan de Universiteit van Hartford, Windsor Castle in Engeland en de
Universiteit van New York. Zelfs Hicks en Lerner toonden tekenen van een
gedeeltelijke terugkeer naar hun vroeger verwaarloosde standpunt. Er
zijn regionale conferenties georganiseerd aan de oostkust, westkust, in
het middenwesten en zuidwesten. Er worden boeken gepubliceerd op dit
vakgebied en, misschien nog belangrijker, er zijn verschillende zeer
bekwame afgestudeerden en jonge professoren opgestaan die zich hebben
toegelegd op de Oostenrijkse economie en die in de toekomst ongetwijfeld
veel zullen bijdragen.

\section{\texorpdfstring{\textbf{GELD EN
INFLATIE}}{GELD EN INFLATIE}}\label{geld-en-inflatie}

Er zijn verschillende zaken die bijna wanhopig verklaard moeten worden:
(1) Waarom is er chronische en versnellende inflatie? (2) Waarom blijft
inflatie bestaan, zelfs tijdens heftige depressies? En ten slotte, (3)
Waarom bestaat de conjunctuurcyclus überhaupt? Wat verklaart de
schijnbaar oneindige cyclus van hoog- en laagconjunctuur? Gelukkig zijn
er antwoorden op deze vragen. Ze komen van de vaak verwaarloosde
Oostenrijkse School van de economie en haar theorie over de geld- en
conjunctuurcyclus. Deze theorie is ontwikkeld door Ludwig von Mises en
zijn volgeling Friedrich A. Hayek in Oostenrijk en werd begin jaren
dertig door Hayek naar de London School of Economics gebracht. De
Oostenrijkse theorie van Hayek over de conjunctuurcyclus maakte indruk
op jonge economen in Groot-Brittannië, omdat alleen deze theorie een
bevredigende uitleg bood voor de Grote Depressie van de jaren dertig.
Toekomstige Keynesiaanse denkers zoals John R. Hicks, Abba P. Lerner,
Lionel Robbins en Nicholas Kaldor in Engeland, en Alvin Hansen in de
Verenigde Staten, waren enkele jaren eerder nog aanhangers van Hayek. Na
1936 kwam de Algemene Theorie van Keynes naar voren in een ware
`Keynesiaanse Revolutie'. Deze beweerde arrogant dat niemand vóór hem
werkelijk een verklaring had durven geven voor de conjunctuurcyclus of
de Grote Depressie. Het is belangrijk op te merken dat de Keynesiaanse
theorie niet als gevolg van een zorgvuldige discussie of weerlegging van
het Oostenrijkse standpunt triomfeerde. Zoals vaak in de sociale
wetenschappen gebeurt, werd het Keynesianisme de nieuwe mode. De
Oostenrijkse theorie werd niet weerlegd, maar simpelweg genegeerd en
vergeten. Vier decennia lang werd de Oostenrijkse theorie door het
grootste deel van de economische wereld genegeerd. Alleen Mises (aan de
NYU) en Hayek (in Chicago), samen met enkele volgelingen, hielden vast
aan deze theorie. Het is dan ook geen toeval dat de huidige heropleving
van de Oostenrijkse economie samenvalt met het verschijnsel stagflatie
en de afbraak van het Keynesiaanse paradigma. In 1974 vond de eerste
conferentie van Oostenrijkse economen in tientallen jaren plaats aan het
Royalton College in Vermont. Later dat jaar was de economische wereld
verbaasd over de Nobelprijs die aan Hayek werd toegekend. Sindsdien zijn
er opmerkelijke conferenties georganiseerd aan de Universiteit van
Hartford, Windsor Castle in Engeland en de Universiteit van New York.
Zelfs Hicks en Lerner toonden tekenen van een gedeeltelijke terugkeer
naar hun eerder verwaarloosde standpunt. Er zijn regionale conferenties
gehouden aan de oostkust, westkust, in het middenwesten en het
zuidwesten. Er worden boeken gepubliceerd over dit onderwerp en,
misschien nog belangrijker, er zijn verschillende capabele
afgestudeerden en jonge professoren opgestaan. Zij hebben zich toegelegd
op de Oostenrijkse economie en zullen ongetwijfeld in de toekomst veel
bijdragen.

Wat heeft deze heropleving van de Oostenrijkse theorie te zeggen over
ons probleem? Het eerste dat we moeten benadrukken, is dat inflatie niet
onontkoombaar is en ook geen vereiste voor een groeiende en bloeiende
economie. Gedurende het grootste deel van de negentiende eeuw, met
uitzondering van de jaren van de Oorlog van 1812 en de Burgeroorlog,
daalden de prijzen. Desondanks groeide en industrialiseerde de economie.
Dalende prijzen vormden geen enkele belemmering voor het bedrijfsleven
of de economische welvaart.

Dalende prijzen lijken dus de normale gang van zaken in een groeiende
markteconomie te zijn. Hoe komt het dan dat het idee van gestaag dalende
prijzen zo haaks staat op onze ervaringen dat het een totaal
onrealistische droomwereld lijkt? Waarom zijn de prijzen in de Verenigde
Staten en de rest van de wereld sinds de Tweede Wereldoorlog voortdurend
en zelfs snel gestegen? Voorheen stegen de prijzen sterk tijdens de
Eerste en Tweede Wereldoorlog. Tussendoor daalden ze licht, ondanks de
grote economische bloei van de jaren twintig, om vervolgens sterk te
dalen tijdens de Grote Depressie in de jaren dertig. Kortom, afgezien
van de periodes van oorlog, kwam het idee van inflatie als norm in
vredestijd pas echt op na de Tweede Wereldoorlog.

De meest voorkomende verklaring voor inflatie is dat hebzuchtige
ondernemers de prijzen blijven verhogen om hun winsten te maximaliseren.
Maar heeft de `hebzucht' van bedrijven sinds de Tweede Wereldoorlog
ineens een enorme sprongetje gemaakt? Waren bedrijven in de negentiende
eeuw en tot 1941 niet even `hebzuchtig'? Waarom was er toen geen sprake
van inflatie? Daarnaast, als ondernemers zo gierig zijn dat ze de
prijzen met 10 procent per jaar willen verhogen, waarom blijven ze dan
daar bij? Waarom wachten ze? Waarom verhogen ze de prijzen niet met 50
procent, of verdubbelen of verdrievoudigen ze ze niet meteen? Wat houdt
hen tegen?

Een vergelijkbare fout weerlegt een andere populaire verklaring voor
inflatie: dat vakbonden aandringen op hogere lonen, waardoor bedrijven
op hun beurt de prijzen verhogen. Behalve dat inflatie al bestond in het
oude Rome, lang voordat vakbonden opkwamen, is er ook weinig bewijs dat
vakbondslonen sneller stijgen dan niet-vakbondslonen of dat de prijzen
van vakbondsproducten sneller toenemen dan die van
niet-vakbondsproducten. Daarbij rijst een vergelijkbare vraag: waarom
verhogen bedrijven hun prijzen eigenlijk niet meer? Wat voorkomt dat ze
de prijzen met een aanzienlijke hoeveelheid verhogen? Als vakbonden zo
machtig zijn en bedrijven zo ontvankelijk, waarom stijgen lonen en
prijzen dan niet met 50 of zelfs 100 procent per jaar? Wat houdt hen
tegen?

Een door de overheid gesteunde TV-propagandacampagne van een paar jaar
geleden kwam dichterbij het doel: consumenten werden beschuldigd van de
inflatie omdat ze te `varkensachtig' zouden zijn, omdat ze te veel aten
en uitgaven. Dit biedt tenminste een begin van een verklaring voor wat
bedrijven of vakbonden weerhoudt om nog hogere prijzen te vragen:
consumenten willen die prijzen niet betalen. Een paar jaar geleden
stegen de koffieprijzen explosief. Een jaar of twee later daalden ze
scherp als gevolg van de weerstand van de consument -- mede door een
opvallende consumentenboycot -- maar vooral door een verschuiving in de
koopgewoonten van koffie naar goedkopere vervangproducten. Een beperking
van de consumentenvraag houdt de prijzen dus onder controle.

Maar daarmee schuiven we het probleem een stap terug. Als de
consumentenvraag, zoals logisch lijkt, op een bepaald moment beperkt is,
hoe kan het dan dat die vraag jaar na jaar blijft toenemen en prijs- en
loonstijgingen mogelijk maakt? En als de vraag met 10 procent kan
stijgen, wat houdt haar dan tegen om met 50 procent te stijgen? Kortom,
wat zorgt ervoor dat de consumentenvraag jaar na jaar blijft groeien,
maar toch een grens kent?

Om verder te gaan in deze speurtocht, moeten we de betekenis van de term
`prijs' analyseren. Wat is precies een prijs? De prijs van een bepaalde
hoeveelheid van een product is het bedrag dat de koper ervoor moet
betalen. Als iemand bijvoorbeeld zeven dollar moet uitgeven voor tien
broden, dan is de `prijs' van die tien broden zeven dollar. Omdat we
prijzen meestal per eenheid aangeven, komt de prijs van één brood neer
op 70 cent. Bij deze uitwisseling zijn er dus twee partijen betrokken:
de koper met geld en de verkoper met brood. Het moge duidelijk zijn dat
de interactie tussen beide partijen de prijs op de markt bepaalt. Als er
meer brood beschikbaar komt, zal de prijs van het brood dalen (een
groter aanbod verlaagt de prijs). Aan de andere kant, als de kopers meer
geld in hun portemonnee hebben, zal de prijs van het brood stijgen (een
grotere vraag verhoogt de prijs).

We hebben nu het belangrijke element geïdentificeerd dat de vraag van de
consument en daarmee de prijs beperkt: de hoeveelheid geld die de
consument beschikking heeft. Als het geld in hun zakken met 20 procent
toeneemt, dan verschaft dat ook 20 procent meer ruimte voor hun vraag.
Als alle andere factoren gelijk blijven, zullen de prijzen eveneens met
20 procent stijgen. We hebben dus de essentiële factor gevonden: de
geldvoorraad.

Als we naar de prijzen in de economie kijken, is de totale geldvoorraad
een cruciale factor. Het belang van de geldvoorraad voor het begrijpen
van inflatie wordt duidelijk als we onze beschouwing uitbreiden van de
brood- of koffiemarkt naar de economie als geheel. Alle prijzen worden
namelijk omgekeerd evenredig bepaald door het aanbod van een goed en
direct door de vraag ernaar. Over het algemeen neemt de goederenaanvoer
jaar na jaar toe in onze nog steeds groeiende economie. Vanuit de
aanbodzijde gezien zouden de meeste prijzen dan moeten dalen, en zouden
we momenteel een gestage prijsdaling, vergelijkbaar met de negentiende
eeuw (`deflatie'), moeten ervaren. Als chronische inflatie het gevolg
was van de aanbodzijde, zoals activiteiten van producenten, bedrijven of
vakbonden, zou het totale aanbod van goederen moeten afnemen, wat
resulteert in stijgende prijzen. Aangezien het aanbod van goederen
duidelijk toeneemt, moet de oorzaak van de inflatie aan de vraagzijde
liggen. En zoals we hebben aangetoond, is de dominante factor aan die
vraagzijde de totale geldhoeveelheid.

Inderdaad, als we naar het verleden en het heden van de wereld kijken,
zien we dat de geldhoeveelheid in hoog tempo is gestegen. In de
negentiende eeuw nam de geldhoeveelheid ook toe, maar veel langzamer dan
de groei van goederen en diensten. Sinds de Tweede Wereldoorlog is de
geldhoeveelheid -- zowel hier als in het buitenland -- echter veel
sneller gestegen dan de voorraad goederen. Dit heeft geleid tot
inflatie.

De centrale vraag is dus wie of wat de geldhoeveelheid controleert en
bepaalt, en waarom deze vooral in de afgelopen decennia steeds verder
toeneemt. Om deze vraag te beantwoorden, moeten we eerst onderzoeken hoe
geld in de markteconomie tot stand komt. Geld ontstaat namelijk wanneer
individuen beginnen te kiezen uit een of meerdere nuttige goederen om
als geld te dienen. De beste geldgoederen zijn die waarvoor veel vraag
is, die een hoge waarde per gewichtseenheid hebben, die duurzaam zijn en
dus lang bewaard kunnen blijven, die mobiel zijn en dus gemakkelijk van
de ene naar de andere plaats verplaatst kunnen worden, die gemakkelijk
herkenbaar zijn en die zonder waardeverlies in kleine delen kunnen
worden opgedeeld. Door de eeuwen heen hebben verschillende markten en
samenlevingen een diverse reeks goederen als geld gekozen, variërend van
zout en suiker tot kaurischelpen, vee, tabak en zelfs sigaretten in
krijgsgevangenenkampen tijdens de Tweede Wereldoorlog. Maar in al die
tijd hebben twee grondstoffen altijd de overhand gehad in de strijd om
als geld te functioneren: goud en zilver.

Metalen worden altijd verhandeld op basis van hun gewicht, zoals een ton
ijzer of een pond koper. Ook de prijzen van goud en zilver worden in
gewichtseenheden berekend. Alle moderne munteenheden zijn ontstaan uit
gewichtseenheden van goud of zilver. Zo is de Britse eenheid, het pond
sterling, vernoemd naar een bedrag van één pond zilver. Als we kijken
naar de waarde van het pond door de jaren heen, moeten we opmerken dat
het Britse pond tegenwoordig op de markt slechts twee vijfde van een
ounce zilver waard is. Dit laat zien wat de inflatie en de devaluatie
van het pond met zich meebrengen. De `dollar' was originally een
Boheemse munt die bestond uit een ons zilver. Later werd de `dollar'
gedefinieerd als een twintigste van een ounce goud.

Wanneer een samenleving of een land een bepaalde grondstof als geld
aanneemt en de gewichtseenheid daarvan vervolgens de munteenheid wordt
-- de rekeneenheid in het dagelijks leven -- dan wordt gezegd dat dat
land die specifieke grondstof als `standaard' hanteert. Omdat goud of
zilver door de markten universeel als de beste standaarden worden
gezien, is het voor deze economieën natuurlijk om op de gouden of
zilveren standaard te opereren. In dat geval wordt het aanbod van goud
bepaald door marktkrachten, zoals de technologische voorwaarden van het
aanbod en de prijzen van andere grondstoffen.

Vanaf het moment dat goud en zilver als geld werden geaccepteerd op de
markt, heeft de staat geprobeerd de controle over de geldvoorziening
over te nemen. Dit betreft het bepalen en creëren van de geldhoeveelheid
in de samenleving. Het is duidelijk waarom de staat dit zou willen: het
zou betekenen dat de controle over de geldhoeveelheid van de markt wordt
weggenomen en wordt overgedragen aan een groep mensen die leidinggeven
aan het staatsapparaat. Hun motivatie is eveneens helder: het biedt een
alternatief voor belastingen, die door de belastingbetalers vaak als
zwaarwegend worden ervaren.

Nu kunnen de heersers van de staat hun eigen geld creëren en dit
uitgeven of uitlenen aan hun favoriete bondgenoten. Dit was nog niet zo
eenvoudig totdat de boekdrukkunst werd ontdekt. Vanaf dat moment kon de
staat de definitie van de `dollar', het `pond', de `mark', en zo verder,
aanpassen van gewichtseenheden van goud of zilver naar simpelweg namen
voor stukken papier die door de centrale regering werden gedrukt. Deze
regering kon die papieren kosteloos en bijna naar eigen willekeur
drukken en vervolgens naar hartenlust uitgeven of uitlenen. Het heeft
eeuwen geduurd voordat deze complexe transformatie compleet was, maar
tegenwoordig is de voorraad en de uitgifte van geld volledig in handen
van elke centrale regering. De gevolgen hiervan zijn steeds duidelijker
zichtbaar in onze omgeving.

Stel je voor wat er zou gebeuren als de overheid een bepaalde groep
mensen zou benaderen -- laten we zeggen de familie Jones -- en tegen hen
zou zeggen: `Hier geven we jullie de absolute en onbeperkte macht om
dollars te drukken en om het aantal dollars in omloop te bepalen. Jullie
hebben een volledig monopolie: iedereen die het waagt om deze macht te
gebruiken, zal voor lange tijd gevangen worden gezet als een
kwaadaardige valsemunter. We hopen dat jullie deze macht verstandig
gebruiken.' We kunnen redelijk goed inschatten wat de familie Jones met
deze nieuwe macht zal doen. In het begin zullen ze het voorzichtig en
terughoudend gebruiken om hun schulden af te lossen en misschien om een
paar gewenste dingen te kopen. Maar naarmate ze gewend raken aan de
verleidelijke mogelijkheid om hun eigen geld te drukken, zullen ze deze
macht steeds meer aangrijpen om luxeartikelen te kopen en hun vrienden
te belonen. Het gevolg hiervan zal een voortdurende en toenemende
geldhoeveelheid zijn, wat leidt tot aanhoudende en versnelde inflatie.

Maar dit is precies wat regeringen -- alle regeringen -- hebben gedaan.
In plaats van de monopoliemacht om te vervalsen aan de Jones of andere
families toe te kennen, heeft de overheid deze macht aan zichzelf
`toegekend'. Net zoals de staat zichzelf een monopolie op gelegaliseerde
ontvoering toeeigent en dit dienstplicht noemt; net zoals de staat een
monopolie op gelegaliseerde roof heeft verworven en dit belastingen
noemt; zo heeft de staat ook het monopolie op valsemunterij in handen
gekregen en het verhogen van de voorraad dollars (of franken, marken of
wat dan ook) genoemd. In plaats van een gouden standaard, in plaats van
geld dat voortkomt uit en waarvan het aanbod wordt bepaald door de vrije
markt, leven we nu onder een papieren fiatstandaard. De dollar, de
franc, enzovoort, zijn gewoon stukjes papier met zulke namen erop
gedrukt, die naar believen door de centrale overheid -- het
staatsapparaat -- worden uitgegeven.

Bovendien, omdat het in het belang van een valsemunter is om zoveel
mogelijk geld te drukken, zal de staat hetzelfde doen. Hij zal net
zoveel geld creëren als hij kan, net zoals hij de macht om belasting te
heffen ook zal gebruiken: om zoveel mogelijk geld binnen te halen zonder
al te veel tegenstand te krijgen.

Overheidscontrole over de geldhoeveelheid is dus altijd inflatoir. Dit
geldt om dezelfde reden als voor elk systeem waarbij een groep mensen de
controle heeft over het drukken van geld.

\section{DE FEDERALE RESERVE EN FRACTIONEEL
RESERVEBANKIEREN}\label{de-federale-reserve-en-fractioneel-reservebankieren}

De Federal Reserve, vaak simpelweg de Fed genoemd, is het centrale
bankysteem van de Verenigde Staten. Het staat bekend om zijn rol in het
reguleren van de geldhoeveelheid en het stabiliseren van de economie.
Een belangrijk onderdeel van dit systeem is het fractioneel
reservebankieren. Bij fractioneel reservebankieren houden banken slechts
een fractie van de deposito's in kas en lenen ze de rest uit aan andere
klanten. Dit zorgt ervoor dat banken meer geld in omloop kunnen brengen
dan ze daadwerkelijk in reserve hebben. Het systeem is ontworpen om
ervoor te zorgen dat er voldoende liquiditeit is, maar het brengt ook
risico's met zich mee. Wanneer mensen massaal hun geld komen opnemen,
kan dit leiden tot een bankrun. Omdat banken niet genoeg reserves hebben
om iedereen te betalen, kunnen ze failliet gaan. De overheid en de
centrale bank proberen dit soort situaties te voorkomen door banken te
begeleiden en hen te voorzien van extra liquiditeit in tijden van
crisis. De Fed heeft de taak om de geldhoeveelheid zo te beheren dat
inflatie en deflatie worden voorkomen. Door hun rentevoeten en andere
beleidsmaatregelen aan te passen, beïnvloeden ze de economie en de
levensstandaard van de Amerikaanse burgers. Het fractioneel
reservebankieren is een controversieel onderwerp. Voorstanders beweren
dat het zorgt voor economische groei en flexibiliteit. Tegenstanders
wijzen op de gevaren van een systeem dat afhankelijk is van krediet en
schulden. Dit maakt het noodzakelijk om het centrale banksysteem en het
bankieren in zijn geheel goed te begrijpen.

Inflatie door simpelweg meer geld te drukken wordt tegenwoordig als
verouderd beschouwd. Ten eerste is het te zichtbaar. Als er veel
biljetten met hoge waardes in omloop zijn, kan het publiek wel eens
denken dat het drukken van al die biljetten door de overheid de oorzaak
is van ongewenste inflatie. Dit zou kunnen leiden tot het idee dat de
overheid deze macht wel eens zou kunnen verliezen. In plaats daarvan
hebben regeringen een veel complexere, geraffineerdere en minder
zichtbare manier gevonden om hetzelfde doel te bereiken: het organiseren
van verhogingen in de geldhoeveelheid. Dit stelt hen in staat om
zichzelf meer geld te geven om uit te geven en om politieke groeperingen
te subsidiëren. Het idee is als volgt: in plaats van zich te richten op
het drukken van geld, behouden ze papieren dollars, marken of franken
als basisvaluta (despectief `wettig betaalmiddel'). Daarbovenop komt een
piramide van mysterieuze en onzichtbare, maar niet minder krachtige
vormen van `chequeboekgeld' of direct opvraagbare bankdeposito's. Het
resultaat is een inflatoire motor die door de overheid wordt
gecontroleerd en die niemand begrijpt behalve bankiers, economen en
centrale bankiers. En dat is ook precies de bedoeling.

Ten eerste is het belangrijk om te begrijpen dat het hele commerciële
banksysteem, zowel in de Verenigde Staten als elders, volledig onder
controle staat van de centrale overheid. De banken zijn blij met deze
controle, omdat het hen in staat stelt om geld te creëren. Ze vallen
onder de volledige supervisie van de centrale bank, een
overheidsinstantie waarvan de controle vooral voortkomt uit het
verplichte monopolie op het drukken van geld. In de Verenigde Staten
vervult het Federal Reserve System deze rol als centrale bank. De
Federal Reserve, oftewel `de Fed', staat commerciële banken toe om hun
vraagdeposito's, ook wel `chequeboekgeld' genoemd, bovenop hun eigen
reserves (deposito's bij de Fed) te piramideren met een verhouding van
ongeveer 6:1. Dit betekent dus dat als de bankreserves bij de Fed met
\$1 miljard toenemen, de banken zelfs \$6 miljard aan deposito's kunnen
creëren. En dat is precies wat ze doen; de banken genereren zo \$6
miljard aan nieuw geld.

Waarom vormen direct opvraagbare bankdeposito's het grootste deel van de
geldhoeveelheid? Officieel zijn ze geen geld of wettig betaalmiddel,
zoals Federal Reserve Notes dat zijn. Ze vertegenwoordigen echter wel
een belofte van een bank om deze direct opvraagbare deposito's in
contanten (Federal Reserve Notes) terug te betalen aan de
depositohouder, wanneer deze dat maar wil. Het probleem is dat de banken
het geld niet echt hebben. Ze zijn namelijk zes keer hun reserves
verschuldigd, wat hun eigen betaalrekening bij de Fed is. Toch wordt het
publiek aangemoedigd om de banken te vertrouwen, dankzij de indruk van
soliditeit en onschendbaarheid die het Federal Reserve System om hen
heen creëert. De Fed kan banken in moeilijkheden helpen en doet dat ook.
Als het publiek het proces zou begrijpen en in een opwelling naar de
banken zou rennen om hun geld op te eisen, dan zou de Fed, als dat nodig
is, altijd genoeg geld kunnen drukken om de banken te steunen.

De Fed controleert de snelheid van de monetaire inflatie door het
veelvoud (6:1) van de geldschepping door banken aan te passen. Nog
belangrijker is het feit dat ze de totale hoeveelheid bankreserves
bepaalt. Als de Fed de totale geldhoeveelheid met \$6 miljard wil
verhogen, drukt ze niet daadwerkelijk \$6 miljard bij. In plaats daarvan
verhoogt ze de bankreserves met \$1 miljard en laat ze het aan de banken
over om \$6 miljard aan nieuw chequeboekgeld te creëren. Ondertussen
blijft het publiek onwetend over het proces en de betekenis ervan.

Hoe creëren banken nieuwe deposito's? Simpelweg door ze uit te lenen
tijdens het creatieproces. Stel je voor dat banken \$1 miljard aan
nieuwe reserves ontvangen. In dat geval zullen ze \$6 miljard uitlenen
en deze nieuwe deposito's creëren terwijl ze de leningen verstrekken.
Kortom, wanneer commerciële banken geld uitlenen aan een persoon,
bedrijf of de overheid, geven ze niet het geld uit dat het publiek met
moeite heeft gespaard en in hun kluizen heeft gedeponeerd, zoals veel
mensen denken. Ze lenen nieuwe direct opvraagbare deposito's uit die ze
zelf tijdens de lening creëren. Ze zijn slechts beperkt door de
`reserveverplichtingen,' oftewel het vereiste maximum van deposito's en
reserves (bijvoorbeeld 6:1). Per slot van rekening drukken ze geen
papieren dollars of graven ze goud op. Ze geven gewoon deposito's of
`chequeboek'-vorderingen op zichzelf uit voor contante vorderingen. Dit
is iets wat ze niet kunnen inlossen als het publiek ooit in opstand zou
komen en een vereffening van hun rekeningen zou eisen.

Hoe slaagt de Fed erin om de totale reserves van commerciële banken te
bepalen en deze bijna altijd te verhogen? Ze kan reserves uitlenen aan
de banken, wat ze ook doet, en dat tegen een kunstmatig laag tarief, de
`rediscount rate'. Toch zijn banken niet altijd geneigd om grote
schulden bij de Fed aan te gaan. Daarom zijn de totale uitstaande
leningen van de Fed aan de banken meestal niet hoog. De belangrijkste
manier waarop de Fed de totale reserves bepaalt, is weinig bekend en
begrepen door het publiek: de methode van `open markt aankopen'. Dit
houdt simpelweg in dat de Federal Reserve Bank actief op de open markt
gaat en activa aankoopt. Het maakt niet zoveel uit wat voor soort activa
de Fed koopt. Zo kan dat bijvoorbeeld een zakcalculator van \$20 zijn.
Stel je voor dat de Fed een zakrekenmachine koopt van XYZ Electronics
voor \$20. De Fed doet de aankoop, maar het belangrijkste voor onze
uitleg is dat XYZ Electronics met de cheque van \$20 naar de Federal
Reserve Bank gaat. De Fed verhandelt niet met particuliere
rekeninghouders, alleen met banken en de federale overheid zelf. Daarom
kan XYZ Electronics maar één ding doen met die cheque van \$20: hem
deponeren bij zijn eigen bank, bijvoorbeeld Acme Bank. Op dat moment
vindt er een nieuwe transactie plaats: XYZ krijgt een toevoeging van
\$20 op zijn betaalrekening in de vorm van direct opvraagbare
deposito's. In ruil daarvoor ontvangt Acme Bank een cheque van de
Federal Reserve Bank, die aan haar wordt overgemaakt.

Het eerste wat er is gebeurd, is dat de geldvoorraad van XYZ met \$20 is
gestegen. Dit is de nieuw gestegen rekening bij de Acme Bank. De
geldvoorraad van niemand anders is veranderd. Aan het eind van deze
eerste fase -- fase I -- is de geldhoeveelheid met \$20 toegenomen, net
zoveel als de aankoop van een activum door de Fed. Als je je afvraagt
waar de Fed de \$20 vandaan heeft gehaald om de rekenmachine te kopen,
dan is het antwoord eenvoudig: de Fed heeft de \$20 uit het niets
gecreëerd door simpelweg een cheque uit te schrijven. Niemand, noch de
Fed, noch iemand anders, had de \$20 voordat het werd gecreëerd in het
proces van de uitgaven van de Fed.

Maar dat is nog niet alles. De Acme Bank ontdekt tot haar vreugde dat ze
een cheque van de Federal Reserve heeft. Ze haast zich naar de Fed,
stort de cheque en krijgt een toename van \$20 in haar reserves, oftewel
haar `direct opvraagbare deposito's bij de Fed'. Met deze stijging van
\$20 kan het banksysteem zijn krediet uitbreiden. Dit betekent dat er
meer vraagdeposito's worden gecreëerd in de vorm van leningen aan
bedrijven, consumenten of de overheid, totdat de totale toename in
chequeboekgeld \$120 bedraagt. Aan het einde van fase II hebben we dus
een stijging van \$20 in bankreserves, die is ontstaan door de aankoop
van de rekenmachine door de Fed. Daarnaast zien we een toename van \$120
in direct opvraagbare bankdeposito's en een stijging van \$100 in
bankleningen aan bedrijven of anderen. De totale geldhoeveelheid is met
\$120 toegenomen, waarvan \$100 is gecreëerd door de banken tijdens het
uitlenen van chequeboekgeld en \$20 door de Fed tijdens de aankoop van
de rekenmachine.

In de praktijk besteedt de Fed niet veel tijd aan het kopen van
willekeurige activa. Haar aankopen zijn zo omvangrijk dat ze de economie
moet stimuleren met regelmatig en zeer liquide activa. Dit betekent in
de praktijk dat ze vooral Amerikaanse staatsobligaties en andere
effecten van de Amerikaanse overheid koopt. De Amerikaanse
staatsobligatiemarkt is enorm en uiterst liquide, waardoor de Fed zich
niet hoeft te bemoeien met de politieke conflicten die ontstaan bij het
kiezen van particuliere aandelen of obligaties. Voor de overheid heeft
dit ook het voordeel dat het helpt om de markt voor staatsobligaties
stabiel te houden en om de prijzen van deze obligaties hoog te houden.

Stel je voor dat een bank, misschien onder druk van haar
depositohouders, een deel van haar giroreserves moet omzetten in harde
valuta. Wat zou er dan gebeuren met de Fed, nu haar cheques nieuwe
bankreserves hebben gecreëerd? Moet ze dan failliet gaan of iets
dergelijks? Nee, dat is niet het geval. De Fed heeft het monopolie op
het drukken van geld en kan -- en zal -- simpelweg haar direct
opvraagbare deposito's aflossen door zoveel Federal Reserve Notes te
drukken als nodig is. Kortom, als een bank naar de Fed komt en \$20 aan
contanten eist voor haar reserves -- of zelfs \$20 miljoen -- dan hoeft
de Fed alleen maar dat bedrag te drukken en uit te betalen. Het feit dat
de Fed haar eigen geld kan drukken, plaatst haar in een unieke en
benijdenswaardige positie.

Hier hebben we eindelijk de sleutel tot het mysterie van het moderne
inflatieproces. Dit proces draait om de voortdurende uitbreiding van de
geldvoorraad, die plaatsvindt door aankopen van staatsobligaties door de
Fed op de open markt. Als de Fed de geldvoorraad met \$6 miljard wil
vergroten, koopt ze op de open markt staatsobligaties tot een totaal van
\$1 miljard, ervan uitgaande dat de geldmultiplicator van
vraagdeposito's tot reserves 6:1 is. Hierdoor wordt het doel snel
bereikt. In feite gaat de Fed week na week, zelfs terwijl deze regels
worden voorgelezen, de open markt op in New York. Ze koopt de
staatsobligaties die ze heeft besloten aan te schaffen en bepaald zo de
mate van monetaire inflatie.

De monetaire geschiedenis van deze eeuw laat zien dat de beperkingen
voor de staat om inflatie te creëren, herhaaldelijk zijn versoepeld. Eén
controle na de andere is verwijderd, waardoor de overheid nu in staat is
om de geldhoeveelheid -- en dus de prijzen -- naar wens te verhogen. In
1913 werd het Federal Reserve System opgericht om dit verfijnde
piramideproces mogelijk te maken. Dit nieuwe systeem leidde tot een
aanzienlijke uitbreiding van de geldhoeveelheid en inflatie, zodat alle
oorlogskosten tijdens de Eerste Wereldoorlog betaald konden worden. In
1933 werd er opnieuw een ingrijpende stap gezet: de regering van de
Verenigde Staten schakelde over van de goudstandaard. Dit betekende dat
dollars, hoewel ze nog steeds wettelijk waren gedefinieerd op basis van
een bepaald gewicht aan goud, niet langer omgewisseld konden worden
tegen goud. Kortom, vóór 1933 was er een belangrijke hinder voor de Fed
om de geldhoeveelheid te verhogen: Federal Reserve Notes waren op dat
moment inwisselbaar voor goud in een equivalente hoeveelheid.

Er is een cruciaal verschil tussen goud en Federal Reserve Notes. De
overheid kan nieuw goud niet zomaar creëren. Goud moet uit de grond
worden gehaald, wat een kostbaar proces is. Federal Reserve Notes
daarentegen kunnen vrijwel zonder kosten aan grondstoffen naar believen
worden uitgegeven. In 1933 schakelde de Amerikaanse overheid over op
fiatgeld en verwijderde ze de beperkingen rondom goud, waardoor de
inflatiepotentie toenam. De papieren dollar zelf werd de geldstandaard,
met de overheid als de monopolistische uitgever van dollars. Het
verlaten van de goudstandaard opende de deur naar de sterke geld- en
prijsinflatie die de Verenigde Staten ervoeren tijdens en na de Tweede
Wereldoorlog.

Maar er was nog één probleem met de inflatoire aanpak van de Amerikaanse
overheid, namelijk een beperking op de neiging tot inflatie. Hoewel de
Verenigde Staten in eigen land geen goud meer hadden, waren ze nog
steeds verplicht om papieren dollars (en uiteindelijk bankdollars) van
buitenlandse regeringen in goud terug te betalen, als die dat zouden
verzoeken. Kortom, we zaten internationaal nog steeds vast aan een
beperkte en verzwakte vorm van de goudstandaard. In de jaren vijftig en
zestig pompten de Verenigde Staten de geldhoeveelheid en prijzen op.
Hierdoor stapelden dollars en dollarclaims (in contanten en chequevorm)
zich op in de handen van Europese regeringen. Na veel economisch
gekonkel en politieke manoeuvres om buitenlandse regeringen ervan te
weerhouden hun recht op goudinwisseling uit te oefenen, verklaarden de
Verenigde Staten in augustus 1971 hun nationale bankroet. Ze zegden hun
contractuele verplichtingen op en sloten `het goudraam'. Het is geen
toeval dat deze afschaffing van de laatste resten van goudrestricties
volgde op de inflatie met dubbele cijfers in 1973-1974, en soortgelijke
inflatieverschijnselen in de rest van de wereld.

We hebben nu de chronische en verergerende inflatie in de huidige wereld
en de Verenigde Staten uitgelegd. Dit ongelukkige gevolg is het
resultaat van een voortdurende verschuiving deze eeuw van goud naar door
de overheid uitgegeven papier als standaardgeld. Daarnaast speelt de
ontwikkeling van centraal bankieren en het stapelen van chequeboekgeld
bovenop opgeblazen papiergeld een rol. Beide samenhangende
ontwikkelingen komen uiteindelijk neer op één ding: de overheid heeft de
controle over de geldhoeveelheid overgenomen.

Als we het probleem van inflatie hebben uitgelegd, hebben we de
conjunctuurcyclus, recessies en het nieuwe mysterieuze fenomeen van
stagflatie nog niet behandeld. Wat is de oorzaak van de
conjunctuurcyclus en wat verklaart stagflatie?

\section{BANKKREDIET EN DE
CONJUNCTUURCYCLUS}\label{bankkrediet-en-de-conjunctuurcyclus}

Er was echter nog één probleem met de inflatoire aanpak van de
Amerikaanse overheid, namelijk een beperking op de inflatietendens.
Hoewel de Verenigde Staten in eigen land geen goud meer hadden, waren ze
nog steeds verplicht om papieren dollars (en uiteindelijk bankdollars)
aan buitenlandse regeringen in goud terug te betalen als dat werd
verzocht. Kortom, we bevonden ons internationaal nog steeds in een
beperkte en verzwakte variant van de goudstandaard. In de jaren vijftig
en zestig pompten de Verenigde Staten de geldhoeveelheid en prijzen op.
Dit leidde ertoe dat dollars en dollarclaims (in contanten en cheques)
zich ophoopten in de handen van Europese regeringen. Na veel economisch
gekonkel en politieke manoeuvres om buitenlandse regeringen te
overtuigen hun recht op goudinwisseling niet uit te oefenen, verklaarden
de Verenigde Staten in augustus 1971 hun nationale bankroet. Ze zegden
hun contractuele verplichtingen op en sloten het `goudraam'. Het is dan
ook geen verrassing dat deze afschaffing van de laatste goudrestricties
samenviel met de inflatie met dubbele cijfers in 1973-1974, evenals
soortgelijke inflatieverschijnselen elders in de wereld. We hebben nu de
chronische en verergerende inflatie in de huidige wereld en de Verenigde
Staten verklaard. Dit is het ongelukkige gevolg van een voortdurende
verschuiving deze eeuw van goud naar door de overheid uitgegeven papier
als standaardgeld. Daarnaast spelen de ontwikkeling van centraal
bankieren en het ophopen van chequeboekgeld bovenop opgeblazen
papiergeld een belangrijke rol. Beide onderling samenhangende
ontwikkelingen leiden uiteindelijk tot één conclusie: de overheid heeft
de controle over de geldhoeveelheid overgenomen. Hoewel we het probleem
van inflatie hebben besproken, hebben we de conjunctuurcyclus, recessies
en het nieuwe fenomeen van stagflatie nog niet behandeld. Wat verklaart
de conjunctuurcyclus en wat is de oorzaak van stagflatie?

De conjunctuurcyclus deed zijn intrede in de westerse wereld aan het
einde van de achttiende eeuw. Het was een opmerkelijk fenomeen, omdat er
eigenlijk geen duidelijke reden voor leek te zijn en het eerder niet had
bestaan. Deze cyclus bestond uit een regelmatig terugkerende (maar niet
strikt periodieke) reeks van pieken en dalen, oftewel een boom-en-bust
cyclus. In de inflatoire periodes waren er toegenomen
bedrijfsactiviteit, hogere werkgelegenheid en stijgende prijzen. Deze
periodes werden scherp gevolgd door recessies of depressies, die
gekenmerkt werden door afnemende bedrijfsactiviteit, stijgende
werkloosheid en dalende prijzen. Na een tijdje van zulke recessies vond
er vervolgens herstel plaats en begon de fase van hoogconjunctuur
opnieuw.

A priori is er geen reden om dit soort cyclische patronen van
economische activiteit te verwachten. Natuurlijk zijn er cyclische
golven in specifieke sectoren; zo zal de cyclus van de zevenjarige
sprinkhaan leiden tot een zevenjarige cyclus in de
bestrijdingsactiviteiten, de productie van bestrijdingsmiddelen en
-apparatuur, enzovoort. Maar er is geen reden om te denken dat
boom-bustcycli ook in de algemene economie zullen optreden. In feite zou
je het tegenovergestelde verwachten, omdat de vrije markt doorgaans
soepel en efficiënt functioneert, zonder grote verzameling fouten. Dit
wordt duidelijk wanneer een hoogconjunctuur plotseling omslaat in een
laagconjunctuur, met zware verliezen als gevolg. Voor het einde van de
achttiende eeuw waren er trouwens ook geen algemene cycli. Over het
algemeen verliepen de zaken soepel en gelijkmatig, totdat er plotseling
een onderbreking optrad. Een hongersnood kon namelijk leiden tot een
ineenstorting in een agrarisch land; de koning kon een groot deel van
het geld van de financiers in beslag nemen, wat een plotselinge
depressie veroorzaakte; of een oorlog kon de handelspatronen verstoren.
In al deze gevallen was er een specifieke schok voor de handel,
veroorzaakt door een gemakkelijk aan te wijzen, eenmalige oorzaak. Er
was geen verdere verklaring nodig.

Waarom het nieuwe fenomeen van de conjunctuurcyclus? Men zag dat deze
cyclus zich voordeed in de economisch meest ontwikkelde gebieden van elk
land: in de havensteden en de regio's die handel dreven met de meest
geavanceerde wereldcentra van productie en activiteit. In deze periode
ontstonden er in West-Europa, vooral in de belangrijkste productie- en
handelscentra, twee belangrijke fenomenen: industrialisatie en
commercieel bankieren. Het commerciële bankieren was hetzelfde soort
`fractionele reserve'-bankieren dat we eerder hebben besproken. Londen
was de locatie van 's werelds eerste centrale bank, de Bank of England,
die rond de eeuwwisseling van de achttiende eeuw tot stand kwam. Tegen
de negentiende eeuw begonnen er in deze nieuwe economische discipline,
onder financiële schrijvers en commentatoren, twee soorten theorieën te
ontstaan. Deze theorieën trachtten het nieuwe en ongewenste fenomeen te
verklaren: sommige legden de schuld bij de opkomst van de industrie,
terwijl anderen zich richtten op het banksysteem. De eerste groep
economen zocht de oorzaak van de conjunctuurcyclus diep in de vrije
markteconomie. Het was voor hen gemakkelijk om te pleiten voor de
afschaffing van de markt (bijvoorbeeld Karl Marx) of om te verlangen
naar strikte controle en regulering door de overheid om de
conjunctuurcyclus te verlichten (bijvoorbeeld Lord Keynes). De andere
groep economen, die de schuld bij het fractionele reserve-banking
systeem legde, plaatst de oorzaak buiten de markteconomie en op een
terrein -- geld en bankieren -- dat zelfs in het Engelse
klassiek-liberalisme nooit onderhevig was aan strenge overheidscontrole.
Zelfs in de negentiende eeuw betekende het wijzen naar de banken dus in
wezen dat men de overheid verantwoordelijk maakte voor de
boom-en-bustcyclus.

We kunnen hier niet in detail ingaan op de vele denkfouten van de
scholen die de markteconomie de schuld geven van de economische cycli.
Het is voldoende om te zeggen dat deze theorieën de prijsstijging
tijdens een piek of de daling tijdens een recessie niet kunnen
verklaren. Daarnaast kunnen ze ook de grote cluster van fouten niet
uitleggen die plotseling optreden in de vorm van aanzienlijke verliezen
wanneer een piek in een dal omslaat.

De eerste economen die een cyclustheorie ontwikkelden waarin het geld-
en banksysteem centraal stond, waren de Engelse klassieke econoom David
Ricardo uit het begin van de negentiende eeuw en zijn navolgers. Zij
ontwikkelden de `monetaire theorie' van de conjunctuurcyclus. De
Ricardiaanse theorie luidt ongeveer als volgt: fractionele
reservebanken, die worden aangemoedigd en gecontroleerd door de overheid
en haar centrale bank, breiden het krediet uit. Naarmate het krediet
toeneemt en bovenop het papiergeld en goud wordt gestapeld, groeit de
geldhoeveelheid. Dit gebeurt niet alleen in de vorm van bankdeposito's,
maar ook, in die historische periode, via bankbiljetten. De expansie van
de geldvoorraad leidt tot stijgende prijzen en zet een inflatoire boom
in gang. Terwijl deze boom aanhoudt, gevoed door de piramide van
bankbiljetten en deposito's die boven op goud zijn gestapeld, stijgen
ook de binnenlandse prijzen. Dit betekent dat de binnenlandse prijzen
hoger worden dan die van geïmporteerde goederen, wat leidt tot een
toename van de import en een afname van de export. Er ontstaat een
tekort op de betalingsbalans dat steeds groter wordt. Dit tekort zal
moeten worden gedekt door goud dat uit het inflatoire land wegstroomt
naar landen met harde valuta. Terwijl het goud wegstroomt, zal de
groeiende geld- en bankpiramide steeds topzwaarder worden, waardoor de
banken in toenemende mate gevaar lopen failliet te gaan. Uiteindelijk
zullen de overheid en de banken hun expansie moeten stoppen. Om zichzelf
te redden, zullen de banken hun leningen en chequeboekgeld moeten
inkrimpen.

De plotselinge verschuiving van bancaire kredietexpansie naar krimp
verandert het economische beeld, waardoor de bust snel volgt op de boom.
De banken moeten hun kredietverlening verminderen en bedrijven en
economische activiteiten moeten onder druk hun schulden aflossen en
inkrimpen. De daling van de geldhoeveelheid leidt op haar beurt tot een
algemene prijsdaling, ook wel `deflatie' genoemd. De recessie- of
depressiefase is begonnen. Maar naarmate de geldhoeveelheid en prijzen
dalen, worden goederen weer concurrerender ten opzichte van buitenlandse
producten. Dit leidt tot een ommekeer in de betalingsbalans, die
overschot in plaats van tekort vertoont. Goud stroomt het land binnen
en, omdat bankbiljetten en deposito's krimpen bovenop een groeiende
goudbasis, verbetert de financiële situatie van de banken en komt het
herstel op gang.

De Ricardiaanse theorie had verschillende kenmerkende aspecten. Ten
eerste verklaarde ze het gedrag van prijzen door te kijken naar
veranderingen in de beschikbaarheid van bankgeld, dat tijdens
economische bloei altijd toenam en tijdens recessies afnam. Daarnaast
gaf de theorie uitleg over de betalingsbalans. Bovendien legde ze een
verbinding tussen de economische hoogtes en laagtes, waarbij de recessie
werd beschouwd als een gevolg van de eerdere bloei. Het was niet alleen
een gevolg, maar ook een noodzakelijke aanpassing om de economie te
herstructureren na de ondoordachte interventies die de inflatoire boom
hadden veroorzaakt.

Kortom, voor het eerst werd de recessie niet gezien als een bezoeking
uit de hel of als een catastrofe die ontstond door de innerlijke werking
van de geïndustrialiseerde markteconomie. De Ricardianen begrepen dat
het grootste kwaad de voorafgaande inflatoire boom was, het gevolg van
overheidsinterventie in het geld- en banksysteem. Ze realiseerden zich
dat de recessie, hoe onwelkom de symptomen ook zijn, in feite een
noodzakelijk aanpassingsproces is waarin de interventionele boom uit het
economische systeem wordt verwijderd. De depressie is het proces waarbij
de markteconomie zich aanpast, de excessen en verstoringen van de
inflatoire boom van zich afwerpt en een gezonde economische toestand
herstelt. De depressie is dus een onaangename, maar noodzakelijke
reactie op de verstoringen en excessen van de vorige hoogconjunctuur.

Waarom komt de conjunctuurcyclus steeds terug? Waarom begint de volgende
boom-en-bustcyclus altijd opnieuw? Om deze vragen te beantwoorden,
moeten we de beweegredenen van banken en de overheid begrijpen.
Commerciële banken leven van het uitbreiden van krediet en het creëren
van nieuwe geldhoeveelheden. Ze zijn er dus van nature op uit om dit te
doen, oftewel `krediet te monetariseren' wanneer dat mogelijk is. De
overheid is ook gebaat bij inflatie, zowel om haar eigen inkomsten te
verhogen (door geld bij te drukken of omdat het banksysteem
overheidstekorten kan financieren) als om bepaalde economische en
politieke groepen te subsidiëren via een bloei en goedkoop krediet. Zo
begrijpen we waarom de eerste booms beginnen. De regering en de banken
trekken zich terug zodra een ramp dreigt en het crisispunt is bereikt.
Maar naarmate er goud het land binnenstromen, verbetert de situatie van
de banken. Wanneer de banken zich redelijk hebben hersteld, bevinden ze
zich in een zelfverzekerde positie om hun natuurlijke neiging tot het
opblazen van de geld- en kredietvoorraad opnieuw op te pakken. En zo
begint de volgende boom, waardoor de kiem voor de volgende
onvermijdelijke bust wordt gelegd\ldots{}

De Ricardiaanse theorie verklaarde dus ook het voortdurende terugkeren
van de conjunctuurcyclus. Echter, er zijn twee dingen die het niet
verklaart. Ten eerste, en dit is het meest wezenlijke punt, legt het
niet uit waarom zakenmensen vaak plotseling een heleboel fouten maken
wanneer de crisis zich aandient en de recessie volgt op de bloei.
Zakelijke professionals zijn immers getraind om goede voorspellingen te
doen. Het is dan ook bijzonder opmerkelijk dat ze plotseling een reeks
ernstige fouten maken die hen dwingen tot wijdverspreide en aanzienlijke
verliezen. Daarnaast is er nog een ander belangrijk kenmerk van elke
conjunctuurcyclus: zowel de hoogtes als de laagtes zijn meestal veel
ernstiger in de kapitaalgoederenindustrie -- de sector die machines,
apparatuur, fabrieken of industriële grondstoffen produceert -- dan in
de consumptiegoederenindustrie. De Ricardiaanse theorie heeft echter
geen verklaring voor dit aspect van de cyclus.

De Oostenrijkse, of Misesiaanse, theorie van de conjunctuurcyclus bouwt
voort op de Ricardiaanse analyse en ontwikkelt een eigen theorie van
`monetaire overinvestering' of, preciezer, `monetaire desinvestering'.
Deze Oostenrijkse theorie kan niet alleen de verschijnselen verklaren
die door de Ricardianen zijn beschreven, maar ook het foutencluster en
de zwaardere schommelingen in de kapitaalgoederenindustrie. Zoals we
zullen zien, is zij bovendien de enige theorie die het moderne fenomeen
van stagflatie begrijpt.

Mises begint, net als de Ricardianen, met de stelling dat de overheid en
haar centrale bank de kredietexpansie aanjagen door activa op te kopen.
Hiermee vergroten ze de bankreserves. De banken kunnen vervolgens hun
kredietverlening uitbreiden, waardoor de geldhoeveelheid van de natie
toeneemt in de vorm van girale deposito's (particuliere bankbiljetten
zijn vrijwel verdwenen). Net als de Ricardianen constateert Mises dat
deze uitbreiding van bankgeld de prijzen opdrijft en inflatie
veroorzaakt.

Maar zoals Mises opmerkte, onderschatten de Ricardianen de nadelige
gevolgen van de inflatie van bankkredieten. Er speelt namelijk iets nog
sinisterders mee. De uitbreiding van bankkredieten verhoogt niet alleen
de prijzen, maar drukt ook kunstmatig de rente. Dit leidt tot
misleidende signalen voor ondernemers, waardoor zij onhoudbare en
niet-economische investeringen doen.

Op de vrije en onbelemmerde markt wordt de rente op leningen uitsluitend
bepaald door de `tijdsvoorkeuren' van alle individuen binnen de
markteconomie. De essentie van elke lening is dat een `huidig goed'
(geld dat nu kan worden gebruikt) wordt geruild voor een `toekomstig
goed' (een schuldbekentenis die op een later moment kan worden
gebruikt). Mensen geven doorgaans de voorkeur aan het hebben van geld
nu, in plaats van in de toekomst hetzelfde bedrag te ontvangen. Hierdoor
is er op de markt altijd een premie te behalen voor huidige goederen ten
opzichte van toekomstige goederen. Deze premie, ook wel `agio' genoemd,
vormt de rentevoet, en de hoogte ervan varieert afhankelijk van de mate
waarin mensen het heden verkiezen boven de toekomst, oftewel de mate van
hun tijdsvoorkeur.

De tijdsvoorkeuren van mensen bepalen in grote mate hoe zij sparen en
investeren voor de toekomst, in vergelijking met hoeveel zij nu willen
consumeren. Wanneer de tijdsvoorkeur van mensen daalt, dat wil zeggen
als hun voorkeur voor het heden ten opzichte van de toekomst afneemt,
zullen ze geneigd zijn minder te consumeren en meer te sparen en te
beleggen. Tegelijkertijd zal, om dezelfde reden, ook de rentevoet dalen,
oftewel de tijdkorting. Economische groei ontstaat voornamelijk doordat
de tijdsvoorkeuren dalen. Dit leidt tot een betere verhouding tussen
sparen en investeren aan de ene kant, en consumptie aan de andere kant,
en tot een lagere rente.

Maar wat gebeurt er als de rente daalt, niet omdat mensen vrijwillig hun
tijdsvoorkeuren verlagen en meer sparen, maar door overheidsingrijpen
dat de uitbreiding van bankkredieten en bankgeld aanmoedigt? Het nieuwe
chequeboekgeld dat ontstaat uit bankleningen aan bedrijven zal op de
markt komen als krediet en zal, althans in eerste instantie, de rente
verlagen. Wat gebeurt er dus als de rente kunstmatig daalt door
interventie, in plaats van op natuurlijke wijze door veranderingen in de
waarderingen en voorkeuren van consumenten?

Wat er dan gebeurt, zijn problemen. Zakenmensen die een daling van de
rente opmerken, reageren zoals ze altijd zouden moeten doen op zulke
veranderingen in marktsignalen: ze zullen meer investeren in
kapitaalgoederen. Investeringen, vooral in langlopende en tijdrovende
projecten, die voorheen onrendabel leken, lijken nu aantrekkelijk door
de lagere rente. Kortom, ondernemers reageren alsof de besparingen
daadwerkelijk zijn toegenomen: ze gaan die vermeende besparingen
investeren. Ze breiden hun investeringen uit in duurzame apparatuur,
kapitaalgoederen, industriële grondstoffen en de bouw, in plaats van
zich enkel te richten op de productie van consumptiegoederen.

Bedrijven lenen graag het nieuwe, toegenomen bankgeld dat tegen
goedkopere tarieven beschikbaar komt. Ze gebruiken dit geld om te
investeren in kapitaalgoederen. Uiteindelijk resulteert dit in hogere
lonen voor werknemers in de kapitaalgoederenindustrie. De grotere vraag
van bedrijven drijft de arbeidskosten op. Bedrijven geloven dat ze deze
hogere kosten kunnen dragen, omdat ze zijn misleid door de interventie
van de overheid en de banken op de leningenmarkt. Ook het verstoren van
het rentesignaal van de markt speelt hierbij een rol. Dit signaal
bepaalt namelijk hoeveel middelen besteed worden aan de productie van
kapitaalgoederen en hoeveel aan consumptiegoederen.

Er ontstaan problemen wanneer de arbeiders het nieuwe bankgeld, dat ze
hebben ontvangen in de vorm van hogere lonen, beginnen uit te geven. De
tijdsvoorkeuren van het publiek zijn niet echt veranderd; de mensen
willen niet meer sparen dan voorheen. Daarom zullen de arbeiders het
grootste deel van hun nieuwe inkomen besteden en hun oude verhoudingen
tussen consumptie en sparen herstellen. Dit betekent dat ze hun uitgaven
weer richten op de consumptiegoederenindustrie. Ze sparen en investeren
niet voldoende om de nieuw geproduceerde machines, kapitaalgoederen en
industriële grondstoffen te kopen. Het gebrek aan genoeg sparen en
investeren om alle nieuwe kapitaalgoederen te verwerven tegen de
verwachte en bestaande prijzen leidt tot een plotselinge en scherpe
depressie in de kapitaalgoederenindustrie. Zodra de consumenten hun
gewenste verhoudingen tussen consumptie en investeren herstellen, blijkt
dat het bedrijfsleven te veel in kapitaalgoederen heeft geïnvesteerd --
vandaar de term `monetaire overinvesteringstheorie' -- en te weinig in
consumptiegoederen. Het bedrijfsleven liet zich misleiden door de
overheidsinterventie en de kunstmatige verlaging van de rente. Dit
leidde tot de illusie dat er meer spaargeld beschikbaar was voor
investeringen dan in werkelijkheid het geval was. Zodra het nieuwe
bankgeld door het systeem was gefilterd en de consumenten hun oude
verhoudingen hadden hersteld, werd duidelijk dat er niet genoeg
spaargeld was om alle goederen van de producenten te kopen. Het
bedrijfsleven had het beperkte beschikbare spaargeld verkeerd
geïnvesteerd, wat aansluit bij de `monetaire malin-investeringstheorie'.
Het resultaat was dat bedrijven te veel in kapitaalgoederen en te weinig
in consumptiegoederen hadden geïnvesteerd.

De inflatoire boom leidt tot verstoringen in het prijs- en
productiesysteem. Tijdens de boom zijn de prijzen van arbeid,
grondstoffen en machines in de kapitaalgoederenindustrie te hoog om
winstgevend te zijn, zodra consumenten hun oude voorkeuren voor
consumptie en investering weer laten gelden. Daarom wordt de `depressie'
gezien -- meer nog dan in de Ricardiaanse theorie -- als een
noodzakelijke en gezonde periode. In deze fase schudt de markteconomie
de ondeugdelijke en oneconomische investeringen van de boom van zich af
en herstelt het de verhoudingen tussen consumptie en investeringen die
de consumenten echt willen. De depressie is een pijnlijk maar
noodzakelijk proces, waarin de vrije markt zich ontdoet van de
overschotten en fouten van de boom. Zo kan de markteconomie opnieuw haar
functie als efficiënte dienaar van de massa consumenten oppakken.
Aangezien de prijzen van de productiefactoren -- land, arbeid, machines
en grondstoffen -- tijdens de boom te hoog waren in de
kapitaalgoederenindustrie, zullen deze prijzen moeten dalen in de
recessie. Dit gaat door totdat de juiste marktverhoudingen van prijzen
en productie zijn hersteld.

Met andere woorden, de inflatoire boom zorgt niet alleen voor een
algemene stijging van de prijzen, maar verstoort ook de relatieve
prijzen. De verhouding tussen verschillende prijstypes komt onder druk
te staan. Een inflatoire kredietexpansie zal inderdaad leiden tot hogere
prijzen. Echter, de prijzen en lonen in de kapitaalgoederenindustrie
zullen sneller stijgen dan die in de consumptiegoederenindustrie. Dit
betekent dat de boom heftiger zal zijn in de kapitaalgoederenindustrie
dan in de consumptiegoederenindustrie. Aan de andere kant zal tijdens de
aanpassingsperiode van de depressie de nadruk liggen op het verlagen van
de prijzen en lonen in de kapitaalgoederenindustrie ten opzichte van de
consumptiegoederenindustrie. Dit is nodig om middelen terug te laten
stromen van de bloeiende kapitaalgoederenindustrie naar de
achterblijvende consumptiegoederenindustrie. Hoewel alle prijzen zullen
dalen door de inkrimping van het bankkrediet, zullen de prijzen en lonen
van kapitaalgoederen sterker dalen dan die van consumptiegoederen.
Kortom, zowel de boom als de recessie zullen intenser zijn in de
kapitaalgoederenindustrie dan in de consumptiegoederenindustrie. Dit
verklaart de grotere intensiteit van de conjunctuurcycli in de eerste
soort industrie.

Er lijkt echter een fout in de theorie te zitten. Als werknemers het
verhoogde geld, in de vorm van hogere lonen, snel ontvangen en
vervolgens hun gewenste verhoudingen tussen consumptie en investeren
weer tot uiting laten komen, hoe kan het dan dat booms jarenlang
aanhouden zonder dat er vergelding plaatsvindt? Waarom komen
ondeugdelijke investeringen of fouten, veroorzaakt door de verstoring
van marktsignalen door banken, niet aan het licht? Kortom, waarom begint
het aanpassingsproces van de depressie zo laat? Het antwoord is dat de
booms inderdaad van zeer korte duur zouden zijn -- laten we zeggen een
paar maanden -- als de kredietexpansie door banken en de daaropvolgende
verlaging van de rente onder het niveau van de vrije markt slechts een
eenmalige gebeurtenis zou zijn. Het cruciale punt is echter dat de
kredietexpansie niet tijdelijk is. Deze blijft maar doorgaan. Hierdoor
krijgen consumenten nooit de kans om hun favoriete verhouding tussen
consumptie en sparen te herstellen. Bovendien kan de kostenstijging in
de kapitaalgoederenindustrie nooit de inflatoire prijsstijging inhalen.
Vergelijk het met het herhaaldelijk drogeren van een paard. De boom
wordt op de been gehouden en voor zijn onvermijdelijke ondergang
beschermd door continue en toenemende doses stimulerend bankkrediet. Pas
wanneer de expansie van het bankkrediet uiteindelijk moet stoppen of
sterk vertraagt -- omdat de banken wankel worden of het publiek onrustig
wordt door de aanhoudende inflatie -- haalt de vergelding de boom in.
Zodra de kredietexpansie stopt, moeten de gevolgen worden ondervonden.
De onvermijdelijke aanpassingen zijn nodig om de ondeugdelijke
overinvesteringen van de boom te liquideren en de economie weer te
richten op de productie van consumptiegoederen. Hoe langer de boom
aanhoudt, hoe groter de slechte investeringen die moeten worden
geliquideerd en hoe ingrijpender de aanpassingen die nodig zijn.

Zo verklaart de Oostenrijkse theorie het grote aantal fouten
(overinvesteringen in de kapitaalgoederenindustrie) dat pas aan het
licht kwam toen de kunstmatige stimulans van de kredietexpansie werd
stopgezet. Ook wordt de grotere intensiteit van de booms en de recessies
in de kapitaalgoederenindustrie, vergeleken met de
consumptiegoederenindustrie, hierin toegelicht. De reden voor de
herhaling en de start van de volgende boom is vergelijkbaar met de
Ricardiaanse theorie. Zodra de liquidaties en faillissementen achter de
rug zijn en de aanpassingen in prijzen en productie zijn voltooid,
herstellen de economie en de banken. Op dat moment kunnen de banken weer
terugkeren naar hun natuurlijke en gewenste koers van kredietexpansie.

Hoe zit het dan met de Oostenrijkse verklaring - de enige bekende
verklaring - voor stagflatie? Waarom blijven de prijzen tijdens recente
recessies stijgen? Laten we hier eerst een correctie aanbrengen. Het
zijn vooral de prijzen van consumptiegoederen die tijdens recessies
blijven stijgen. Dit veroorzaakt verwarring bij het publiek, omdat ze
tegelijkertijd met het ergste van twee werelden worden geconfronteerd:
hoge werkloosheid en stijgende kosten van levensonderhoud. Tijdens de
laatste depressie van 1974-1976 stegen de prijzen van consumptiegoederen
snel, terwijl de groothandelsprijzen gelijk bleven en de prijzen van
industriële grondstoffen snel en aanzienlijk daalden. Hoe komt het dat
de kosten van levensonderhoud in de huidige recessies blijven stijgen?

Laten we terugkijken en onderzoeken wat er gebeurde met de prijzen in de
`klassieke' of ouderwetse boom-bustcyclus (van voor de Tweede
Wereldoorlog). Tijdens de booms nam de geldvoorraad toe, waardoor de
prijzen in het algemeen stegen. De prijzen van kapitaalgoederen stegen
echter sterker dan die van consumptiegoederen. Dit resulteerde in een
verschuiving van middelen van consumptiegoederen naar kapitaalgoederen.
Als we de algemene prijsstijgingen buiten beschouwing laten, zien we dat
de prijzen van kapitaalgoederen onderling stegen, terwijl de
consumentenprijzen tijdens de booms daalden. Wat gebeurde er tijdens de
bust? De situatie keerde om: de geldvoorraad daalde, en daardoor daalden
de prijzen in het algemeen. De prijzen van kapitaalgoederen daalden
echter sterker dan die van consumptiegoederen, waardoor middelen uit de
kapitaalgoederenindustrie naar de consumptiegoederenindustrie stroomden.
Dus, als we de algemene prijsdalingen buiten beschouwing laten, daalden
de prijzen van kapitaalgoederen onderling, terwijl de prijzen van
consumptiegoederen tijdens de bust stegen.

Het Oostenrijkse standpunt is dat dit scenario met relatieve prijzen dus
nog steeds bestaat tijdens booms en recessies. Tijdens de booms blijven
de prijzen van kapitaalgoederen stijgen, terwijl de prijzen van
consumptiegoederen dalen ten opzichte van elkaar. Dit geldt ook
omgekeerd tijdens de recessies. Het verschil is dat we nu leven in een
nieuwe monetaire wereld, zoals eerder in dit hoofdstuk is aangegeven.
Met de afschaffing van de goudstandaard kan de Fed de geldhoeveelheid
continu verhogen, ongeacht of we ons in een boom of recessie bevinden,
en dat doet zij ook. Sinds begin jaren dertig is er geen krimp van de
geldhoeveelheid geweest, en het is onwaarschijnlijk dat dit in de nabije
toekomst zal gebeuren. Daarom, nu de geldhoeveelheid altijd toeneemt,
stijgen de prijzen in het algemeen steeds, soms langzamer, soms sneller.

Kortom, tijdens een klassieke recessie stegen de prijzen van
consumptiegoederen altijd ten opzichte van die van kapitaalgoederen. Dus
als de prijzen van consumptiegoederen in een bepaalde recessie met 10
procent daalden en de prijzen van kapitaalgoederen met 30 procent, dan
stegen de consumentenprijzen relatief gezien aanzienlijk. Voor de
consument was de daling van de kosten van levensonderhoud echter zeer
welkom. Het was zelfs het zoete suikerlaagje op de bittere pil van een
recessie of depressie. Zelfs tijdens de Grote Depressie van de jaren
dertig, met extreem hoge werkloosheid, profiteerde 75 tot 80 procent van
de beroepsbevolking die nog werk had van spotprijzen voor hun
consumptiegoederen.

Maar nu Keynesiaanse fine-tuning in werking is, is het suikerlaagje van
de pil verdwenen. Aangezien de geldhoeveelheid - en dus ook de algemene
prijzen - nooit mogen dalen, zal de stijging van de relatieve prijzen
van consumptiegoederen tijdens een recessie de consument ook
rechtstreeks treffen als een merkbare stijging van de nominale prijzen.
Zijn kosten van levensonderhoud stijgen nu zelfs tijdens een depressie.
Hierdoor ervaart hij het slechtste van twee werelden; in de klassieke
conjunctuurcyclus, vóór de dominantie van Keynes en de Raad van
Economisch Adviseurs, hoefde hij tenminste maar één ramp tegelijk te
doorstaan.

Wat zijn de beleidsconclusies die snel en gemakkelijk voortvloeien uit
de Oostenrijkse analyse van de conjunctuurcyclus? Ze zijn precies het
tegenovergestelde van die van het Keynesiaanse establishment. Omdat
verstoringen in productie en prijzen voortkomen uit inflatoire
kredietexpansie door banken, luidt het Oostenrijkse recept voor de
conjunctuurcyclus als volgt: Ten eerste, als we ons in een periode van
hoogconjunctuur bevinden, moeten de overheid en haar banken onmiddellijk
stoppen met inflatie. Het is waar dat deze stopzetting van kunstmatige
stimulans onvermijdelijk een einde zal maken aan de inflatoire boom en
een recessie of depressie zal inluiden. Maar hoe langer de overheid dit
proces uitstelt, hoe harder de noodzakelijke aanpassingen zullen zijn.
Snelle aanpassing aan de depressie is beter voor de economie. Dit
betekent ook dat de overheid nooit mag proberen het proces van depressie
uit te stellen; het moet zo snel mogelijk plaatsvinden, zodat het echte
herstel kan beginnen. Daarnaast moet de overheid bijzonder voorzichtig
zijn met interventies, iets waar Keynesianen vaak op aandringen. Ze mag
nooit ongezonde bedrijfssituaties ondersteunen of borg staan voor, of
geld lenen aan, bedrijven in moeilijkheden. Als ze dat doet, verlengt ze
alleen maar de lijdensweg en verandert ze een scherpe en snelle
depressiefase in een slepende en chronische situatie. De overheid moet
ook nooit proberen om lonen of prijzen te verhogen, vooral niet in de
kapitaalgoederenindustrie. Dit zal het aanpassingsproces van de
depressie onbepaalde tijd vertragen. Bovendien mag de overheid niet
opnieuw proberen te inflateren om uit de depressie te komen. Zelfs als
deze poging succesvol is (wat allesbehalve zeker is), zullen er later
alleen maar grotere problemen en een hernieuwde, langdurige depressie
ontstaan. De overheid moet niets doen om de consumptie aan te moedigen
en mag haar eigen uitgaven niet verhogen. Dit zal de verhouding tussen
sociale consumptie en investeringen verder opdrijven. Het enige dat het
aanpassingsproces kan versnellen, is een verlaging van de
consumptie/spaarratio, zodat meer van de huidige ongezonde investeringen
economisch worden gevalideerd. De enige manier waarop de overheid kan
bijdragen aan dit proces is door haar eigen budget te verlagen. Dit zal
de verhouding tussen investeringen en consumptie in de economie doen
toenemen, aangezien overheidsuitgaven kunnen worden beschouwd als
consumptieve uitgaven voor bureaucraten en politici.

Volgens de Oostenrijkse analyse van de depressie en de conjunctuurcyclus
zou de overheid helemaal niets moeten doen. Ze moet de inflatie stoppen
en vervolgens een strikt hands-off, laissez-faire beleid voeren. Alles
wat ze onderneemt, zal de aanpassingsprocessen van de markt vertragen en
belemmeren. Hoe minder ze doet, hoe sneller de markt zal aanpassen en
een gezond economisch herstel zal plaatsvinden.

Het Oostenrijkse recept voor een depressie staat dus haaks op het
Keynesiaanse: de overheid moet zich volledig terugtrekken uit de
economie en zich beperken tot het stoppen van haar eigen inflatie en het
verlagen van haar begroting.

Het is duidelijk dat de Oostenrijkse analyse van de conjunctuurcyclus
goed aansluit bij de libertarische kijk op de overheid en een vrije
economie. De staat heeft altijd de neiging om te inflateren en zich met
de economie te bemoeien. Een libertarisch voorstel zou daarom het belang
van een duidelijke scheiding tussen geld, bankieren en de staat
benadrukken. Dit zou in elk geval de afschaffing van het Federal Reserve
System vereisen en een terugkeer naar goederengeld, zoals goud of
zilver. Zo zou de geldeenheid weer fungeren als een gewichtseenheid van
een door de markt geproduceerd goed, in plaats van slechts de naam van
een stuk papier dat door de staat wordt geprint.

\bookmarksetup{startatroot}

\chapter{De publieke sector, I: De overheid in het
bedrijfsleven}\label{de-publieke-sector-i-de-overheid-in-het-bedrijfsleven}

Mensen hebben de neiging om in gewoonten en onbetwiste sleur te
vervallen, vooral als het om de overheid gaat. In de markteconomie en de
maatschappij passen we ons snel aan veranderingen aan. We omarmen de
eindeloze wonderen en verbeteringen van onze beschaving. Nieuwe
producten, levensstijlen en ideeën worden vaak gretig verwelkomd. Maar
op het gebied van de overheid volgen we blindelings eeuwenlange
tradities. We zijn tevreden met de gedachte dat wat ooit was, wel goed
moet zijn. In de Verenigde Staten en elders heeft de overheid ons
eeuwenlang, en schijnbaar sinds mensenheugenis, voorzien van essentiële
diensten. Iedereen erkent dat deze diensten belangrijk zijn: defensie
(waaronder leger, politie, justitie en rechtbanken), brandbestrijding,
wegenbouw, watervoorziening, riolering, afvalverwerking, post,
enzovoort. De staat is zo sterk verbonden geraakt met deze diensten dat
een aanval op de staatsfinanciering voor velen aanvoelt als een aanval
op de dienstverlening zelf. Wanneer iemand daarom stelt dat de staat
geen gerechtelijke diensten moet leveren en dat particuliere
ondernemingen deze diensten efficiënter en moreel beter kunnen bieden,
interpreteren mensen dit vaak als een ontkenning van het belang van de
rechtbanken.

De libertariër die de overheid op bovenstaande terreinen wil vervangen
door particuliere ondernemingen, wordt op dezelfde manier behandeld als
wanneer de overheid al sinds mensenheugenis schoenen zou leveren als een
door belastingen gefinancierd monopolie. Stel je voor dat de overheid en
alleen de overheid het monopolie had op de productie en verkoop van
schoenen. Hoe zou het grootste deel van het publiek dan reageren op de
libertariër die pleit voor het verlaten van de schoenenindustrie door de
overheid en het openstellen voor particuliere ondernemingen? Zonder
twijfel zou de libertariër op de volgende manier tegemoetgetreden
worden: mensen zouden uitroepen: `Hoe is het mogelijk? Je bent tegen het
feit dat mensen, en vooral arme mensen, schoenen dragen! Wie zou er
schoenen leveren als de overheid zich er niet mee bemoeit? Vertel ons
dat eens! Wees constructief! Het is makkelijk om negatief en cynisch
over de overheid te zijn; maar wie zouden er nu schoenen leveren? Welke
mensen? Hoeveel schoenenwinkels zouden er in elke stad zijn? Hoe zouden
de schoenenbedrijven gefinancierd worden? Hoeveel merken zouden er
bestaan? Welk materiaal zouden ze gebruiken? Welke leesten? Wat zouden
de prijsafspraken voor schoenen zijn? Is er geen regulering van de
schoenenindustrie nodig om de kwaliteit te waarborgen? En wie zorgt
ervoor dat de armen schoenen krijgen? Wat als iemand zonder geld een
paar schoenen wil kopen?'

Deze vragen, hoe belachelijk ze ook lijken in het kader van de
schoenenindustrie, zijn even absurd wanneer ze worden toegepast op de
libertariër die pleit voor een vrije markt in de brandweer, politie,
post of andere overheidsdiensten. Het punt is dat een voorstander van
een vrije markt niet van tevoren een `constructieve' blauwdruk kan
presenteren voor zo'n markt. De essentie en de kracht van de vrije markt
is dat individuele bedrijven, die met elkaar concurreren, zorgen voor
een voortdurend veranderende samenstelling van efficiënte en
vooruitstrevende goederen en diensten. Ze verbeteren producten en
markten, bevorderen technologie, besparen kosten en spelen snel en
efficiënt in op de veranderende wensen van consumenten. Een
libertarische econoom kan enkele richtlijnen geven over hoe markten zich
zouden kunnen ontwikkelen op plekken waar dat nu wordt verhinderd of
beperkt. Maar hij kan niet veel meer doen dan wijzen op de voordelen van
vrijheid en oproepen tot het afschaffen van overheidsbelemmeringen.
Niemand kan het aantal bedrijven, de grootte van elk bedrijf of de
prijsstellingen in toekomstige markten voor welke dienst of grondstof
dan ook voorspellen. We weten alleen, op basis van economische theorie
en historisch inzicht, dat een vrije markt veel beter zal presteren dan
het gedwongen monopolie van de bureaucratische overheid.

Hoe zullen de armen betalen voor defensie, brandbestrijding, post
enzovoort? Deze vraag kan in principe worden beantwoord met de
wedervraag: hoe betalen de armen voor alles wat ze nu op de markt kunnen
krijgen? Het verschil is dat we weten dat een vrije particuliere markt
deze goederen en diensten veel goedkoper, in grotere hoeveelheden en van
veel hogere kwaliteit zal leveren dan de monopolistische overheid dat nu
doet. Iedereen in de samenleving zou hiervan profiteren, vooral de
armen. Bovendien zou de enorme belastingdruk om deze en andere
activiteiten te financieren, van de schouders van iedereen in de
samenleving worden gehaald, inclusief die van de armen.

We hebben hierboven gezien dat de algemeen erkende en dringende
problemen in onze samenleving allemaal voortkomen uit
overheidsoperaties. We hebben ook ontdekt dat de grote sociale
conflicten die het openbaar onderwijssysteem met zich meebrengt, zouden
verdwijnen als elke groep ouders het onderwijs van hun kinderen zelf zou
kunnen financieren en ondersteunen. De ernstige inefficiënties en de
intense conflicten zijn inherent aan de werking van de overheid. Wanneer
de overheid bijvoorbeeld monopoliediensten aanbiedt, zoals op het gebied
van onderwijs of watervoorziening, worden alle beslissingen die de
overheid neemt opgelegd aan de ongelukkige minderheid. Dit geldt voor
het onderwijsbeleid (integratie of segregatie, progressief of
traditioneel, gelovig of seculier, enzovoort) en zelfs voor het soort
water dat geleverd moet worden (bijvoorbeeld gefluorideerd of
niet-gefluorideerd). Het is duidelijk dat dergelijke heftige conflicten
niet ontstaan wanneer elke groep consumenten de goederen of diensten kan
kopen die zij wensen. Er zijn bijvoorbeeld geen ruzies tussen
consumenten over welk type kranten er gedrukt moeten worden, welke
kerken er opgericht moeten worden, welke boeken er gepubliceerd moeten
worden, welke platen op de markt komen of welke auto's er geproduceerd
moeten worden. Wat op de markt wordt aangeboden, weerspiegelt zowel de
diversiteit als de kracht van de consumentenvraag.

Op de vrije markt is de consument koning. Elk bedrijf dat winst wil
maken en verliezen wil vermijden, doet zijn uiterste best om de
consument zo efficiënt en goedkoop mogelijk te bedienen. In
overheidsbedrijven verandert echter alles. Bij overheidsoperaties is er
altijd een ernstige en fatale scheiding tussen de dienst en de betaling,
tussen het verlenen van een dienst en het ontvangen van de betaling
daarvoor. Een overheidsbureau haalt zijn inkomen niet, zoals een
particulier bedrijf, uit goede dienstverlening aan de consument of uit
de verkoop van producten die zijn kosten dekken. In plaats daarvan komt
het inkomen voort uit het belasten van de belastingbetaler. Dit leidt
tot inefficiëntie en hoge kosten, omdat overheidsbureaus zich geen
zorgen hoeven te maken over verliezen of faillissement. Ze kunnen hun
verliezen eenvoudig dekken met extra geld uit de staatskas. Daarnaast
wordt de consument, in plaats van als een waardevolle klant te worden
behandeld, vaak gezien als een last voor de overheid. Hij wordt ervaren
als iemand die de schaarse middelen van de overheid `verspilt'. In
overheidsoperaties wordt de consument gezien als ongewenst, als een
verstoring van het rustige leven van de bureaucraten met hun vaste
inkomen.

Als de vraag van consumenten naar de goederen of diensten van een
privébedrijf toeneemt, is het bedrijf blij en verwelkomt het de nieuwe
klanten. Het breidt zijn activiteiten gretig uit om aan de nieuwe vraag
te voldoen. De overheid reageert echter meestal heel anders. Zij dringt
er bij de consumenten op aan, of beveelt hen zelfs aan, minder te
`kopen'. Dit leidt tot tekorten en een afname van de kwaliteit van de
dienstverlening. Bijvoorbeeld, als het gebruik van overheidswegen door
stadsbewoners toeneemt, worden de verkeersopstoppingen verergerd. Om dit
tegen te gaan, komen er voortdurend klachten en dreigementen tegen
mensen die met hun eigen auto rijden. Het stadsbestuur van New York
dreigt bijvoorbeeld vaak met een verbod op het gebruik van privé-auto's
in Manhattan, waar de congestie het grootste probleem vormt. Het is
opvallend dat alleen de overheid op deze manier consumenten aanpakt.
Geen enkel bedrijf zou het durven om verkeersproblemen `op te lossen'
door privé-auto's (of vrachtwagens, taxi's, of wat dan ook) van de weg
te halen. Volgen we deze logica, dan is de `ideale' oplossing voor
verkeersopstoppingen simpelweg het verbieden van alle voertuigen!

Maar deze houding tegenover de consument beperkt zich niet tot het
verkeer op straat. New York City kampt bijvoorbeeld al jarenlang met een
`watertekort'. Dit probleem ontstaat omdat het stadsbestuur al die tijd
een dwangmonopolie heeft op de levering van water aan haar inwoners.
Doordat New York niet in staat is om voldoende water te leveren en de
prijs niet zo vaststelt dat deze de markt laat functioneren --- iets wat
particuliere bedrijven automatisch zouden doen --- reageert het
stadsbestuur bij watertekorten steevast door de schuld bij de consument
te leggen. De consument wordt verweten `te veel' water te gebruiken. In
plaats van zijn eigen tekortkomingen onder ogen te zien, reageert het
stadsbestuur door het besproeien van gazons te verbieden, het gebruik
van water te beperken en mensen aan te sporen minder water te drinken.
Zo wijst het bestuur de gebruiker aan als zondebok, terwijl het zelf
niet wordt aangesproken. In plaats van hen op een goede en efficiënte
manier te helpen, bedreigt de overheid de consument.

De overheid heeft in New York City op een vergelijkbare manier
gereageerd op het toenemende misdaadprobleem. In plaats van zorg te
dragen voor efficiënte politiebescherming, heeft de stad de onschuldige
burger gedwongen om misdaadgevoelige gebieden te mijden. Toen Central
Park in Manhattan een berucht centrum werd voor berovingen en andere
criminaliteit 's nachts, besloot New York City een avondklok in te
stellen. Hierdoor werd het verboden om het park in die uren te betreden.
Kortom, als een onschuldige burger 's nachts in Central Park wil
blijven, kan hij worden gearresteerd voor het overtreden van de
avondklok. Het is blijkbaar makkelijker om hem te arresteren dan om het
park daadwerkelijk van criminaliteit te ontdoen.

Kortom, terwijl het aloude motto van het bedrijfsleven is dat `de klant
altijd gelijk heeft', is de impliciete regel van het overheidsbeleid dat
de klant altijd de schuld krijgt.

Natuurlijk hebben de politieke bureaucraten standaardreacties op de
toenemende klachten over slechte en inefficiënte dienstverlening: `De
belastingbetalers moeten ons meer geld geven!' Het is niet voldoende dat
de publieke sector en de daarmee samenhangende belastingen dit eeuw veel
sneller zijn gegroeid dan het nationaal inkomen. Evenmin is het genoeg
dat de tekortkomingen en de hoofdpijn van de overheid zijn toegenomen,
samen met de druk op het overheidsbudget. We worden verondersteld om nog
meer geld in het riool van de overheid te stoppen!

Het juiste tegenargument voor de politieke oproep om meer belastinggeld
is de vraag: `Waarom hebben particuliere ondernemingen deze problemen
niet?' Waarom hebben hifi-fabrikanten, fotokopieerbedrijven,
computerfirma's of andere bedrijven geen moeite om kapitaal te vinden om
hun productie uit te breiden? Waarom maken zij geen verwijten naar
investeerders omdat ze niet meer geld krijgen om aan de behoeften van
consumenten te voldoen? Het antwoord is simpel: consumenten betalen voor
hifi-sets, kopieermachines en computers, waardoor investeerders weten
dat ze kunnen profiteren door in die bedrijven te investeren. Op de
privémarkt kunnen bedrijven die het publiek succesvol bedienen
gemakkelijk kapitaal aantrekken om uit te breiden; inefficiënte en
onsuccesvolle bedrijven hebben die mogelijkheid niet en zullen
uiteindelijk failliet gaan. Bij de overheid ontbreekt echter het winst-
en verliesmechanisme dat investeringen in efficiënte activiteiten
stimuleert en inefficiënte of verouderde activiteiten sanctioneert.
Overheidsactiviteiten zorgen niet voor winsten of verliezen die
uitbreiding of inkrimping aanmoedigen. Bij de overheid `investeert'
niemand echt en er is geen garantie dat succesvolle activiteiten zullen
groeien of dat onsuccesvolle activiteiten zullen verdwijnen. Daarom is
de overheid genoodzaakt om haar `kapitaal' te verhogen door
belastingheffing te gebruiken als dwangmechanisme.

Veel mensen, ook sommige overheidsfunctionarissen, denken dat deze
problemen opgelost zouden kunnen worden als `de overheid als een bedrijf
geleid zou worden'. De overheid zou dan een soort
pseudo-bedrijfsmonopolie oprichten, dat geacht wordt de zaken op een
`zakelijke basis' te regelen. Dit is bijvoorbeeld geprobeerd met het
postkantoor, nu de U.S. Postal Service, en met de steeds verder
afbrokkelende New York City Transit Authority. De `bedrijven' krijgen de
opdracht om hun chronische tekorten aan te pakken en mogen obligaties
uitgeven op de obligatiemarkt. Het klopt dat directe gebruikers zo een
deel van de financiële last wegnemen van de massa belastingbetalers,
inclusief zowel gebruikers als niet-gebruikers. Maar er zijn ingrijpende
tekortkomingen inherent aan elke overheidsoperatie, die niet oplossen
door dit pseudo-bedrijfsmodel. Ten eerste is overheidsdienst altijd een
monopolie of semimonopolie. Vaak, zoals bij de Post of de Transit
Authority, is het zelfs een verplicht monopolie; alle of bijna alle
private concurrentie is verboden. Dit monopolie betekent dat de overheid
haar dienstverlening veel duurder en van lagere kwaliteit aanbiedt dan
op de vrije markt het geval zou zijn. Private ondernemingen maken winst
door kosten te besparen. De overheid, die nooit failliet kan gaan of
verlies kan lijden, heeft daarentegen geen prikkel om in de kosten te
snijden. Beschermd tegen concurrentie en verliezen, hoeft ze alleen maar
haar dienstverlening te verminderen of simpelweg de prijzen te verhogen.
Een tweede fatale tekortkoming is dat een overheidsbedrijf, hoezeer het
ook probeert, nooit zoals een echt bedrijf kan worden geleid, omdat het
kapitaal altijd afkomstig is van de belastingbetaler. Er is geen manier
om dit te omzeilen. Het feit dat een overheidsbedrijf obligaties kan
uitgeven op de markt, berust nog steeds op de uiteindelijke verplichting
van de belastingbetaler om deze obligaties terug te betalen.

Tot slot is er nog een ander kritiekpunt dat inherent is aan elke
overheidsbedrijfsvoering. Een van de redenen waarom particuliere
bedrijven vaak efficiënt opereren, is dat de vrije markt prijzen
vaststelt. Deze prijzen helpen hen bij het bepalen van hun kosten en
zorgen ervoor dat ze weten wat ze moeten doen om winst te maken en
verliezen te vermijden. Dankzij dit prijssysteem en de motivatie om
winsten te verhogen en verliezen te voorkomen, worden goederen en
diensten efficiënt verdeeld over de complexe sectoren van de moderne
industriële `kapitalistische' economie. Economische berekening maakt
deze prestaties mogelijk. Centrale planning, zoals die onder socialisme
wordt toegepast, heeft echter geen nauwkeurige prijsbepaling en kan
daarom kosten en prijzen niet adequaat berekenen. Dit is de
belangrijkste reden waarom centrale socialistische planning steeds vaker
is gefaald, vooral naarmate de communistische landen industrialiseerden.
Omdat centrale planning geen goede prijs- en kostenberekening kan maken,
zijn de communistische landen in Oost-Europa snel overgestapt van
socialistische planning naar een vrije markteconomie.

Als centrale planning de economie in een hopeloze rekenchaos stort en
leidt tot irrationele toewijzingen en productieprocessen, dan brengt de
groei van overheidsactiviteiten onvermijdelijk steeds grotere gebieden
van deze chaos met zich mee. Dit maakt het steeds moeilijker om kosten
te berekenen en productiemiddelen rationeel toe te wijzen. Naarmate de
overheidsactiviteiten toenemen en de markteconomie verder verzwakt,
wordt de berekeningschaos steeds ontwrichtender en raakt de economie
steeds minder werkbaar.

Het ultieme libertarische programma kan in één zin worden samengevat: de
afschaffing van de publieke sector en de omzetting van alle activiteiten
en diensten die de overheid uitvoert, naar activiteiten die vrijwillig
door de particuliere economie worden uitgevoerd. Laten we nu de algemene
beschouwingen over de overheid tegenover particuliere activiteiten
verlaten en ons richten op enkele belangrijke gebieden van
overheidsoperaties. We onderzoeken hoe deze uitgevoerd kunnen worden
door de vrije markteconomie.

\bookmarksetup{startatroot}

\chapter{De publieke sector, II: Straten en
wegen}\label{de-publieke-sector-ii-straten-en-wegen}

\section{DE STRATEN BESCHERMEN}\label{de-straten-beschermen}

Afschaffing van de publieke sector houdt in dat alle stukken land,
inclusief straten en wegen, privébezit zouden worden van individuen,
bedrijven, coöperaties of andere vrijwillige groeperingen van mensen en
kapitaal. Het feit dat alle straten en landgebieden privé-eigendom zijn,
zou veel van de schijnbaar onoplosbare problemen van particuliere
exploitatie oplossen. We moeten ons denken heroriënteren en nadenken
over een wereld waarin al het land privébezit is.

Neem bijvoorbeeld politiebescherming. Hoe zou dit geregeld zijn in een
volledig private economie? Een deel van het antwoord wordt duidelijk als
we kijken naar een wereld met uitsluitend privébezit van land en
straten. Denk aan het Times Square-gebied in New York City, dat bekend
staat om zijn criminaliteit en waar het stadsbestuur weinig
politiebescherming biedt. Iedere New Yorker weet eigenlijk dat hij op
straat leeft en zich, niet alleen op Times Square, maar overal, in een
bijna `anarchistische' situatie bevindt, waarbij hij alleen afhankelijk
is van de normale vreedzaamheid en goede wil van zijn medeburgers. De
politiebescherming in New York is minimaal. Dit kwam op dramatische
wijze naar voren tijdens een recente politiestaking van een week. In die
periode nam de criminaliteit geen enkele vorm aan die verschilde van de
normale toestand wanneer de politie zogenaamd alert en aan het werk is.
Stel je nu voor dat het Times Square-gebied, inclusief de straten,
privébezit is van bijvoorbeeld de `Times Square Merchants Association'.
De winkeliers zouden goed beseffen dat als de criminaliteit in hun
gebied zou toenemen met veel overvallen en berovingen, hun klanten weg
zouden blijven en voor concurrerende gebieden zouden kiezen. Het zou dus
in het economische belang van de winkeliersvereniging zijn om te zorgen
voor efficiënte en overvloedige politiebescherming. Op die manier zouden
klanten aangetrokken worden naar hun buurt in plaats van afgestoten.
Particuliere bedrijven proberen immers altijd klanten aan te trekken en
te behouden. Maar wat heb je aan aantrekkelijke winkeldisplays, mooie
verlichting en vriendelijke service, als klanten het risico lopen
beroofd of aangevallen te worden terwijl ze door de buurt lopen?

De winkeliersvereniging zou door hun winstbejag en de wens om verliezen
te voorkomen gestimuleerd worden om niet alleen voldoende
politiebescherming te bieden, maar ook vriendelijke en klantgerichte
service. Overheidspolitieagenten hebben geen stimulans om efficiënt te
werken of zich in te leven in de behoeften van hun `klanten'. Bovendien
lopen ze constant het risico om hun macht op een brutale en dwingende
manier te misbruiken. Politiegeweld is een bekend probleem binnen het
politiesysteem en wordt alleen ingedamd door de klachten van de
getreiterde burgers. Maar als de particuliere handelarenpolitie de
verleiding toegeeft om de klanten van de winkeliers te intimideren,
zullen deze klanten snel wegblijven en ergens anders hun aankopen doen.
Daarom zal de winkeliersvereniging ervoor zorgen dat hun politie zowel
vriendelijk als goed vertegenwoordigd is.

Zulke efficiënte en hoogwaardige politiebescherming zou in heel
Nederland gelden, voor alle privéstraten en terreinen. Fabrieken zouden
de toegang tot hun straten bewaken, handelsfirma's voor hun gebieden
zorgen en wegbedrijven veilige en effectieve politiebescherming bieden
voor hun tolwegen en andere wegen in privébezit. Hetzelfde geldt voor
woonwijken. We kunnen ons twee mogelijke typen privébezit van straten in
zulke buurten voorstellen. In het eerste type zouden alle landeigenaren
in een bepaald blok gezamenlijk eigenaar kunnen zijn van dat blok, laten
we zeggen als de `85th St.~Block Company.' Deze maatschappij zou dan de
politiebescherming verzorgen, waarbij de kosten ofwel rechtstreeks door
de huiseigenaren betaald worden, of via de huur van de huurders als het
blok huurappartementen bevat. Huiseigenaren hebben natuurlijk direct
belang bij de veiligheid van hun blok, terwijl verhuurders hun huurders
willen aantrekken door veilige straten aan te bieden, naast
gebruikelijke diensten zoals verwarming, water en conciërgediensten.
Vragen waarom huisbazen voor veilige straten moeten zorgen in een
libertarische, volledig autonome maatschappij is net zo absurd als
vragen waarom ze hun huurders zouden moeten voorzien van verwarming of
warm water. Door concurrentie en de vraag van consumenten zijn zij
verplicht om zulke diensten te leveren. Bovendien zal, of we het nu
hebben over huiseigenaren of huurwoningen, de waarde van het kapitaal
van grond en woning afhankelijk zijn van de veiligheid van de straat en
van andere bekende kenmerken van het huis en de buurt. Veilige en goed
bewaakte straten verhogen de waarde van het land en de huizen van de
landeigenaar op dezelfde manier als goed onderhouden huizen dat doen.
Straten met veel criminaliteit zullen de waarde van het land en de
huizen echter zeker verlagen, net zoals vervallen huizen dat doen.
Aangezien landeigenaren altijd de voorkeur geven aan hogere marktwaarden
voor hun eigendom, is er een ingebouwde prikkel om te zorgen voor
efficiënte, goed onderhouden en veilige straten.

Een andere vorm van privébezit van straten in woonwijken zou kunnen
bestaan uit privé-straatbedrijven. Deze bedrijven zijn uitsluitend
eigenaar van de straten en niet van de huizen of gebouwen die erop
staan. De straatbedrijven zouden de grondeigenaren laten betalen voor
het onderhoud, de verbetering en de beveiliging van hun straten.
Veilige, goed verlichte en goed onderhouden straten zouden landeigenaren
en huurders aanmoedigen om zich daar te vestigen. In tegenstelling
daarmee zouden onveilige, slecht verlichte en verwaarloosde straten hen
afschrikken. Een levendig gebruik van de straten door landeigenaren en
auto's zou de winsten en aandelenwaarden van de straatbedrijven
verhogen. Een slechte uitstraling van de straten zou daarentegen
gebruikers afschrikken en de winsten en aandelenwaarden van de
privé-straatbedrijven doen dalen. Daarom zullen de straateigenaren hun
best doen om een effectieve straatdienst te bieden, inclusief
politiebescherming, om tevreden gebruikers te garanderen. Ze worden
gedreven door de behoefte om winst te maken, de waarde van hun kapitaal
te verhogen en verliezen te vermijden. Het is veel verstandiger om te
vertrouwen op de economische belangen van landeigenaren of
straatbedrijven dan op het twijfelachtige `altruïsme' van bureaucraten
en overheidsfunctionarissen.

Op dit punt in de discussie zal iemand ongetwijfeld de vraag stellen:
Wat als straten eigendom zijn van straatbedrijven en een of andere
vreemde of tirannieke straateigenaar plotseling besluit om de toegang
tot zijn straat te blokkeren voor een aangrenzende huiseigenaar? Hoe kan
die persoon dan in- of uitgaan? Verliest hij dan voorgoed toegang, of
moet hij een enorm bedrag betalen om naar binnen of naar buiten te
mogen? Het antwoord op deze vraag is hetzelfde als bij een vergelijkbaar
probleem met grondbezit. Stel je voor dat iedereen die huizen bezit
rondom iemands eigendom deze persoon plotseling zou verbieden om zijn
terrein te betreden. Het antwoord is dat iedereen die in een
libertarische samenleving een huis of straat koopt, ervoor zorgt dat het
koop- of huurcontract volledige toegang garandeert voor welke termijn
dan ook. Met zo'n vooraf vastgelegde `erfdienstbaarheid' zou een
plotselinge blokkade niet geaccepteerd worden, omdat dit een inbreuk op
het eigendomsrecht van de landeigenaar zou zijn.

Er is niets nieuws of verrassends aan het principe van deze beoogde
libertarische samenleving. We zijn al bekend met de positieve effecten
van concurrentie tussen lokale en interregionale transportmiddelen. In
de negentiende eeuw bijvoorbeeld, zorgden de particuliere spoorwegen die
door het hele land zijn aangelegd voor een aanzienlijke stimulans voor
de ontwikkeling van hun gebieden. Elke spoorweg deed zijn best om
immigratie en economische groei in zijn regio te bevorderen. Dit had als
doel de winst, landwaarde en het kapitaal te verhogen.
Spoorwegmaatschappijen waren hiertoe genoodzaakt, want anders zouden
mensen en markten hun gebied verlaten ten gunste van havens, steden en
gebieden die door concurrerende spoorwegen werden bediend. Hetzelfde
principe geldt als alle straten en wegen ook privébezit zijn. We kennen
ook de beveiliging die particuliere handelaren en organisaties bieden.
Winkels hebben bewakers en toezichters in dienst; banken zorgen voor
beveiligers; fabrieken hebben ook toezichters; winkelcentra hebben hun
bewakingspersoneel, enzovoort. In een libertarische samenleving zou dit
gezonde en functionerende systeem eenvoudigweg ook op straat worden
toegepast. Het is geen toeval dat er veel meer overvallen en berovingen
plaatsvinden op straat voor winkels dan in de winkels zelf. Dit komt
omdat de winkels zijn beveiligd door waakzame privébewakers, terwijl we
op straat moeten vertrouwen op de `anarchie' van de politie van de
overheid. In verschillende blokken van New York City zijn in de
afgelopen jaren, als reactie op het toenemende criminaliteitsprobleem,
particuliere bewakers ingehuurd. Deze bewakers patrouilleren in de
blokken met vrijwillige bijdragen van huiseigenaren en verhuurders. De
criminaliteit in deze blokken is hierdoor al aanzienlijk verminderd. Het
probleem is echter dat deze initiatieven slechts sporadisch en
inefficiënt zijn geweest. Straten zijn immers geen eigendom van de
bewoners, waardoor er geen effectief mechanisme bestaat om kapitaal te
verzamelen voor permanente en efficiënte bescherming. Bovendien kunnen
de patrouillerende wachtposten wettelijk gezien niet gewapend zijn,
omdat ze zich niet op het terrein van hun werkgevers bevinden. Ze
kunnen, in tegenstelling tot eigenaren van winkels en andere panden, ook
niet iedereen aan de tand voelen die zich verdacht gedraagt maar nog
geen misdaad heeft gepleegd. Kortom, ze kunnen niet de financiële of
administratieve acties ondernemen die eigenaren met hun eigendom wel
kunnen uitvoeren.

Bovendien zou een politie die betaald wordt door de landeigenaren en
bewoners van een blok of buurt niet alleen een einde maken aan
politiegeweld tegen klanten. Dit systeem zou ook ervoor zorgen dat de
politie niet langer wordt gezien als een vreemde `imperiale'
kolonisator, die er niet is om de gemeenschap te dienen, maar om deze te
onderdrukken. In Amerika hebben we momenteel de situatie dat zwarte
gebieden worden bewaakt door politie die is ingehuurd door centrale
stedelijke overheden, overheden die als vreemd worden ervaren door de
zwarte gemeenschappen. Als de politie zou worden gesteund, gecontroleerd
en betaald door de bewoners en landeigenaren zelf, zou dat een heel
ander verhaal zijn. Dan zouden zij hun klanten daadwerkelijk van dienst
zijn, in plaats van hen te onderdrukken namens een vreemde autoriteit.

Eén blok in Harlem laat een scherp contrast zien tussen de voordelen van
publieke en private bescherming. Aan West 135th Street, tussen Seventh
en Eighth Avenues, staat het politiebureau van het 82nd Precinct van de
New York City Police Department. Desondanks kon de aanwezigheid van de
politie een golf van nachtelijke overvallen op verschillende winkels in
het blok niet voorkomen. Uiteindelijk besloten in de winter van 1966 15
winkeliers in het blok samen te werken en een bewaker in te huren. Deze
bewaker patrouilleerde de hele nacht in het blok. Hij werd ingehuurd via
het beveiligingsbedrijf Leroy V. George om de taken te vervullen die de
politie eerder niet kon bieden, ondanks hun onroerendgoedbelasting.1

De succesvolste en best georganiseerde privépolitie in de Amerikaanse
geschiedenis was de spoorwegpolitie. Veel spoorwegmaatschappijen stelden
deze in om letsel of diefstal van reizigers en vracht te voorkomen. De
moderne spoorwegpolitie werd aan het einde van de Eerste Wereldoorlog
opgericht door de Protection Section van de American Railway
Association. Zij functioneerden zo goed dat in 1929 het aantal
schadeclaims voor diefstal met maar liefst 93 procent was gedaald.
Arrestaties door de spoorwegpolitie, die tijdens het grote onderzoek
naar hun activiteiten in het begin van de jaren dertig in totaal 10.000
agenten telde, resulteerden in een veel hoger percentage veroordelingen
dan de reguliere politie. Dit percentage varieerde van 83 tot 97
procent. De spoorwegpolitie was gewapend, kon normale arrestaties
verrichten en werd door een onsympathieke criminoloog afgeschilderd als
een organisatie met een uitstekende reputatie op het gebied van zowel
karakter als bekwaamheid.2

\section{Straatregels}\label{straatregels}

Een van de onbetwistbare gevolgen van het feit dat alle grond in het
land in handen zou zijn van particuliere eigenaren en bedrijven, zou de
grotere rijkdom en diversiteit van Amerikaanse buurten zijn. De manier
waarop politiebescherming wordt verleend en de regels die door
privépolitie worden toegepast, zouden afhangen van de wensen van de
eigenaren van het land of de straat. In verdachte woonwijken zou het
bijvoorbeeld kunnen vereisen dat alle mensen of auto's die het gebied
willen betreden, eerst toestemming vragen aan een bewoner of goedgekeurd
worden via een telefoontje vanaf de poort. Kortom, dezelfde
eigendomsregels die nu vaak gelden in particuliere appartementen of
familielandgoederen, zouden hier van toepassing zijn. In andere, meer
losbandige gebieden zou iedereen vrijelijk naar binnen mogen en zou er
een uiteenlopende mate van toezicht kunnen zijn. Commerciële gebieden
zouden waarschijnlijk toegankelijk zijn voor iedereen, om klanten niet
af te schrikken. Dit zou ruimte bieden voor de wensen en waarden van de
bewoners en eigenaren in de verschillende gebieden van het land.

Men zou kunnen stellen dat dit alles de vrijheid biedt `om te
discrimineren' op het gebied van huisvesting of het gebruik van de
straat. Daarover bestaat geen twijfel. Essentieel voor het libertarische
credo is het recht van iedereen om te bepalen wie zijn eigendom mag
betreden of gebruiken, mits de ander daar ook mee instemt.

`Discriminatie' in de zin van het maken van een gunstige of ongunstige
keuze, ongeacht de criteria die iemand gebruikt, is een essentieel
onderdeel van keuzevrijheid en daarmee van een vrije samenleving. In de
vrije markt is dergelijke discriminatie echter kostbaar en zal de
betrokken eigenaar daarvoor moeten betalen.

Stel je voor dat iemand in een vrije samenleving een huis of een
appartementencomplex verhuurt. Hij zou gewoon de markthuur kunnen vragen
en verder niets doen. Maar dat brengt risico's met zich mee. Misschien
besluit hij om geen woningen te verhuren aan gezinnen met jonge
kinderen, omdat hij vreest dat zijn eigendom daardoor beschadigd raakt.
Aan de andere kant kan hij ervoor kiezen om een hogere huurprijs te
vragen om het verhoogde risico te compenseren. Dit betekent dat de vrije
markthuur voor zulke gezinnen dan hoger zal zijn dan voor anderen. In de
meeste gevallen zal dit op de vrijemarkteconomie gebeuren. Maar hoe zit
het met persoonlijke, in plaats van puur economische, `discriminatie'
door de verhuurder? Stel dat de verhuurder een groot bewonderaar is van
twee meter lange Zweeds-Amerikanen en besluit zijn appartementen alleen
aan gezinnen uit die groep te verhuren. In een vrije maatschappij heeft
hij daar het recht toe, maar het is duidelijk dat hij hierdoor een
aanzienlijk financieel verlies zou lijden. Hij zou huurder na huurder
afwijzen in een eindeloze zoektocht naar lange Zweeds-Amerikanen. Hoewel
dit als een extreem voorbeeld kan worden gezien, geldt hetzelfde effect
--- zij het in verschillende mate --- voor elke vorm van persoonlijke
discriminatie op de markt. Als de verhuurder bijvoorbeeld een hekel
heeft aan roodharigen en besluit zijn appartementen niet aan hen te
verhuren, zal hij ook verliezen, zij het niet zo ernstig als in het
eerste geval.

Hoe dan ook, wanneer iemand dergelijke `discriminatie' toepast op de
vrije markt, moet hij de kosten daarvan dragen. Dit kan betekenen dat
hij winst misloopt of dat hij als consument diensten verliest. Als een
consument ervoor kiest om goederen te boycotten die worden verkocht door
mensen die hij niet mag---ongeacht of die afkeer terecht is of
niet---zal hij verstoken blijven van goederen of diensten die hij anders
zou hebben gekocht.

In een vrije samenleving stellen alle eigenaren van onroerend goed
regels op voor het gebruik van hun eigendom of de toegang daartoe. Hoe
strikter de regels, hoe minder mensen ervan gebruik zullen maken. De
eigenaar zal dan moeten afwegen of de striktheid van de toelating
opweegt tegen het verlies aan inkomsten. Een verhuurder kan bijvoorbeeld
`discrimineren' door te eisen --- zoals George Pullman deed in zijn
`company town' in Illinois aan het einde van de negentiende eeuw --- dat
al zijn huurders altijd in jasje en das verschijnen. Hij zou dit kunnen
doen, maar het is twijfelachtig of veel huurders ervoor zouden kiezen om
in een dergelijk gebouw te wonen of daar te blijven. In dat geval zou de
verhuurder ernstige verliezen lijden.

Het principe dat eigendom door de eigenaren wordt beheerd, vormt ook een
weerlegging van een gangbaar argument voor overheidsinterventie in de
economie. Vaak wordt gesteld: `De overheid stelt tenslotte
verkeersregels op - zoals rood en groen licht, rechtsrijden en
maximumsnelheid. Iedereen kan zich voorstellen dat het verkeer in chaos
zou ontaarden zonder zulke regels. Waarom zou de overheid dan niet
ingrijpen in de rest van de economie?' De denkfout hier is niet dat
verkeer gereguleerd moet worden; zulke regels zijn zeker nodig. Maar het
cruciale punt is dat deze regels altijd worden vastgesteld door de
personen die eigenaar zijn van de wegen en deze beheren. De overheid
heeft verkeersregels opgesteld omdat zij de eigenaar is van de straten
en wegen en deze daarom beheert. In een libertarische samenleving met
privé-eigendom zouden de privé-eigenaren de regels opstellen voor het
gebruik van hun wegen.

Maar zouden de verkeersregels niet `chaotisch' zijn in een puur vrije
maatschappij? Zouden sommige wegowners niet rood aanwijzen voor `stop',
anderen misschien groen of blauw? En zouden sommige wegen niet rechts
worden bereden terwijl andere links worden gebruikt? Zulke vragen zijn
absurd. Het is in het belang van alle wegbeheerders om uniforme regels
te hebben, zodat het verkeer soepel en zonder problemen kan verlopen.
Elke eigenzinnige eigenaar die zou aandringen op links rijden of groen
voor `stop' in plaats van `ga' zou al snel geconfronteerd worden met
talloze ongelukken en een afname van klanten. De privéspoorwegen in het
negentiende-eeuwse Amerika hadden vergelijkbare problemen en losten deze
harmonieus op. Ze lieten elkaars wagons toe op hun sporen, maakten
onderlinge verbindingen voor wederzijds voordeel, stemden de
spoorbreedtes op elkaar af en stelden uniforme regionale
goederenclassificaties op voor 6.000 artikelen. Bovendien waren het de
spoorwegen en niet de overheid die het initiatief namen om de chaotische
lappendeken van tijdzones te consolideren. Voor een nauwkeurige
dienstregeling en tijdschema's moesten ze samenwerken, en in 1883
stemden ze ermee in om de 54 bestaande tijdzones in het hele land samen
te voegen tot de vier die we nu hebben. De financiële krant van New
York, de \emph{Commercial and Financial Chronicle}, constateerde dat `de
wetten van de handel en het instinct voor zelfbehoud hervormingen en
verbeteringen teweegbrengen die alle wetgevende instanties samen niet
konden verwezenlijken'.

\section{\#\#\# TARIFERING VAN STRATEN EN
WEGEN}\label{tarifering-van-straten-en-wegen}

In een vrije samenleving stellen eigenaren van onroerend goed regels op
voor het gebruik van hun eigendom of de toegang ertoe. Hoe strikter de
regels, hoe minder mensen ervan gebruik zullen maken. De eigenaar moet
dan afwegen of de stringente toelatingseisen opwegen tegen het verlies
aan inkomsten. Een verhuurder kan bijvoorbeeld `discrimineren' door te
eisen --- zoals George Pullman deed in zijn `company town' in Illinois
aan het einde van de negentiende eeuw --- dat al zijn huurders altijd in
jasje en das verschijnen. Hij kan dit doen, maar het is de vraag of veel
huurders bereid zouden zijn om in zo'n gebouw te wonen. In dat geval zou
de verhuurder ernstige verliezen lijden. Het principe dat eigendom door
de eigenaren wordt beheerd, vormt ook een weerlegging van een
veelvoorkomend argument voor overheidsinterventie in de economie.
Sommigen stellen: `De overheid stelt tenslotte verkeersregels op ---
zoals rood en groen licht, rechtsrijden en maximumsnelheid. Iedereen kan
zich voorstellen dat het verkeer in chaos zou ontaarden zonder zulke
regels. Waarom zou de overheid dan niet ingrijpen in de rest van de
economie?' De denkfout hier is niet dat verkeer gereguleerd moet worden;
zulke regels zijn zeker nodig. Maar het cruciale punt is dat deze regels
altijd worden vastgesteld door degenen die eigenaar zijn van de wegen en
deze beheren. De overheid heeft verkeersregels ingevoerd omdat zij de
eigenaar van de straten en wegen is en deze daardoor moet beheren. In
een libertarische samenleving met privé-eigendom zouden de
privé-eigenaren de regels voor het gebruik van hun wegen zelf opstellen.
Maar zouden de verkeersregels niet `chaotisch' zijn in een puur vrije
maatschappij? Zou de ene wegbeheerder niet rood voor `stop' aanwijzen,
terwijl een andere groen of blauw kiest? En zouden sommige wegen niet
rechts worden bereden terwijl andere links worden gebruikt? Zulke vragen
zijn absurd. Het is in het belang van alle wegbeheerders om uniforme
regels te hebben, zodat het verkeer soepel en zonder problemen kan
verlopen. Elke eigenzinnige eigenaar die zou aandringen op links rijden
of groen voor `stop' in plaats van `ga,' zou al snel te maken krijgen
met talloze ongelukken en een afname van gebruikers. De privéspoorwegen
in het negentiende-eeuwse Amerika hadden vergelijkbare problemen en
losten deze probleemloos op. Ze lieten elkaars wagons toe op hun sporen,
maakten onderlinge verbindingen voor wederzijds voordeel, stemden de
spoorbreedtes op elkaar af, en ontwikkelden uniforme regionale
goederenclassificaties voor 6.000 artikelen. Bovendien waren het de
spoorwegen, en niet de overheid, die het initiatief namen om de
chaotische lappendeken van tijdzones samen te voegen. Voor een
nauwkeurige dienstregeling moesten ze samenwerken. In 1883 stemden ze
ermee in om de 54 bestaande tijdzones in het hele land te consolideren
tot de vier die we nu hebben. De financiële krant van New York, de
\emph{Commercial and Financial Chronicle}, merkte op dat `de wetten van
de handel en het instinct voor zelfbehoud hervormingen en verbeteringen
teweegbrengen die alle wetgevende instanties samen niet kunnen
verwezenlijken.'

Als we kijken naar de prestaties van de openbare wegen en snelwegen in
Amerika, is het moeilijk te begrijpen hoe privé-eigendom een slechtere
of inefficiëntere staat van dienst zou kunnen hebben. Het is
tegenwoordig algemeen erkend dat federale en staatsregeringen, onder
druk van lobbygroepen van auto-industrieën, oliemaatschappijen,
bandenfabrikanten, aannemers en vakbonden, hebben gezorgd voor een
enorme uitbreiding van snelwegen. Deze snelwegen bieden aanzienlijke
subsidies aan gebruikers en hebben bijgedragen aan het ineenstorten van
de spoorwegen als levensvatbare onderneming. Vrachtwagens rijden
namelijk op wegen die zijn aangelegd en onderhouden door
belastingbetalers, terwijl de spoorwegen zelf verantwoordelijk zijn voor
de aanleg en het onderhoud van hun spoor. Bovendien hebben de
gesubsidieerde snelwegen en wegenprogramma's geleid tot een
ongecontroleerde groei van voorsteden waar veel autos rijden. Dit ging
vaak gepaard met het gedwongen afbreken van talloze huizen en bedrijven
en drukte de belasting op de centrale steden. De kosten voor
belastingbetalers en de economie zijn enorm geweest.

Vooral de forens die in een autostad woont, profiteert van subsidies.
Juist in de steden is het aantal verkeersopstoppingen toegenomen, samen
met deze subsidie voor het overmatige verkeer. Professor William Vickrey
van de Universiteit van Columbia heeft geschat dat stedelijke snelwegen
zijn aangelegd tegen kosten van 6 tot 27 cent per voertuigkilometer. De
gebruikers betalen echter maar zo'n 1 cent per voertuigkilometer aan
benzine- en andere autobelastingen. Hierdoor betaalt de algemene
belastingbetaler, in plaats van de automobilist, voor het onderhoud van
de straten in de stad. Bovendien wordt de benzinebelasting per kilometer
geheven, ongeacht welke straat of snelweg wordt gebruikt, en dat geldt
ook voor het tijdstip van de rit. Wanneer snelwegen dus gefinancierd
worden vanuit het algemene benzinebelastingfonds, zijn de gebruikers van
de goedkopere landelijke snelwegen verplicht om de gebruikers van de
veel duurdere stedelijke snelwegen te subsidiëren. Landelijke snelwegen
kosten doorgaans slechts 2 cent per voertuigkilometer om aan te leggen
en te onderhouden.

Bovendien is de benzinebelasting geen rationeel prijssysteem voor het
gebruik van wegen. Geen enkel privébedrijf zou wegen ooit op deze manier
prijzen. Privébedrijven bepalen hun tarieven om de `markt vrij te
maken'. Dit houdt in dat het aanbod aansluit op de vraag, zodat er geen
tekorten of onverkochte goederen ontstaan. Het feit dat benzineaccijnzen
per kilometer worden geheven, ongeacht welke weg je gebruikt, betekent
dat de drukste straten en snelwegen in steden een prijs betalen die veel
lager is dan de prijs in een vrije markt. Het gevolg hiervan zijn enorme
verkeersopstoppingen op druk bereden wegen, vooral tijdens de spits,
terwijl het plattelandswegennet vrijwel ongebruikt blijft. Een rationeel
prijssysteem zou de winst voor de wegbeheerder maximaliseren en
tegelijkertijd zorgen voor vrije straten zonder files. In het huidige
systeem houdt de overheid de prijs voor gebruikers van overbelaste wegen
extreem laag. Deze ligt ver onder de vrije marktprijs, wat resulteert in
een chronisch tekort aan wegruimte dat zich manifesteert in
verkeersopstoppingen. In plaats van dit groeiende probleem op te lossen
door prijzen rationeel aan te passen, heeft de overheid geprobeerd het
op te lossen door nog meer wegen aan te leggen. Dit brengt alleen maar
extra kosten met zich mee voor de belastingbetalers, die nog meer
subsidies moeten financieren voor automobilisten, waardoor het tekort
alleen maar toeneemt. Het wanhopig vergroten van het aanbod terwijl de
gebruiksprijs onder de marktprijs blijft, leidt simpelweg tot chronische
en verergerende files. Het doet denken aan een hond die achter een
mechanisch konijn aanziet. Zo heeft de \emph{Washington Post} de impact
van het federale snelwegprogramma in de hoofdstad van het land in kaart
gebracht:

\begin{quote}
De Capital Beltway in Washington was een van de eerste grote onderdelen
van het systeem die voltooid werd. Toen het laatste stuk in de zomer van
1964 werd geopend, werd het geprezen als een van de mooiste snelwegen
die ooit zijn aangelegd.

Er werd verwacht dat het (a) de verkeersopstoppingen in het centrum van
Washington zou verminderen door een rondweg te creëren voor
noord-zuidverkeer en (b) de voorstedelijke gemeenten en steden rond de
hoofdstad met elkaar zou verbinden.

Wat de ringweg in werkelijkheid werd, was (a) een snelweg voor forensen
en een rondweg voor lokaal verkeer en (b) de oorzaak van een enorme
bouwboom die de vlucht van blanken en welgestelden uit de binnenstad
versnelde.

In plaats van de verkeersopstoppingen te verminderen, heeft de Beltway
deze juist verergerd. Samen met I-95, I-70 en I-66 heeft het forenzen in
staat gesteld zich steeds verder van hun werk in het centrum te
verwijderen.

Dit heeft er ook toe geleid dat overheidsinstanties en bedrijven in de
detailhandel en dienstverlening van het centrum naar de buitenwijken
zijn verhuisd. Hierdoor liggen de banen die deze bedrijven creëren vaak
buiten het bereik van veel inwoners van de binnenstad.6
\end{quote}

Hoe zou een rationeel prijssysteem, dat ingesteld wordt door
particuliere wegbeheerders, eruitzien? In de eerste plaats zouden
snelwegen tolkosten in rekening brengen, vooral bij belangrijke
toegangspunten tot steden, zoals bruggen en tunnels, maar niet op de
huidige manier. De toltarieven zouden bijvoorbeeld veel hoger zijn
tijdens de spitsuren en andere drukke periodes (zoals zondagen in de
zomer) dan daarbuiten. In een vrije markt zou de grotere vraag in de
spits leiden tot hogere toltarieven, totdat de verkeersdrukte afneemt en
de doorstroming verbetert. Maar wat als mensen dan toch naar hun werk
moeten, vraagt de lezer zich misschien af? Natuurlijk hoeven ze niet met
hun eigen auto te gaan. Sommige forenzen zullen helemaal overstappen en
teruggaan naar de stad. Anderen zullen carpoolen, terwijl weer anderen
gebruikmaken van snelbussen of treinen. Op deze manier wordt het gebruik
van de wegen tijdens de spits beperkt tot degenen die bereid zijn om de
marktconforme prijs te betalen voor hun rit. Bovendien zullen mensen
proberen hun werktijden te verschuiven, zodat ze op andere momenten
kunnen aankomen en vertrekken. Ook weekendreizigers zullen minder gaan
rijden of hun reistijden spreiden. Tot slot zullen de hogere winsten die
bijvoorbeeld gemaakt kunnen worden op bruggen en tunnels, particuliere
bedrijven stimuleren om meer van dergelijke infrastructuur te bouwen. De
aanleg van wegen zal niet langer afhangen van lobby's van actiegroepen
en gebruikers die om subsidies vragen, maar van de efficiënte vraag- en
kostenanalyses van de markt.

Terwijl veel mensen zich de werking van particuliere snelwegen kunnen
voorstellen, raakt men in de war bij het idee van particuliere straten
in de stad. Hoe zouden die geprijsd worden? Zouden er tolpoorten bij
elke straat komen? Natuurlijk niet, want een dergelijk systeem zou
duidelijk onrendabel en onbetaalbaar zijn, zowel voor de eigenaar als
voor de bestuurder. Allereerst zullen de straateigenaren het parkeren
veel doordachter prijzen dan nu het geval is. Ze zullen het parkeren in
drukke straten in het stadscentrum fors duurder maken, als reactie op de
grote vraag. In tegenstelling tot de huidige praktijken zullen ze juist
veel meer, in plaats van minder, rekenen voor langer parkeren, de hele
dag door. Kortom, de straateigenaren zullen zich inzetten voor een
betere doorstroming in de drukke gebieden. Dit is begrijpelijk voor het
parkeren, maar hoe zit het met autorijden op drukke straten in de stad?
Hoe kan dit geprijsd worden? Er zijn verschillende mogelijke manieren.
Ten eerste zouden de straateigenaren in de binnenstad kunnen eisen dat
iedereen die hun straten gebruikt, een vergunning aanschaft, die op de
auto kan worden aangebracht, zoals nu met stickers en vergunningen
gebeurt. Bovendien zouden ze kunnen verlangen dat iedereen die tijdens
piekuren rijdt, een extra, dure vergunning koopt en laat zien. Daarnaast
zijn er andere mogelijkheden. Moderne technologie kan ervoor zorgen dat
auto's worden uitgerust met een meter die niet alleen per kilometer
meetelt, maar ook sneller registreert op drukke straten en wegen tijdens
piekuren. Hierdoor zou de auto-eigenaar aan het eind van de maand een
rekening kunnen ontvangen. Een soortgelijk plan werd tien jaar geleden
voorgesteld door professor A.A. Walters:

\begin{quote}
De specifieke administratieve instrumenten die gebruikt kunnen worden,
omvatten speciale kilometertellers, vergelijkbaar met die van taxi's.
Deze kilometertellers zouden het aantal gereden kilometers registreren
wanneer de `vlag' omhoog is; over deze kilometers zou een heffing worden
toegepast. Dit systeem zou goed werken in grote stedelijke gebieden
zoals New York, Londen en Chicago. Straten met een `vlag omhoog' zouden
voor bepaalde uren van de dag kunnen worden aangegeven. Voertuigen
zouden op deze straten mogen rijden zonder speciale kilometertellers,
mits zij een dagelijkse `sticker' kopen en zichtbaar tonen. Voor
incidenteel verkeer met een `sticker' zou een hoger bedrag gelden dan
het maximumbedrag voor voertuigen met een kilometerteller. Het toezicht
op deze regeling zou relatief eenvoudig zijn. Camera's zouden kunnen
worden geïnstalleerd om auto's zonder sticker of vlag te registreren.
Bij overtredingen kan een passende boete worden opgelegd.7
\end{quote}

Professor Vickrey heeft ook voorgesteld om verkeerscamera's te plaatsen
op de kruispunten van de drukste straten. Deze camera's zouden de
kentekens van alle auto's registreren. Automobilisten zouden dan elke
maand een rekening ontvangen voor alle keren dat ze het kruispunt
oversteken. Als alternatief stelde hij voor dat iedere auto uitgerust
zou kunnen worden met het elektronische meetapparaat van Oxford. Elke
auto zou dan een uniek signaal uitzenden, dat door het apparaat op het
betreffende kruispunt wordt opgevangen.8

Hoe dan ook, het probleem van een eerlijke prijsstelling voor straten en
snelwegen kan eenvoudig worden opgelost door particuliere bedrijven en
moderne technologie. Ondernemers op de vrije markt hebben al veel
moeilijkere vraagstukken aangepakt. Het enige wat nodig is, is hen de
ruimte geven om hun werk te doen.

Als al het transport volledig vrij zou zijn en de wegen,
luchtvaartmaatschappijen, spoorwegen en waterwegen bevrijd zouden zijn
van hun ingewikkelde netwerken van subsidies, controles en voorschriften
binnen een puur privaat systeem, hoe zouden consumenten hun
transportbudget dan verdelen? Zouden we bijvoorbeeld weer de trein
nemen? De beste schattingen van de kosten en de vraag naar vervoer geven
aan dat de spoorwegen het belangrijkste vervoermiddel zouden worden voor
vrachtvervoer over lange afstanden. Luchtvaartmaatschappijen zouden de
voorkeur genieten voor passagiersvervoer op lange afstanden,
vrachtwagens voor vrachtvervoer over korte afstanden en bussen voor
openbaar woon-werkverkeer. In feite zouden de spoorwegen een heropleving
doormaken voor langeafstandsgoederenvervoer, maar ze zouden niet opnieuw
opbloeien voor veel passagiersvervoer. In de afgelopen jaren hebben
verschillende liberalen, die teleurgesteld zijn over de overbelasting
van snelwegen, gepleit voor het massaal ontmoedigen van snelweggebruik.
Ze pleiten voor grootschalige subsidies en bouwprojecten voor metro's en
pendeltreinen voor stedelijk verkeer. Maar deze ambitieuze plannen
negeren de hoge kosten en verspilling die dit met zich meebrengt. Hoewel
veel snelwegen misschien niet gebouwd hadden moeten worden, zijn ze er
wel en het zou onzinnig zijn om er geen gebruik van te maken. Enkele
slimme transporteconomen hebben in de afgelopen jaren hun kritiek laten
horen op de enorme verspilling die gepaard gaat met de aanleg van nieuwe
snelwegen, zoals in de regio van de San Francisco Bay. Ze hebben in
plaats daarvan voorgesteld om de bestaande snelwegen beter te benutten
door snelbussen voor woon-werkverkeer in te zetten.9

Het is niet moeilijk om je een netwerk voor te stellen van particuliere,
ongesubsidieerde en ongereguleerde spoorwegen en
luchtvaartmaatschappijen. Maar kan er ook een systeem van particuliere
wegen bestaan? Is zo'n systeem überhaupt haalbaar? Eén antwoord is dat
privéwegen in het verleden goed hebben gefunctioneerd. In Engeland voor
de achttiende eeuw bijvoorbeeld waren de wegen, die doorgaans eigendom
waren van en beheerd werden door lokale overheden, slecht aangelegd en
nog slechter onderhouden. Deze openbare wegen konden nooit de krachtige
industriële revolutie van de achttiende eeuw ondersteunen, de
`revolutie' die de moderne tijd inluidde. De cruciale taak om de bijna
onbegaanbare Engelse wegen te verbeteren, werd uitgevoerd door
particuliere tolwegmaatschappijen. Deze maatschappijen, die vanaf 1706
actief waren, organiseerden en legden een uitgebreid netwerk aan dat
Engeland tot de afgunst van de wereld maakte. De eigenaren van deze
particuliere tolwegmaatschappijen waren meestal landeigenaren,
kooplieden en industriëlen uit de regio die door de weg bediend werd. Ze
vergoedden hun kosten door tol te heffen bij geselecteerde tolpoorten.
Vaak werd het innen van de tol voor een jaar of langer verhuurd aan
individuen die via een veiling geselecteerd waren. Het waren deze
privéwegen die een interne markt in Engeland ontwikkelden en de
transportkosten van kolen en ander volumineus materiaal aanzienlijk
verlaagden. Doordat ze er wederzijds voordeel uit haalden, sloten de
tolwegmaatschappijen zich aan bij elkaar om een onderling verbonden
wegennet door het hele land te vormen. Dit alles was het resultaat van
particulier ondernemerschap in actie.10

Zoals in Engeland, zo ook in de Verenigde Staten, zij het iets later.
Geconfronteerd met vrijwel onbegaanbare wegen die door lokale overheden
waren aangelegd, bouwden en financierden particuliere maatschappijen
tussen 1800 en 1830 een uitgebreid netwerk van tolwegen door de
noordoostelijke staten. Ook hier bleek particuliere ondernemingen
superieur aan de achterhaalde activiteiten van de overheid als het gaat
om wegenbouw en -beheer. De wegen werden aangelegd en geëxploiteerd door
particuliere tolwegbedrijven, en de gebruikers moesten tol betalen. Net
als in Engeland werden deze tolwegmaatschappijen grotendeels
gefinancierd door kooplieden en landeigenaren langs de routes. Zij
sloten zich vrijwillig aan bij een onderling verbonden netwerk van
wegen. Deze tolwegen vormden de eerste echt goede wegen in de Verenigde
Staten.11

\bookmarksetup{startatroot}

\chapter{De publieke sector, III: Politie, rechtspraak en
rechtbanken}\label{de-publieke-sector-iii-politie-rechtspraak-en-rechtbanken}

\section{\texorpdfstring{\textbf{POLITIEBESCHERMING}}{POLITIEBESCHERMING}}\label{politiebescherming}

De markt en privaat ondernemerschap zijn aanwezig, waardoor de meeste
mensen zich gemakkelijk een vrije markt voor de meeste goederen en
diensten kunnen voorstellen. Maar het lastigste onderdeel is
waarschijnlijk de afschaffing van overheidsactiviteiten die bescherming
bieden: politie, rechtbanken, enzovoort. Dit betreft de verdediging van
personen en eigendommen tegen aanvallen of inbreuken. Hoe kunnen
particuliere ondernemingen en de vrije markt zo'n dienst verlenen? Hoe
kunnen politie, rechtssystemen, gerechtelijke diensten en wetshandhaving
bestaan in een vrije markt? We hebben al gezien dat een groot deel van
de politiebescherming mogelijk geleverd kan worden door verschillende
eigenaren van straten en landerijen. Nu moeten we echter dit geheel
systematisch onder de loep nemen.

In de eerste plaats bestaat er een veel voorkomende misvatting, die
zelfs door de meeste voorstanders van laissez-faire wordt gedeeld: dat
de overheid `politiebescherming' moet bieden, alsof dit een enkele,
vaste dienst is die voor iedereen beschikbaar is. In werkelijkheid is er
geen absoluut goed dat `politiebescherming' heet, net zoals er geen
absoluut goed is dat `voedsel' of `onderdak' wordt genoemd. Iedereen
betaalt belasting voor een ogenschijnlijk vaste hoeveelheid bescherming,
maar dat is een mythe. Er zijn namelijk bijna ontelbare vormen van
bescherming beschikbaar. Voor een bepaalde persoon of een bedrijf kan de
politie variëren van een agent die één keer per nacht patrouilleert tot
twee agenten die continu in een wijk aanwezig zijn. Dit kan zelfs verder
gaan naar rondrijdende patrouillewagens of één of meerdere persoonlijke
lijfwachten die 24 uur per dag beschikbaar zijn. Daarnaast zijn er nog
tal van beslissingen die de politie moet nemen. De complexiteit hiervan
wordt duidelijk als we de mythe van de absolute `bescherming' nader
bekijken. Hoe zal de politie haar middelen toewijzen? Die zijn namelijk
altijd beperkt, net zoals de middelen van andere personen, organisaties
en instanties. Hoeveel moet de politie investeren in elektronische
apparatuur? In apparatuur voor het nemen van vingerafdrukken? In
detectives tegenover geüniformeerde agenten? In patrouillewagens versus
voetpatrouilles, en ga zo maar door?

Het probleem is dat de overheid geen rationele manier heeft om middelen
toe te wijzen. Ze weet alleen dat ze een beperkt budget heeft. De
toewijzing van middelen is daardoor afhankelijk van politieke
spelletjes, gierigheid en bureaucratische inefficiëntie. Dit gebeurt
zonder enige garantie dat het politiedepartement de behoeften van de
burgers op een efficiënte manier vervult. De situatie zou anders zijn
als politiediensten op een vrije, concurrerende markt zouden worden
aangeboden. In dat geval zouden consumenten betalen voor de bescherming
die zij wensen. Degenen die af en toe een politieagent willen zien,
zouden minder betalen dan degenen die voortdurend patrouilles willen. En
mensen die 24 uur per dag lijfwachten wensen, zouden nog meer moeten
betalen. Op de vrije markt zou bescherming geleverd worden naar rato van
wat consumenten bereid zijn te betalen. De verplichting om winst te
maken en verliezen te vermijden zou ervoor zorgen dat efficiëntie
gewaarborgd is, wat resulteert in lage kosten en een hoge kwaliteit die
aan de wensen van de consumenten voldoet. Politiebedrijven die
onverantwoord omgaan met middelen zouden snel failliet gaan en
verdwijnen.

Een groot probleem waar een overheids-politiemacht altijd mee te maken
heeft, is de vraag welke wetten echt gehandhaafd moeten worden.
Politieafdelingen krijgen de opdracht om `alle wetten te handhaven',
maar in de praktijk dwingt een beperkt budget hen om hun personeel en
middelen te richten op de meest dringende misdaden. Dit absolute dictaat
belemmert een rationele toewijzing van middelen. Op de vrije markt zou
de handhaving afgedwongen worden door wat klanten bereid zijn ervoor te
betalen. Stel je voor dat meneer Jones een kostbaar juweel bezit waarvan
hij vreest dat het binnenkort gestolen zal worden. Hij kan dan 24-uurs
politiebescherming aanvragen en betalen tegen een tarief dat hij met het
politiebedrijf afspreekt. Aan de andere kant heeft hij misschien ook een
privéweg op zijn landgoed waar hij niet wil dat er veel mensen op
rijden. In dat geval zal het hem waarschijnlijk weinig interesseren dat
er mensen op die weg komen. Hij zal dan geen politie inschakelen om de
weg te beschermen. Net als in de algemene markt is het aan de consument.
Aangezien wij allemaal consumenten zijn, betekent dit dat iedereen
individueel beslist hoeveel en wat voor soort bescherming hij wil en kan
kopen.

Alles wat we hebben gezegd over de politie van landeigenaren geldt ook
voor particuliere politie in het algemeen. De politie op de vrije markt
zou niet alleen efficiënt zijn, maar zou ook een sterke prikkel hebben
om hoffelijk te zijn en zich te onthouden van brutaliteit tegenover hun
klanten, of de vrienden en klanten van deze klanten. Een privé Central
Park zou efficiënt worden beveiligd om het park maximaal te laten
renderen, in plaats van dat onschuldige betalende klanten een avondklok
opgelegd krijgen. Op een vrije politiemarkt zouden efficiënte en
hoffelijke politiediensten beloond worden, terwijl elke afwijking van
deze norm bestraft zou worden. Er zou niet langer een scheiding bestaan
tussen dienstverlening en betaling zoals die nu bij overheidsoperaties
gebruikelijk is. Deze scheiding betekent dat de politie, net als andere
overheidsinstellingen, haar inkomsten niet op vrijwillige en
concurrerende basis van consumenten verwerft, maar onder dwang van de
belastingbetalers.

Omdat de overheidspolitie steeds inefficiënter wordt, wenden consumenten
zich steeds vaker tot particuliere vormen van bescherming. We hebben het
al gehad over buurt- of blokbeveiliging. Daarnaast zijn er
privébewakers, verzekeringsmaatschappijen, privédetectives en
geavanceerdere apparatuur zoals kluizen, sloten, beveiligingscamera's en
inbraakalarmen. De President's Commission on Law Enforcement and the
Administration of Justice schatte in 1969 dat de overheids politie het
Amerikaanse publiek \$2,8 miljard per jaar kostte. Tegelijkertijd gaven
consumenten \$1,35 miljard uit aan privébeschermingsdiensten en nog eens
\$200 miljoen aan apparatuur. Dit betekent dat de uitgaven voor
privébescherming meer dan de helft van die voor overheidspolitie
uitmaakten. Deze cijfers zouden zelfs de meest sceptische mensen aan het
denken moeten zetten, vooral degenen die geloven dat politiebescherming
op de een of andere mysterieuze manier, door een of ander recht of
macht, noodzakelijkerwijs en voor altijd een kenmerk van
staatssoevereiniteit is.

Elke lezer van detectivefictie weet dat privédetectives van
verzekeringsmaatschappijen veel efficienter zijn dan de politie in het
terugvinden van gestolen goederen. De verzekeringsmaatschappij heeft
immers een economische prikkel om de consument te helpen en probeert zo
uitkeringen te vermijden. De focus van de verzekeringsmaatschappij
verschilt bovendien wezenlijk van die van de politie. De politie, die
namens een abstracte `maatschappij' opereert, richt zich voornamelijk op
het vangen en straffen van criminelen. Het teruggeven van de gestolen
buit aan het slachtoffer is daarbij van secundair belang. Voor de
verzekeringsmaatschappij en haar rechercheurs daarentegen is het
terugkrijgen van de buit de hoogste prioriteit. De arrestatie en
bestraffing van de dader komen pas op de tweede plaats, want het
hoofddoel is om het slachtoffer van het misdrijf te helpen. Hier zien we
opnieuw het verschil tussen een particuliere onderneming die gedwongen
is de klant-slachtoffer van een misdrijf te bedienen, en de openbare
politie, die niet onder een dergelijke economische druk staat.

We kunnen geen blauwdruk maken voor een markt die momenteel slechts in
hypothesen bestaat. Toch is het redelijk te veronderstellen dat in een
libertarische samenleving politiediensten geleverd zouden worden door
landeigenaren of verzekeringsmaatschappijen. Aangezien
verzekeringsmaatschappijen uitkeringen uitkeren aan slachtoffers van
criminaliteit, is het zeer waarschijnlijk dat zij ook politiediensten
zouden aanbieden om criminaliteit te bestrijden en zo hun uitkeringen te
beperken. In elk geval lijkt het waarschijnlijk dat de politiediensten
zouden worden gefinancierd door regelmatige maandelijkse premies. De
dienst kan dan, of het nu een verzekeringsmaatschappij betreft of niet,
worden ingeschakeld wanneer dat nodig is.

Dit biedt een eenvoudig antwoord op de typische nachtmerrievraag van
mensen die voor het eerst over het idee van een volledig particuliere
politie horen: `Maar betekent dat dan dat je, als je aangevallen of
beroofd wordt, naar een politieagent moet rennen en moet kibbelen over
de kosten van je verdediging?' Een moment van nadenken laat zien dat
geen enkele dienst op deze manier op de vrije markt wordt aangeboden.
Het is duidelijk dat iemand die bescherming wil van Agentschap A of
Verzekeringsmaatschappij B reguliere premies zal betalen, in plaats van
te wachten tot hij wordt aangevallen voordat hij bescherming aanschaft.
`Maar stel dat er een noodgeval is en een politieagent van maatschappij
A ziet iemand overvallen; zal hij dan stoppen om te vragen of het
slachtoffer een verzekering heeft bij maatschappij A?' Ten eerste wordt
dit soort straatcriminaliteit, zoals eerder vermeld, afgehandeld door de
politie die ingehuurd is door de eigenaar van de straat. Wat gebeurt er
in het onwaarschijnlijke geval dat een buurt geen straatpolitie heeft en
een agent van bedrijf A toevallig ziet dat iemand wordt aangevallen? Zal
hij het slachtoffer te hulp schieten? Dat hangt natuurlijk af van
bedrijf A, maar het lijkt onwaarschijnlijk dat particuliere
politiebedrijven geen goodwill zouden opbouwen door beleid te voeren dat
hen aanmoedigt om gratis hulp te verlenen aan slachtoffers in nood.
Misschien vragen ze het geredde slachtoffer achteraf om een vrijwillige
donatie. Als een huiseigenaar wordt beroofd of aangevallen, zal hij
uiteraard het politiebedrijf inschakelen dat hij heeft gekozen. Hij zal
politiebedrijf A bellen in plaats van `de politie' waar hij nu op
terugvalt.

Concurrentie leidt tot efficiëntie, lage prijzen en hoge kwaliteit. Er
is daarom geen reden om aan te nemen dat het hebben van slechts één
politieagentschap in een bepaald gebied iets goddelijks is, wat veel
mensen wel denken. Economen hebben vaak betoogd dat de productie van
bepaalde goederen of diensten een `natuurlijk monopolie' is. Dit zou
betekenen dat meer dan één politiebureau in een bepaald gebied niet zou
kunnen overleven. Misschien, maar alleen een volledig vrije markt kan
hierover definitief beslissen. De markt kan bepalen welke bedrijven, in
welke hoeveelheid, grootte en kwaliteit kunnen concurreren en overleven.
Er is echter geen reden om van tevoren aan te nemen dat
politiebescherming een `natuurlijk monopolie' is.
Verzekeringsmaatschappijen zijn dat immers ook niet. Als we bedrijven
zoals Metropolitan, Equitable en Prudential naast elkaar kunnen laten
bestaan, waarom zouden we dan geen Metropolitan, Equitable en Prudential
als politiebeschermingsmaatschappijen kunnen hebben? Gustave de
Molinari, een negentiende-eeuwse Franse vrijmarkteconoom, was de eerste
die het idee van een vrije markt voor politiebescherming overwoog en
bepleitte. Molinari schatte dat er uiteindelijk verschillende
particuliere politiebureaus in steden zouden komen en één particulier
bureau in elk landelijk gebied. Dat lijkt mogelijk, vooral nu moderne
technologie het makkelijker maakt voor grote stedelijke bedrijven om ook
in afgelegen plattelandsgebieden actief te zijn. Iemand die in een klein
dorp in Wyoming woont, kan kiezen voor de diensten van een lokaal
beschermingsbedrijf of gebruikmaken van een nabijgelegen filiaal van de
Metropolitan Protection Company.

`Maar hoe kan een arm persoon zich privébescherming veroorloven,
waarvoor hij zou moeten betalen, in plaats van gratis bescherming te
ontvangen, zoals nu het geval is?' Dit is een veelgehoorde kritiek op
het idee van volledig particuliere politiebescherming. Een van de
antwoorden is dat dit probleem geldt voor elk goed of dienst in een
libertarische samenleving, en niet alleen voor de politie. Is
bescherming niet noodzakelijk? Misschien, maar dat geldt ook voor
voedsel, kleding, onderdak, en dergelijke. Deze zijn zeker minstens zo
belangrijk, zo niet belangrijker, dan politiebescherming. Toch zegt
bijna niemand dat de overheid voedsel, kleding, en onderdak moet
nationaliseren en gratis moet aanbieden als verplicht monopolie. Zeer
arme mensen zouden meestal worden ondersteund door particuliere
liefdadigheid, zoals we zagen in het hoofdstuk over welzijn. Bovendien
zijn er ongetwijfeld manieren om gratis politiebescherming aan de
bevolking te bieden. Dit kan bijvoorbeeld door de politiebedrijven zelf,
uit welwillendheid, zoals ziekenhuizen en artsen dat nu doen. Ook zouden
er speciale `politiehulp'-verenigingen kunnen ontstaan die vergelijkbaar
werk verrichten als de huidige `rechtshulp'-verenigingen. Deze
organisaties bieden vrijwillig gratis juridisch advies aan mensen in
nood die problemen hebben met de autoriteiten.

Er zijn belangrijke aanvullende overwegingen. Zoals we hebben gezien,
zijn politiediensten niet `gratis'; ze worden gefinancierd door de
belastingbetaler, en vaak is die belastingbetaler de arme persoon zelf.
Het is goed mogelijk dat hij nu meer belasting betaalt voor de politie
dan hij zou betalen aan particuliere, en veel efficiëntere,
politiebedrijven. Bovendien zouden deze bedrijven een massamarkt
aanspreken. Door de schaalvoordelen van zo'n ruime markt zou
politiebescherming ongetwijfeld veel goedkoper worden. Geen enkel
politiebedrijf kan het zich leisten om een groot deel van zijn
klantenbasis te verliezen, en de kosten van bescherming zouden niet veel
hoger zijn dan de huidige verzekeringspremies. (Sterker nog, het zou
veel goedkoper zijn dan de huidige verzekering, aangezien de
verzekeringssector momenteel sterk wordt gereguleerd door de overheid,
wat goedkope concurrentie uitsluit).

Er is nog een laatste nachtmerrie die veel mensen hebben als ze nadenken
over particuliere beveiligingsbedrijven. Ze beschouwen dit als een
doorslaggevende reden om het idee af te wijzen. Zouden die bureaus niet
voortdurend met elkaar in conflict komen? Zou er geen `anarchie'
ontstaan, met voortdurende ruzies tussen politiekorpsen wanneer de ene
persoon `zijn' politie inschakelt en de ander `de zijne'?

Er zijn verschillende niveaus van antwoorden op deze belangrijke vraag.
Ten eerste, omdat er geen overkoepelende staat zou zijn, geen centrale
of enkele lokale regering, zouden we in ieder geval gevrijwaard blijven
van de ellende van oorlogen tussen staten, met hun enorme, destructieve
en inmiddels nucleaire wapens. Wanneer we terugkijken in de
geschiedenis, is het dan niet overduidelijk dat het aantal mensen dat
omkomt bij geïsoleerde buurtconflicten vergeleken kan worden met de
totale massavernietiging die voortkomt uit interstatelijke oorlogen?
Daar zijn goede redenen voor. Laten we om emotionele discussies te
vermijden, twee hypothetische landen bekijken: `Ruritanië' en
`Walldavië'. Als beide landen zouden worden opgedeeld in een
libertarische samenleving, zonder regering en met talloze particuliere
burgers, bedrijven en politiebureaus, zouden de enige conflicten lokaal
zijn. Bovendien zou de omvang en destructiviteit van wapens
noodzakelijkerwijs sterk beperkt zijn. Stel je voor dat in een stad in
Ruritanië twee politiebureaus met elkaar in botsing komen en beginnen te
schieten. In het slechtste geval zouden zij geen massabombardementen of
nucleaire vernichtung of bacteriële oorlogsvoering kunnen inzetten,
omdat ze anders zichzelf in de lucht zouden blazen. Het zijn juist de
territoriale gebieden die zijn onderverdeeld in enkele
regeringsmonopolies die leiden tot massavernietiging. Wanneer de
enkelvoudige monopolistische regering van Walldavië haar oude rivaal, de
regering van Ruritanië, confronteert, kunnen beide regeringen
massavernietigingswapens en zelfs nucleaire wapens gebruiken, omdat het
altijd de `andere man' en het `andere land' zijn die ze schade
toebrengen. Bovendien wordt elke persoon, als onderdaan van een
monopolistische regering, in de ogen van andere regeringen
onherroepelijk geïdentificeerd met `zijn' regering. De burger van
Frankrijk wordt bijvoorbeeld gezien als een vertegenwoordiger van `zijn'
regering. Daarom zal, als een andere regering Frankrijk aanvalt, zowel
de burgerij als de regering van Frankrijk onder vuur komen te liggen.
Maar wanneer bedrijf A het conflict aangaat met bedrijf B, kunnen alleen
de klanten van deze bedrijven in het conflict worden meegesleurd, en
niemand anders. Het zou dus duidelijk moeten zijn dat zelfs als het
ergste zou gebeuren en een libertarische wereld inderdaad een wereld van
`anarchie' werd, we nog steeds veel beter af zouden zijn dan nu. We
zouden niet overgeleverd zijn aan de genade van ongebreidelde,
`anarchistische' natiestaten, die elk een angstaanjagend monopolie op
massavernietigingswapens bezitten. We moeten nooit vergeten dat we
allemaal leven, en altijd hebben geleefd, in een wereld van
`inter-nationale anarchie'. Dit is een wereld van dwingende natiestaten
zonder controle van een algemene wereldregering, en er is geen
vooruitzicht dat deze situatie zal veranderen.

Een libertarische wereld zou, ook al is het anarchistisch, nog steeds
niet lijden onder de brute oorlogen, de massavernietiging en de
A-bombardementen die onze door de staat geteisterde wereld al eeuwenlang
teisteren. Zelfs als de lokale politie voortdurend met elkaar in
conflict komt, zouden er geen tragedies meer plaatsvinden zoals in
Dresden en Hiroshima.

Maar er is nog veel meer te zeggen. We moeten nooit aannemen dat deze
lokale `anarchie' waarschijnlijk zal optreden. Laten we het probleem van
politieconflicten opsplitsen in twee delen: eerlijke meningsverschillen
en de pogingen van een of meerdere politiekorpsen om `bandiet' te
worden, fondsen te werven of hun heerschappij met dwang op te leggen.
Laten we er even van uitgaan dat de politiekorpsen eerlijk zijn en
gedreven worden door echte meningsverschillen. We laten de kwestie van
de bandietenpolitie voorlopig buiten beschouwing. Een van de
belangrijkste aspecten van de bescherming die de politie hun klanten
biedt, is stille bescherming. Elke consument of koper van
politiebescherming wil vooral efficiënte en stille bescherming, zonder
conflicten of verstoringen. Elke politieorganisatie is zich ten volle
bewust van dit essentiële feit. De veronderstelling dat de politie
voortdurend met elkaar zou botsen en strijden, is absurd. Dit negeert
het verwoestende effect dat zo'n chaotische `anarchie' zou hebben op de
activiteiten van alle politiebedrijven. Eenvoudig gezegd: zulke oorlogen
en conflicten zouden extreem schadelijk zijn voor de zaken. Op de vrije
markt zouden de politiebedrijven er daarom alles aan doen om conflicten
te voorkomen en zouden alle meningsverschillen in privérechtbanken
worden opgelost, beslist door privérechters of arbiters.

Om specifieker te zijn: in de eerste plaats zouden botsingen, zoals we
al zeiden, minimaal zijn. De eigenaar van een straat zou zijn eigen
bewakers hebben, de winkelier de zijne, en de huisbaas en huiseigenaar
ook hun eigen politiebedrijf. Realistisch gezien zou er in het dagelijks
leven weinig ruimte zijn voor directe conflicten tussen politiediensten.
Maar stel je voor dat, zoals soms gebeurt, twee naburige huiseigenaren
ruzie krijgen. De een beschuldigt de ander van geweldpleging en schakelt
zijn eigen politiebedrijf in, terwijl ze toevallig bij verschillende
bedrijven zijn aangesloten. Wat dan? Het zou zinloos en zowel economisch
als fysiek zelfvernietigend zijn voor de twee politiebedrijven om op
elkaar te schieten. In plaats daarvan zou elk politiebedrijf, om te
kunnen blijven functioneren, als essentieel onderdeel van zijn
dienstverlening het gebruik van particuliere rechtbanken of arbiters
aankondigen. Zo zouden zij kunnen beslissen wie er fout zit.

\section{De rechtbanken}\label{de-rechtbanken-1}

Stel je voor dat de rechter of arbiter oordeelt dat Smith in een geschil
fout zat en Jones heeft aangevallen. Als Smith dit vonnis accepteert,
dan zijn er, ongeacht de schadevergoeding of straf die wordt opgelegd,
geen problemen voor de theorie van de libertarische bescherming. Maar
wat als hij het vonnis niet accepteert? Laten we een ander voorbeeld
bekijken: Jones wordt beroofd. Hij schakelt zijn politiebedrijf in om de
dader op te sporen. Het bedrijf komt tot de conclusie dat een zekere
Brown de crimineel is. Wat gebeurt er dan? Als Brown zijn schuld erkent,
is er opnieuw geen probleem en volgt er een gerechtelijke straf, waarbij
de focus ligt op het dwingen van de crimineel om het slachtoffer
schadeloos te stellen. Maar wat als Brown zijn schuld ontkent?

Deze kwesties brengen ons buiten het domein van politiebescherming en
naar een ander essentieel gebied van bescherming: juridische
dienstverlening. Dit houdt in dat er, volgens algemeen aanvaarde
procedures, een methode wordt aangeboden om vast te stellen wie de
misdadiger is of wie de contractbreuk pleegt, ongeacht het soort
misdrijf of geschil. Veel mensen, zelfs degenen die erkennen dat er op
een vrije markt een particuliere en concurrerende politiedienst kan
bestaan, verzetten zich tegen het idee van volledig particuliere
rechtbanken. Hoe zouden rechtbanken überhaupt privé kunnen zijn? Hoe
zouden ze geweld kunnen toepassen in een wereld zonder regering? Zouden
er dan niet eindeloze conflicten en `anarchie' ontstaan?

Ten eerste ondervinden de monopolistische overheidsrechtbanken dezelfde
ernstige problemen als andere overheidsoperaties, zoals inefficiëntie en
minachting voor de consument. We weten allemaal dat rechters niet
gekozen worden op basis van hun wijsheid, eerlijkheid of
klantgerichtheid, maar dat ze vaak politieke maquettes zijn die via het
politieke proces zijn benoemd. Bovendien zijn rechtbanken monopolies.
Als de rechtbanken in een stad corrupt, verraderlijk, onderdrukkend of
inefficiënt worden, heeft de burger momenteel geen andere optie. De
benadeelde inwoner van Deep Falls, Wyoming, is gedwongen zich te voegen
naar de lokale rechtbank in Wyoming of helemaal niets te doen. In een
libertarische samenleving zouden er meerdere rechtbanken en rechters
beschikbaar zijn waar hij terecht kan. Er is bovendien geen reden om aan
te nemen dat er zoiets als een `natuurlijk monopolie' van rechterlijke
wijsheid bestaat. De inwoner van Deep Falls zou bijvoorbeeld kunnen
kiezen voor de plaatselijke afdeling van de Prudential Judicial Company.

Hoe zouden rechtbanken gefinancierd worden in een vrije samenleving? Er
zijn verschillende mogelijkheden. Zo zou ieder individu zich kunnen
abonneren op een juridische dienst, waarbij hij maandelijks een premie
betaalt en de rechtbank inschakelt als dat nodig is. Aangezien
rechtbanken waarschijnlijk veel minder vaak nodig zijn dan
politieagenten, kan men ook kiezen om per gebruik te betalen wanneer de
rechtbank ingeschakeld wordt. In dat geval vergoedt de crimineel of de
overtreder uiteindelijk het slachtoffer of de eiser. Een derde
mogelijkheid is dat rechtbanken door politiebureaus worden gebruikt om
geschillen op te lossen. Er zouden zelfs `verticaal geïntegreerde'
bedrijven kunnen ontstaan die zowel politie- als justitiediensten
aanbieden. De Prudential Judicial Company zou bijvoorbeeld zowel een
politie- als een justitiële afdeling kunnen hebben. Uiteindelijk kan
alleen de markt bepalen welke van deze methoden het meest effectief is.

We zouden beter op de hoogte moeten zijn van het toenemende gebruik van
particuliere arbitrage in onze maatschappij. De overheidsrechtbanken
zijn zo druk, inefficiënt en kostbaar geworden dat steeds meer partijen
zich tot particuliere arbiters wenden. Dit biedt een goedkopere en
minder tijdrovende manier om geschillen op te lossen. De afgelopen jaren
is private arbitrage uitgegroeid tot een bloeiende en succesvolle
sector. Omdat arbitrage vrijwillig is, kunnen de regels snel door de
betrokken partijen zelf worden vastgesteld. Dit gebeurt zonder de
complicaties van een lastig, ingewikkeld wettelijk kader dat voor
iedereen geldt. Hierdoor kunnen uitspraken gedaan worden door mensen die
deskundig zijn in het betreffende vakgebied of beroep. Momenteel heeft
de American Arbitration Association, met het motto `The pen is mightier
than the sword', 25 regionale kantoren door het hele land. Ze beschikt
over zo'n 23.000 arbiters. In 1969 voerde de vereniging meer dan 22.000
arbitrages uit. Daarnaast regelen verzekeringsmaatschappijen meer dan
50.000 claims per jaar via vrijwillige arbitrage. Steeds vaker worden
privé-arbiters ook succesvol ingeschakeld bij schadeclaims door
auto-ongelukken.

Men zou kunnen aanvoeren dat de beslissingen van particuliere arbiters,
terwijl ze steeds meer gerechtelijke taken op zich nemen, nog altijd
door rechtbanken moeten worden afgedwongen. Dit betekent dat zodra de
betrokken partijen het eens zijn over een arbiter, zijn uitspraak
wettelijk bindend wordt. Dit klopt, maar vóór 1920 was dit nog niet het
geval. Tussen 1900 en 1920 groeide de arbitrage juist net zo snel. De
moderne arbitragebeweging begon in Engeland tijdens de Amerikaanse
Burgeroorlog op stoom te komen. Handelaars maakten steeds vaker gebruik
van de `privérechtbanken' die door vrijwillige arbiters waren opgericht,
ook al waren de uitspraken toen nog niet wettelijk bindend. Tegen 1900
kreeg vrijwillige arbitrage ook in de Verenigde Staten steeds meer voet
aan de grond. In middeleeuws Engeland ontstond de hele structuur van het
koopmansrecht, dat door de overheidsrechtbanken vaak onhandig en
inefficiënt werd afgehandeld, binnen particuliere koopmansrechtbanken.
Deze bankiers waren puur vrijwillige arbiters en hun uitspraken waren
niet wettelijk bindend. Hoe konden ze dan toch succesvol zijn?

Het antwoord is dat de kooplieden in de Middeleeuwen en tot 1920
uitsluitend vertrouwden op uitsluiting en boycot door andere kooplieden
in de regio. Met andere woorden, als een koopman weigerde zich aan
arbitrage te onderwerpen of een beslissing negeerde, maakten de andere
kooplieden dit openbaar in de handel. Vervolgens weigerden ze zaken te
doen met de recalcitrante koopman, waardoor deze snel in de problemen
kwam. Wooldridge geeft een middeleeuws voorbeeld:

\begin{quote}
Kooplieden zorgden ervoor dat hun rechtbanken functioneerden door
simpelweg akkoord te gaan met de uitspraken. Een koopman die de
afspraken overtrad, werd niet naar de gevangenis gestuurd, maar het was
wel het einde van zijn carrière als handelaar. De naleving die zijn
collega's van hem eisten en hun invloed op zijn goederen waren
effectiever dan fysieke dwang. Neem bijvoorbeeld John van Homing, die
zijn inkomens verdiende met de verkoop van grote hoeveelheden vis. Toen
John een partij haring verkocht met de claim dat deze voldeed aan een
monster van drie vaten, maar die in werkelijkheid vermengd was met
stekelbaarsjes en rotte haring, werd hij gedwongen om het tekort te
herstellen, anders zou hij economisch uitgesloten worden.
\end{quote}

In moderne tijden werd uitsluiting nog effectiever. Het inzicht dat
iemand die een arbitrale uitspraak negeerde, nooit meer gebruik kon
maken van de diensten van een arbiter, speelde hierbij een grote rol. De
industrieel Owen D. Young, hoofd van General Electric, concludeerde dat
de morele afkeuring van andere zakenlieden een veel effectievere straf
was dan juridische handhaving. Tegenwoordig zouden moderne technologie,
computers en kredietbeoordelingen zo'n landelijke afkeuring nog
effectiever maken dan ooit tevoren.

Maar zelfs als vrijwillige arbitrage voldoende is voor commerciële
geschillen, hoe zit het dan met criminele activiteiten zoals overvallen,
verkrachtingen en bankovervallen? In deze gevallen moeten we erkennen
dat uitsluiting waarschijnlijk niet voldoende zou zijn. Het is bovendien
belangrijk om te beseffen dat particuliere straateigenaren dergelijke
criminelen niet in hun gebied willen toelaten. Voor criminele zaken zijn
rechtbanken en wetshandhaving noodzakelijk.

Hoe zouden de rechtbanken functioneren in een libertarische samenleving?
En vooral, hoe zouden ze hun beslissingen kunnen afdwingen? Bij al hun
handelingen moeten ze bovendien rekening houden met de essentiële
libertarische regel: er mag geen fysiek geweld worden gebruikt tegen
iemand die niet als crimineel is veroordeeld. Anders zouden de
gebruikers van dergelijk geweld --- of het nu de politie of de
rechtbanken zijn --- zelf als daders kunnen worden gezien, mocht later
blijken dat de persoon tegen wie zij geweld gebruikten onschuldig was.
In tegenstelling tot statistische systemen kan geen enkele politieagent
of rechter speciale immuniteit krijgen om meer dwang uit te oefenen dan
ieder ander in de samenleving zou kunnen doen.

Laten we het voorbeeld nemen dat we eerder hebben besproken. Meneer
Jones wordt beroofd, en zijn ingehuurde detectivebureau is van mening
dat Brown de dader is. Brown weigert echter zijn schuld te erkennen. Wat
nu? Ten eerste moeten we beseffen dat er op dit moment geen wereldwijd
gerechtshof of een wereldregering is die haar decreten kan afdwingen.
Desondanks hebben we, terwijl we in een staat van `internationale
anarchie' leven, weinig of geen problemen met geschillen tussen burgers
van verschillende landen. Stel bijvoorbeeld dat een inwoner van Uruguay
beweert opgelicht te zijn door een inwoner van Argentinië. Naar welke
rechtbank trekt hij dan? Hij gaat naar zijn eigen rechtbank, die van het
slachtoffer of die van de eiser. De zaak wordt voor de Uruguayaanse
rechtbank gebracht, en de Argentijnse rechtbank erkent de uitspraak.
Ditzelfde geldt als een Amerikaan meent opgelicht te zijn door een
Canadees, en ga zo maar door. In het Europa na het Romeinse Rijk, waar
verschillende Duitse stammen samenleefden, kon een Visigoot, als hij
zich benadeeld voelde door een Frank, naar zijn eigen rechtbank stappen.
De beslissing werd doorgaans door de Franken geaccepteerd. Naar de
rechtbank van de eiser gaan is ook de logische libertarische manier,
omdat het slachtoffer of de eiser degene is die benadeeld is en dus
vanzelfsprekend zijn zaak bij zijn eigen rechtbank indient. In het geval
van Jones zou hij naar de Prudential Court Company moeten gaan om Brown
aan te klagen wegens diefstal.

Het is natuurlijk mogelijk dat Brown ook cliënt is van de Prudential
Court. In dat geval is er geen probleem. De beslissing van de Prudential
is bindend voor beide partijen. Een belangrijke regel is echter dat er
geen dwingende dagvaarding tegen Brown kan worden gebruikt, omdat hij
als onschuldig moet worden beschouwd totdat hij is veroordeeld. Brown
zou met een vrijwillige dagvaarding worden opgeroepen. Dit is een
bericht waarin staat dat hij voor de rechter moet verschijnen in verband
met bepaalde aanklachten, en waarin hij of zijn wettelijke
vertegenwoordiger uitgenodigd wordt om aanwezig te zijn. Als hij niet
verschijnt, wordt hij bij verstek berecht. Dit is uiteraard minder
gunstig voor Brown, omdat zijn kant van het verhaal dan niet in de
rechtszaal naar voren zal komen. Als Brown schuldig wordt bevonden,
zullen de rechtbank en haar agenten geweld gebruiken om hem te
arresteren en de opgelegde straf af te dwingen. Deze straf zal in eerste
instantie gericht zijn op het vergoeden van het slachtoffer.

Maar wat als Brown de Prudential Court niet erkent? Wat als hij klant is
bij de Metropolitan Court Company? Dan wordt de situatie ingewikkelder.
Wat gebeurt er in dat geval? Eerst doet slachtoffer Jones zijn zaak voor
de Prudential Court. Als Brown onschuldig wordt bevonden, is de kwestie
opgelost. Maar stel dat beklaagde Brown schuldig wordt verklaard. Doet
hij niks, dan zit hij met het vonnis tegen hem. Stel nu dat Brown de
zaak voorlegt aan de Metropolitan Court Company en beweert dat er sprake
is van inefficiëntie of valsheid in geschrifte bij de Prudential. In dat
geval behandelt Metropolitan de zaak. Als Metropolitan ook tot de
conclusie komt dat Brown schuldig is, is de controverse voorbij en zal
Prudential snel stappen ondernemen tegen Brown. Maar wat als
Metropolitan Brown onschuldig acht aan de aanklacht? Wat gebeurt er dan?
Zullen de twee rechtbanken en hun gewapende agenten de confrontatie op
straat aangaan?

Nogmaals, dit zou van de rechtbanken duidelijk irrationeel en
zelfdestructief gedrag zijn. Een essentieel onderdeel van hun juridische
dienstverlening is het leveren van rechtvaardige, objectieve en
vreedzaam functionerende beslissingen. Dit is de beste manier om achter
de waarheid te komen over wie de misdaad heeft gepleegd. Als ze een
beslissing nemen en vervolgens een chaotisch vuurgevecht toestaan, zou
dat door hun klanten nauwelijks als waardevolle juridische
dienstverlening worden gezien. Een belangrijk aspect van de
dienstverlening van elke rechtbank aan haar klanten zou daarom een
beroepsprocedure moeten zijn. Kortom, elke rechtbank zou ermee instemmen
zich te houden aan een beroepsprocedure, vastgesteld door een
vrijwillige arbiter waartoe Metropolitan en Prudential zich nu zouden
wenden. De beroepsrechter zou zijn beslissing nemen en het resultaat van
dit derde proces zou bindend zijn voor de schuldige. De
Prudential-rechtbank zou dan overgaan tot handhaving.

Een hof van beroep! Maar is dit niet opnieuw het creëren van een
verplichte monopolistische overheid? Nee, want het systeem vereist niet
dat één persoon of rechtbank als hof van beroep fungeert. In de
Verenigde Staten is momenteel het Hooggerechtshof het hof van beroep. De
rechters van het Hooggerechtshof zijn daarmee de ultieme arbiters,
ongeacht wat de eiser of de gedaagde willen. In een libertarische
samenleving kunnen verschillende concurrerende privérechtbanken hun
beroep doen op elke beroepsrechter die zij eerlijk, deskundig en
objectief vinden. Geen enkele beroepsrechter of groep rechters zou onder
dwang aan de samenleving opgelegd worden.

Hoe worden de beroepsrechters gefinancierd? Er zijn verschillende
mogelijkheden, maar de meest voor de hand liggende is dat ze worden
betaald door de oorspronkelijke rechtbanken. Deze rechtbanken zouden hun
klanten kunnen laten bijdragen in hun premies of vergoedingen voor de
diensten in hoger beroep.

Maar stel je voor dat Brown blijft aandringen op nog een rechter in
hoger beroep, en dan nog een? Kan hij dan niet aan zijn vonnis
ontsnappen door maar door te procederen? Het is duidelijk dat juridische
procedures in elke maatschappij niet eindeloos kunnen doorgaan; er moet
een eindpunt zijn. In de huidige statistische samenleving, waar de
overheid de rechterlijke macht monopoliseert, is het Hooggerechtshof
willekeurig aangewezen als dat eindpunt. In een libertarische
samenleving moet er ook een afgesproken scheidslijn zijn. Aangezien er
bij een misdaad of geschil slechts twee partijen betrokken zijn - de
eiser en de gedaagde - lijkt het logisch om in de regelgeving vast te
leggen dat een beslissing van twee willekeurige rechtbanken bindend is.
Dit dekt zowel de situatie waarin de rechtbank van de eiser als die van
de gedaagde tot dezelfde beslissing komen, als het geval waarin een hof
van beroep oordeelt over een meningsverschil tussen de twee
oorspronkelijke rechtbanken.

\section{\#\# DE WET EN DE RECHTBANKEN}\label{de-wet-en-de-rechtbanken}

Een hof van beroep! Maar betekent dit niet opnieuw dat we een
verplichte, monopolistische overheid creëren? Nee, want het systeem
verplicht ons niet om één persoon of rechtbank als hof van beroep aan te
wijzen. Momenteel fungeert in de Verenigde Staten het Hooggerechtshof
als dat hof, waardoor de rechters van het Hooggerechtshof de
uiteindelijke beslissingen nemen, ongeacht wat de eiser of de gedaagde
willen. In een libertarische samenleving hebben verschillende
concurrerende privérechtbanken de vrijheid om een beroepsrechter te
kiezen die zij eerlijk, deskundig en objectief vinden. Geen enkele
beroepsrechter of groep rechters zou aan de samenleving opgelegd worden.
Hoe worden de beroepsrechters gefinancierd? Er zijn verschillende
mogelijkheden, maar de meest voor de hand liggende is dat zij betaald
worden door de oorspronkelijke rechtbanken. Deze rechtbanken zouden hun
klanten kunnen laten bijdragen aan de premies of vergoedingen voor de
diensten in hoger beroep. Maar stel je voor dat Brown blijft aandringen
op meer rechters in hoger beroep. Kan hij dan niet blijven procederen en
zo aan zijn vonnis ontsnappen? Het is duidelijk dat juridische
procedures in elke maatschappij niet eindeloos kunnen doorgaan; er moet
een eindpunt zijn. In de huidige samenleving, waar de overheid de
rechterlijke macht monopoliseert, is het Hooggerechtshof gekozen als dat
eindpunt. Ook in een libertarische samenleving moet er een gezamenlijke
grens zijn. Aangezien bij een misdaad of geschil maar twee partijen
betrokken zijn -- de eiser en de gedaagde -- lijkt het logisch om vast
te leggen in de regelgeving dat een beslissing van twee willekeurige
rechtbanken bindend is. Dit dekt zowel de situatie waarin de rechtbank
van de eiser als die van de gedaagde tot dezelfde beslissing komen, als
het geval waarin een hof van beroep oordeelt over een meningsverschil
tussen de twee oorspronkelijke rechtbanken.

Het is nu duidelijk dat er een rechtscode moet komen in de libertarische
samenleving. Maar hoe? Hoe kan er een wetboek zijn en een rechtssysteem
zonder een regering die het uitgeeft, een aangewezen groep rechters, of
een wetgevende macht die over wetten stemt? Om te beginnen, is een wet
wel in overeenstemming met de libertarische principes?

Om eerst de laatste vraag te beantwoorden: het is duidelijk dat een
wetboek nodig is om precieze richtlijnen vast te leggen voor de
privérechtbanken. Als rechtbank A bijvoorbeeld beslist dat alle
roodharigen van nature slecht zijn en gestraft moeten worden, dan is het
evident dat zulke uitspraken in strijd zijn met de libertarische
principes. Een dergelijke wet zou een inbreuk vormen op de rechten van
roodharigen en daarom onwettig zijn volgens het libertarische beginsel.
De rest van de samenleving zou hier ook niet achter kunnen staan. Daarom
is het noodzakelijk om een wettelijke code te hebben die algemeen
geaccepteerd wordt en waar de rechtbanken zich aan houden. De wet zou
nadrukkelijk het libertarische principes van geen agressie tegen
personen of eigendom moeten benadrukken, eigendomsrechten definiëren
conform deze principes, bewijsregels opstellen (zoals die nu bestaan) om
te bepalen wie de verantwoordelijke partij is in een geschil, en een
code opstellen met maximale straffen voor specifieke misdrijven. Binnen
zo'n framework zouden de rechtbanken met elkaar concurreren op het
gebied van de meest efficiënte procedures. De markt zou dan bepalen of
rechters, jury's, en andere juridische structuren de meest efficiënte
methoden zijn om juridische diensten te verlenen.

Zijn zulke stabiele en consistente wetboeken mogelijk, met alleen
concurrerende rechters om ze te ontwikkelen en toe te passen, en zonder
regering of wetgevende macht? Ja, ze zijn niet alleen mogelijk, maar de
beste en meest succesvolle onderdelen van ons rechtssysteem zijn door de
jaren heen juist op deze manier ontstaan. Wetgevende machten en koningen
zijn vaak grillig, invasief en inconsequent geweest. Ze hebben juist
anomalieën en despotisme in het rechtssysteem geïntroduceerd. In feite
is de regering net zo min gekwalificeerd om wetten te ontwikkelen en toe
te passen als om andere diensten te verlenen. Net zoals religie van de
staat is gescheiden en de economie van de staat dat kan zijn, kunnen ook
andere staatsfuncties zoals politie, rechtbanken en de wet zelf
ditzelfde pad volgen.

Zoals eerder aangegeven, werd het hele koopmansrecht niet door de staat
of in staatsrechtbanken ontwikkeld, maar door particuliere
koopmansrechtbanken. Pas veel later nam de overheid het koopvaardijrecht
over van de ontwikkeling die had plaatsgevonden in de
koopmansrechtbanken. Het zelfde gebeurde met het admiraliteitsrecht en
de hele structuur van het zeerecht, waaronder scheepvaart, berging,
enzovoort. Ook hier was de staat niet geïnteresseerd en had hij geen
jurisdictie op volle zee. Daarom namen de schippers zelf de
verantwoordelijkheid om het zeerecht niet alleen toe te passen, maar ook
verder uit te werken in hun eigen privérechtbanken. Pas later eigende de
overheid zich het admiraliteitsrecht toe voor haar eigen rechtbanken.

Tot slot werd het belangrijkste Angelsaksische recht, het
gerechtvaardigd geachte gewoonterecht (common law), door de eeuwen heen
ontwikkeld door concurrerende rechters die oude principes toepasten in
plaats van de veranderende decreten van de staat. Deze principes werden
niet willekeurig vastgesteld door een koning of wetgever; ze groeiden
door de eeuwen heen door rationele en vaak libertarische principes toe
te passen op de zaken die aan hen werden voorgelegd. Het idee om
precedenten te volgen kwam voort uit de wetenschap dat rechters uit het
verleden hun beslissingen hadden genomen door de algemeen aanvaarde
common law-principes toe te passen op specifieke gevallen en problemen.
Er was algemeen consensus dat de rechter geen wetten maakte (zoals vaak
het geval is in de huidige tijd); de verantwoordelijkheid van de
rechter, zijn expertise, lag in het vinden van de wet binnen de
geaccepteerde common law-principes en het toepassen van die wet op
specifieke gevallen of op nieuwe technologische of institutionele
omstandigheden. De roem van de eeuwenlange ontwikkeling van de common
law getuigt van hun succes.

De rechters van het common law leken bovendien veel op particuliere
arbiters, die als experts in de wet fungeerden. Partijen gingen naar hen
toe met hun geschillen. Er was geen opgelegd `hooggerechtshof' wiens
beslissing bindend was, en precedent, hoewel gerespecteerd, werd niet
automatisch als bindend beschouwd. Zo schreef de libertarische
Italiaanse jurist Bruno Leoni:

\begin{quote}
De rechtbanken in Engeland konden niet zomaar arbitraire regels
uitvaardigen. Ze bevonden zich nooit in de positie om dit rechtstreeks
te doen, zoals wetgevers dat gewoonlijk deden, op een plotselinge,
wijdverspreide en dwingende manier. Bovendien waren er zoveel
rechtbanken in Engeland, en waren ze zo jaloers op elkaar, dat zelfs het
beroemde principe van het bindende precedent pas in relatief recente
tijden openlijk door hen werd erkend. Daarnaast konden ze alleen
beslissingen nemen over zaken die eerder door privépersonen aan hen
waren voorgelegd. Tot slot waren er relatief weinig mensen die naar de
rechtbanken gingen om hen te vragen naar de regels die op hun situaties
van toepassing waren.

En over de afwezigheid van `hooggerechtshoven':

Het kan niet worden ontkend dat de wet op de advocatuur of de
rechterlijke macht de neiging heeft om wetgevende kenmerken te vertonen.
Dit omvat ook ongewenste aspecten, vooral wanneer juristen of rechters
in laatste instantie over een zaak moeten beslissen. In onze tijd leidt
het systeem van de rechterlijke macht, in landen met
`hooggerechtshoven', er soms toe dat de persoonlijke standpunten van de
leden van deze hoven, of van een meerderheid daarvan, worden opgelegd
aan alle andere betrokkenen. Dit gebeurt vaak als er veel onenigheid
bestaat tussen hun mening en de overtuigingen van anderen. Deze
mogelijkheid is echter, verre van een onvermijdelijk kenmerk van het
juristenrecht of het gerechtelijk recht, eerder een afwijking
ervan.\^{}5
\end{quote}

Afgezien van dergelijke afwijkingen waren de persoonlijke opvattingen
van de rechters tot een minimum beperkt. Dit kwam doordat: (a) rechters
alleen beslissingen konden nemen wanneer particuliere burgers zaken bij
hen indienen; (b) de beslissingen van elke rechter alleen betrekking
hadden op het specifieke geval; en (c) de rechters en advocaten altijd
de juridische geschiedenis in overweging namen. Bovendien merkt Leoni op
dat rechters, in tegenstelling tot wetgevende of uitvoerende instanties
waar dominante meerderheden vaak minderheden negeren, door hun rol
verplicht zijn de argumenten van beide partijen te horen en te wegen.
`Partijen staan gelijk ten opzichte van de rechter, omdat ze vrij zijn
om argumenten en bewijzen in te dienen. Ze vormen geen groep waarin
afwijkende minderheden het afleggen tegen zegevierende meerderheden.'
Leoni wijst ook op de overeenkomst met een vrije markteconomie:
`Argumenten kunnen inderdaad sterker of zwakker zijn, maar het feit dat
elke partij deze kan aandragen, is vergelijkbaar met het idee dat
iedereen individueel met iedereen kan concurreren op de markt om te
kopen en verkopen.'\^{}6

Professor Leoni ontdekte dat de oude Romeinse rechters in het
privaatrecht op dezelfde manier te werk gingen als de rechtbanken van de
Engelse common law:

\begin{quote}
De Romeinse jurist was in wezen een wetenschapper. Hij bestudeerde
oplossingen voor zaken die burgers hem voorlegden, vergelijkbaar met hoe
industriëlen tegenwoordig een natuurkundige of ingenieur om hulp vragen
bij technische problemen in hun fabrieken of productieprocessen. Het
Romeinse privaatrecht was dus iets dat beschreven of ontdekt moest
worden, niet iets dat simpelweg werd uitgevaardigd. Het vormde een
wereld van bestaande zaken, deel van het gemeenschappelijke erfgoed van
alle Romeinse burgers. Niemand vaardigde dit recht uit en niemand kon
het veranderen door een persoonlijke wil. Dit is het
lange-termijnconcept, of het Romeinse concept, van de zekerheid van de
wet.\^{}7
\end{quote}

Tot slot kon professor Leoni zijn kennis van het oude recht en het
gewoonterecht inzetten om de belangrijke vraag te beantwoorden: In een
libertarische samenleving, `wie zal de rechters benoemen?'

\ldots om hen de taak te laten uitvoeren om de wet te definiëren?' Zijn
antwoord was: de mensen zelf. Mensen die naar de rechters zouden gaan
met de beste reputatie op het gebied van deskundigheid en wijsheid in
het kennen en toepassen van de fundamentele juridische principes die in
de samenleving gelden.

\begin{quote}
In feite is het niet van groot belang wie de rechters benoemt, omdat in
zekere zin iedereen dat zou kunnen doen. Dit gebeurt al tot op zekere
hoogte wanneer mensen particuliere bemiddelaars inschakelen om hun
geschillen te beslechten. Het aanstellen van rechters is immers geen
bijzonder probleem, zoals het `aanstellen' van natuurkundigen, artsen of
andere deskundigen. Het ontstaan van goede professionals in een
samenleving lijkt alleen te danken aan officiële benoemingen, als die er
al zijn. In werkelijkheid berust het op de brede instemming van klanten,
collega's en het grote publiek. Zonder deze instemming is geen enkele
benoeming echt effectief. Natuurlijk kunnen mensen zich vergissen in de
waarde die ze aan iemand toekennen, maar deze moeilijkheden zijn
onvermijdelijk bij elke vorm van keuze. \^{}8
\end{quote}

Natuurlijk zou in een toekomstige libertarische samenleving de
basiswetgeving niet zomaar gebaseerd zijn op gewoonten, zeker niet als
veel daarvan antilibertarisch van aard is. Het wetboek zou moeten worden
opgesteld op basis van erkende libertarische principes, zoals het
beginsel van non-agressie tegen de persoon of het eigendom van anderen.
Kortom, het zou moeten steunen op de rede in plaats van op traditie, hoe
goed die ook moge zijn. Aangezien we echter beschikken over een corpus
van common law-principes om uit te putten, zou het corrigeren en
aanpassen van de common law een stuk eenvoudiger zijn dan het vanuit het
niets ontwikkelen van een nieuw geheel aan systematische juridische
principes.

Het meest opmerkelijke historische voorbeeld van een samenleving met
libertarische wetten en rechtbanken is echter tot voor kort door
historici verwaarloosd. Deze samenleving had niet alleen libertaire
rechtbanken en wetten, maar opereerde ook volledig zonder een staat. We
hebben het hier over het oude Ierland -- een Ierland dat ongeveer
duizend jaar lang dit libertarische pad volgde, tot de brutale
verovering door Engeland in de zeventiende eeuw. In tegenstelling tot
veel primitieve stammen die op een vergelijkbare manier functioneerden,
zoals de Ibo's in West-Afrika en verschillende Europese stammen, was het
Ierland van vóór de verovering in geen enkel opzicht `primitief'. Het
was juist een zeer complexe samenleving die eeuwenlang de meest
geavanceerde, meest geleerde en meest beschaafde van heel West-Europa
was.

Duizend jaar lang kende het oude Keltische Ierland geen staat of iets
dat daarop leek. Zoals de belangrijkste autoriteit op het gebied van het
oude Ierse recht eens opmerkte: `Er was geen wetgevende macht, geen
deurwaarders, geen politie, geen openbare handhaving van het
recht\ldots{} Er was geen spoor van door de staat bestuurde
rechtspraak.'\^{}9

Hoe werd rechtvaardigheid dan gewaarborgd? De politieke basiseenheid van
het oude Ierland was de tuath. Alle `vrije mensen' die grond bezaten,
evenals beroepsbeoefenaars en ambachtslieden, hadden het recht om lid te
worden van een tuath. De leden van elke tuath kwamen jaarlijks bijeen om
gezamenlijk besluiten te nemen over gemeenschappelijk beleid, oorlog of
vrede te verklaren tegen andere tuatha en hun `koningen' te kiezen of af
te zetten. Een belangrijk punt is dat, in tegenstelling tot primitieve
stammen, niemand vastzat of gebonden was aan een specifieke tuath, noch
door verwantschap, noch door geografische ligging. Individuele leden
waren vrij om zich af te scheiden van een tuath en zich aan te sluiten
bij een concurrerende tuath, iets dat regelmatig gebeurde. Het kwam
zelfs voor dat twee of meer tuatha besloten om samen te smelten tot een
enkele, efficiëntere eenheid. Zoals professor Peden opmerkt: `De tuath
is dus een lichaam van personen die vrijwillig verenigd zijn voor
sociaal nuttige doeleinden, en de som van de landeigendommen van haar
leden vormde haar territoriale dimensie.'\^{}10 Kortom, ze beschikten
niet over een moderne staat met een aanspraak op soevereiniteit over een
specifiek (meestal uitbreidend) grondgebied, dat gescheiden was van de
landeigendomsrechten van zijn onderdanen. Tuatha waren daarentegen
vrijwillige verenigingen die alleen de eigendommen van hun vrijwillige
leden omvatten. In Ierland bestonden er op elk moment ongeveer 80 tot
100 tuatha naast elkaar.

Maar hoe zit het met de gekozen `koning'? Was hij een soort
staatsheerser? De koning fungeerde vooral als een religieuze
hogepriester, die de erediensten van de tuath leidde. Deze tuath was een
vrijwillige religieuze, sociale en politieke organisatie. Net als in
heidense, voorchristelijke priesterrijken was de koningsfunctie
erfelijk. De koning werd gekozen door de tuath uit een koninklijke
verwantengroep, de derbfine, die deze erfelijke priestelijke rol
vervulde. Politiek gezien waren de functies van de koning echter strikt
beperkt. Hij was de militaire leider van de tuath en voorzat de
vergaderingen. Oorlog voeren of vredesonderhandelingen kon hij alleen
als vertegenwoordiger van de vergadering. Hij was op geen enkele manier
soeverein en had geen rechten om recht te spreken over de leden van de
tuath. Daarnaast mocht hij zelf geen recht spreken; als hij partij was
in een rechtszaak, moest hij zijn zaak voorleggen aan een onafhankelijke
rechter.

Nogmaals, hoe werd het recht dan ontwikkeld en gehandhaafd? In de eerste
plaats was de wet gebaseerd op eeuwenoude gebruiken die mondeling werden
doorgegeven en later schriftelijk zijn vastgelegd door een groep
professionele juristen, de brehons. De brehons waren geen openbare of
overheidsfunctionarissen; ze werden simpelweg geselecteerd door de
partijen in een geschil, op basis van hun reputatie op het gebied van
wijsheid, kennis van het gewoonterecht en de integriteit van hun
beslissingen. Zoals professor Peden opmerkt:

\begin{quote}
De professionele juristen werden door partijen in geschillen
geraadpleegd voor advies over de inhoud van de wet in specifieke
gevallen. Deze juristen traden vaak ook op als scheidsrechters tussen
rechtzoekenden. Ze bleven te allen tijde privépersonen en waren geen
ambtenaren. Hun functioneren hing af van hun kennis van de wet en de
integriteit van hun juridische reputatie. \^{}11
\end{quote}

Bovendien hadden de brehons geen enkele connectie met de individuele
tuatha of hun koningen. Ze waren volledig onafhankelijk en nationaal
georiënteerd, en werden ingeschakeld door geschilpunten in heel Ierland.
Een belangrijk punt is dat de brehon, in tegenstelling tot het systeem
van Romeinse privéadvocaten, de enige was die er was; er waren geen
andere rechters, geen `openbare' rechters van welke aard dan ook in het
oude Ierland.

Het waren de brehons die geschoold waren in de wet en die glans en
toepassing aan de wet gaven, aangepast aan veranderende omstandigheden.
Daarnaast was er geen monopolie van de brehon-juristen; er waren
verschillende concurrerende scholen van jurisprudentie die streden om de
gewoonten van het Ierse volk.

Hoe werden de beslissingen van de brehons afgedwongen? Dit gebeurde via
een uitgebreid, vrijwillig ontwikkeld systeem van `borgstelling'. Mensen
waren met elkaar verbonden door verschillende borgstellerrelaties,
waarmee ze elkaar garanties boden voor het rechtzetten van misstanden en
de handhaving van gerechtigheid, evenals de beslissingen van de brehons.
Kort samengevat waren de brehons zelf niet betrokken bij de uitvoering
van hun beslissingen; die verantwoordelijkheid lag bij privépersonen die
door borgstelling met elkaar verbonden waren. Er bestonden verschillende
soorten borgstellers. Een borgsteller stond bijvoorbeeld met zijn eigen
bezittingen garant voor de betaling van een schuld. In dat geval voegde
hij zich bij de eiser om een vonnis uit te voeren wanneer de schuldenaar
weigerde te betalen. In zo'n situatie moest de schuldenaar een dubbele
schadevergoeding betalen: een aan de oorspronkelijke schuldeiser en een
aan zijn borgsteller. Dit systeem gold voor alle overtredingen,
agressies en aanrandingen, maar ook voor commerciële overeenkomsten.
Kortom, het was van toepassing op alle gevallen die wij `burgerlijk' en
`strafrecht' zouden noemen. Alle criminelen werden gezien als
`schuldenaars' die restitutie en compensatie verschuldigd waren aan hun
slachtoffers, die daarmee hun `schuldeisers' werden. Het slachtoffer
verzamelden zijn borgstellers om zich heen en ging verder met het
aanhouden van de misdadiger of het publiekelijk aankondigen van zijn
beschuldigingen. Hij eiste dat de gedaagde zich overgaf aan de
berechting van hun geschil bij de brehons. De misdadiger kon zijn eigen
borgstellers sturen om te onderhandelen over een schikking, of instemmen
om het geschil voor te leggen aan de brehons. Weigerde hij dit, dan werd
hij door de hele gemeenschap als `vogelvrij' beschouwd; hij kon dan geen
enkele eigen vordering meer afdwingen in de rechtbanken en werd door de
gemeenschap met afkeuring bejegend. \^{}12

Er waren zeker af en toe `oorlogen' in de duizend jaar dat Keltisch
Ierland bestond, maar deze bestonden vaak uit kleine gevechten die
verwaarloosbaar waren in vergelijking met de verwoestende oorlogen die
de rest van Europa teisterden. Zoals professor Peden opmerkt,

\begin{quote}
Zonder het dwangapparaat van de staat, die via belastingen en
dienstplicht grote hoeveelheden wapens en mankracht kan mobiliseren,
waren de Ieren niet in staat om een grootschalig leger langdurig in het
veld te houden. Ierse oorlogen waren, vergeleken met Europese
maatstaven, armzalige vechtpartijen en veediefstallen.
\end{quote}

Hierboven hebben we aangegeven dat het theoretisch en historisch heel
goed mogelijk is om een efficiënte en hoffelijke politie, competente en
goed opgeleide rechters en een systematische en sociaal geaccepteerde
wetgeving te hebben, zonder dat een dwingende overheid hierbij betrokken
is. De overheid, die een verplicht monopolie op bescherming over een
geografisch gebied opeist en haar inkomsten met geweld verwerft, kan
worden gescheiden van het hele spectrum van bescherming. De overheid is
net zo min noodzakelijk voor het leveren van vitale beschermingsdiensten
als voor andere taken. Bovendien hebben we nog niet benadrukt dat er een
cruciaal feit is over de overheid: haar verplichte monopolie op de
wapens van dwang heeft door de eeuwen heen geleid tot ontelbare
slachtpartijen en veel grotere tirannie en onderdrukking dan enige
gedecentraliseerde, particuliere instantie ooit zou kunnen
bewerkstelligen. Als we kijken naar het zwarte record van massamoord,
uitbuiting en tirannie dat overheden door de eeuwen heen op de
samenleving hebben uitgeoefend, hoeven we niet terughoudend te zijn in
onze beslissing om de Leviathaanse staat te verlaten en te streven naar
vrijheid.

\section{Criminele beschermers}\label{criminele-beschermers}

Er waren ongetwijfeld af en toe `oorlogen' in de duizend jaar dat
Keltisch Ierland bestond, maar deze bestonden vaak uit kleine gevechten
die te verwaarlozen waren in vergelijking met de verwoestende oorlogen
die de rest van Europa teisteren. Zoals professor Peden opmerkt: zonder
het dwangapparaat van de staat, dat via belastingen en dienstplicht
grote hoeveelheden wapens en mankracht kan mobiliseren, waren de Ieren
niet in staat om een grootschalig leger langdurig in het veld te houden.
Ierse oorlogen waren, gemeten naar Europese maatstaven, armzalige
gevechten en veediefstallen. We hebben hierboven aangegeven dat het
zowel theoretisch als historisch heel goed mogelijk is om een efficiënte
en hoffelijke politie, competente en goed opgeleide rechters, en een
systematische en sociaal geaccepteerde wetgeving te hebben, zonder
tussenkomst van een dwingende overheid. De overheid, die een verplicht
monopolie op bescherming over een bepaald geografisch gebied opeist en
haar inkomsten met geweld verwerft, kan eigenlijk gescheiden worden van
het hele domein van bescherming. De overheid is niet meer nodig voor het
leveren van essentiële beschermingsdiensten dan voor andere taken.
Bovendien willen we nog benadrukken dat er een cruciaal feit is over de
overheid: haar verplichte monopolie op dwang heeft door de eeuwen heen
geleid tot ontelbare slachtpartijen en veel grotere tirannie en
onderdrukking dan welke gedecentraliseerde particuliere instantie dan
ook ooit zou kunnen veroorzaken. Als we kijken naar het treurige
verleden van massamoord, uitbuiting en tirannie dat overheden door de
geschiedenis heen op de samenleving hebben uitgeoefend, hoeven we niet
terughoudend te zijn in onze keuze om de Leviathaanse staat achter ons
te laten en te streven naar vrijheid.

We hebben het moeilijkste probleem voor het laatst bewaard: wat als de
politie, rechters en rechtbanken oneerlijk en partijdig zijn? Wat als ze
hun beslissingen bijvoorbeeld bevoordelen op basis van financiële
invloed van rijke cliënten? We hebben aangetoond hoe een libertarisch
juridisch en gerechtelijk systeem zou kunnen functioneren binnen een
pure vrije markt, waarin eerlijke meningsverschillen centraal staan.
Maar wat als één of meerdere politieagenten of rechtbanken in de
praktijk vogelvrij worden? Wat nu?

Libertariërs deinzen niet terug voor deze vragen. In tegenstelling tot
utopisten zoals marxisten of linkse anarchisten (anarchocommunisten of
anarcho-syndicalisten) gaan zij er niet van uit dat de komst van de
ideale vrije maatschappij automatisch een nieuwe, magisch
getransformeerde libertarische mens met zich meebrengt. Ze verwachten
niet dat de leeuw vredig naast het lam zal liggen of dat niemand
kwaadwillende of frauduleuze bedoelingen zal hebben ten opzichte van
zijn buurman. Natuurlijk geldt: hoe `beter' de mensen zijn, hoe beter
elk sociaal systeem zal functioneren. Dat betekent ook dat de politie en
de rechtbanken minder werk te doen zullen hebben. Maar libertariërs
nemen deze aanname niet als vanzelfsprekend aan. Wat wij beweren, is
dat, ongeacht de mate van `goedheid' of `slechtheid' onder de mensen,
een puur libertarische samenleving tegelijkertijd de meest morele,
efficiënte, minst criminele en veiligste is voor zowel de persoon als
zijn eigendom.

Laten we eerst kijken naar het probleem van de oneerlijke of corrupte
rechter of rechtbank. Wat als een rechtbank haar eigen rijke klant
bevoordeelt? In de eerste plaats is zo'n bevoordeling uiterst
onwaarschijnlijk, vanwege de voordelen en beloningen van de vrije
markteconomie. Het bestaan van de rechtbank en het inkomen van een
rechter hangen af van zijn reputatie op het gebied van integriteit,
eerlijkheid, objectiviteit en het zoeken naar de waarheid in elke zaak.
Dit is zijn `merknaam'. Als er ook maar iets uitlekt over oneerlijkheid,
verliest hij direct klanten. De rechtbanken zullen dan ook geen klanten
meer hebben, want zelfs cliënten die zelf crimineel zijn, zullen
nauwelijks een rechtbank sponsoren die niet serieus genomen wordt door
de rest van de samenleving. Ook als zij zelf in de gevangenis zitten
vanwege oneerlijke en frauduleuze handelingen, zullen zij niet
investeren in een rechtbank met een gebrek aan geloofwaardigheid. Neem
bijvoorbeeld Joe Zilch, die beschuldigd wordt van een misdaad of
contractbreuk en naar een `rechtbank' gaat onder leiding van zijn
zwager. Niemand, zeker andere eerlijke rechtbanken niet, zal de
beslissing van deze `rechtbank' serieus nemen. In de ogen van iedereen
behalve Joe Zilch en zijn familie zal het niet meer als een `rechtbank'
worden beschouwd.

Vergelijk dit ingebouwde correctiemechanisme eens met de huidige
overheidsrechtbanken. Rechters worden voor lange termijnen benoemd of
gekozen, soms zelfs levenslang. Ze hebben het monopolie op de
besluitvorming in hun specifieke gebied. Het is bijna onmogelijk om iets
te doen aan de oneerlijke beslissingen van rechters, behalve in gevallen
van grove corruptie. Jaar na jaar blijft hun macht om beslissingen te
nemen en uit te voeren ongecontroleerd. Hun salarissen worden betaald,
onder dwang van de ongelukkige belastingbetaler. In een volledig vrije
maatschappij zorgt echter elke verdenking van een rechter of rechtbank
ervoor dat hun klanten verdwijnen en dat hun `beslissingen' niet meer
serieus worden genomen. Dit systeem is veel efficiënter om rechters
eerlijk te houden dan het huidige mechanisme van de overheid.

Bovendien zou de verleiding tot oneerlijkheid en partijdigheid om een
andere reden veel kleiner zijn: bedrijven in de vrije markt verdienen
hun geld niet door rijke klanten te bedienen, maar door een massamarkt
van consumenten. Macy's haalt haar omzet uit de brede bevolking, niet
uit een handvol rijke klanten. Dit geldt vandaag de dag ook voor
Metropolitan Levensverzekeringen, en hetzelfde zou morgen gelden voor
ieder `Metropolitan'-rechtssysteem. Het zou namelijk onzinnig zijn voor
rechtbanken om het verlies van de steun van de meerderheid van hun
klanten te riskeren voor de gunsten van enkele rijke klanten. Dit staat
in schril contrast met het huidige systeem, waar rechters, net als
andere politici, afhankelijk kunnen zijn van rijke donateurs die de
campagnes van hun politieke partijen financieren.

Er bestaat een mythe dat het `Amerikaanse systeem' een uitstekende reeks
`checks and balances' biedt. Daarbij zouden de uitvoerende macht, de
wetgevende macht en de rechtbanken elkaar in evenwicht houden en
controleren, zodat de macht zich niet onnodig verzamelt in handen van
enkelen. Maar in werkelijkheid is het Amerikaanse systeem van `checks
and balances' grotendeels bedrog. Elke instelling vormt namelijk een
dwangmonopolie binnen zijn eigen domein, en ze maken allemaal deel uit
van één overheid, die op elk moment wordt geleid door één politieke
partij.

Bovendien zijn er in het beste geval maar twee partijen die dicht bij
elkaar staan in ideologie en personeel. Ze werken vaak samen, en de
dagelijkse gang van zaken van de overheid wordt geleid door een
ambtenarenbureaucratie die niet door de kiezers kan worden veranderd.
Stel je deze mythische checks and balances tegenover de echte checks and
balances van de vrije markteconomie! Wat zorgt ervoor dat A\&P eerlijk
blijft, is de werkelijke en potentiële concurrentie van Safeway, Pioneer
en talloze andere kruidenierswinkels. Hun eerlijkheid komt voort uit de
mogelijkheid voor consumenten om hun klantenkring te verlaten. Wat
rechters en rechtbanken in de vrije markt eerlijk zou houden, is de
optie om naar een andere rechter of rechtbank te gaan als er twijfels
zijn over een bepaalde rechters of rechtbank. De aanzienlijke kans dat
klanten hun diensten intrekken, zou hen ook eerlijk houden. Dit zijn de
echte, actieve checks and balances van de vrije markteconomie en de
vrije samenleving.

Dezelfde analyse geldt voor de mogelijkheid dat een particuliere
politiemacht crimineel wordt, haar dwangmiddelen gebruikt om betalingen
af te dwingen of zelfs een afpersingstruc opzet om slachtoffers uit te
schakelen. Natuurlijk is dat niet uitgesloten. Maar in tegenstelling tot
de huidige maatschappij zouden er meteen controles en tegenwichten
aanwezig zijn. Er zouden andere politiekorpsen zijn die hun wapens
kunnen gebruiken om samen te werken en de agressors tegen hun klanten
aan te pakken. Als de Metropolitan Police Force inderdaad gangsters zou
worden en een afpersing zou eisen, dan zou de rest van de samenleving
zich kunnen wenden tot de politie van Prudential of Equitable. Zij
zouden zich kunnen verenigen om deze gangsters te bestrijden. Dit staat
in schril contrast met de situatie waarin de staat zich bevindt. Wanneer
een groep gangsters het staatsapparaat overneemt, dat beschikt over het
monopolie op geweld, is er op dat moment niets dat hen kan tegenhouden,
behalve het uiterst complexe proces van een revolutie. In een
libertarische samenleving zou er geen massale revolutie nodig zijn om de
plunderingen van gangsterstaten te stoppen. In plaats daarvan zou er
snel een beroep gedaan kunnen worden op eerlijke politiediensten om de
criminele macht te controleren en te verwerpen.

En inderdaad, wat is de staat eigenlijk anders dan georganiseerde
misdaad? Wat is belastingheffing anders dan diefstal op enorme,
ongecontroleerde schaal? Wat is oorlog anders dan massamoord op een
schaal die een particuliere politiemacht nooit zou kunnen bereiken? Wat
is dienstplicht anders dan dwangmatige onderwerping? Kan iemand zich
voorstellen dat een particuliere politiemacht wegkomt met een fractie
van wat staten doen, en dat ook nog eens jaar na jaar, eeuw na eeuw?

Er is nog een andere belangrijke overweging die het voor een criminele
politiemacht bijna onmogelijk zou maken om zich te gedragen zoals
moderne regeringen. Een van de cruciale factoren die regeringen in staat
stelt om de vreselijke dingen te doen die ze vaak doen, is het gevoel
van legitimiteit bij het publiek. De gemiddelde burger hoeft de
beleidsmaatregelen van zijn regering misschien niet leuk te vinden of
zelfs tegen te zijn, maar hij is doordrenkt van het idee - zorgvuldig
ingeprent door eeuwen van regeringspropaganda - dat de regering de
legitieme soeverein is. Het zou slecht of zelfs stapelgek zijn om haar
dictaten te negeren. Dit gevoel van legitimiteit is door de eeuwen heen
gekoesterd door de intellectuelen van de staat, ondersteund door
allerlei middelen om legitimiteit te creëren: vlaggen, rituelen,
ceremonies, onderscheidingen, grondwetten, enzovoort. Een bende van
criminelen - zelfs als alle politiekorpsen zich zouden verenigen in één
grote bende - zou nooit een dergelijke legitimiteit kunnen afdwingen.
Het publiek zou hen simpelweg als bandieten beschouwen. Hun afpersingen
en `eerbetuigingen' zouden niet als legitieme belasting worden gezien
die je automatisch moet betalen. Het publiek zou zich snel verzetten
tegen deze onrechtmatige eisen en de bandieten zouden worden
tegengehouden en omvergeworpen. Zodra het publiek de geneugten,
welvaart, vrijheid en efficiëntie van een libertarische, staatloze
samenleving heeft ervaren, wordt het bijna onmogelijk voor een staat om
het publiek nog aan zich te binden. Als je eenmaal volledig van de
vrijheid hebt genoten, is het geen gemakkelijke opgave om mensen te
dwingen dit op te geven.

Maar stel je eens voor dat, ondanks al deze handigheidjes en obstakels,
ondanks de liefde voor hun pas ontdekte vrijheid en de inherente checks
and balances van de vrije markt, de staat er toch in slaagt om zichzelf
te herstellen. Wat dan? Het enige wat dan zou gebeuren, is dat we weer
een staat zouden hebben. We zouden er niet slechter voor staan dan nu
met onze huidige staat. En zoals een libertaire filosoof het verwoordde:
`De wereld zal tenminste een glorieuze vakantie hebben gehad.' De
klinkende belofte van Karl Marx geldt veel meer voor een libertarische
samenleving dan voor het communisme: door vrijheid te omarmen en de
staat af te schaffen, hebben we niets te verliezen en alles te winnen.

\section{\# Nationale Defensie}\label{nationale-defensie}

De staat wordt vaak gezien als de hoeder van de nationale defensie. Maar
wat betekent dat eigenlijk? Voor velen lijkt het erop dat nationale
defensie voornamelijk draait om militaire macht en wapengekletter. Maar
is het niet veel meer dan dat? Is de rol van de staat in de nationale
defensie niet eerder een poging om controle uit te oefenen over haar
burgers? Wanneer we kijken naar de geschiedenis, zien we dat de meeste
conflicten niet zozeer voortkwamen uit een oprechte noodzaak om het volk
te beschermen, maar eerder uit politieke ambities en machtsstrategieën.
De meeste oorlogen zijn gevoerd om territorium te veroveren of
economische voordelen te behalen. Als we de staat beschouwen als een
georganiseerde entiteit die voortdurend zijn invloed wil uitbreiden,
kunnen we hieruit de conclusie trekken dat de nationale defensie vaak
een dekmantel is voor andere belangen. Bovendien zijn er veel
alternatieven voor de traditionele staatgerichte aanpak van defensie. In
een libertarische samenleving zouden gemeenschappen zelf
verantwoordelijk zijn voor hun bescherming. Dit betekent dat burgers
actief betrokken zijn en samen beslissen hoe ze hun veiligheid willen
waarborgen. Daarbij komt ook de vraag van efficiëntie en effectiviteit.
Is een grote, bureaucratische defensiemacht wel de beste oplossing voor
de moderne dreigingen waar we mee te maken hebben? Veel experts zijn
ervan overtuigd dat kleinere, flexibele eenheden effectiever zijn dan
een massale militaire aanwezigheid. Uiteindelijk komt het erop neer dat
nationale defensie niet zomaar een kwestie is van troepen en wapens. Het
heeft te maken met de verantwoordelijkheden van de staat en de rol die
deze speelt in onze levens. Het is tijd om deze rol kritisch te bekijken
en te overwegen of er niet betere manieren zijn om onszelf en onze
waarden te verdedigen.

We komen nu bij wat vaak het laatste argument tegen het libertarische
standpunt is. Iedere libertariër heeft wel eens een sympathieke maar
kritische luisteraar horen zeggen: `Oké, ik zie hoe dit systeem
succesvol kan worden toegepast op lokale politie en rechtbanken. Maar
hoe zou een libertarische samenleving ons kunnen beschermen tegen de
Russen?'

Een dergelijke vraag is natuurlijk gebaseerd op een aantal betwiste
veronderstellingen. Er wordt bijvoorbeeld aangenomen dat de Russen uit
zijn op een militaire invasie van de Verenigde Staten, wat op zijn best
twijfelachtig is. Daarnaast is er de veronderstelling dat zo'n verlangen
zou voortpersisten nadat de Verenigde Staten een puur libertaire
samenleving is geworden. Dit idee negeert de les die de geschiedenis ons
leert: oorlogen ontstaan uit conflicten tussen natiestaten, die allemaal
zwaar bewapend zijn en elkaar wantrouwen. Echter, een libertair Amerika
zou voor niemand een bedreiging vormen. Niet omdat het geen wapens zou
hebben, maar omdat het geen agressie tegen anderen of tegen welk land
dan ook zou koesteren. Aangezien het niet langer een natiestaat is, die
inherent bedreigend is, zou de kans klein zijn dat een land ons aanvalt.
Een van de grote nadelen van de natiestaat is dat elke staat zijn
onderdanen als eigen mensen beschouwt. Hierdoor lijden onschuldige
burgers, de onderdanen van elk land, vaak onder de agressie van de
vijandelijke staat tijdens interstatelijke oorlogen. In een
libertarische samenleving zou er geen dergelijke identificatie zijn en
dus ook geen verwoestende oorlog. Neem als voorbeeld onze criminele
Metropolitan Police. Stel dat deze niet alleen agressie heeft gepleegd
tegen Amerikanen, maar ook tegen Mexicanen. Als Mexico een regering had,
zou deze goed weten dat de Amerikanen als geheel niet betrokken waren
bij de misdaden van de Metropolitan en er geen enkele band mee hadden.
Als de Mexicaanse autoriteiten een strafexpeditie zouden opzetten om de
Metropolitan te straffen, zouden ze niet in oorlog zijn met de
Amerikanen in het algemeen, zoals nu het geval zou zijn. Het is zelfs
zeer waarschijnlijk dat andere Amerikaanse troepen zich bij de Mexicanen
zouden voegen om de agressor te verslaan. Daarom lijkt het idee van een
oorlog tussen staten tegen een libertair land of gebied
hoogstwaarschijnlijk uit te doven.

Bovendien zit er een ernstige filosofische fout in het stellen van
vragen over de Russen. Als we een nieuw systeem overwegen, wat het ook
moge zijn, moeten we eerst beslissen of we dat willen. Of het nu
libertarisme, communisme, links anarchisme, of een theocratie is, we
moeten aannemen dat het al bestaat. Daarna kunnen we ons afvragen of het
systeem zou kunnen functioneren, of het levensvatbaar is en hoe
efficiënt het zou zijn. Ik denk dat we hebben aangetoond dat een
libertarisch systeem, eenmaal geïmplementeerd, kan werken. Het kan
levensvatbaar zijn en tegelijkertijd veel efficiënter, welvarender,
moreler en vrijer dan welk ander sociaal systeem dan ook. Maar we hebben
nog niets gezegd over hoe we van het huidige systeem naar het ideaal
kunnen komen. Dit zijn immers twee totaal verschillende vragen: wat ons
ideale doel is en de strategie om van het huidige systeem naar dat doel
te komen. De kwestie rond Rusland vermengt deze twee niveaus van
discussie. Het idee gaat niet uit van een libertarisme dat overal ter
wereld is gevestigd, maar van een systeem dat om de een of andere reden
alleen in Amerika bestaat en nergens anders. Maar waarom zouden we dat
aannemen? Waarom niet eerst veronderstellen dat het overal gevestigd is
en dan kijken of het werkt? De libertarische filosofie is immers een
tijdloze filosofie, niet gebonden aan tijd of plaats. Wij pleiten voor
vrijheid voor iedereen, overal, niet alleen in de Verenigde Staten. Als
iemand het ermee eens is dat een libertarische wereldgemeenschap,
eenmaal gevestigd, het beste is dat hij zich kan voorstellen---werkbaar,
efficiënt en moreel---laat hem dan libertariër worden. Laat hem samen
met ons vrijheid omarmen als ons ideale doel en samen met ons werken aan
de uitdaging---natuurlijk een moeilijke taak---om uit te vinden hoe we
dit ideaal kunnen bereiken.

Als we het toch over strategie hebben, dan is het duidelijk dat de
overlevingskansen groter zijn naarmate de vrijheid over een groter
gebied wordt gevestigd. Als vrijheid wereldwijd onmiddellijk wordt
uitgeroepen, zijn er natuurlijk geen problemen met `nationale
verdediging'. Alle problemen zouden dan lokale kwesties van politie
zijn. Maar als alleen Deep Falls, Wyoming, libertair wordt en de rest
van Amerika en de wereld hetzelfde blijft, dan zijn de overlevingskansen
zeer klein. Als Deep Falls zich afscheidt van de Verenigde Staten en een
vrije samenleving opricht, is de kans groot dat de VS, gezien hun
historische vijandigheid tegenover afscheidingsbewegingen, deze nieuwe
vrije samenleving snel zullen aanvallen en onderdrukken. Tussen deze
twee uitersten ligt een oneindig continuüm van gradaties. Het is
duidelijk dat hoe groter het gebied van vrijheid, hoe beter het bestand
is tegen externe bedreigingen. De `Russische kwestie' is daarom eerder
een strategische kwestie dan een keuze over basisprincipes. Het gaat om
het doel waar we onze inspanningen op willen richten.

Laten we, na alles wat we hebben besproken, toch de kwestie van de
Russen aankaarten. Stel je voor dat de Sovjet-Unie echt vastbesloten is
om een libertarische bevolking binnen de huidige grenzen van de
Verenigde Staten aan te vallen (waarbij er duidelijk geen Amerikaanse
regering meer zou zijn om als natiestaat op te treden). In de eerste
plaats zouden de Amerikaanse consumenten zelf bepalen hoe ze hun
defensie-uitgaven willen vormgeven en hoeveel ze daarin willen
investeren. Amerikanen die voorstanders zijn van Polaris-onderzeeërs en
bang zijn voor een Sovjetdreiging, zouden geld beschikbaar stellen voor
de financiering van dergelijke schepen. Degenen die de voorkeur geven
aan een ABM-systeem, zouden investeren in verdediging met raketten. Aan
de andere kant zouden degenen die om zo'n dreiging lachen of die
toegewijde pacifisten zijn, helemaal geen bijdrage leveren aan een
`nationale' verdediging. Verschillende verdedigingstheorieën zouden
worden toegepast, afhankelijk van degenen die deze ondersteunen en
erachter staan. Gezien de enorme verspilling in alle oorlogen en
defensievoorbereidingen door de geschiedenis heen, is het zeker niet
onredelijk om te stellen dat vrijwillige defensie-inspanningen veel
efficiënter zouden zijn dan overheidsbudgetten. Bovendien zouden deze
inspanningen ongetwijfeld een stuk moreler zijn.

Maar laten we van het ergste uitgaan. Stel je voor dat de Sovjet-Unie
Amerika binnenvalt en het land veroverd. Wat dan? We moeten ons
realiseren dat de moeilijkheden voor de Sovjet-Unie dan pas echt
beginnen. De belangrijkste reden waarom een veroverend land over een
verslagen land kan heersen, is dat dit laatste een bestaand
staatsapparaat heeft om de bevelen van de overwinnaar over te brengen
aan een ondergeschikte bevolking. Groot-Brittannië, dat veel kleiner was
in oppervlakte en bevolking, kon eeuwenlang over India heersen. Dit kwam
doordat de Britten hun bevelen konden doorgeven aan de heersende Indiase
prinsen, die ze op hun beurt oplegden aan de onderdanen. In de gevallen
in de geschiedenis waar de veroverden geen regering hadden, vonden de
veroveraars het extreem moeilijk om over hen te heersen. Zo hadden de
Britten bij de verovering van West-Afrika grote moeite om over de
Ibo-stam (de latere basis voor Biafra) te heersen. Deze stam was in
wezen libertair en had geen heersende regering met stamhoofden om
bevelen aan de inboorlingen door te geven. Een andere belangrijke reden
waarom het eeuwen duurde voordat de Engelsen het oude Ierland
veroverden, is dat de Ieren geen staat hadden, waardoor er geen
regerende structuur bestond om verdragen te sluiten of bevelen door te
geven. Om deze reden noemden de Engelsen de `wilde' en `onbeschaafde'
Ieren `ongelovig,' omdat ze zich niet aan de verdragen met de Engelse
veroveraars hielden. De Engelsen konden nooit begrijpen dat, zonder
enige vorm van staat, de Ierse krijgers die verdragen sloten met de
Engelsen, enkel voor zichzelf spraken. Ze konden nooit een andere groep
binnen de Ierse bevolking binden.

Bovendien zou het leven voor de Russische bezetters nog moeilijker
worden door de onvermijdelijke uitbraak van guerrillaoorlog door de
Amerikaanse bevolking. De twintigste eeuw heeft ons immers geleerd - een
les die voor het eerst duidelijk werd toen de succesvolle Amerikaanse
revolutionairen het krachtige Britse Rijk bestreden - dat geen enkele
bezettingsmacht een vastberaden inheemse bevolking lang kan
onderdrukken. Als de enorme Verenigde Staten, met zijn grotere
productiviteit en vuurkracht, niet kon overwinnen tegen een kleine,
relatief ongewapende Vietnamese bevolking, hoe zou de Sovjetunie dan
ooit in staat zijn om het Amerikaanse volk te bedwingen? Geen enkele
Russische bezettingssoldaat zou veilig zijn voor de woede van een verzet
biedend Amerikaans volk. Guerrillaoorlogvoering heeft bewezen een
onweerstaanbare kracht te zijn. Dit komt omdat het voortkomt uit het
volk zelf, dat vecht voor zijn vrijheid en onafhankelijkheid tegen een
buitenlandse overheerser. De gedachte aan deze enorme problemen en de
onvermijdelijke kosten en verliezen die hieruit voort zouden vloeien,
zou zelfs een hypothetische Sovjetregering die uit was op militaire
verovering, vanuit het begin al tegenhouden.

\bookmarksetup{startatroot}

\chapter{Natuurbehoud, ecologie en
groei}\label{natuurbehoud-ecologie-en-groei}

\section{LIBERALE KLACHTEN}\label{liberale-klachten}

Links-liberale intellectuelen vormen vaak een opmerkelijke groep. In de
afgelopen drie of vier decennia, een relatief korte periode in de
geschiedenis van de mensheid, hebben ze als wervelende derwisjen een
reeks felle klachten tegen het vrijemarktkapitalisme geuit. Wat opvalt,
is dat elke nieuwe klacht vaak in tegenspraak is met een of meerdere
eerdere beschouwingen. Tegenstrijdige uitingen schijnen hen echter niet
te verontrusten of hun vasthoudendheid te verminderen; vaak zijn het
dezelfde intellectuelen die zich razendsnel van mening kunnen
veranderen. Deze ommezwaai lijkt geen enkele afbreuk te doen aan hun
zelfingenomenheid of het zelfvertrouwen in hun standpunt.

\begin{quote}
Laten we de resultaten van de afgelopen decennia eens onder de loep
nemen:
\end{quote}

Eind jaren dertig en begin jaren veertig concludeerden liberale
intellectuelen dat het kapitalisme leed aan een onvermijdelijke
`seculiere stagnatie'. Deze stagnatie zou veroorzaakt worden door een
afname van de bevolkingsgroei, het verdwijnen van de oude westelijke
grenzen en de veronderstelling dat er geen nieuwe uitvindingen meer
mogelijk waren. Dit alles leidde tot de verwachting van eeuwige
stagnatie, permanente massawerkloosheid en de noodzaak van socialisme,
oftewel grondige staatsplanning, ter vervanging van het
vrijemarktkapitalisme. En dat op de vooravond van de grootste groei in
de Amerikaanse geschiedenis!

In de jaren vijftig, ondanks de grote naoorlogse boom in Amerika, bleven
de liberale intellectuelen hun visie verbreden. De cultus van de
`economische groei' kwam toen op. Natuurlijk groeide het kapitalisme,
maar niet snel genoeg. Hierdoor moest het vrijemarktkapitalisme worden
losgelaten. Tijd voor socialisme of overheidsinterventie om de economie
aan te stoten, investeringen te bevorderen en hogere besparingen af te
dwingen, zodat we de groeisnelheid konden maximaliseren, ook al was er
geen dringende behoefte aan snelle groei. Conservatieve economen zoals
Colin Clark bekritiseerden dit liberale programma en noemden het
`groeimanisme'.

Plotseling kwam John Kenneth Galbraith in 1958 op de voorgrond met zijn
bestseller \emph{The Affluent Society}. Even onverwacht keerden de
liberale intellectuelen hun kritiek om. Het probleem met het
kapitalisme, zo bleek, was niet langer stagnatie, maar juist dat we te
welvarend waren geworden. De mens had zijn spiritualiteit verloren te
midden van supermarkten en autostartvinnen. Wat nodig was, was ingrijpen
van de overheid, hetzij door massale interventie, hetzij door
socialisme, en een zware belasting van de consumenten om hun opgeblazen
welvaart te verminderen.

De cultus van overmatige welvaart leek zijn langste tijd gehad te hebben
en werd vervangen door een tegenstrijdige zorg om armoede, aangewakkerd
door Michael Harringtons \emph{The Other America} in 1962. Plotseling
was het probleem in Amerika niet langer buitensporige welvaart, maar
toenemende en schrijnende armoede. En opnieuw bleek de oplossing dat de
regering moest ingrijpen, grondig moest plannen en de rijken moest
belasten om de armen te ondersteunen. Zo kregen we een aantal jaren te
maken met de Oorlog tegen Armoede.

Stagnatie, beperkte groei, overvloed en extreme armoede: de
intellectuele mode veranderde sneller dan de zoomlijnen van dames. In
1964 publiceerde het kort levende Ad Hoc Comité voor de Drievoudige
Revolutie zijn beroemde manifest, dat de cirkel voor ons en de liberale
denkers rondmaakte. Twee tot drie jaar lang waren we opgelucht met de
gedachte dat het probleem in Amerika niet stagnatie was, maar juist het
tegenovergestelde. Binnen een paar jaar zouden alle
productiefaciliteiten in Amerika geautomatiseerd en cybernetisch zijn.
De inkomens en productie zouden enorm en overvloedig zijn, maar iedereen
zou zonder werk komen te zitten. Wederom zou het vrijemarktkapitalisme
leiden tot permanente massawerkloosheid, die alleen verholpen kon worden
- je raadt het al - door ingrijpen van de staat of door socialisme. En
zo leden we enkele jaren, halverwege de jaren zestig, aan wat terecht de
`Automatiseringshysterie' werd genoemd.

Tegen het einde van de jaren zestig was voor iedereen duidelijk dat de
automatiseringshysterie volledig misplaatst was. Automatisering ging
namelijk niet sneller dan de ouderwetse `mechanisatie', en de recessie
van 1969 leidde zelfs tot een daling van de productiviteitsgroei.
Tegenwoordig hoor je nergens meer over de gevaren van automatisering; we
bevinden ons nu in de zevende fase van liberale economische flip-flops.

De welvaart is opnieuw buitensporig. In naam van behoud, ecologie en de
toenemende schaarste aan grondstoffen groeit het vrijemarktkapitalisme
veel te snel. Staatsplanning of socialisme moet ingrijpen om al deze
groei te beëindigen en een nulgroei-samenleving en -economie op te
richten. Dit zou negatieve groei of achteruitgang in de toekomst moeten
voorkomen! We zijn nu terug bij een super-Galbraithiaanse stelling,
waarin wetenschappelijk jargon over afvalwater, ecologie en `Spaceship
Earth' is verwerkt. Daarnaast is er een felle aanval op technologie, die
wordt gezien als een grote bron van vervuiling. Het kapitalisme heeft
geleid tot technologie, groei -- inclusief bevolkingsgroei, industriële
ontwikkeling en vervuiling -- en de overheid wordt verondersteld in te
grijpen en deze problemen op te lossen.

Het is helemaal niet ongewoon om dezelfde mensen te treffen die nu met
tegenstrijdige standpunten 5 en 7 pronken. Ze beweren tegelijkertijd dat
(a) we in een `post-schaarste' tijdperk leven, waarin we geen
privé-eigendom, kapitalisme of materiële prikkels voor productie meer
nodig hebben, en (b) dat kapitalistische hebzucht onze grondstoffen
uitput en een dreigende wereldwijde schaarste veroorzaakt. Het liberale
antwoord op beide, of eigenlijk op alle, problemen blijkt natuurlijk
steeds hetzelfde te zijn: socialisme of staatsplanning ter vervanging
van het vrijemarktkapitalisme. De grote econoom Joseph Schumpeter vatte
het hele rommelige optreden van liberale intellectuelen een generatie
geleden treffend samen:

\begin{quote}
Het kapitalisme staat terecht voor rechters die het doodvonnis in handen
hebben. Ze zullen hun uitspraak doen, ongeacht welk verweer er ook wordt
ingebracht. Het enige wat een succesvolle verdediging zou kunnen
bereiken, is een wijziging van de aanklacht.
\end{quote}

En dus kunnen de aanklachten en beschuldigingen veranderen en elkaar
tegenspreken, maar het antwoord blijft altijd moeizaam hetzelfde.

\section{\#\#\# DE AANVAL OP TECHNOLOGIE EN
GROEI}\label{de-aanval-op-technologie-en-groei}

Het is helemaal niet ongewoon om mensen tegen te komen die een
tegenstrijdige mix van standpunten 5 en 7 aanhangen. Ze beweren tegelijk
dat (a) we in een `post-schaarste'-tijdperk leven, waarin we geen
privé-eigendom, kapitalisme of materiële prikkels voor productie meer
nodig hebben, en (b) dat de hebzucht van het kapitalisme onze
grondstoffen uitput en leidt tot een dreigende wereldwijde schaarste.
Het liberale antwoord op beide, of eigenlijk op alle, problemen is
steevast hetzelfde: socialisme of staatsplanning in de plaats van
vrijemarktkapitalisme. Joseph Schumpeter, een grote econoom, vatte het
rommelige gedrag van liberale intellectuelen een generatie geleden
treffend samen: `Het kapitalisme staat terecht voor rechters die het
doodvonnis in handen hebben. Ze zullen hun uitspraak doen, ongeacht welk
verweer er ook wordt ingebracht: het enige wat een succesvolle
verdediging zou kunnen opleveren, is een wijziging van de aanklacht.' Zo
kunnen de aanklachten en beschuldigingen weliswaar veranderen en elkaar
tegenspreken, maar het antwoord blijft altijd moeizaam hetzelfde.

De modieuze aanval op groei en welvaart komt vooral van welgestelde,
tevreden liberalen uit de hogere klasse. Aangezien zij genieten van
materiële genoegens en een levensstandaard waarvan zelfs de rijksten in
het verleden niet konden dromen, is het voor deze liberalen gemakkelijk
om te spotten met `materialisme' en te pleiten voor het bevriezen van
verdere economische vooruitgang. Voor de miljoenen mensen die nog steeds
in ellende leven, is zo'n oproep om de groei te stoppen echt beledigend.
Zelfs in de Verenigde Staten is er weinig bewijs van verzadiging en
overvloed. Ook de liberalen uit de hogere klasse hebben niet veel van
hun salaris afgedragen als bijdrage aan hun strijd tegen `materialisme'
en welvaart.

De wijdverspreide aanval op technologie is nog onverantwoordelijker. Als
technologie werd teruggedraaid naar het niveau van de `stam' en het
preïndustriële tijdperk, zouden massale hongersnoden en sterfte op grote
schaal het resultaat zijn. De overgrote meerderheid van de
wereldbevolking is voor haar voortbestaan afhankelijk van moderne
technologie en industrie. In de tijd vóór Columbus kon het
Noord-Amerikaanse continent ongeveer een miljoen Indianen herbergen,
allemaal op een bestaansminimum. Tegenwoordig is het in staat om
honderden miljoenen mensen te huisvesten, die allemaal op een veel hoger
niveau leven. De reden hiervoor is de moderne technologie en industrie.
Als we die zouden afschaffen, zouden we ook de mensen afschaffen.
Misschien is deze `oplossing' voor het bevolkingsvraagstuk aantrekkelijk
voor fanatieke antipopulatie-aanhangers, maar voor de meerderheid van
ons zou dit een draconische `definitieve oplossing' zijn.

De onverantwoordelijke aanval op technologie is weer zo'n voorbeeld van
de liberale dubbelzinnigheid. Deze kritiek komt van dezelfde
intellectuelen die dertig jaar geleden het kapitalisme bekritiseren
omdat het moderne technologie niet volledig inzet voor staatsplanning en
pleitte voor absolute controle door een moderne `technocratische' elite.
Nu proberen dezelfde intellectuelen, die nog niet zo lang geleden
verlangden naar een technocratische dictatuur over elk aspect van ons
leven, ons te beroven van de essentiële vruchten van de technologie
zelf.

Toch sterven de verschillende tegenstrijdige stromingen binnen het
liberalisme nooit volledig uit. Veel van dezelfde antitechnologen voeren
nu, in een volledig tegengestelde reactie op de automatiseringshysterie,
ook vol vertrouwen aan dat we binnenkort te maken krijgen met
technologische stagnatie. Ze schetsen een sombere toekomst voor de
mensheid, omdat ze ervan uitgaan dat technologie niet zal blijven
verbeteren en versnellen. Dit is kenmerkend voor de
pseudowetenschappelijke voorspellingen die in het veelbesproken
antigroei rapport van de Club van Rome zijn gedaan. Zoals Passell,
Roberts en Ross opmerken in hun kritiek op het rapport: `Als de
telefoonmaatschappij zich zou moeten beperken tot de technologie van
rond de eeuwwisseling, dan zouden er 20 miljoen operatoren nodig zijn om
de hoeveelheid telefoontjes van vandaag te verwerken.' Britse redacteur
Norman Macrae voegde hieraan toe: `Een extrapolatie van de trends uit de
jaren 1880 zou ons de steden van vandaag verbergen onder paardenmest.'

\begin{quote}
Terwijl het model van het team {[}de Club van Rome{]} uitgaat van
exponentiële groei voor industriële en agrarische behoeften, stelt het
arbitraire, niet-exponentiële grenzen aan de technische vooruitgang die
deze behoeften zou kunnen vervullen.

Thomas Malthus maakte twee eeuwen geleden een soortgelijk punt, zonder
de voordelen van computeranalyses.

Malthus stelde dat mensen de neiging hebben zich exponentieel te
vermenigvuldigen, terwijl de voedselvoorraad in het beste geval slechts
gelijkblijvend toeneemt. Hij verwachtte dat honger en oorlog periodiek
de balans zouden herstellen. . . .

Maar er is geen specifiek criterium, buiten bijziendheid, waarop die
speculatie gebaseerd kan worden. Malthus had ongelijk; de
voedselcapaciteit heeft gelijke tred gehouden met de groei van de
bevolking. Hoewel niemand het zeker weet, vertoont de technische
vooruitgang geen tekenen van vertraging. De beste econometrische
schattingen wijzen erop dat deze daadwerkelijk exponentieel groeit.
\end{quote}

Wat we nodig hebben, is meer economische groei, niet minder; meer en
betere technologie, en niet de onmogelijke en absurde poging om
technologie af te schaffen en terug te keren naar een primitieve
levenswijze. Verbeterde technologie en grotere investeringen in kapitaal
zullen zorgen voor een hogere levensstandaard voor iedereen. Daarbij
bieden ze meer materieel comfort en de vrije tijd om de `spirituele'
kant van het leven na te streven en ervan te genieten. Er is bitter
weinig ruimte voor cultuur of beschaving voor mensen die lange dagen
moeten werken om in hun levensonderhoud te voorzien. Het echte probleem
is dat productieve investeringen worden weggeleid door belastingen,
beperkingen en overheidscontracten voor onproductieve en verkwistende
overheidsuitgaven, waaronder militaire en ruimtevaartprojecten.
Bovendien wordt de kostbare technische hulp van wetenschappers en
ingenieurs steeds vaker ingezet voor de overheid, in plaats van voor
`civiele' consumentenproductie. Wat we nodig hebben, is dat de overheid
zich terugtrekt, haar belastingdruk en uitgaven uit de economie
verwijdert en de productieve en technische middelen weer volledig inzet
voor het verbeteren van het welzijn van de massa consumenten. We hebben
behoefte aan groei, een hogere levensstandaard en technologie die
aansluit bij de wensen en eisen van de consument. Dit kunnen we alleen
bereiken door de verstorende invloed van het statisme te verwijderen en
de energie van de gehele bevolking de ruimte te geven om zich te uiten
in de vrije markteconomie. Kortom, we hebben een economische en
technologische groei nodig die vrij voortkomt uit de vrije
markteconomie, zoals Jane Jacobs heeft aangetoond, en niet uit de
verstoringen en verspilling die de wereldeconomie zijn opgelegd door de
dwingende maatregelen van de jaren vijftig. We hebben, met andere
woorden, een echte vrije markteconomie nodig.

\section{BEHOUD VAN HULPBRONNEN}\label{behoud-van-hulpbronnen}

Zoals we al hebben opgemerkt, zijn het dezelfde liberalen die beweren
dat we het `postschaarste'-tijdperk zijn binnengetreden en geen verdere
economische groei nodig hebben, die vooraan staan in de klacht over
`kapitalistische hebzucht' die onze schaarse natuurlijke hulpbronnen
vernietigt. De pessimisten van de Club van Rome, bijvoorbeeld, stellen,
door simpelweg de huidige trends in het gebruik van grondstoffen door te
trekken, vol vertrouwen dat essentiële grondstoffen binnen 40 jaar
uitgeput zullen zijn. Maar zelfverzekerde en volledig misplaatste
voorspellingen van grondstofuitputting zijn de afgelopen eeuwen al
talloze keren gedaan.

Wat de waarzeggers vaak over het hoofd zien, is de belangrijke rol die
het mechanisme van de vrije markteconomie speelt in het behouden en
aanvullen van natuurlijke hulpbronnen. Laten we bijvoorbeeld eens kijken
naar een typische kopermijn. Waarom is het kopererts niet al uitgeput
door de onverbiddelijke eisen van onze industriële samenleving? Waarom
ontginnen kopermijnwerkers, zodra ze een ertsader hebben gevonden, niet
meteen al het koper? Waarom kiezen ze ervoor om de kopermijn te
conserveren, deze aan te vullen en geleidelijk elk jaar koper te
onttrekken? Omdat de mijn eigenaren zich realiseren dat als ze de
koperproductie van dit jaar verdrievoudigen, ze weliswaar het inkomen
van dit jaar kunnen verhogen, maar ook de mijn uitputten. Dit betekent
dat hun toekomstig inkomen uit de mijn zal afnemen. Op de markt wordt
dit verlies aan toekomstige inkomsten direct weerspiegeld in de
monetaire waarde -- de prijs -- van de mijn als geheel. Deze waarde, die
zichtbaar is in de verkoopprijs van de mijn en vervolgens van
individuele aandelen in mijnbouwbedrijven, is gebaseerd op het verwachte
toekomstig inkomen dat met de koperproductie wordt verdiend. Elke
uitputting van de mijn zal dus de waarde van de mijn verlagen, en
daarmee ook de prijs van de aandelen. Elke mijn eigenaar moet daarom de
voordelen van onmiddellijke inkomsten uit koperproductie afwegen tegen
het verlies van de `kapitaalwaarde' van de mijn als geheel, en dus tegen
het verlies van de waarde van zijn aandelen.

De beslissingen van mijneigenaren worden bepaald door hun verwachtingen
over toekomstige koperopbrengsten en de vraag, evenals door de bestaande
en verwachte rentepercentages. Stel bijvoorbeeld dat men verwacht dat
koper binnen een paar jaar verouderd zal zijn door een nieuw synthetisch
metaal. In dat geval zullen kopermijneigenaren zich haasten om nu meer
koper te produceren, terwijl het nog waardevol is. Ze zullen minder
koper opslaan voor de toekomst, wanneer het waarschijnlijk minder waarde
zal hebben. Dit komt de consumenten en de economie ten goede, omdat er
nu meer koper beschikbaar is, precies op het moment dat het harder nodig
is. Aan de andere kant, als men verwacht dat er in de toekomst een
kopertekort zal ontstaan, dan zullen mijneigenaren nu minder produceren.
Ze zullen wachten met de productie totdat de prijzen hoger zijn, zodat
ze later meer koper kunnen leveren. Dit is weer voordelig voor de
maatschappij, omdat er in de toekomst meer geproduceerd kan worden,
wanneer de vraag naar koper stijgt. We zien dus dat de markteconomie een
uitstekend ingebouwd mechanisme heeft. De beslissingen van
grondstofeigenaren om nu of later te produceren komen niet alleen hun
eigen inkomen en rijkdom ten goede, maar ook de massa consumenten en de
economie als geheel.

Maar er zit veel meer achter dit vrijemarkmechanisme. Stel je voor dat
er een groeiend tekort aan koper wordt verwacht in de toekomst. Dit zou
ertoe leiden dat er nu meer koper wordt achtergehouden en opgeslagen
voor toekomstige productie. Daarom zal de prijs van koper stijgen. Deze
prijsstijging heeft verschillende `conserverende' effecten. Ten eerste
is de hogere koperprijs een signaal voor de gebruikers dat het metaal
schaarser en duurder wordt. Hierdoor zullen kopergebruikers zuiniger
omgaan met dit duurdere materiaal. Ze zullen minder koper gebruiken en
dit zoveel mogelijk vervangen door goedkopere metalen of kunststoffen.
Koper zal beter worden bewaard voor toepassingen waarvoor geen
bevredigend alternatief bestaat. Bovendien zullen de hogere kosten van
koper\ldots{}

\begin{enumerate}
\def\labelenumi{(\alph{enumi})}
\tightlist
\item
  een stormloop op het vinden van nieuwe koperertsen teweegbrengen; en
  (b) een zoektocht naar goedkopere substituten aanmoedigen, mogelijk
  door nieuwe technologische doorbraken. Hogere koperprijzen stimuleren
  bovendien initiatieven om het metaal te besparen en te recyclen. Dit
  prijsmechanisme binnen de vrije markt is precies de oorzaak dat koper
  en andere natuurlijke hulpbronnen nog lang niet zijn verdwenen. Zoals
  Passell, Roberts en Ross opmerken in hun kritiek op de Club van Rome:
\end{enumerate}

\begin{quote}
De reserves en behoeften aan natuurlijke hulpbronnen in het model worden
berekend zonder prijzen als variabele in de projectie van hoe
hulpbronnen zullen worden gebruikt. In de echte wereld fungeren
stijgende prijzen echter als een economisch signaal om schaarse
hulpbronnen te besparen. Ze bieden stimulansen om goedkopere materialen
te gebruiken, stimuleren onderzoeksinspanningen naar nieuwe manieren om
hulpbronnen te besparen en maken nieuwe exploitatiepogingen
winstgevender.
\end{quote}

In tegenstelling tot de pessimisten zijn de prijzen van grondstoffen en
natuurlijke hulpbronnen laag gebleven en zelfs over het algemeen gedaald
in vergelijking met andere prijzen. Voor liberale en marxistische
intellectuelen is dit vaak een teken van kapitalistische `uitbuiting'
van landen die meestal de producenten van deze grondstoffen zijn. Maar
het betekent iets heel anders: dat natuurlijke hulpbronnen niet
schaarser, maar juist overvloediger zijn geworden, en dat verklaart hun
relatief lage kosten. De ontwikkeling van goedkope alternatieven, zoals
plastic en synthetische vezels, heeft ervoor gezorgd dat natuurlijke
hulpbronnen goedkoop en overvloedig zijn gebleven. In de komende
decennia kunnen we verwachten dat moderne technologie een opmerkelijk
goedkope energiebron - kernfusie - zal ontwikkelen. Dit zou automatisch
resulteren in een aanzienlijke overvloed aan grondstoffen voor het werk
dat daarvoor nodig is.

De ontwikkeling van synthetische materialen en goedkopere energie
benadrukt een belangrijke dimensie van moderne technologie die
doemdenkers vaak over het hoofd zien: technologie en industriële
productie creëren hulpbronnen die eerder niet als waardevol werden
beschouwd. Voorheen was aardolie, voordat de kerosinelamp en vooral de
auto werden uitgevonden, geen grondstof, maar eerder ongewenst afval,
een soort gigantisch vloeibaar zwart `onkruid'. Pas door de moderne
industrie werd aardolie een waardevolle grondstof. Bovendien heeft
moderne technologie, dankzij verbeterde geologische technieken en de
prikkels van de markt, in hoog tempo nieuwe aardoliereserves ontdekt.

Voorspellingen over de dreigende uitputting van natuurlijke hulpbronnen
zijn, zoals we hebben gezien, niets nieuws. In 1908 waarschuwde
president Theodore Roosevelt op een conferentie van gouverneurs voor de
`dreigende uitputting' van deze bronnen. Ook staalindustrieel Andrew
Carnegie voorspelde op diezelfde conferentie dat het ijzerertsreservoir
van Lake Superior in 1940 uitgeput zou zijn. Spoorwegmagnaat James J.
Hill voegde daar aan toe dat een groot deel van onze houtreserves binnen
tien jaar op zou raken. En dat is nog niet alles: Hill voorspelde zelfs
een dreigend tekort aan tarweproductie in de Verenigde Staten, terwijl
we in dat land nog steeds worstelen met de tarweoverschotten die ons
landbouwsubsidieprogramma genereert. De huidige onheilsprognoses zijn op
dezelfde grondslagen gebaseerd: een ernstige onderschatting van de
mogelijkheden van moderne technologie en een gebrek aan inzicht in de
werking van de markteconomie.8

Het klopt dat sommige specifieke natuurlijke hulpbronnen in het verleden
en ook nu te maken hebben gehad met uitputting. Maar in alle gevallen
lag de oorzaak niet in `kapitalistische hebzucht'; integendeel, het
probleem was het falen van de overheid om privé-eigendom van de hulpbron
mogelijk te maken. Met andere woorden, er was geen goede toepassing van
het principe van privé-eigendomsrechten.

Een voorbeeld hiervan zijn de houtvoorraden. In het Amerikaanse Westen
en in Canada zijn de meeste bossen in handen van de federale of
provinciale overheid. Deze overheid verlaat het gebruik van de bossen
aan particuliere houtbedrijven. Dit betekent dat privébezit alleen geldt
voor het jaarlijkse gebruik van de grondstof, maar niet voor het bos
zelf. In deze situatie heeft het particuliere houtbedrijf geen eigendom
over de kapitaalwaarde en hoeft het zich dus geen zorgen te maken over
de uitputting van de bron. Het houtbedrijf heeft geen economische
prikkel om de bron te behouden of om bomen te herplanten. De enige
motivatie is om zo snel mogelijk zoveel mogelijk bomen te kappen, omdat
het behouden van de kapitaalwaarde van het bos voor hen geen economische
waarde heeft. In Europa, waar privébezit van bossen veel gebruikelijker
is, zijn er weinig klachten over de vernietiging van houtbronnen.
Wanneer privébezit in het bos zelf is toegestaan, is het in het belang
van de eigenaar om de boomgroei te behouden en te herstellen terwijl hij
hout kapt. Zo kan hij uitputting van de kapitaalwaarde van het bos
voorkomen.9

In de Verenigde Staten is de Forest Service van het Amerikaanse
Ministerie van Landbouw een belangrijke boosdoener. Deze instantie bezit
bossen en verhuurt jaarlijkse kaprechten, wat leidt tot verwoesting van
de bomen. Particuliere bossen, zoals die van grote houtbedrijven zoals
Georgia-Pacific en U.S. Plywood, worden daarentegen op een
wetenschappelijke manier beheerd. Deze bedrijven kappen en herbebossen
hun bomen om hun toekomstige voorraad op peil te houden.10

Een ander ongelukkig gevolg van het falen van de Amerikaanse overheid om
privébezit toe te staan, was de vernietiging van de westelijke
graslanden aan het eind van de negentiende eeuw. Elke kijker van
westernfilms is bekend met de mystiek van de `open range' en de vaak
gewelddadige conflicten tussen rundveehouders, schapenhoeders en boeren
over percelen ranchgrond. De `open range' was het gevolg van het falen
van de federale overheid om het beleid van homesteading aan te passen
aan de veranderde omstandigheden van het drogere klimaat ten westen van
de Mississippi. In het oosten bood de 160 acres (ongeveer 65 hectare),
die gratis aan boeren op overheidsgrond werden toegewezen, een
levensvatbare basis voor landbouw in een natter klimaat. In het droge
westen daarentegen kon je met slechts 160 acres geen succesvolle
veeteelt of schapenfokkerij opzetten. Desondanks weigerde de federale
overheid om deze eenheid te vergroten, zodat er grotere veeboerderijen
konden ontstaan. Dit leidde tot de `open landerijen', waar particuliere
vee- en schapeneigenaren vrijelijk hun dieren lieten grazen op
overheidsgrond. Het probleem was dat niemand eigenaar was van de
weidegrond zelf, wat betekende dat het economisch voordelig was voor
elke rundvee- of schapeneigenaar om het land zo snel mogelijk te
begrazen. Anders zou een ander zijn dieren op datzelfde grasland laten
grazen. Het resultaat van deze kortzichtige weigering om privébezit van
graasland toe te staan, was overbegrazing. Dit leidde tot schade aan het
grasland, omdat er te vroeg in het seizoen werd gegraasd. Bovendien was
er niemand die het gras kon herstellen of herplanten. Degenen die dat
probeerden, moesten machteloos toezien hoe anderen hun vee of schapen
lieten grazen. Dit resulteerde in de overbegrazing van het westen en de
vorming van de `dust bowl'. Ook leidde het tot de illegale pogingen van
talloze veehouders, boeren en schapenherders om het recht in eigen hand
te nemen en het land om te heinen tot privé-eigendom, wat vaak
resulteerde in de `Ranch Wars'.

Professor Samuel P. Hays schrijft in zijn gezaghebbende verslag over de
natuurbeschermingsbeweging in Amerika over het probleem van het
verspreidingsgebied:

\begin{quote}
Een groot deel van de vee-industrie in het westen was voor haar
voortbestaan afhankelijk van de `open' weidegronden die in handen waren
van de federale overheid, maar die door iedereen vrij konden worden
gebruikt. Het Congres had nooit wetgeving vastgesteld om de begrazing te
reguleren of om veehouders de mogelijkheid te geven om weidegronden aan
te kopen. Vee- en schapenhouders zwierven door het publieke domein.
Sommige veehouders omheinden weilanden voor exclusief gebruik, maar
concurrenten knipten de draden door. Hierdoor ontstonden er gewelddadige
conflicten; schaapherders en cowboys `losten' hun geschillen over
weidegronden op door rivaliserende veestapels af te slachten en
tegenstanders te vermoorden. Het gebrek aan de meest elementaire regels
omtrent eigendom leidde tot verwarring, verbittering en vernietiging.

Te midden van deze onrust verslechterde het aanbod van openbaar grasland
snel. Oorspronkelijk was het voer overvloedig en weelderig, maar door
het toenemende gebruik kwam het onder zware druk te staan. Het openbare
domein werd bevolkt met meer dieren dan het land kon dragen. Elke
veehouder vreesde dat anderen hem voor zouden zijn met het beschikbare
ruwvoer. Hierdoor begon hij vroeg in het jaar te grazen, waardoor het
jonge gras niet de kans kreeg om te rijpen en opnieuw aan te groeien.
Onder deze omstandigheden namen zowel de kwaliteit als de kwantiteit van
het beschikbare voer snel af. Sterke vaste planten maakten plaats voor
eenjarige planten, en die weer voor onkruid.
\end{quote}

Hays concludeert dat de landerijen die tot het publieke domein behoren
door dit proces meer dan twee derde van hun oorspronkelijke omvang
hebben verloren.

Er is een cruciaal gebied waar het ontbreken van privébezit van
hulpbronnen niet alleen heeft geleid tot uitputting, maar ook tot het
falen van de ontwikkeling van enorme potentiële hulpbronnen. Dat betreft
de zeer productieve hulpbronnen van de oceanen. De oceanen maken deel
uit van het internationale publieke domein. Dit betekent dat geen enkele
persoon, bedrijf of nationale overheid eigendomsrechten heeft op delen
van de oceaan. Hierdoor zijn de oceanen in een primitieve staat
gebleven, vergelijkbaar met hoe het land eruitzag in de pre-beschaafde
tijd, vóór de ontwikkeling van de landbouw. De primitieve mens beschikte
over een productiewijze van `jagen en verzamelen': hij jaagde op wilde
dieren en verzamelde fruit, bessen, noten, wilde zaden en groenten. Deze
mensen werkten passief binnen hun omgeving en veranderden deze niet. Ze
leefden van het land zonder pogingen om het te hervormen. Dit
resulteerde in onproductieve grond, waardoor slechts een relatief klein
aantal stamleden op een sober niveau kon overleven. Pas met de
ontwikkeling van de landbouw, waarbij de grond werd bewerkt en het
landschap werd getransformeerd, kon de productiviteit en levensstandaard
enorm verbeteren. Alleen door de landbouw kon de beschaving zich
ontwikkelen. Om echter de landbouw mogelijk te maken, waren particuliere
eigendomsrechten nodig, in eerste instantie op de velden en gewassen, en
later op het land zelf.

Met betrekking tot de oceaan bevinden we ons echter nog steeds in een
primitief, onproductief stadium van jagen en verzamelen. Iedereen kan
vis vangen of andere hulpbronnen uit de oceaan halen, maar dat gebeurt
steeds op de vlucht, als jagers en verzamelaars. Niemand kan de oceaan
bewerken of zich bezighouden met aquacultuur. Hierdoor worden we beroofd
van het gebruik van de enorme vis- en mineraalbronnen van de zeeën. Als
iemand bijvoorbeeld zou proberen de zee te bewerken en de productiviteit
van de visserij met kunstmest te verhogen, zou hij direct het
slachtoffer worden van vissers die op hem afstormen en zijn vis zouden
grijpen. Daarom probeert niemand de oceanen te bemesten zoals we dat met
het land doen. Bovendien ontbreekt het aan economische prikkels -
sterker nog, er is geen enkele aanleiding voor iemand om technologisch
onderzoek te doen naar manieren om de productiviteit van de visserij te
verbeteren of om de mineralen in de oceanen te ontginnen. Dergelijke
stimulansen ontstaan alleen als eigendomsrechten op delen van de oceaan
net zo volledig worden toegestaan als eigendomsrechten op land.
Eigenlijk is er al een eenvoudige maar effectieve techniek die de
visproductiviteit zou kunnen verhogen: delen van de oceaan zouden
elektronisch omheind kunnen worden. Met zo'n elektronisch hekwerk zouden
vissen op grootte gescheiden kunnen worden. Door te voorkomen dat grote
vissen kleine vissen opeten, kan de visproductie aanzienlijk toenemen.
Als privébezit in delen van de oceaan zou worden toegestaan, zou er een
enorme bloei van aquacultuur ontstaan, waardoor we oceanen op talloze
manieren zouden kunnen benutten die we nu nog niet kunnen voorzien.

Nationale regeringen hebben tevergeefs geprobeerd het probleem van
visuitputting aan te pakken. Ze leggen vaak irrationele en oneconomische
beperkingen op aan de totale vangst of aan de duur van het seizoen. Bij
zalm, tonijn en heilbot blijven technologische vismethoden daardoor
primitief en onproductief. De seizoenen worden onnodig verkort, wat de
kwaliteit van de vangst schaadt. Bovendien stimuleren deze beperkingen
overproductie en onderbenutting van de vissersvloten gedurende het jaar.
Uiteraard dragen dergelijke overheidsmaatregelen niet bij aan de groei
van aquacultuur. Zoals professoren North en Miller schrijven:

\begin{quote}
Vissers leven in armoede omdat ze gedwongen worden om inefficiënte
apparatuur te gebruiken en slechts een beperkt deel van de tijd te
vissen door de overheidsregels. Bovendien zijn er te veel vissers
actief. De consument betaalt hierdoor veel meer voor rode zalm dan nodig
zou zijn als er efficiënte methoden werden toegepast. Ondanks de
toenemende regelgeving is het behoud van de zalmstand nog steeds niet
verzekerd.

De wortel van het probleem ligt in de huidige regeling omtrent eigendom.
Het is voor een individuele visser niet in zijn belang om zich bezig te
houden met het voortbestaan van de zalmstand. Sterker nog, het is juist
in zijn voordeel om tijdens het seizoen zoveel mogelijk vis te vangen.
\end{quote}

North en Miller wijzen er echter op dat privé-eigendomsrechten in de
oceaan nu haalbaarder zijn dan ooit. Eigenaars kunnen de goedkoopste en
meest efficiënte technologie gebruiken om de natuurlijke hulpbron te
behouden en productief te maken. `De uitvinding van moderne
elektronische meetapparatuur heeft het toezicht op grote watermassa's
relatief goedkoop en eenvoudig gemaakt.'

De groeiende internationale conflicten over delen van de oceaan
onderstrepen het belang van privé-eigendomsrechten in dit vitale gebied.
Nu de Verenigde Staten en andere landen hun soevereiniteit tot 200 mijl
van hun kusten claimen, en particuliere bedrijven en regeringen met
elkaar in strijd zijn over delen van de oceaan, worden eigendomsrechten
steeds crucialer. Trawlers, visnetten, olieboorders en mineraalgravers
strijden allemaal om dezelfde gebieden, wat de situatie verder
compliceert. Zoals Francis Christy schrijft:

\begin{quote}
Steenkool wordt gewonnen uit schachten onder de zeebodem. Olie wordt
aangeboord vanaf platforms die aan de bodem zijn bevestigd en boven het
water uitsteken. Mineralen kunnen van het oppervlak van de oceaanbodem
worden gebaggerd\ldots{}

Sedentaire dieren worden van de bodem geschraapt, waar telefoonkabels
kunnen liggen. Bodemvoedende dieren raken gevangen in vallen of
sleepnetten. Midwatersoorten kunnen worden gevangen met haken en lijnen
of met sleepnetten, die af en toe in de weg zitten van onderzeeboten.
Oppervlaktesoorten worden gevangen met netten en harpoenen. Bovendien
wordt het oppervlak zelf gebruikt voor de scheepvaart en voor de schepen
die hulpbronnen ontginnen.14
\end{quote}

Dit groeiende conflict leidt Christy tot de voorspelling dat `de zeeën
zich in een overgangsfase bevinden. Ze bewegen van een situatie waarin
eigendomsrechten vrijwel niet bestaan, naar een situatie waarin deze
rechten op de een of andere manier zullen worden toegeëigend of
beschikbaar gesteld.' Uiteindelijk concludeert Christy: 'Naarmate de
rijkdommen van de zee waardevoller worden, zullen er exclusieve rechten
worden verworven.'15

\section{VERVUILING}\label{vervuiling}

Steenkool wordt gewonnen uit schachten onder de zeebodem. Olie wordt
aangeboord vanaf platforms die aan de zeebodem zijn bevestigd en boven
het water uitsteken. Mineralen kunnen van de oceaanbodem worden
gebaggerd. Sedentaire dieren worden van de bodem geschraapt, waar vaak
telefoonkabels liggen. Bodemvoedende dieren raken verstrikt in vallen of
sleepnetten. Midwatersoorten kunnen worden gevangen met haken en lijnen,
of met sleepnetten die soms in de weg zitten van onderzeeboten.
Oppervlaktesoorten worden gevangen met netten en harpoenen. Daarnaast
wordt het wateroppervlak zelf gebruikt voor de scheepvaart en voor de
schepen die de hulpbronnen ontginnen. Dit groeiende conflict leidt
Christy tot de voorspelling dat `de zeeën zich in een overgangsfase
bevinden. Ze bewegen van een situatie waarin eigendomsrechten vrijwel
niet bestaan naar een situatie waarin deze rechten op de een of andere
manier zullen worden toegeëigend of beschikbaar gesteld.' Uiteindelijk
concludeert Christy: `Naarmate de rijkdommen van de zee waardevoller
worden, zullen er exclusieve rechten worden verworven.'

Oké: Zelfs als we aannemen dat volledig privébezit van grondstoffen en
de vrije markt deze beter behouden en creëren dan overheidsregulering,
hoe zit het dan met het probleem van vervuiling? Zouden we dan niet te
maken krijgen met een toename van vervuiling door ongecontroleerde
`kapitalistische hebzucht'?

Allereerst is er dit grimmige feit: overheidseigendom, zelfs socialisme,
heeft niet bewezen de oplossing te zijn voor het probleem van
vervuiling. Zelfs de meest visionaire voorstanders van overheidsplanning
erkennen dat de vergiftiging van het Baikalmeer in de Sovjetunie een
symbool is van achteloze industriële vervuiling van een waardevolle
natuurlijke hulpbron. Maar er speelt nog veel meer. Denk bijvoorbeeld
aan de twee cruciale gebieden waar vervuiling een groot probleem is
geworden: de lucht en de waterwegen, met name de rivieren. Dit zijn
precies de vitale gebieden in de samenleving waar privébezit niet heeft
kunnen functioneren.

Eerst de rivieren. Rivieren en oceanen zijn algemeen eigendom van de
overheid; volledig privébezit is niet toegestaan in het water. In wezen
is de overheid dus de eigenaar van de rivieren. Maar overheidseigendom
is geen echt eigendom, omdat overheidsfunctionarissen, hoewel ze de
hulpbron kunnen beheren, de economische waarde ervan niet op de markt
kunnen verzilveren. Ze kunnen de rivieren niet verkopen of aandelen
ervan uitgeven. Hierdoor hebben ze geen economische prikkel om de
zuiverheid en waarde van de rivieren te behouden. Economisch gezien zijn
rivieren `geen eigendom'; daarom hebben overheidsfunctionarissen
corruptie en vervuiling toegestaan. Iedereen kan vervuilend afval in het
water dumpen. Stel je eens voor wat er zou gebeuren als particuliere
bedrijven de rivieren en meren wel konden bezitten. Als een privébedrijf
bijvoorbeeld eigenaar zou zijn van Lake Erie, dan zou iedereen die afval
in het meer dumpt onmiddellijk aangeklaagd worden voor aanvallen op
privé-eigendom. De rechter zou dergelijke personen dwingen om
schadevergoeding te betalen en te stoppen met hun vuiligheid. Alleen
privé-eigendomsrechten kunnen zorgen voor het beëindigen van de
vervuiling van hulpbronnen. Omdat de rivieren geen eigendom zijn, is er
niemand die op kan staan om zijn kostbare bron te verdedigen tegen
aanvallen. Als iemand daarentegen afval of vervuilende stoffen zou
dumpen in een meer dat privébezit is (zoals veel kleinere meren), dan
zou dat niet lang onopgemerkt blijven; de eigenaar zou zich krachtig
verdedigen. Professor Dolan schrijft:

\begin{quote}
Als General Motors de rivier de Mississippi zou bezitten, zou je er
zeker van kunnen zijn dat er hoge heffingen voor afvalwater zouden
worden opgelegd aan de industrieën en gemeenten langs de oevers. Het
water zou dan schoon genoeg worden gehouden om de inkomsten uit
pachtovereenkomsten met bedrijven die rechten willen op drinkwater,
recreatie en commerciële visserij te maximaliseren.
\end{quote}

Als de overheid als eigenaar de vervuiling van de rivieren toestaat, dan
is zij ook de grootste actieve vervuiler, vooral door haar rol als
gemeentelijke afvalwaterverwerker. Er zijn al goedkope chemische
toiletten beschikbaar die toiletafval kunnen verbranden zonder de lucht,
de grond of het water te vervuilen. Maar wie zal er investeren in
chemische toiletten als de lokale overheden het afvalwater gratis voor
hun inwoners afvoeren?

Dit voorbeeld wijst op een probleem dat vergelijkbaar is met de
belemmeringen voor aquacultuurtechnologie door het ontbreken van
privé-eigendom. Als overheden als eigenaren van de rivieren
watervervuiling toestaan, dan wordt industriële technologie een bron van
vervuiling---en dat gebeurt al.~Wanneer productieprocessen
ongecontroleerd de rivieren kunnen vervuilen door hun eigenaren, is dat
het soort productietechnologie dat we zullen krijgen.

Als het probleem van watervervuiling kan worden opgelost door
privé-eigendomsrechten op water, hoe zit het dan met luchtvervuiling?
Hoe kunnen libertariërs een oplossing bedenken voor dit ernstige
probleem? Er kan toch geen privé-eigendom in de lucht zijn? Maar het
antwoord is: ja, dat kan wel. We hebben al gezien hoe radio- en
tv-frequenties privébezit kunnen zijn. Dat geldt ook voor routes van
luchtvaartmaatschappijen. Commerciële luchtvaartmaatschappijen zouden
bijvoorbeeld privé-eigendom kunnen zijn; er is geen Civil Aeronautics
Board nodig om routes tussen verschillende steden te verdelen en te
reguleren. Bij luchtvervuiling gaat het echter niet zozeer om privébezit
in de lucht, maar om het beschermen van privébezit in iemands longen,
velden en boomgaarden. Het essentiële kenmerk van luchtvervuiling is dat
de vervuiler ongewenste en schadelijke stoffen, zoals rook, nucleaire
straling en zwaveloxiden, de lucht in blaast en zo in de longen van
onschuldige slachtoffers terechtkomt, evenals op hun materiële eigendom.
Al deze uitstoot die schade toebrengt aan personen of eigendommen is
agressie tegen het privébezit van de slachtoffers. Luchtvervuiling is
net zo goed agressie als brandstichting van andermans eigendom of het
lichamelijk verwonden van iemand. Luchtvervuiling die anderen schade
berokkent, is pure agressie. De belangrijkste taak van de overheid---via
rechtbanken en politie---is om agressie te stoppen. In plaats daarvan
heeft de overheid gefaald in deze taak en heeft zij ernstig
tekortgeschoten in haar verantwoordelijkheden om de slachtoffers van
luchtvervuiling te beschermen.

Het is belangrijk te beseffen dat dit falen geen louter kwestie van
onwetendheid is geweest. Het betreft eerder een tijdsverschil tussen het
herkennen van een nieuw technologisch probleem en het daadwerkelijk
aanpakken ervan. Sommige moderne vervuilende stoffen zijn pas recent
ontdekt, maar fabrieksrook en de schadelijke effecten daarvan zijn al
sinds de Industriële Revolutie bekend. Zelfs aan het eind en het begin
van de negentiende eeuw namen Amerikaanse rechtbanken de weloverwogen
beslissing om de schending van eigendomsrechten door industriële rook
toe te staan. Om dit mogelijk te maken, moesten de rechtbanken de
verdediging van eigendomsrechten, die in het Angelsaksische
gewoonterecht was verankerd, systematisch aanpassen en verzwakken. Voor
het midden en het eind van de negentiende eeuw werd elke schadelijke
luchtvervuiling beschouwd als een onrechtmatige daad. Slachtoffers
konden schadevergoeding eisen en een gerechtelijk bevel uitvaardigen om
verdere inbreuken op hun eigendomsrechten te stoppen. In de loop van de
negentiende eeuw begonnen de rechtbanken echter systematisch de wet op
nalatigheid en de wet op overlast aan te passen. Zo werd luchtvervuiling
toegestaan zolang deze niet aanzienlijk groter was dan die van
vergelijkbare productiebedrijven en niet groter dan de gebruikelijke
praktijk van collega-vervuilers.

Toen de eerste fabrieken verrezen en rook begonnen uit te stoten, die de
boomgaarden van de naburige boeren verwoestten, daagden de boeren de
fabrikanten voor de rechter. Ze vroegen om schadevergoeding en een
verbod om hun eigendom verder te beschadigen. Maar de rechters zeiden in
feite: `Sorry, we weten dat industriële rook (oftewel luchtvervuiling)
jullie eigendomsrechten aantast en verstoort. Maar er is iets
belangrijker dan enkel eigendomsrechten, en dat is de publieke politiek,
het 'algemeen belang'. En het algemeen belang stelt dat de industrie een
goede zaak is en dat industriële vooruitgang gewenst is. Daarom moeten
jouw privé-eigendomsrechten terzijde worden geschoven ten gunste van het
algemeen welzijn.' Vandaag de dag betalen we allemaal de zware prijs
voor deze verwaarlozing van privébezit, in de vorm van longziekten en
talloze andere kwalen. En dat allemaal voor het `algemeen welzijn'!

Dat dit principe ook in het luchttijdperk door de rechtbanken is
toegepast, blijkt uit een uitspraak van de rechtbank van Ohio in de zaak
Antonik v. Chamberlain (1947). Bewoners van een buitenwijk in de buurt
van Akron spanden een rechtszaak aan om de gedaagden te verbieden een
particulier vliegveld te exploiteren. Ze beriepen zich op schending van
hun eigendomsrechten door excessief lawaai. De rechtbank weigerde het
verzoek om een verbod en verklaarde:

\begin{quote}
Bij het oordelen in deze zaak, als een rechtbank van billijkheid, moeten
we niet alleen het belangenconflict tussen de eigenaar van het vliegveld
en de nabijgelegen landeigenaren afwegen. We moeten ook het openbaar
beleid van onze generatie in acht nemen. We moeten erkennen dat de
oprichting van een vliegveld van groot belang is voor het publiek. Als
zo'n vliegveld wordt tegengehouden, of als de oprichting ervan wordt
verhinderd, zal dit niet alleen ernstige schade toebrengen aan de
eigenaar van de haven. Het kan ook een aanzienlijk verlies betekenen van
een waardevol bezit voor de hele gemeenschap.
\end{quote}

Om de misdaden van de rechters te onderstrepen, kwamen zowel de federale
als de statelijke wetgevende machten tussenbeide. Ze versterkten de
agressie door slachtoffers van luchtvervuiling te verbieden om
collectieve rechtszaken aan te spannen tegen de vervuilers. Als een
fabriek de lucht van een stad vervuilt en tienduizenden mensen schade
lijden, is het onpraktisch voor elk slachtoffer om afzonderlijk een
rechtszaak te beginnen tegen de vervuiler. Terwijl een gerechtelijk
bevel effectief kan zijn voor een enkel slachtoffer, erkent het
gewoonterecht de mogelijkheid van collectieve rechtszaken. Dit houdt in
dat één of enkele slachtoffers de vervuiler kunnen aanklagen, niet
alleen namens zichzelf, maar ook namens de hele groep slachtoffers. Toch
heeft de wetgever dergelijke collectieve rechtszaken in vervuilingszaken
systematisch verboden. Dit betekent dat een slachtoffer met succes een
vervuiler kan aanklagen voor de schade die hij persoonlijk heeft
geleden, in een één-op-één `privé-hinder' procedure. Maar het is bij wet
verboden om op te treden tegen een grote vervuiler die schade berokkent
aan een groot aantal mensen in een bepaald gebied. Zoals Frank Bubb
schrijft: `Het is alsof de overheid je vertelt dat ze je zal (proberen
te) beschermen tegen een dief die alleen van jou steelt, maar je niet
zal beschermen als de dief ook van alle anderen in de buurt steelt.'

Ook lawaai geldt als een vorm van luchtvervuiling. Lawaai ontstaat door
geluidsgolven die door de lucht bewegen en de eigendommen en personen
van anderen binnendringen. Artsen onderzoeken pas sinds kort de
schadelijke effecten van lawaai op de menselijke fysiologie. Bovendien
zou een libertair rechtssysteem schadeclaims, collectieve rechtszaken en
verbodsacties toestaan tegen buitensporig en schadelijk lawaai, oftewel
`lawaaivervuiling'.

De oplossing voor luchtvervuiling is eenvoudig en staat los van
miljarden kostende overheidsprogramma's, gefinancierd door de
belastingbetaler, die het echte probleem niet aanpakken. De oplossing
ligt in het herstellen van de rol van de rechtbanken: hun taak is het
beschermen van persoonlijke en eigendomsrechten tegen inbreuken. Daarom
moeten ze iedereen verbieden om vervuilende stoffen in de lucht te
lozen. Maar wat betekent dit voor de voorvechters van industriële
vooruitgang? En hoe zit het met de hogere kosten die de consument
mogelijk moet dragen? Wat gebeurt er met onze huidige vervuilende
technologie?

Het argument dat een verbod op vervuiling de kosten van industriële
productie zou verhogen, is even ongegrond als het argument vóór de
Burgeroorlog dat de afschaffing van de slavernij de kosten van de
katoenteelt zou verhogen. Daarom zou afschaffing, hoe moreel juist ook,
`onpraktisch' zijn. Dit impliceert dat vervuilers in staat zijn om alle
hoge kosten van vervuiling te laten betalen door degenen wiens longen en
eigendomsrechten ze zonder gevolgen hebben geschonden.

Bovendien negeert het kosten- en technologieargument het belangrijke
feit dat er geen economische prikkel is om vervuilende technologieën te
ontwikkelen als luchtvervuiling ongestraft blijft. Sterker nog, de
prikkel om te verduurzamen zal verder afnemen, net zoals dat al een eeuw
lang het geval is. Stel je bijvoorbeeld voor dat de rechtbanken in de
tijd dat auto's en vrachtwagens voor het eerst werden gebruikt, op de
volgende manier hadden beslist:

\begin{quote}
Normaal gesproken zouden we tegen vrachtwagens zijn die het gazon van
mensen betreden, omdat dit een inbreuk is op privé-eigendom. We zouden
erop staan dat vrachtwagens zich aan de weg houden, ongeacht de
verkeersopstoppingen. Maar vrachtwagens spelen een cruciale rol voor het
algemeen welzijn. Daarom bepaalt men dat vrachtwagens alle gazons mogen
oversteken die ze willen, zolang ze geloven dat dit hun
verkeersproblemen zal verlichten.
\end{quote}

Als de rechtbanken op deze manier hadden geoordeeld, zouden we nu een
transportsysteem hebben waarin gazons systematisch door vrachtwagens
worden beschadigd. Elke poging om dit te stoppen zou worden afgeserveerd
in naam van de moderne transportbehoeften! Het punt is dat dit precies
is hoe de rechtbanken omgaan met luchtvervuiling -- vervuiling die veel
schadelijker is voor ons allemaal dan het vertrappen van gazons. Op deze
manier heeft de overheid vanaf het begin het groene licht gegeven voor
vervuilende technologieën. Het is dan ook geen verrassing dat dit het
soort technologie is dat we momenteel hebben. De enige oplossing is om
de vervuilende indringers te dwingen hun invasie te stoppen, zodat
technologie kan worden omgevormd naar niet-vervuilende of zelfs
antivervuilende middelen.

Zelfs in het noodzakelijke primitieve stadium van
antipollutietechnologie zijn er al technieken ontwikkeld om
luchtvervuiling en geluidsoverlast tegen te gaan. Zo kunnen er
geluidsdempers worden geïnstalleerd op lawaaierige machines. Deze
dempers zenden geluidsgolven uit die precies het tegenovergestelde zijn
van de golven van de machines, waardoor het hinderlijke geluid
vermindert. Luchtafval kan nu zelfs al worden opgevangen zodra het de
schoorsteen verlaat en gerecycled worden tot bruikbare producten voor de
industrie. Zwaveldioxide, een belangrijke luchtvervuiler, kan worden
opgevangen en omgezet in economisch waardevol zwavelzuur. De sterk
vervuilende verbrandingsmotor moet ofwel `genezen' worden met nieuwe
apparaten, of volledig vervangen worden door niet-vervuilende
alternatieven zoals diesel-, gas- of stoommotoren, of door elektrische
auto's. Zoals libertarische systeemingenieur Robert Poole Jr.~opmerkt,
zullen de kosten van de installatie van niet- of antivervuilende
technologie uiteindelijk `dragen door de consumenten van de producten
van de firma, dat wil zeggen degenen die ervoor kiezen om met de firma
in zee te gaan, in plaats van door te schuiven naar onschuldige derden
in de vorm van vervuiling (of belastingen).'

Robert Poole definieert vervuiling als `de overdracht van schadelijke
materie of energie naar de persoon of het eigendom van een ander, zonder
diens toestemming.' De libertarische, en enige volledige, oplossing voor
het probleem van luchtvervuiling is om de rechtbanken en de juridische
structuur in te zetten om dergelijke inbreuken te bestrijden en te
voorkomen. Er zijn recente signalen dat het rechtssysteem in deze
richting begint te bewegen, zoals nieuwe gerechtelijke uitspraken en de
intrekking van wetten die collectieve rechtszaken verbieden. Maar dit is
slechts het begin.

Onder conservatieven, in tegenstelling tot libertariërs, zijn er twee
vergelijkbare reacties op het probleem van luchtvervuiling. De eerste
reactie, van onder anderen Ayn Rand en Robert Moses, is om te ontkennen
dat het probleem bestaat. Zij schrijven de hele discussie toe aan links,
die volgens hen het kapitalisme en de technologie willen afschaffen ten
gunste van een tribale vorm van socialisme. Hoewel een deel van deze
beschuldiging wellicht terecht is, is het ontkennen van het bestaan van
het probleem een ontkenning van de wetenschap zelf. Dit biedt een
belangrijke vrijbrief voor de linkse kritiek dat verdedigers van het
kapitalisme `eigendomsrechten boven mensenrechten stellen'. Bovendien is
een verdediging van luchtvervuiling niet eens een verdediging van
eigendomsrechten. Sterker nog, het geeft conservatieven de indruk dat
zij goedkeuring verlenen aan industriëlen die de eigendomsrechten van de
meeste burgers schenden.

Een tweede, meer geraffineerde conservatieve reactie komt van
vrijemarkteconomen zoals Milton Friedman. De aanhangers van Friedman
erkennen het bestaan van luchtvervuiling, maar zij stellen voor om dit
probleem niet aan te pakken door eigendomsrechten te verdedigen. In
plaats daarvan pleiten ze voor een utilitaristische `kosten-baten'
berekening door de overheid. De overheid zou dan een `sociale
beslissing' nemen en afdwingen over de hoeveelheid vervuiling die
acceptabel is. Deze beslissing zou worden gehandhaafd door een bepaalde
hoeveelheid vervuiling toe te staan, bijvoorbeeld door
`vervuilingsrechten' te vergeven. Daarnaast zou dit kunnen gebeuren via
een gradatieschaal van belastingen of door belastingbetalers te laten
opdraaien voor bedrijven die besluiten om niet te vervuilen. Deze
voorstellen zouden de overheid een enorme bureaucratische macht
toekennen, onder het mom van het beschermen van de `vrije markt.'
Tegelijkertijd zou dit eigendomsrechten ondermijnen, aangezien de staat
een collectieve beslissing afdwingt. Dit is allesbehalve een echte
`vrije markt' en laat zien dat het, net als op veel andere economische
gebieden, onmogelijk is om vrijheid en de vrije markt te verdedigen
zonder ook de rechten van privé-eigendom te waarborgen. Friedmans
groteske uitspraak dat stadsbewoners die geen emfyseem willen krijgen,
maar naar het platteland moeten verhuizen, doet sterk denken aan
Marie-Antoinettes beroemde `Laat ze taart eten.' Dit getuigt van een
gebrek aan gevoel voor mensenrechten en eigendomsrechten. Zijn opmerking
is vergelijkbaar met de typisch conservatieve uitspraak `Als het je hier
niet bevalt, vertrek dan.' Deze uitspraak impliceert dat de overheid het
gehele gebied van `hier' rechtmatig bezit en dat iedereen die bezwaar
heeft tegen haar heerschappij, het gebied moet verlaten. Robert Poole's
libertarische kritiek op de voorstellen van Friedman biedt een
verfrissend contrast:

\begin{quote}
Helaas illustreert dit een van de grootste tekortkomingen van
conservatieve economen: nergens in het voorstel wordt er gesproken over
rechten. Dit is dezelfde tekortkoming die voorstanders van het
kapitalisme al 200 jaar ondermijnt. Zelfs vandaag de dag roept de term
`laissez-faire' beelden op van achttiende-eeuwse Engelse fabriekssteden,
overspoeld door rook en bedekt met roet. De vroege kapitalisten stemden
met de rechtbanken in dat rook en roet de `prijs' waren die betaald
moest worden voor de voordelen van de industrie. Maar laissez-faire
zonder rechten is een contradictio in terminis; het laissez-faire
standpunt is gebaseerd op de mensenrechten en kan alleen bestaan als die
rechten onschendbaar zijn. Nu, in een tijdperk van toenemend
milieubewustzijn, duikt deze oude tegenstrijdigheid weer op om het
kapitalisme te achtervolgen.

Het klopt dat lucht een schaars goed is, zoals de Friedmanites stellen.
Maar het is de vraag waarom dit schaars is. Als lucht schaars is door
systematische schending van rechten, dan is de oplossing niet om de
prijs van de huidige situatie te verhogen, waardoor deze schendingen
worden goedgekeurd. De oplossing ligt in het handhaven van de rechten en
het eisen dat ze worden beschermd. Wanneer een fabriek een grote
hoeveelheid zwaveldioxide loost die in iemands longen terechtkomt en
longoedeem veroorzaakt, hebben de eigenaren van die fabriek hem net zo
hard beschadigd als wanneer ze zijn been hadden gebroken. Dit punt is
cruciaal en moet worden benadrukt, omdat het essentieel is voor het
libertarische laissez-faire standpunt. Een laissez-faire vervuiler is
een contradictio in terminis en moet als zodanig worden herkend. Een
libertarische samenleving zou een samenleving zijn waarin volledige
aansprakelijkheid bestaat, waarin iedereen volledig verantwoordelijk is
voor zijn daden en de schadelijke gevolgen daarvan.
\end{quote}

Naast het verraad aan haar veronderstelde functie om privébezit te
beschermen, heeft de overheid ook op een actieve manier bijgedragen aan
luchtvervuiling. Niet zo lang geleden voerde het Ministerie van Landbouw
massale DDT-bespuitingen met helikopters uit over grote gebieden,
waarbij het de bezwaren van individuele boeren negeerde. Nog steeds
worden er tonnen giftige en kankerverwekkende insecticiden verspreid
over het hele Zuiden in een dure en vergeefse poging om de vuurmier uit
te roeien. En de Commissie voor Atoomenergie heeft radioactief afval in
de lucht en in de grond gedumpt via haar kerncentrales en atoomproeven.
Gemeentelijke elektriciteits- en watercentrales, evenals nutsbedrijven
met een monopoliepositie, vervuilen de atmosfeer enorm. Een van de
belangrijkste taken van de staat op dit gebied is daarom het beëindigen
van haar eigen bijdrage aan de vergiftiging van de atmosfeer.

Als we de verwarring en de ondeugdelijke filosofie van moderne ecologen
ontleden, ontdekken we een fundamentele kwestie tegen het huidige
systeem. Maar die kwestie is niet tegen kapitalisme, privébezit, groei
of technologie op zichzelf. Het is een kwestie van het falen van de
overheid om de rechten van privébezit te waarborgen en te beschermen
tegen invasies. Als eigendomsrechten volledig werden beschermd, zowel
tegen privé-invasies als tegen inmenging door de overheid, dan zouden we
zien dat privé-ondernemingen en moderne technologie, net als op andere
terreinen van onze economie en samenleving, niet als een vloek, maar als
een redding voor de mensheid worden beschouwd.

\bookmarksetup{startatroot}

\chapter{Oorlog en buitenlands
beleid}\label{oorlog-en-buitenlands-beleid}

\section{\texorpdfstring{\textbf{ISOLATIONISME, LINKS EN
RECHTS}}{ISOLATIONISME, LINKS EN RECHTS}}\label{isolationisme-links-en-rechts}

`Isolationisme' werd oorspronkelijk gebruikt als een lasterterm voor
degenen die tegen de Amerikaanse deelname aan de Tweede Wereldoorlog
waren. Dit woord kreeg vaak een negatieve bijklank door de associatie
met pro-Nazi opvattingen. Daarom werd `isolationist' vaak geassocieerd
met `rechts' en kreeg het een negatieve connotatie. Zelfs als iemand
niet actief pro-Nazi was, werd hij of zij gezien als een bekrompen en
onwetende geest, in tegenstelling tot de verfijnde, wereldwijze en
zorgzame `internationalisten'. Deze laatste groep pleitte voor
Amerikaanse interventies over de hele wereld. In het afgelopen decennium
werden anti-oorlogskrachten vaak als `links' bestempeld.
Interventionisten, van Lyndon Johnson tot Jimmy Carter, hebben
voortdurend geprobeerd om het huidige linkse kamp het etiket
`isolationist,' of op zijn minst `neo-isolationist,' op te plakken.

\textbf{Links of rechts?} Tijdens de Eerste Wereldoorlog werden
tegenstanders van de oorlog, net als nu, hard aangevallen als `linksen',
ook al bevonden zich onder hen libertariërs en voorstanders van
laissez-faire kapitalisme. Het belangrijkste centrum van verzet tegen de
Amerikaanse oorlog met Spanje en de oorlog om de Filippijnse opstand aan
het begin van de twintigste eeuw bestond uit laissez-faire liberalen.
Mannen zoals socioloog en econoom William Graham Sumner en de koopman
Edward Atkinson uit Boston, die de `Anti-Imperialist League' oprichtte,
leidde dit verzet. Atkinson en Sumner stonden bovendien stevig in de
traditie van de klassieke Engelse liberalen van de achttiende en
negentiende eeuw, met name de laissez-faire `extremisten' zoals Richard
Cobden en John Bright van de `Manchester School'. Cobden en Bright nam
het voortouw in het verzet tegen elke Britse oorlog en buitenlandse
politieke interventie in hun tijd. Cobden werd dan ook niet gezien als
een `isolationist', maar als de `International Man'. Tot de
lastercampagne van de late jaren dertig werden tegenstanders van oorlog
gezien als de echte `internationalisten'. Dit waren mannen die zich
verzetten tegen de uitbreiding van de natiestaat en pleitten voor vrede,
vrije handel, vrije migratie en vreedzame culturele uitwisselingen
tussen alle volkeren. Buitenlandse interventie kan alleen als
`internationaal' worden beschouwd in de zin dat oorlog internationaal
is: dwang, of het nu de dreiging van geweld of de daadwerkelijke inzet
van troepen is, zal altijd grenzen overschrijden tussen de ene natie en
de andere.

`Isolationisme' klinkt rechts, terwijl `neutralisme' en `vreedzame
coëxistentie' links lijken. Maar de essentie is hetzelfde: het verzet
tegen oorlog en politieke interventie tussen landen. Dit standpunt wordt
al twee eeuwen lang ingenomen door anti-oorlogskrachten, of het nu gaat
om klassieke liberalen uit de achttiende en negentiende eeuw, de
`linksen' van de Eerste Wereldoorlog en de Koude Oorlog, of de
`rechtsen' van de Tweede Wereldoorlog. In zeer weinig gevallen pleitten
deze anti-interventionisten voor daadwerkelijke `isolatie'. Wat ze
meestal nastreven, is politieke niet-inmenging in de aangelegenheden van
andere landen, gecombineerd met economisch en cultureel
internationalisme. Dit houdt in dat zij vreedzame vrijheid van handel,
investeringen en uitwisselingen tussen de burgers van alle landen
steunen. En dat is ook de kern van het libertarische standpunt.

\section{\texorpdfstring{\textbf{BEPERKING VAN DE
OVERHEID}}{BEPERKING VAN DE OVERHEID}}\label{beperking-van-de-overheid}

`Isolationisme' werd oorspronkelijk als een lasterterm gebruikt voor
degenen die tegen de Amerikaanse deelname aan de Tweede Wereldoorlog
waren. Dit woord kreeg vaak een negatieve bijklank door de associatie
met pro-Nazi opvattingen. Daarom werd `isolationist' vaak gezien als een
term die bij de `rechtse' hoek hoorde. Zelfs als iemand niet actief
pro-Nazi was, werd hij of zij beschouwd als een bekrompen en onwetende
geest, in tegenstelling tot de verfijnde, wereldwijze en zorgzame
`internationalisten'. Deze groep pleitte voor Amerikaanse interventies
over de hele wereld. In het afgelopen decennium werden
anti-oorlogskrachten vaak als `links' bestempeld. Interventionisten, van
Lyndon Johnson tot Jimmy Carter, hebben voortdurend geprobeerd om het
huidige linkse kamp het etiket `isolationist' of `neo-isolationist' op
te plakken. \textbf{Links of rechts?} Tijdens de Eerste Wereldoorlog
werden tegenstanders van de oorlog, net als nu, fel aangevallen als
`linksen', ook al waren er onder hen libertariërs en voorstanders van
laissez-faire kapitalisme. Het belangrijkste centrum van verzet tegen de
Amerikaanse oorlog met Spanje en de oorlog om de Filippijnse opstand aan
het begin van de twintigste eeuw bestond uit laissez-faire liberalen.
Mannen zoals socioloog en econoom William Graham Sumner en de koopman
Edward Atkinson uit Boston, die de `Anti-Imperialist League' oprichtte,
leidden dit verzet. Atkinson en Sumner stonden bovendien stevig in de
traditie van de klassieke Engelse liberalen van de achttiende en
negentiende eeuw, met name de laissez-faire `extremisten' zoals Richard
Cobden en John Bright van de `Manchester School'. Cobden en Bright namen
het voortouw in het verzet tegen elke Britse oorlog en buitenlandse
politieke interventie in hun tijd. Cobden werd dan ook niet gezien als
een `isolationist', maar als de `International Man'. Tot de
lastercampagne van de late jaren dertig werden tegenstanders van oorlog
gezien als de echte `internationalisten': mannen die zich verzetten
tegen de uitbreiding van de natiestaat en pleitten voor vrede, vrije
handel, vrije migratie en vreedzame culturele uitwisselingen tussen alle
volkeren. Buitenlandse interventie kan alleen als `internationaal'
worden gezien in de zin dat oorlog internationaal is: dwang, of dat nu
de dreiging van geweld of de daadwerkelijke inzet van troepen betreft,
zal altijd grenzen tussen naties overschrijden. `Isolationisme' klinkt
rechts, terwijl `neutralisme' en `vreedzame coëxistentie' vaak als links
worden gezien. Maar de kern van deze begrippen is hetzelfde: het verzet
tegen oorlog en politieke ingrepen tussen landen. Dit standpunt wordt al
twee eeuwen lang ingenomen door anti-oorlogskrachten, of het nu gaat om
klassieke liberalen uit de achttiende en negentiende eeuw, de `linksen'
van de Eerste Wereldoorlog en de Koude Oorlog, of de `rechtsen' van de
Tweede Wereldoorlog. In de meeste gevallen pleitten deze
anti-interventionisten echter niet voor daadwerkelijke `isolatie'. Wat
zij over het algemeen nastreefden, was politieke niet-inmenging in de
aangelegenheden van andere landen, gecombineerd met economisch en
cultureel internationalisme. Dit houdt in dat zij vreedzame vrijheid van
handel, investeringen en uitwisselingen tussen burgers van alle landen
steunen. En dat vormt ook de kern van het libertarische standpunt.

Libertariërs pleiten voor de afschaffing van alle staten wereldwijd en
voor het overnemen van legitieme taken die momenteel slecht door
overheden worden uitgevoerd, zoals politie en rechtbanken, door middel
van de vrije markt. Ze beschouwen vrijheid als een natuurlijk
mensenrecht en pleiten daarvoor, niet alleen voor Amerikanen, maar voor
iedereen. In een puur libertarische wereld zou er dus geen `buitenlands
beleid' zijn, omdat er geen staten zouden bestaan en geen overheden die
een monopolie op dwang uitoefenen over bepaalde gebieden. Gezien het
feit dat we in een wereld van natiestaten leven en het onwaarschijnlijk
is dat dit systeem in de nabije toekomst zal verdwijnen, wat is dan de
houding van libertariërs ten opzichte van buitenlands beleid in deze
door staten gedomineerde wereld?

In afwachting van de ontbinding van staten willen libertariërs de
overheidsmacht zoveel mogelijk inperken. We hebben al aangetoond hoe het
principe van `de-statizering' kan functioneren in verschillende
belangrijke `binnenlandse' kwesties. Hier is het doel om de rol van de
overheid te verminderen en de vrijwillige en spontane krachten van vrije
mensen volledig tot hun recht te laten komen door middel van vreedzame
interactie, vooral in de vrije markteconomie. In buitenlandse
aangelegenheden is het doel vergelijkbaar: voorkomen dat de overheid
zich mengt in de zaken van andere regeringen of landen. Politiek
`isolationisme' en vreedzame coëxistentie -- het niet ingrijpen in
andere landen -- vormen dan ook de libertarische tegenhanger van het
pleiten voor een laissez-faire beleid op nationaal niveau. Het idee is
om overheden te belemmeren in hun buitenlandse optreden, net zoals we
dat thuis proberen te doen. Isolationisme of vreedzame coëxistentie is
de equivalente aanpak in het buitenlands beleid van een strikte
beperking van de overheid in eigen land.

Concreet is het totale landoppervlak van de wereld verdeeld over
verschillende staten. Elke staat wordt geregeerd door een centrale
regering die een geweldsmonopolie heeft over het betreffende gebied. In
de relaties tussen staten is het libertarische doel om te voorkomen dat
elke staat zijn geweld uitbreidt naar andere landen. Op die manier
blijft de tirannie van elke staat tenminste beperkt tot zijn eigen
gebied. Libertariërs willen de staatsagressie tegen privépersonen zoveel
mogelijk inperken. De enige manier om dit te bereiken in internationale
aangelegenheden is voor de mensen van elk land om druk uit te oefenen op
hun eigen staat. Ze moeten eisen dat hun regering zich beperkt tot het
gebied dat zij monopoliseert en dat zij geen andere staten aanvalt of
agressie pleegt tegen hun onderdanen. Kortom, het doel van de
libertariër is om elke bestaande staat zoveel mogelijk te beperken in
zijn inbreuk op de vrijheid van personen en eigendommen. Dit houdt in
dat oorlog volledig moet worden vermeden. De mensen binnen elke staat
zouden `hun' respectieve regeringen onder druk moeten zetten om elkaar
niet aan te vallen. In het geval dat er toch een conflict ontstaat,
moeten zij zich zo snel mogelijk terugtrekken.

Laten we eens uitgaan van een wereld met twee hypothetische landen:
Graustark en Belgravia. Elk land wordt geregeerd door zijn eigen
overheid. Wat gebeurt er als de regering van Graustark het grondgebied
van Belgravia binnendringt? Vanuit een libertarisch perspectief treden
er meteen twee problemen op. Ten eerste begint het leger van Graustark
onschuldige burgers in Belgravia te doden, mensen die niets te maken
hebben met de misdaden die de regering van Belgravia heeft begaan.
Oorlog is dus massamoord. Deze grove schending van het recht op leven en
zelfeigenaarschap van een groot aantal mensen is niet alleen een
misdaad, maar voor de libertariër ook de ultieme misdaad. Ten tweede,
omdat alle regeringen hun inkomsten verkrijgen via de dwangmatige
belastingheffing, leidt elke mobilisatie en inzet van troepen
onvermijdelijk tot een hogere belastingdruk in Graustark. Om beide
redenen---oorlogen tussen staten leiden tot zowel massamoord als een
toename van de belastingdruk---is de libertariër tegen oorlog. Punt uit.

Dat is niet altijd zo geweest. In de Middeleeuwen waren oorlogen veel
beperkter van omvang. Vóór de opkomst van moderne wapens was de
bewapening zo eenvoudig dat regeringen hun geweld zich meestal beperkten
tot de legers van rivaliserende regeringen. En dat deden ze vaak ook.
Het klopt dat de belastingdruk toenam, maar er vond in ieder geval geen
massamoord op onschuldigen plaats. Niet alleen was de vuurkracht laag
genoeg om het geweld te beperken tot de strijdende legers, ook bestond
er in het premoderne tijdperk geen centrale natiestaat die namens alle
inwoners van een bepaald gebied sprak. Wanneer een groep koningen of
baronnen het tegen een andere groep opnam, werd er niet van iedereen
verwacht dat hij of zij een toegewijde partizaan was. Bovendien
bestonden legers uit kleine bendes van huurlingen, en waren het geen
massale, dienstplichtige troepen die slaven waren van hun heersers. Vaak
was het voor de bevolking een favoriete bezigheid om een veldslag te
aanschouwen vanaf de veilige stadsmuren. Oorlog werd dan ook gezien als
een soort sportwedstrijd. Maar met de opkomst van de centraliserende
staat en moderne massavernietigingswapens is het afslachten van burgers,
evenals het inzetten van dienstplichtige legers, een essentieel
onderdeel geworden van de oorlogsvoering tussen staten.

Stel je voor dat, ondanks mogelijk libertarisch verzet, de oorlog is
uitgebroken. Het is duidelijk dat het libertarische standpunt moet zijn
dat, zolang de oorlog voortduurt, de aanvallen op onschuldige burgers
zoveel mogelijk beperkt moeten worden. Het traditionele internationale
recht bood twee uitstekende middelen om dit doel te bereiken: de
`oorlogswetten' en de `neutraliteitswetten' of `neutralenrechten'. De
neutraliteitswetten waren ontworpen om een oorlog beperkt te houden tot
de strijdende staten zelf, zonder aanvallen op niet-oorlogvoerende
staten en, in het bijzonder, zonder agressie tegen de bevolking van
andere naties. Dit benadrukt het belang van oude, bijna vergeten
Amerikaanse principes zoals `vrijheid van de zee' en strenge beperkingen
op het recht van oorlogvoerende staten om neutrale handel met het
vijandige land te blokkeren. Kortom, de libertariër streeft ernaar
neutrale staten te bewegen om neutraal te blijven in een interstatelijk
conflict. Tegelijkertijd probeert hij de oorlogvoerende staten ertoe te
brengen de rechten van neutrale burgers volledig te respecteren. De
`oorlogswetten' zijn op hun beurt bedoeld om de inbreuk van
oorlogvoerende staten op de rechten van burgers in hun gebied zoveel
mogelijk te beperken. Zoals de Britse jurist F.J.P. Veale het
verwoordde:

\begin{quote}
Het basisprincipe van deze code was dat vijandelijkheden tussen
beschaafde volkeren beperkt moesten blijven tot de daadwerkelijk
betrokken strijdkrachten. Het maakte een onderscheid tussen strijders en
niet-strijders door vast te leggen dat de enige taak van de strijders is
om elkaar te bestrijden. Niet-strijders moeten daarom worden uitgesloten
van militaire operaties.²
\end{quote}

In de gewijzigde vorm van een verbod op het bombarderen van steden die
zich niet aan de frontlinie bevonden, heeft deze regel standgehouden in
de oorlogen in West-Europa gedurende de afgelopen eeuwen. Dit bleef zo
totdat Groot-Brittannië in de Tweede Wereldoorlog begon met het
strategisch bombarderen van burgers. Tegenwoordig wordt het hele concept
nauwelijks nog herinnerd, aangezien de aard van moderne nucleaire
oorlogsvoering is gebaseerd op de vernietiging van burgers.

Om terug te keren naar ons hypothetische Graustark en Belgravia: stel je
voor dat Graustark Belgravia is binnengevallen en dat een derde
regering, Walldavia, nu de wapens oppakt om Belgravia te verdedigen
tegen `Graustarkische agressie'. Is deze actie te rechtvaardigen? Hier
ligt inderdaad de wortel van de problematische twintigste-eeuwse theorie
van `collectieve veiligheid'. Dit idee stelt dat wanneer een regering
`agressie' pleegt tegen een andere regering, het de morele plicht van
alle andere regeringen in de wereld is om zich te verenigen en de
`slachtofferstaat' te verdedigen.

Er zitten verschillende fatale fouten in het concept van collectieve
veiligheid tegen `agressie'. Een daarvan is dat wanneer Walldavië of
andere staten zich in de strijd mengen, ze zelf de omvang van de
agressie uitbreiden en verergeren. Dit gebeurt omdat ze (1) op een
onrechtvaardige manier massa's Graustarkische burgers doden en (2) de
belastingdruk op Walldavische burgers verhogen. Bovendien (3), in een
tijd waarin staten en hun onderdanen goed te identificeren zijn, stelt
Walldavia de Walldavische burgers bloot aan vergelding door
Graustarkische bommenwerpers of raketten. De deelname van de
Walldavische regering aan de oorlog brengt dus de levens en eigendommen
van de Walldavische burgers in gevaar, die de regering geacht wordt te
beschermen. Tot slot zal (4) de dienstplicht en slavernij van
Walldavische burgers meestal toenemen.

Als dit soort `collectieve veiligheid' echt wereldwijd zou worden
toegepast, waarbij alle `Walldavias' zich in elk lokaal conflict zouden
mengen en dit zouden laten escaleren, dan zou iedere lokale
schermutseling al snel uitgroeien tot een wereldwijde vuurzee.

Er is nog een andere belangrijke fout in het concept van collectieve
veiligheid. Het idee om een oorlog te beginnen om `agressie' te stoppen,
is duidelijk gebaseerd op de analogie van een individu dat agressie
uitoefent tegen een ander. Stel je voor dat Smith wordt gezien terwijl
hij Jones in elkaar slaat---dat is agressie tegen Jones. De politie in
de buurt komt dan in actie om Jones te beschermen. Ze gebruiken
`politieoptreden' om de agressie te stoppen. President Truman verwees
bijvoorbeeld voortdurend naar de Amerikaanse deelname aan de Koreaanse
oorlog als een `politieactie', als een collectieve inspanning van de VN
om `agressie' af te weren.

Maar `agressie' heeft alleen betekenis op het niveau van een individu,
zoals bij Smith en Jones, net zoals de term `politieoptreden'. Deze
termen zijn irrelevant op interstatelijk niveau. Ten eerste hebben we
gezien dat regeringen die een oorlog beginnen, zelf agressors worden
tegen onschuldige burgers; sterker nog, ze worden massamoordenaars. De
juiste analogie bij individuele acties zou zijn: Smith slaat Jones in
elkaar, de politie komt om Jones te helpen, maar terwijl ze Smith
proberen te arresteren, bombarderen ze een stadsblok en doden duizenden
mensen, of schieten ze met machinegeweren op een onschuldige menigte.
Dit is een veel nauwkeuriger vergelijking, want zo handelt een
oorlogvoerende regering. In de twintigste eeuw gebeurt dit op
monumentale schaal. Elk politiebureau dat zich zo gedraagt, wordt zelf
een criminele agressor, vaak veel meer dan de oorspronkelijke Smith die
de zaak begon.

Maar er is nog een andere fatale fout in de analogie met individuele
agressie. Wanneer Smith Jones in elkaar slaat of zijn eigendom steelt,
kunnen we Smith identificeren als de agressor die het persoonlijke of
eigendomsrecht van zijn slachtoffer schendt. Maar als de Graustarkiaanse
staat het grondgebied van de Belgraviaanse staat binnenvalt, is het
onterecht om op dezelfde manier naar `agressie' te verwijzen. Voor de
libertariër heeft geen enkele regering recht op eigendom of
`soevereiniteit' in een bepaald territorium. De aanspraak van de
Belgraviaanse staat op zijn grondgebied verschilt daarom totaal van de
aanspraak van meneer Jones op zijn eigendom (hoewel de laatste ook het
resultaat van diefstal kan blijken te zijn). Geen enkele staat heeft
legitiem eigendom; al zijn grondgebied is het resultaat van een of
andere vorm van agressie en gewelddadige verovering. De invasie van de
Graustarkiaanse staat is daarom noodzakelijkerwijs een conflict tussen
twee groepen dieven en agressors. Het enige probleem is dat aan beide
zijden onschuldige burgers daaronder lijden.

Afgezien van dit algemene voorbehoud met betrekking tot regeringen,
heeft de zogenaamde `agressor'-staat vaak een plausibele claim op zijn
`slachtoffer'. Dit is plausibel binnen de context van het systeem van de
natiestaat. Stel je voor dat Graustark de grens met Belgravia
overschrijdt omdat Belgravia een eeuw eerder Graustark is binnengevallen
en de noordoostelijke provincies van het land heeft veroverd. De
inwoners van deze provincies zijn cultureel, etnisch en taalkundig
Graustarkisch. Graustark valt nu binnen om eindelijk herenigd te worden
met zijn medelandgenoten. In deze situatie zou de libertariër, hoewel
hij beide regeringen veroordeelt voor hun oorlogsvoering en het doden
van burgers, de kant van Graustark moeten kiezen als de meest
rechtvaardige of minst onrechtvaardige claim. Laten we het zo
formuleren: in het onwaarschijnlijke geval dat de twee landen zouden
terugkeren naar premoderne oorlogsvoering, met (a) wapens die beperkt
zijn zodat er geen burgers gewond raken aan hun persoon of eigendom; (b)
legers die uit vrijwilligers bestaan in plaats van uit dienstplichtigen;
en (c) financiering via vrijwillige bijdragen in plaats van belasting,
dan zou de libertariër in onze context onvoorwaardelijk de kant van
Graustark kunnen kiezen.

Van alle recente oorlogen voldoet geen enkele, al is het niet helemaal,
zo goed aan de drie criteria voor een `rechtvaardige oorlog' als de
Indiase oorlog van eind 1971 voor de bevrijding van Bangladesh. De
regering van Pakistan was ontstaan als de laatste gruwelijke erfenis van
het Britse keizerrijk op het Indiase subcontinent. De Pakistaanse natie
bestond vooral uit de onwettige heerschappij van de Punjabi's uit
West-Pakistan over de talrijkere en productievere Bengali's uit
Oost-Pakistan, evenals over de Pathanen van het noordwestelijke
grensgebied. De Bengali's streden al lange tijd voor onafhankelijkheid
van hun imperialistische onderdrukkers. Begin 1971 werd het parlement
opgeschort na een overweldigende Bengaalse verkiezingswinst. Vanaf dat
moment slachtten de Punjabi-troepen systematisch de burgerbevolking van
Bengalen af. De Indiase betrokkenheid bij het conflict hielp de
populaire Bengaalse verzetsgroepen van de Mukhti Bahini. Hoewel
belastingheffing en dienstplicht een rol speelden, gebruikten de Indiase
legers hun wapens niet tegen Bengaalse burgers. Integendeel, dit was een
echte revolutionaire oorlog van het Bengaalse volk tegen een Punjabische
bezettingsstaat. Alleen de Punjabi-soldaten vielen ten prooi aan Indiase
kogels.

Dit voorbeeld wijst op een ander kenmerk van oorlogsvoering: een
revolutionaire guerrillaoorlog kan veel beter aansluiten bij
libertarische principes dan enige interstatelijke oorlog. Door de aard
van hun activiteiten beschermen guerrillastrijders de burgerbevolking
tegen de plunderingen van een staat. Daarom kunnen guerrillastrijders,
die in hetzelfde land wonen als de vijandelijke staat, geen nucleaire of
andere massavernietigingswapens gebruiken. Bovendien zijn
guerrillastrijders voor hun overwinning afhankelijk van de steun en hulp
van de bevolking. Als essentieel onderdeel van hun strategie moeten ze
burgers beschermen tegen schade en hun aanvallen uitsluitend richten op
het staatsapparaat en zijn strijdkrachten. Guerrillaoorlog roept ons
terug naar de oude en eervolle deugd van het bestrijden van de vijand en
het sparen van onschuldige burgers. Daarnaast zien guerrillastrijders,
om enthousiaste steun van de bevolking te verwerven, vaak af van
dienstplicht en belasting. Ze vertrouwen op vrijwillige bijdragen voor
manschappen en materieel.

De libertarische kwaliteiten van guerrillaoorlog zijn alleen te vinden
aan de revolutionaire kant. Voor de contrarevolutionaire strijdkrachten
van de staat is het een ander verhaal. Hoewel de staat niet zover gaat
dat hij zijn eigen onderdanen `bombardeert', vertrouwt hij noodgedwongen
op campagnes van massaterreur. Dit houdt in dat hij massa's burgers
doden, terroriseren en oppakken. Om succesvol te zijn, hebben
guerrilla's de steun van een groot deel van de bevolking nodig. Doordat
de staat zijn oorlog wil voeren, richt hij zich op het vernietigen van
deze bevolking. Dit kan betekenen dat hij massa's burgers in
concentratiekampen stuurt om ze te scheiden van hun
guerrillabondgenoten. Deze tactiek werd onder andere gebruikt door de
Spaanse generaal `Slager' Weyler tegen de Cubaanse rebellen in de jaren
`90 van de 19e eeuw. Later werd ze voortgezet door Amerikaanse troepen
op de Filipijnen en door de Britten in de Boerenoorlog. Ook tot aan het
recente mislukte 'strategisch gehucht'-beleid in Zuid-Vietnam zijn deze
methoden steeds weer toegepast.

Het libertarische buitenlands beleid is dus geen pacifistisch beleid.
Wij geloven niet, zoals pacifisten, dat niemand het recht heeft om
geweld te gebruiken ter verdediging tegen een gewelddadige aanval. Wat
wij wel vinden, is dat niemand het recht heeft om anderen te
onderwerpen, te belasten of te doden, of om geweld jegens anderen te
gebruiken als verweer. Aangezien alle staten tot stand komen en
voortbestaan door agressie tegen hun onderdanen en het verwerven van hun
grondgebied, en aangezien oorlogen tussen staten onschuldige burgers
doden, zijn deze oorlogen altijd onrechtvaardig. Sommige oorlogen kunnen
zelfs onrechtvaardiger zijn dan andere. Guerrillaoorlogen tegen staten
voldoen op zijn minst aan de libertarische beginselen, omdat
guerrillastrijders zich richten op staatsfunctionarissen en legers en
vrijwillige methoden gebruiken om hun strijd te bemannen en te
financieren.

\section{AMERIKAANS BUITENLANDS
BELEID}\label{amerikaans-buitenlands-beleid}

We hebben gezien dat libertariërs als eerste verantwoordelijkheid hebben
om zich te concentreren op de invasies en agressies van hun eigen staat.
De libertariërs van Graustark moeten hun aandacht richten op het
beperken en afschalen van de Graustarkse staat. De Walldavische
libertariërs moeten proberen hun eigen staat onder controle te krijgen,
en zo verder. Bij buitenlandse zaken moeten de libertariërs in elk land
druk uitoefenen op hun regering om oorlog en buitenlandse inmenging te
vermijden, en om zich terug te trekken uit elk conflict waarin ze
betrokken zijn. Als er geen andere reden is, moeten libertariërs in de
Verenigde Staten hun kritische blik richten op de imperialistische en
oorlogszuchtige activiteiten van hun eigen regering.

Er zijn nog andere redenen voor libertariërs om zich te richten op de
invasies en buitenlandse interventies van de Verenigde Staten. Empirisch
gezien zijn de Verenigde Staten in de twintigste eeuw de meest
oorlogszuchtige, interventionistische en imperialistische regering
geweest. Zo'n uitspraak zal veel Amerikanen choqueren. We zijn immers al
tientallen jaren blootgesteld aan intense propaganda van de gevestigde
orde over de onwrikbare heiligheid, vreedzame bedoelingen en toewijding
aan gerechtigheid van de Amerikaanse regering in buitenlandse
aangelegenheden.

De expansionistische drang van de Amerikaanse staat begon aan het eind
van de negentiende eeuw steeds duidelijker vorm aan te nemen. Amerika
maakte een grote sprong over de oceaan met de oorlog tegen Spanje. Het
overheerste Cuba, veroverde Puerto Rico en de Filippijnen, en
onderdrukte op brute wijze een Filippijnse opstand voor de
onafhankelijkheid. De imperialistische expansie van de Verenigde Staten
kwam tot volle bloei tijdens de Eerste Wereldoorlog. Het ingrijpen van
president Woodrow Wilson verlengde de oorlog en de massaslachtingen, wat
onbedoeld leidde tot de verschrikkelijke verwoestingen die directe
gevolgen hadden voor de overwinning van de bolsjewieken in Rusland en de
nazi's in Duitsland. Wilson's bijzondere gave was om een piëtistische en
moralistische schuilnaam te geven aan een nieuw Amerikaans beleid van
wereldwijde interventie en overheersing. Dit beleid probeerde elk land
in het Amerikaans stramien te gieten door enerzijds radicale of
marxistische regimes te onderdrukken en anderzijds verouderde
monarchieën te handhaven. Het was Woodrow Wilson die de uitgangspunten
van het Amerikaanse buitenlandse beleid voor de rest van de eeuw
vastlegde. Bijna elke opvolgende president beschouwde zichzelf als een
Wilsoniaan en volgde zijn koers. Het was dan ook geen toeval dat zowel
Herbert Hoover als Franklin D. Roosevelt -- lange tijd gezien als
tegenpolen -- een belangrijke rol speelden in Amerika's eerste
wereldwijde kruistocht tijdens de Eerste Wereldoorlog. Beide mannen
baseerden hun toekomstige buitenlandse en binnenlandse beleid op hun
ervaringen met interventies en planning uit de oorlog. Een van de eerste
dingen die Richard Nixon deed als president, was een foto van Woodrow
Wilson op zijn bureau plaatsen.

In naam van `nationale zelfbeschikking' en `collectieve veiligheid'
heeft de Amerikaanse regering consequent gestreefd naar
wereldheerschappij en de gewelddadige onderdrukking van elke opstand
tegen de status quo, waar ter wereld die ook mag plaatsvinden. Terwijl
ze beweert `agressie' overal te bestrijden en de `politieman' van de
wereld te zijn, is ze zelf uitgegroeid tot een grote en voortdurende
agressor.

Iedereen die zo'n beschrijving van het Amerikaanse beleid afkeurt, moet
eens kijken naar de typische Amerikaanse reactie op binnenlandse of
buitenlandse crises, waar ook ter wereld. Zelfs in afgelegen gebieden
die geen directe of indirecte bedreiging vormen voor de levens en de
veiligheid van het Amerikaanse volk, wordt er groot alarm geslagen.
Wanneer de militaire dictator van `Bumblestan' in gevaar komt,
bijvoorbeeld omdat zijn onderdanen het zat zijn om door hem en zijn
collega's te worden uitgebuit, maken de Verenigde Staten zich
onmiddellijk zorgen. Journalisten die goed bevriend zijn met het State
Department of het Pentagon luiden de noodklok over de mogelijke gevolgen
voor de `stabiliteit' van Bumblestan en de omliggende regio als de
dictator omvergeworpen zou worden. Hij is immers een `pro-Amerikaanse'
of `pro-Westerse' dictator; hij is één van `ons' in plaats van `van
hen'. In allerijl worden er miljoenen, zelfs miljarden, dollars aan
militaire en economische hulp gestuurd om de Bumblestaanse
veldmaarschalk in het zadel te houden. Als `onze' dictator gered is,
slaakt men een zucht van verlichting en feliciteert men hem met de
redding van `onze' staat. De voortdurende of voorgenomen onderdrukking
van de Amerikaanse belastingbetalers en de Bumblestaanse burgers wordt
daarbij uiteraard niet meegenomen. Stel dat de Bumblestaanse dictator
toch valt; dan zou de hysterie het Amerikaanse nieuws en het
ambtenarenapparaat tijdelijk kunnen treffen. Maar na verloop van tijd
blijkt dat het Amerikaanse volk zijn leven na het `verlies' van
Bumblestan net zo goed kan voortzetten als daarvoor -- misschien zelfs
beter, omdat het betekent dat er een paar miljard minder aan
buitenlandse hulp wordt onttrokken om de Bumblestaanse staat overeind te
houden.

Als men begrijpt en verwacht dat de Verenigde Staten in iedere crisis
wereldwijd zullen proberen hun wil op te leggen, is dat een duidelijke
aanwijzing dat Amerika een grote interventiemacht en imperialistische
kracht is. De enige plek waar de Verenigde Staten momenteel niet hun
invloed proberen uit te oefenen, is de Sovjetunie en andere
communistische landen. In het verleden hebben ze dat echter wel
geprobeerd. Woodrow Wilson werkte jarenlang samen met Groot-Brittannië
en Frankrijk om het bolsjewisme te bestrijden. Amerikaanse en
geallieerde troepen werden naar Rusland gestuurd om de Tsaristische
(`Witte') strijdkrachten te ondersteunen in hun pogingen om de Roden te
verslaan. Na de Tweede Wereldoorlog deden de Verenigde Staten hun
uiterste best om de Sovjets uit Oost-Europa te verdrijven. Ze slaagden
erin om ze uit Azerbeidzjan in het noordwesten van Iran te verjagen. Ook
hielpen ze de Britten om een communistisch regime in Griekenland te
verpletteren. Daarnaast probeerden de Verenigde Staten de dictatoriale
heerschappij van Chiang Kai-shek in China te handhaven. Ze vlogen met
veel van Chiangs troepen naar het noorden om Mantsjoerije te bezetten,
toen de Russen zich na de oorlog terugtrokken. Ook blijven ze
verhinderen dat de Chinezen hun nabijgelegen eilanden, Quemoy en Matsu,
bezetten. Nadat ze dictator Batista in Cuba vrijwel hadden
geïnstalleerd, deden de Verenigde Staten er alles aan om het
communistische regime van Castro omver te werpen. Dit varieerde van de
door de CIA georganiseerde invasie in de Varkensbaai tot de pogingen van
de CIA en de maffia om Castro te vermoorden.

Van alle recente oorlogen was de oorlog in Vietnam zonder twijfel de
meest ingrijpende voor Amerikanen en hun opvattingen over buitenlands
beleid. Deze imperialistische oorlog vormde een microkosmos van wat er
tragisch misging met het Amerikaanse buitenlands beleid in deze eeuw. De
Amerikaanse interventie in Vietnam begon, in tegenstelling tot wat veel
mensen denken, niet met Kennedy, Eisenhower of zelfs Truman. Het begon
al op 26 november 1941, toen de Amerikaanse regering onder Franklin
Roosevelt een scherp en beledigend ultimatum aan Japan stelde. Dit
ultimatum eiste dat Japan zijn strijdkrachten uit China en Indochina,
het latere Vietnam, zou terugtrekken. Dit zette definitief de toon voor
Pearl Harbor. Verweven in een oorlog in de Stille Oceaan, die gericht
was op het verdrijven van Japan van het Aziatische continent, steunden
de Verenigde Staten en hun OSS (de voorloper van de CIA) de door
communisten geleide nationale verzetsbeweging van Ho Chi Minh. Na de
Tweede Wereldoorlog had de communistische Viet Minh de controle over
heel Noord-Vietnam. Echter, Frankrijk, dat eerder de imperialistische
heerser over Vietnam was, verraadde zijn overeenkomst met Ho Chi Minh en
bloedde de Viet Minh-strijdkrachten uit. Dit verraad werd ondersteund
door Groot-Brittannië en de Verenigde Staten.

Toen de Fransen verloren van de heropgerichte Viet
Minh-guerrillabeweging onder leiding van Ho, steunden de Verenigde
Staten de Overeenkomst van Genève in 1954. Deze overeenkomst voorzag in
een snelle hereniging van Vietnam als één natie. Het was namelijk
algemeen erkend dat de naoorlogse verdeling van het land in Noord en
Zuid puur willekeurig was en enkel voor militair gemak was bedoeld. Maar
nadat de Verenigde Staten erin slaagden om de Viet Minh met een list uit
de zuidelijke helft van Vietnam te verdrijven, schonden ze de
Overeenkomst van Genève. Ze vervingen de Fransen en hun marionetkeizer
Bao Dai door hun eigen handlangers, Ngo Dinh Diem en zijn familie, die
een dictatoriaal regime installeerden in Zuid-Vietnam. Toen Diem in
diskrediet raakte, organiseerde de CIA een staatsgreep om hem te laten
vermoorden en te vervangen door een ander dictatorial regime. Om de Viet
Cong, de door communisten geleide nationale onafhankelijkheidsbeweging
in het Zuiden, te onderdrukken, veroorzaakten de Verenigde Staten enorme
verwoestingen in zowel Zuid- als Noord-Vietnam. Ze bombardeerden en
doodden een miljoen Vietnamezen en stuurden een half miljoen Amerikaanse
soldaten naar de moerassen en jungles van Vietnam.

Tijdens het tragische conflict in Vietnam hielden de Verenigde Staten de
schijn op dat het ging om een oorlog van `agressie' van de
communistische Noord-Vietnamese staat tegen een bevriende,
`pro-Westerse' Zuid-Vietnamese staat. Dit laatste maakte ook niet veel
duidelijk. De Zuid-Vietnamese leiders hadden immers om hulp gevraagd. In
werkelijkheid was de oorlog een langdurige, maar gedoemde poging van de
Verenigde Staten om de wensen van de meeste Vietnamese mensen te
onderdrukken. Ze wilden impopulaire, dictatoriale heersers in het zuiden
handhaven, desnoods door middel van bijna genocide.

Amerikanen zijn vaak terughoudend om de term `imperialisme' te gebruiken
in verband met de acties van hun regering, maar het beschrijft hun
gedrag behoorlijk treffend. Imperialisme kan in de breedste zin worden
gedefinieerd als de agressie van staat A tegen de bevolking van land B,
gevolgd door de gedwongen handhaving van zo'n buitenlandse heerschappij.
In ons voorbeeld zou de permanente controle van de staat Graustark over
het voormalige noordoostelijke Belgravia een voorbeeld zijn van dit
soort imperialisme. Imperialisme hoeft echter niet altijd de vorm van
directe heerschappij over een buitenlandse bevolking aan te nemen. In de
twintigste eeuw heeft indirecte invloed, ook wel `neo-imperialisme'
genoemd, steeds meer de overhand gekregen. Deze vorm is subtieler en
minder zichtbaar, maar daarom niet minder effectief. In deze situatie
oefent de imperialistische staat controle uit over de buitenlandse
bevolking via inheemse cliënt-heersers. De libertaire historicus Leonard
Liggio heeft deze moderne versie van westers imperialisme scherp
gedefinieerd:

\begin{quote}
De imperialistische macht van de westerse landen legde de volkeren van
de wereld een dubbel of versterkt systeem van uitbuiting op. Hierbij
ondersteunden de westerse regeringen de lokale heersende klassen in ruil
voor de mogelijkheid om de westerse uitbuiting te combineren met de
bestaande uitbuiting door lokale staten.
\end{quote}

Deze visie op Amerika als een wereldmacht met een langdurige
imperialistische geschiedenis heeft de afgelopen jaren aan populariteit
gewonnen onder historici. Dit komt dankzij het overtuigende en
wetenschappelijke werk van een vooraanstaande groep revisionistische
historici van Nieuw Links, geïnspireerd door professor William Appleman
Williams. Interessant genoeg deelden zowel conservatieve als
klassiek-liberale `isolationisten' deze opvatting tijdens de Tweede
Wereldoorlog en in de vroege jaren van de Koude Oorlog.

\section{ISOLATIONISTISCHE KRITIEK}\label{isolationistische-kritiek}

Amerikanen zijn vaak terughoudend om de term `imperialisme' toe te
passen op de acties van hun regering, maar dit woord past bijzonder
goed. In de breedste zin kan imperialisme worden gedefinieerd als de
agressieve maatregelen van staat A tegen de bevolking van land B,
gevolgd door de gedwongen handhaving van zo'n buitenlandse controle. In
ons voorbeeld zou de blijvende heerschappij van de staat Graustark over
het voormalige noordoostelijke Belgravia zo'n voorbeeld zijn van
imperialisme. Imperialisme hoeft echter niet altijd te resulteren in
directe heerschappij over een buitenlandse bevolking. In de twintigste
eeuw nam de indirecte vorm van `neo-imperialisme' in toenemende mate de
plaats in van het traditionele imperialisme. Deze subtiele en minder
zichtbare vorm is echter niet minder effectief. In dit geval oefent de
imperialistische staat controle uit over de buitenlandse bevolking via
lokale heersers. Leonard Liggio, een libertaire historicus, heeft deze
moderne vorm van westers imperialisme als volgt gedefinieerd: `De
imperialistische macht van de westerse landen legde de volkeren van de
wereld een dubbel of versterkt systeem van uitbuiting op, waarbij de
westerse regeringen de lokale heersende klasse ondersteunt in ruil voor
de mogelijkheid om de westerse uitbuiting te combineren met de bestaande
uitbuiting door lokale staten.' Deze opvatting over Amerika als een
wereldmacht met een lange geschiedenis van imperialisme heeft de
afgelopen jaren aan populariteit gewonnen onder historici. Dit is te
danken aan het overtuigende en wetenschappelijke werk van een
vooraanstaande groep revisionistische historici van de Nieuwe Linkse
School, geïnspireerd door professor William Appleman Williams.
Interessant genoeg deelden zowel conservatieve als klassiek-liberale
`isolationisten' deze visie tijdens de Tweede Wereldoorlog en in de
vroege jaren van de Koude Oorlog.

De laatste uiting van anti-interventionisme en anti-imperialisme van de
oude conservatieve en klassiek-liberale isolationisten vond plaats
tijdens de Koreaanse Oorlog. De conservatief George Morgenstern,
hoofdredacteur van de Chicago Tribune en auteur van het eerste
revisionistische boek over Pearl Harbor, publiceerde toen een artikel in
het rechtse weekblad Human Events in Washington. In dit artikel
beschreef hij uitgebreid de gruwelijke imperialistische staat van dienst
van de Amerikaanse regering, vanaf de Spaans-Amerikaanse Oorlog tot aan
Korea. Morgenstern opmerkte dat de `verheven onzin' waarmee president
McKinley de oorlog tegen Spanje rechtvaardigde\ldots{}

\begin{quote}
Iedereen die later de evangelische rationalisaties hoorde van Wilson om
in te grijpen in de Europese oorlog, de beloften van Roosevelt over het
millennium, de `kruistocht in Europa' die Eisenhower koesterde - die
uiteindelijk verkeerd afliep - of de heilige oorlog in Korea gepredikt
door Truman, Stevenson, Paul Douglas of de New York Times, was hiervan
op de hoogte.
\end{quote}

In een bekende toespraak, tijdens het dieptepunt van de Amerikaanse
nederlaag in Noord-Korea door de Chinezen eind 1950, riep de
conservatieve isolationist Joseph P. Kennedy op tot een terugtrekking
van de Verenigde Staten uit Korea. Kennedy stelde: `Natuurlijk ben ik
tegen het communisme, maar als delen van Europa of Azië communistisch
willen worden of zelfs het communisme willen omarmen, kunnen we dat niet
tegenhouden. Het resultaat van de Koude Oorlog, de Truman Doctrine en
het Marshallplan is een ramp geweest --- een mislukking om vrienden te
winnen en een dreigende landoorlog in Europa of Azië.' Kennedy
waarschuwde dat:

\begin{quote}
De helft van deze wereld zal zich nooit onderwerpen aan het dictaat van
de andere helft. Wat hebben wij ermee te maken om de Franse koloniale
politiek in Indochina te steunen of om de democratische ideeën van de
heer Syngman Rhee in Korea te verwezenlijken? Zullen we nu de mariniers
de bergen van Tibet insturen om de Dalai Lama op zijn troon te houden?
\end{quote}

Economisch gezien voegde Kennedy hieraan toe dat we onszelf onnodige
schulden hebben opgelegd door het beleid van de Koude Oorlog. Blijven we
onze economie verzwakken `met buitensporige uitgaven aan buitenlandse
naties of in buitenlandse oorlogen, dan lopen we het risico een nieuw
1932 te veroorzaken en het systeem te vernietigen dat we juist proberen
te redden.'

Kennedy concludeerde dat het enige redelijke alternatief voor Amerika
was om het buitenlands beleid van de Koude Oorlog volledig stop te
zetten: `Weg uit Korea' en weg uit Berlijn en Europa. De Verenigde
Staten konden de Russische legers niet tegenhouden als ze ervoor kozen
Europa binnen te vallen. Mocht Europa communistisch worden, dan zou het
communisme\ldots{}

\begin{quote}
van zichzelf kunnen breken als een verenigde kracht. Hoe meer mensen er
geregeerd moeten worden, hoe noodzakelijker het wordt voor de heersers
om zich te verantwoorden tegenover de onderdanen. Hoe meer volkeren
onder het juk van een machthebber vallen, des te groter is de kans op
opstand.
\end{quote}

En hier, in een tijd waarin koude strijders voorspelden dat de wereld
zou uitgroeien tot een monolithisch communisme, verwees Joseph Kennedy
naar Tito als de voorbode van het uiteenvallen van de communistische
wereld. `Het is niet waarschijnlijk dat Mao in China zijn orders van
Stalin zal aannemen.'

\begin{quote}
Kennedy realiseerde zich dat het enige redelijke alternatief voor
Amerika was om het buitenlands beleid van de Koude Oorlog volledig stop
te zetten. `Weg uit Korea', `weg uit Berlijn' en `weg uit Europa'. De
Verenigde Staten konden de Russische legers niet tegenhouden als ze
kozen om Europa binnen te vallen. Mocht Europa communistisch worden, dan
zou het communisme als het ware van binnenuit kunnen instorten als een
verenigde kracht. Hoe meer mensen er geregeerd moeten worden, hoe
belangrijker het voor de heersers wordt om zich te verantwoorden
tegenover de onderdanen. Hoe meer volkeren onder het juk van een
machthebber leven, hoe groter de kans op opstand. En juist in een tijd
waarin koude strijders voorspelden dat de wereld zou veranderen in een
monolithisch communisme, verwees Joseph Kennedy naar Tito als de
voorbode van het uiteenvallen van de communistische wereld. `Het is niet
waarschijnlijk dat Mao in China zijn orders van Stalin zal aannemen.'

Dit beleid zal natuurlijk bekritiseerd worden als verzoening. Maar het
is ook verzoening als je je terugtrekt uit onverstandige verbintenissen.
Als het verstandig is om geen verplichtingen aan te gaan die onze
veiligheid in gevaar brengen, en dit wordt als verzoening beschouwd, dan
sta ik achter verzoening.
\end{quote}

Kennedy concludeerde dat 'de suggesties die ik doe {[}Amerikaanse levens
zouden sparen voor Amerikaanse doelen en niet verspild zouden worden in
de ijskoude heuvels van Korea of op de door gevechten geteisterde
vlaktes van West-Duitsland.'6

Een van de scherpste en krachtigste aanvallen op het Amerikaanse
buitenlandse beleid na de Koreaanse Oorlog kwam van de ervaren
klassiek-liberale journalist Garet Garrett. Hij begon zijn pamflet,
\emph{The Rise of Empire} (1952), met de verklaring: `We hebben de grens
overschreden tussen de republiek en het rijk.' Hij koppelde deze
stelling expliciet aan zijn opvallende pamflet uit de jaren dertig,
\emph{The Revolution Was}, waarin hij het optreden van uitvoerende en
statelijke tirannie binnen de republikeinse structuur onder de New Deal
aan de kaak stelde. Garrett zag opnieuw een `revolutie binnen de vorm'
van de oude constitutionele republiek. Zo noemde hij Trumans interventie
in Korea, zonder oorlogsverklaring, een `usurpatie' van de macht van het
Congres.

In zijn pamflet somde Garrett de kenmerken op die een imperium
definiëren. Het eerste kenmerk is de overheersing van de uitvoerende
macht. Dit wordt duidelijk in de ongeoorloofde interventie van de
president in Korea. Het tweede kenmerk is de ondergeschiktheid van het
binnenlands beleid aan het buitenlands beleid. Het derde kenmerk is de
`opgang van de militaire geest.' Het vierde kenmerk betreft een `systeem
van satellietnaties.' Tot slot is er het vijfde kenmerk: `een complex
van grootspraak en angst.' Dit omvat een opschepperij over onbeperkte
nationale macht, gecombineerd met een voortdurende angst voor de vijand,
de `barbaar,' en de onbetrouwbaarheid van satellietbondgenoten. Garrett
vond dat elk van deze criteria volledig van toepassing was op de
Verenigde Staten.

Nadat hij had vastgesteld dat de Verenigde Staten alle kenmerken van een
imperium hadden ontwikkeld, voegde Garrett eraan toe dat zij zich, net
als eerdere imperia, `een gevangene van de geschiedenis' voelden. Achter
deze angst schuilt namelijk `collectieve veiligheid' en de vervulling
van de vermeende voorbestemde Amerikaanse rol op het wereldtoneel.
Garrett concludeerde:

\begin{quote}
Het is aan ons.

Onze beurt om iets te doen?

Het is aan ons om de verantwoordelijkheden van moreel leiderschap in de
wereld op ons te nemen.

Het is aan ons om een machtsevenwicht te handhaven tegen de krachten van
het kwaad, waar die zich ook bevinden: in Europa, Azië, Afrika, de
Atlantische Oceaan of de Stille Oceaan. Dit kwaad, in dit geval, is de
Russische barbaar.

Onze beurt om de vrede in de wereld te bewaren.

Het is aan ons om de beschaving te redden.

Het is onze beurt om de mensheid te dienen.

Maar dit is de taal van het Rijk. Het Romeinse Rijk heeft nooit
getwijfeld aan zijn rol als verdediger van de beschaving. Zijn
bedoelingen waren gebaseerd op vrede, recht en orde. Het Spaanse Rijk
voegde daar verlossing aan toe. Het Britse Rijk introduceerde de nobele
mythe van de last van de westerse wereld. Wij hebben vrijheid en
democratie toegevoegd. Maar hoe meer men toevoegt, hoe meer het dezelfde
taal blijft. Een taal van macht.
\end{quote}

\section{OORLOG ALS GEZONDHEID VAN DE
STAAT}\label{oorlog-als-gezondheid-van-de-staat}

Het is onze beurt om de beschaving te redden. Het is aan ons om de
mensheid te dienen. Maar dit is de taal van het Rijk. Het Romeinse Rijk
twijfelde nooit aan zijn rol als verdediger van de beschaving. De
intenties waren gericht op vrede, recht en orde. Het Spaanse Rijk voegde
daar verlossing aan toe. Het Britse Rijk introduceerde de nobele mythe
van de last van de westerse wereld. Wij hebben vrijheid en democratie
toegevoegd. Maar hoe meer we toevoegen, desto meer blijft het dezelfde
taal. Een taal van macht.

Veel libertariërs voelen zich ongemakkelijk als het gaat om buitenlands
beleid. Ze besteden hun energie liever aan fundamentele vraagstukken
binnen de libertarische theorie of aan `binnenlandse' zaken zoals de
vrije markt, het privatiseren van de post of de vuilnisophaaldienst.
Toch is een kritiek op oorlog en oorlogszuchtig buitenlands beleid van
groot belang voor libertariërs. Daarvoor zijn twee belangrijke redenen.
De eerste is inmiddels een cliché, maar blijft zeer relevant: het
voorkomen van een nucleaire holocaust is cruciaal. Naast de traditionele
morele en economische argumenten tegen een interventionistisch
buitenlands beleid, staat nu de constante dreiging van
wereldvernietiging op de agenda. Als de wereld zou worden vernietigd,
zou geen enkel ander probleem of ideologie --- of het nu socialisme,
kapitalisme, liberalisme of libertarisme is --- nog van belang zijn.
Daarom is een vreedzaam buitenlands beleid en het beëindigen van de
nucleaire dreiging van groot belang.

De andere reden is dat oorlog, afgezien van de nucleaire dreiging,
volgens de libertariër Randolph Bourne `de gezondheid van de staat' is.
Oorlog heeft altijd geleid tot een grote, vaak blijvende, versnelling en
intensivering van de staatsmacht over de samenleving. Het is het
perfecte excuus om alle energie en middelen van de natie te mobiliseren,
in naam van patriottische retoriek, met de overheid aan het roer. Het is
tijdens oorlog dat de staat echt tot bloei komt: hij zwelt aan in macht,
aantal, trots en absolute controle over de economie en de samenleving.
De maatschappij verandert in een kudde die haar vermeende vijanden
probeert te verslaan. Alle tegenstanders van de officiële
oorlogsinspanning worden uitgeschakeld en onderdrukt, terwijl de
waarheid vrolijk wordt verraden in naam van het zogenaamd publieke
belang. De maatschappij transformeert in een gewapend kamp, met waarden
en moraal -- zoals de libertariër Albert Jay Nock het ooit verwoorde --
van een `leger op mars.'

Het is bijzonder ironisch dat oorlog de staat altijd in staat stelt om
de energie van zijn burgers te mobiliseren, onder het mom van de
verdediging van het land tegen een beestachtige bedreiging van buitenaf.
De mythe die de basis vormt voor het idee dat oorlog een verdediging van
de staat is voor zijn onderdanen, stelt de staat echter in staat om te
groeien door oorlog. De feiten zijn echter precies omgekeerd. Oorlog kan
weliswaar worden gezien als de gezondheid van de staat, maar is
tegelijkertijd ook zijn grootste gevaar. Een staat kan alleen `sterven'
door een nederlaag in een oorlog of door een revolutie. Tijdens een
oorlog mobiliseert de staat zijn burgers dus fanatiek om voor hem te
vechten tegen een andere staat, daarbij onder het voorwendsel dat hij
hen verdedigt.

In de geschiedenis van de Verenigde Staten is oorlog doorgaans de
belangrijkste aanleiding geweest voor de vaak blijvende versterking van
de macht van de staat over de samenleving. Tijdens de oorlog van 1812
tegen Groot-Brittannië werd voor het eerst op grote schaal het moderne
inflatoire bankstelsel met fractionele reserve opgezet. Dit ging gepaard
met beschermende invoerrechten, binnenlandse federale belastingen en de
oprichting van een permanent leger en marine. Een direct gevolg van de
inflatie in oorlogstijd was bovendien de heroprichting van een centrale
bank, de Second Bank of the United States. Vrijwel al deze
beleidsmaatregelen en instellingen bleven na de oorlog bestaan. De
Burgeroorlog en het feit dat er een vrijwel eenpartijstelsel ontstond,
leidden tot de blijvende invoering van een neomercantilistisch beleid
gericht op een grotere rol voor de overheid. Dit omvatte de subsidiëring
van diverse grote zakelijke belangen door middel van beschermende
invoerrechten, enorme landtoelagen en andere subsidies voor spoorwegen,
federale accijnzen en een federaal gecontroleerd banksysteem. Ook bracht
het de eerste federale dienstplicht en de invoering van een
inkomstenbelasting met zich mee, wat gevaarlijke precedenten voor de
toekomst schept.

De Eerste Wereldoorlog markeerde de ingrijpende en ongelukkige overgang
van een relatief vrije en laissez-faire economie naar het huidige
systeem van staatsmonopolies in ons eigen land, en naar een permanente
wereldwijde interventie. De collectivistische economische mobilisatie
tijdens de oorlog, onder leiding van Bernard Baruch, voorzitter van de
War Industries Board, vervulde de opkomende droom van leiders uit het
grootkapitaal en vooruitstrevende intellectuelen. Zij streefden naar een
gekartelde en gemonopoliseerde economie, die gepland werd door de
federale overheid in nauwe samenwerking met het grootbedrijf. Dit
tijdelijke collectivisme voedde en ontwikkelde bovendien een landelijke
arbeidersbeweging die gretig haar plek wilde innemen als junior partner
in de nieuwe staats-economie. Dit collectivisme fungeerde tevens als een
model voor leiders van grote bedrijven en corporatistische politici. Zij
zagen dit als het soort permanente economie in vredestijd dat ze de
Verenigde Staten wilden opleggen. Als voedseltsaar, minister van Handel
en later president hielp Herbert C. Hoover mee aan de totstandkoming van
deze gemonopoliseerde statistische economie. Deze visie werd verder
gerealiseerd met de oprichting van verschillende agentschappen en de
aanstelling van oorlogspersoneel door Franklin D. Roosevelts New Deal.
De Eerste Wereldoorlog leidde ook tot een aanhoudende Wilsoniaanse
interventie in het buitenland, een vastlegging van het nieuw opgelegde
Federal Reserve-systeem en een permanente inkomstenbelasting op de
burgers. Daarnaast resulteerde het in hoge federale budgetten, massale
dienstplicht en nauwe banden tussen economische bloei, oorlogscontracten
en leningen aan westerse landen.

De Tweede Wereldoorlog markeerde het hoogtepunt en de vervulling van al
deze trends. Franklin D. Roosevelt bracht de onstuimige belofte van het
Wilsoniaanse binnenlandse en buitenlandse programma eindelijk tot leven.
Dit resulteerde in een permanent partnerschap tussen grote overheden,
grote bedrijven en grote vakbonden; een steeds verder uitbreidend
militair-industrieel complex; dienstplicht; voortdurende en toenemende
inflatie; en een eindeloze, dure rol als contrarevolutionaire
`politieagent' voor de hele wereld. De periode van Roosevelt, Truman,
Eisenhower, Kennedy, Johnson, Nixon, Ford en Carter (en er is weinig
wezenlijk verschil tussen deze regeringen) valt samen met wat wij
`bedrijfsliberalisme' noemen: de verwezenlijking van de bedrijfsstaat.

Het is bijzonder ironisch dat conservatieven, die in hun retoriek
pleiten voor een vrije markteconomie, zo meegaand en zelfs bewonderend
zijn ten opzichte van ons enorme militair-industrieel complex. Er is
geen grotere verstoring van de vrije markt in het huidige Amerika. Het
grootste deel van de wetenschappers en ingenieurs richt zich op
fundamenteel onderzoek voor civiele doeleinden, zoals het verhogen van
de productiviteit en de levensstandaard van consumenten. Dit wordt
echter verstoord door de verkwistende, inefficiënte en niet-productieve
militaire en ruimtevaartondernemingen. Deze uitgaven zijn net zo
verspillend, maar ook veel destructiever dan de enorme piramides van de
farao. Het is geen toeval dat de economie van Lord Keynes de ideale
vertegenwoordiger is van de liberale corporatieve staat. Keynesiaanse
economen hechten namelijk dezelfde waarde aan alle vormen van
overheidsuitgaven, of het nu gaat om piramides, raketten of
staalfabrieken. Al deze uitgaven dragen per definitie bij aan de groei
van het bruto nationaal product, hoe verkwistend ze ook zijn. Pas
recentelijk zijn veel liberalen zich bewust geworden van de negatieve
gevolgen van de verspilling, inflatie en het militarisme dat het
Keynesiaanse bedrijfsliberalisme Amerika heeft gebracht.

Naarmate de overheidsuitgaven, zowel voor militaire als civiele
doeleinden, zijn toegenomen, zijn wetenschap en industrie steeds vaker
afgeleid naar onproductieve doelen en zeer inefficiënte processen. Het
streven om consumenten zo efficiënt mogelijk tevreden te stellen, is in
toenemende mate vervangen door het vragen om gunsten door
overheidsaannemers. Dit gebeurt vaak in de vorm van zeer verspillende
`cost plus'-contracten. Politiek heeft op elk vlak de rol van de
economie overgenomen als leidraad voor de activiteiten van de industrie.
Doordat hele sectoren en regio's van ons land afhankelijk zijn geworden
van overheids- en militaire contracten, is er bovendien een groot belang
ontstaan bij het voortzetten van deze programma's. Dit gebeurt ongeacht
of er zelfs maar het geringste excuus van militaire noodzaak voorhanden
is. Onze economische welvaart is afhankelijk gemaakt van het
voortbestaan van de verslavende onproductiviteit van antiproductieve
overheidsuitgaven.

Eén van de meest scherpzinnige en profetische critici van Amerika's
deelname aan de Tweede Wereldoorlog was de klassiek-liberale schrijver
John T. Flynn. In zijn boek \emph{As We Go Marching}, geschreven midden
in de oorlog die hij zo fel had geprobeerd te voorkomen, beschuldigde
Flynn de New Deal ervan dat deze, met de belichaming in oorlogstijd,
eindelijk had geleid tot de oprichting van de corporatieve staat.
Belangrijke elementen van het grootkapitaal hadden hier sinds de
eeuwwisseling van de twintigste eeuw naar gestreefd. `Het algemene
idee,' schreef Flynn, was\ldots{}

\begin{quote}
om de samenleving opnieuw te ordenen door deze om te vormen tot een
geplande en gedwongen economie, in plaats van een vrije. Bedrijven
zouden worden samengebracht in grote gilden of een immense corporatieve
structuur. Hierin zouden de elementen van zelfbestuur en
overheidstoezicht worden gecombineerd met een nationaal economisch
politiesysteem om deze decreten af te dwingen. Dit staat tenslotte niet
zo ver af van waar de zakenwereld het over had.11
\end{quote}

De New Deal had voor het eerst geprobeerd een nieuwe maatschappij te
creëren via de National Recovery Administration en de Agricultural
Adjustment Administration. Deze krachtige motoren van `regimentatie'
werden zowel door arbeiders als door de zakenwereld geprezen. Met de
komst van de Tweede Wereldoorlog werd dit collectivistische programma
opnieuw leven ingeblazen. Het betrof `een economie ondersteund door
grote stromen schulden, onder volledige controle, waarbij bijna alle
planningsinstanties functioneren met bijna totalitaire macht binnen een
enorme bureaucratie.' Na de oorlog voorspelde Flynn dat de New Deal zou
proberen dit systeem permanent uit te breiden naar internationale
aangelegenheden. Hij stelde wijs vast dat het zwaartepunt van de grote
overheidsuitgaven na de oorlog militair zou blijven. Dit kwam omdat
miltaire uitgaven de enige vorm van overheidsuitgaven zijn waar
conservatieven nooit bezwaar tegen zouden maken. Ook arbeiders zouden
deze uitgaven verwelkomen vanwege het creëren van banen. 'Militarisme is
dus het enige grote glamoureuze project van openbare werken waarover
verschillende elementen in de gemeenschap het eens kunnen worden.'12

Flynn voorspelde dat het naoorlogse beleid van Amerika
`internationalistisch' zou zijn, maar dan met een imperialistische
inslag. Imperialisme `is natuurlijk internationaal \ldots{} in die zin
dat oorlog internationaal is,' en het zal voortkomen uit een beleid van
militarisme. `We zullen doen wat andere landen hebben gedaan; we zullen
de angst van ons volk voor de agressieve ambities van andere landen
levend houden, en zelf beginnen aan imperialistische ondernemingen.
Imperialisme zal ervoor zorgen dat de Verenigde Staten eeuwige
'vijanden' hebben, waarmee we wat Charles A. Beard later zou noemen
`eeuwigdurende oorlog voor eeuwigdurende vrede' kunnen voeren.' Want,
zei Flynn,

\begin{quote}
We zijn erin geslaagd om bases over de hele wereld te verwerven. Er is
geen enkel deel van de wereld waar problemen kunnen uitbreken zonder dat
we kunnen beweren dat onze belangen in het geding zijn. Op deze manier
moet er na de oorlog een voortdurend argument overblijven in handen van
de imperialisten voor een enorme marine en een groot leger, dat
klaarstaat om overal aan te vallen of een aanval van alle vijanden die
we onvermijdelijk zullen hebben, af te wijzen.13
\end{quote}

Een van de meest aangrijpende portretten van de veranderingen in het
Amerikaanse leven door de Tweede Wereldoorlog is geschreven door John
Dos Passos. Hij was een levenslange radicale denker en individualist,
die door de opkomst van de New Deal van `extreem links' naar `extreem
rechts' werd geduwd. Dos Passos uitte zijn bitterheid in zijn naoorlogse
roman \emph{The Grand Design}:

\begin{quote}
Thuis organiseerden we bloedbanken en burgerbescherming. We imiteerden
de rest van de wereld door concentratiekampen op te zetten, maar noemden
ze `herhuisvestingscentra'. Daar stopten we onze\ldots{}

Amerikaanse burgers van Japanse afkomst\ldots{} zonder het voordeel van
habeas corpus\ldots{}

De president van de Verenigde Staten sprak met de oprechte democrat, en
ook de leden van het Congres deden dat. In de regering waren er
toegewijde gelovigen in burgerlijke vrijheden. `We zijn nu bezig met
oorlog voeren; later zullen we ons inzetten voor alle vier de
vrijheden,' zeiden ze\ldots{}

Oorlog is een tijd van Caesars\ldots{}

En het Amerikaanse volk werd verondersteld `dank u' te zeggen voor de
eeuw van de gewone man, die werd overgeleverd aan herhuisvesting achter
prikkeldraad, zo help hem God.

We hebben geleerd. Er zijn dingen die we hebben opgestoken.

Maar we hebben niet geleerd, ondanks de Grondwet, de
Onafhankelijkheidsverklaring en de belangrijke debatten in Richmond en
Philadelphia.

hoe je de controle over het leven van mensen in handen van één man legt

en hem verstandig laat gebruiken.14
\end{quote}

\section{SOVJET BUITENLANDS BELEID}\label{sovjet-buitenlands-beleid}

In het vorige hoofdstuk hebben we het probleem van de nationale defensie
besproken, zonder ons bezig te houden met de vraag of de Russen
daadwerkelijk van plan zijn de Verenigde Staten militair aan te vallen.
Sinds de Tweede Wereldoorlog is het militaire en buitenlandse beleid van
de Verenigde Staten, althans in retorisch opzicht, gebaseerd op de
veronderstelling van een dreigende Russische aanval. Deze
veronderstelling heeft geleid tot publieke goedkeuring voor wereldwijde
Amerikaanse interventie en heeft bijgedragen aan tientallen miljarden
aan militaire uitgaven. Maar hoe realistisch is deze veronderstelling?
Hoe goed is deze gefundeerd?

Ten eerste staat het buiten kijf dat de Sovjets, samen met alle andere
marxistisch-leninisten, alle bestaande sociale systemen zouden willen
vervangen door communistische regimes. Maar zo'n verlangen betekent
natuurlijk niet dat er een reële dreiging van een aanval bestaat. Het is
vergelijkbaar met hoe een slechte wens in het privéleven niet
rechtvaardigt dat je een directe agressie verwacht. Integendeel, het
marxisme-leninisme gelooft dat de overwinning van het communisme
onvermijdelijk is. Dit gebeurt niet door de invloed van buitenaf, maar
door de opstapeling van spanningen en `tegenstellingen' binnen elke
samenleving. Volgens het marxisme-leninisme is een interne revolutie
(of, in de hedendaagse `eurocommunistische' variant, democratische
verandering) noodzakelijk voor de installatie van het communisme.
Tegelijkertijd wordt elke gedwongen oplegging van communisme van
buitenaf als verdacht, of zelfs als verstorend en contraproductief voor
echte organische sociale verandering, beschouwd. Het idee om het
communisme naar andere landen te `exporteren' met behulp van het
Sovjetleger staat volledig haaks op de marxistisch-leninistische
theorie.

We beweren zeker niet dat de Sovjetleiders nooit iets doen dat afwijkt
van de marxistisch-leninistische theorie. Maar voor zover ze zich
gedragen als de gebruikelijke heersers van een sterke Russische
natiestaat, verzwakt de suggestie van een dreigende Sovjetaanval op de
Verenigde Staten aanzienlijk. De enige basis voor zo'n dreiging, zoals
die door onze koude strijders wordt geschetst, is de vermeende
toewijding van de Sovjet-Unie aan de marxistisch-leninistische theorie
en haar uiteindelijke doel van wereldwijde communistische overwinningen.
Als de leiders van de Sovjet-Unie simpelweg als Russische dictators
zouden optreden die alleen de belangen van hun eigen natiestaat
behartigen, zou de hele redenering om de Sovjet-Unie te beschouwen als
een unieke bron van dreiging voor een militaire aanval in elkaar
storten.

Toen de Bolsjewieken in 1917 de macht grepen in Rusland, hadden ze
nauwelijks nagedacht over een toekomstig buitenlands beleid voor de
Sovjet-Unie. Ze waren ervan overtuigd dat de communistische revolutie
snel zou uitbreken in de geavanceerde industrielanden van West-Europa.
Toen deze hoop na de Eerste Wereldoorlog vervloog, namen Lenin en zijn
medebolsjewieken de theorie van `vreedzame coëxistentie' aan als het
basisbuitenlands beleid voor de communistische staat. Het idee
hierachter was als volgt: de eerste succesvolle communistische beweging
zou Sovjet-Rusland moeten dienen als een steunpunt voor andere
communistische partijen over de hele wereld. De Sovjetstaat zelf zou
zich vooral richten op vreedzame relaties met andere landen en zou niet
proberen het communisme te exporteren door middel van interstatelijke
oorlogen. Dit beleid volgde niet alleen de marxistisch-leninistische
theorie, maar had ook een zeer praktische insteek: het belangrijkste
doel van het buitenlands beleid was het voortbestaan van de bestaande
communistische staat. Dit hield in dat de Sovjetstaat nooit in gevaar
mocht worden gebracht door het uitlokken van oorlogen met andere landen.
Andere landen zou men geacht worden communistisch te worden door hun
eigen interne processen.

Zo kwamen de Sovjets, door een mix van eigen theoretische en praktische
overwegingen, al vroeg tot wat libertariërs beschouwen als de enige
juiste en principiële buitenlandse politiek. Naarmate de tijd vorderde,
werd dit beleid bovendien versterkt door een soort `conservatisme'. Dit
fenomeen treft immers alle bewegingen die enige tijd aan de macht zijn
en waarbij de belangen van het behoud van de eigen natiestaat steeds
meer prioriteit krijgen boven het oorspronkelijke ideaal van de
wereldrevolutie. Dit toenemende conservatisme onder Stalin en zijn
opvolgers versterkte het niet-agressieve beleid van `vreedzame
coëxistentie'.

De bolsjewieken beginnen hun succesverhaal door als enige politieke
partij in Rusland vanaf het begin van de Eerste Wereldoorlog te pleiten
voor een onmiddellijke terugtrekking uit de oorlog. Ze gingen zelfs nog
verder en maakten zich daarmee zeer impopulair bij het publiek door op
te roepen tot de nederlaag van `hun eigen' regering, wat ze
`revolutionair defaitisme' noemden. Toen Rusland enorme verliezen begon
te lijden en er massale desertie aan het front plaatsvond, werd de
oorlog extreem onpopulair. De bolsjewieken, onder leiding van Lenin,
bleven echter de enige partij die opriep tot een onmiddellijk einde van
de oorlog. De andere partijen zwoeren nog steeds door te vechten tegen
de Duitsers. Toen de bolsjewieken aan de macht kwamen, drong Lenin --
ondanks het hysterische verzet van zelfs de meerderheid van het centraal
comité van de bolsjewieken -- aan op het sluiten van de `verzoenende'
vrede van Brest-Litovsk in maart 1918. Hij slaagde erin Rusland uit de
oorlog te halen, zelfs tegen de prijs dat het Duitse leger alle delen
van het Russische rijk kreeg die het toen bezette, waaronder Wit-Rusland
en Oekraïne. De bolsjewieken begonnen hun regeerperiode dus niet alleen
als een vredespartij, maar bijna als een `vrede tegen elke
prijs'-partij.

Na de Eerste Wereldoorlog en de nederlaag van Duitsland viel de nieuwe
Poolse staat Rusland aan en slaagde erin een groot deel van Wit-Rusland
en Oekraïne te veroveren. Profiterend van de onrust en de burgeroorlog
in Rusland aan het einde van de oorlog, besloten verschillende andere
nationale groepen -- Finland, Estland, Letland en Litouwen -- zich los
te maken van het Russische rijk van voor de Eerste Wereldoorlog en hun
nationale onafhankelijkheid uit te roepen. Hoewel het Leninisme
nationale zelfbeschikking lijkt te steunen, was het voor de
Sovjetheersers vanaf het begin duidelijk dat de grenzen van de oude
Russische staat intact moesten blijven. Het Rode Leger heroverde
Oekraïne, niet alleen van de Witten, maar ook van de Oekraïense
nationalisten en het inheemse Oekraïense anarchistische leger van Nestor
Makhno. Daarnaast was het evident dat Rusland, net als Duitsland in de
jaren twintig en dertig, een `revisionistisch' land was ten opzichte van
de naoorlogse regeling van Versailles. Het belangrijkste doel van zowel
het Russische als het Duitse buitenlandse beleid was het heroveren van
hun grenzen van voor de Eerste Wereldoorlog, grenzen die zij beiden
beschouwden als de `ware' grenzen van hun respectievelijke staten. Het
is belangrijk op te merken dat elke politieke partij of stroming in
Rusland en Duitsland, of ze nu aan de macht waren of in de oppositie,
het eens was over dit doel van volledig herstel van het nationale
territorium.

Maar het is belangrijk te benadrukken dat terwijl Duitsland onder Hitler
krachtige maatregelen nam om verloren gebieden te heroveren, de
voorzichtige en conservatieve Sovjetheersers helemaal niets deden. Pas
na het Stalin-Hitlerpact en de Duitse verovering van Polen heroverden de
Sovjets, die nu geen gevaar meer zagen, hun verloren gebieden. Ze
heroverden Estland, Letland en Litouwen, evenals de oude Russische
gebieden van Wit-Rusland en Oekraïne, die vroeger het oosten van Polen
waren. Dit gebeurde zonder enige weerstand. Het oude Rusland van voor de
Eerste Wereldoorlog was nu hersteld, met uitzondering van Finland.
Finland was echter bereid om te vechten. De Russen eisten niet de
volledige herintegratie van Finland, maar alleen van delen van de
Karelische Isthmus die etnisch Russisch waren. Toen de Finnen deze eis
weigerden, ontstond de `Winteroorlog' (1939-1940) tussen Rusland en
Finland, die eindigde met het opgeven door de Finnen van enkel Russisch
Karelië.

Op 22 juni 1941 lanceerde Duitsland, dat zich in het Westen
triomfantelijk voelde tegenover iedereen behalve Engeland, een
plotselinge, massale en ongeproviseerde aanval op Sovjet-Rusland. Deze
daad van agressie werd ondersteund door andere pro-Duitse staten in
Oost-Europa, zoals Hongarije, Roemenië, Bulgarije, Slowakije en Finland.
De invasie van Rusland door de Duitse en geallieerde troepen werd al
snel een van de belangrijkste gebeurtenissen in de Europese
geschiedenis. Stalin was totaal onvoorbereid op de aanval. Hij
vertrouwde zo op de rationaliteit van het Duits-Russische akkoord voor
vrede in Oost-Europa, dat hij het Russische leger in verval had laten
raken. In feite was Stalin zo weinig oorlogszuchtig dat Duitsland bijna
in staat was om Rusland te veroveren, ondanks de enorme tegenslagen die
ze te verduren hadden. Doordat Duitsland anders wellicht in staat zou
zijn geweest om de controle over Europa voor onbepaalde tijd vast te
houden, werd Hitler gedreven door de sirene van de anticommunistische
ideologie. Dit weerhield hem ervan een rationele en voorzichtige koers
te varen en leidde hem naar wat uiteindelijk zijn nederlaag zou worden.

De mythologie van de koude strijders erkent vaak dat de Sovjets tot de
Tweede Wereldoorlog niet internationaal agressief waren. Ze zijn zelfs
gedwongen deze bewering te doen, omdat de meeste koude strijders de
alliantie van de Verenigde Staten met Rusland tegen Duitsland tijdens de
oorlog van harte goedkeuren. Volgens hen werd Rusland pas tijdens en
onmiddellijk na de oorlog expansionistisch en drong het Oost-Europa
binnen.

Wat deze beschuldiging over het hoofd ziet, is het centrale feit van de
Duitse en geallieerde aanval op Rusland in juni 1941. Er is geen twijfel
dat Duitsland en haar bondgenoten deze oorlog hebben gestart. Om de
indringers te verslaan, was het daarom essentieel dat de Russen de
binnenvallende legers terugdrongen en Duitsland, evenals de andere
oorlogvoerende landen in Oost-Europa, veroverden. Het is gemakkelijker
te beweren dat de Verenigde Staten expansionistisch waren doordat ze
Italië en een deel van Duitsland veroverden en bezetten, dan het is om
de acties van Rusland te rechtvaardigen; tenslotte werden de Verenigde
Staten nooit direct door de Duitsers aangevallen.

Tijdens de Tweede Wereldoorlog waren de Verenigde Staten,
Groot-Brittannië en Rusland, de drie grote geallieerden, overeengekomen
om gezamenlijk alle veroverde gebieden militair te bezetten. De
Verenigde Staten waren de eersten die deze overeenkomst tijdens de
oorlog schonden door Rusland geen enkele rol te geven in de militaire
bezetting van Italië. Ondanks deze ernstige schending toonde Stalin zijn
blijvende voorkeur voor de conservatieve belangen van de Russische
natiestaat boven de steun voor de revolutionaire ideologie. Hij verried
herhaaldelijk de inheemse communistische bewegingen. Om vreedzame
relaties tussen Rusland en het Westen te behouden, probeerde Stalin
voortdurend het succes van verschillende communistische bewegingen te
ondermijnen. In Frankrijk en Italië had hij succes, waar de
communistische partizanen gemakkelijk de macht hadden kunnen grijpen na
de Duitse terugtrekking. Maar Stalin beval hen dit niet te doen en
overtuigde hen in plaats daarvan zich aan te sluiten bij coalitieregimes
onder leiding van anticommunistische partijen. In beide landen werden de
communisten al snel uit de coalitie gezet. In Griekenland, waar de
communistische partizanen bijna de macht grepen, verzwakte Stalin hen
door hen in de steek te laten. Hij spoorde hen aan de macht over te
dragen aan de binnenvallende Britse troepen.

In andere landen, vooral daar waar communistische partizanen sterk
waren, wezen de communisten Stalins verzoeken resoluut af. In
Joegoslavië weigerde de zegevierende Tito om aan Stalins eis te voldoen
dat hij zich ondergeschikt moest maken aan de anticommunist Mihailovich
in een regeringscoalitie. Evenzo weigerde Mao een soortgelijke eis van
Stalin om zich ondergeschikt te stellen aan Chiang Kai-shek. Deze
afwijzingen markeerden onmiskenbaar het begin van latere, cruciale
scheuringen binnen de wereldcommunistische beweging.

Rusland regeerde Oost-Europa als militaire bezetter, nadat het een
oorlog had gewonnen die tegen haar was begonnen. Het oorspronkelijke
doel van Rusland was niet om Oost-Europa te communiseren met behulp van
het Sovjetleger. Het was eerder belangrijk voor haar om ervoor te zorgen
dat Oost-Europa geen snelweg zou worden voor een aanval op Rusland,
zoals in de afgelopen halve eeuw drie keer was gebeurd. De laatste keer
resulteerde in een oorlog waarbij meer dan 20 miljoen Russen omkwamen.
Kortom, Rusland wilde landen aan haar grenzen die militair gezien niet
anticommunistisch zouden zijn en die niet als springplank voor een
nieuwe invasie zouden worden gebruikt. De politieke situatie in
Oost-Europa was zodanig dat er alleen in het meer gemoderniseerde
Finland niet-communistische politici waren die Rusland kon vertrouwen om
een vreedzaam buitenlands beleid te volgen. In Finland was deze situatie
te danken aan één vooruitziende staatsman, de agrarische leider Julio
Paasikivi. Omdat Finland sindsdien de `Paasikivi-lijn' vastberaden heeft
gevolgd, was Rusland bereid zijn troepen uit Finland terug te trekken en
niet te dringen op de communisering van dat land, ook al had het in de
zes jaar daarvoor twee oorlogen met Finland gevoerd.

Zelfs in andere Oost-Europese landen hield Rusland na de oorlog nog
enkele jaren vast aan coalitieregeringen. Pas in 1948 werden deze landen
volledig gecommuniseerd, na drie jaar van onophoudelijke Amerikaanse
druk in het kader van de Koude Oorlog om Rusland uit deze gebieden te
verdrijven. In andere regio's trok Rusland zonder moeite zijn troepen
terug uit Oostenrijk en Azerbeidzjan.

De Koude Oorlog-strijders vinden het lastig om de Russische acties in
Finland te verklaren. Als Rusland altijd vastbesloten is om waar
mogelijk een communistische heerschappij op te leggen, waarom dan die
`zachte lijn' in Finland? De enige plausibele verklaring is dat de
motivatie voortkomt uit de behoefte aan veiligheid van de Russische
natiestaat tegen aanvallen. Het succes van het wereldcommunisme speelt
hierbij slechts een ondergeschikte rol.

In feite zijn de koude strijders nooit in staat geweest om de diepere
verdeeldheid binnen de wereldwijde communistische beweging te verklaren
of te begrijpen. Als alle communisten immers worden geleid door dezelfde
ideologie, zou je verwachten dat elke communist overal deel uitmaakt van
één verenigde monolith. Dit zou hen, gezien het vroege succes van de
bolsjewieken, tot ondergeschikten of `agenten' van Moskou maken. Maar
als communisten vooral worden gedreven door hun verbondenheid met het
marxisme-leninisme, hoe verklaar je dan de diepe kloof tussen China en
Rusland? Rusland heeft bijvoorbeeld een miljoen troepen aan de grens met
China gelegerd. En wat te denken van de vijandschap tussen de
communistische staten Joegoslavië en Albanië? Waarom is er een militair
conflict tussen de Cambodjaanse en Vietnamese communisten? Het antwoord
ligt voor de hand: zodra een revolutionaire beweging de staatsmacht
grepen, beginnen ze snel de kenmerken van een heersende klasse aan te
nemen. Hun belang verschuift naar het behouden van die staatsmacht,
waardoor de wereldrevolutie in hun ogen aan betekenis verliest. Omdat
elites binnen een staat vaak tegenstrijdige belangen hebben op het
gebied van macht en rijkdom, is het niet verwonderlijk dat
intercommunistische conflicten endemisch zijn geworden.

Sinds hun overwinning op de Duitse en aanverwante militaire agressie
tijdens de Tweede Wereldoorlog zijn de Sovjets in hun militaire beleid
conservatief gebleven. Ze gebruikten hun troepen voornamelijk om hun
grondgebied binnen het communistische blok te verdedigen, in plaats van
dit uit te breiden. Toen Hongarije in 1956 dreigde het Sovjetblok te
verlaten, of Tsjecho-Slowakije in 1968, grepen de Sovjets in met
troepen. Dat was ongetwijfeld defensief en conservatief, en niet
expansionistisch. (De Sovjets overwegen blijkbaar om Joegoslavië binnen
te vallen toen Tito het land uit het Sovjetblok haalde, maar ze werden
afgeschrikt door de indrukwekkende kwaliteiten van het Joegoslavische
leger in guerrillagevechten.) In geen enkel geval heeft Rusland zijn
troepen ingezet om het blok uit te breiden of extra gebieden te
veroveren.

Professor Stephen F. Cohen, directeur van het programma Russische
studies aan Princeton, heeft onlangs de aard van het
Sovjet-conservatisme in de buitenlandse politiek omschreven:

\begin{quote}
Het lijkt misschien vreemd dat een systeem dat is voortgekomen uit een
revolutie en nog steeds revolutionaire ideeën uitdraagt, inmiddels één
van de meest conservatieve systemen ter wereld is geworden. Toch hebben
alle factoren die als belangrijk worden beschouwd in de Sovjetpolitiek
bijgedragen aan dit conservatisme. Dit betreft onder andere de
bureaucratische traditie van de Russische regering vóór de revolutie, de
bureaucratisering van het Sovjetleven die conservatieve normen
verspreidde en een diepgewortelde klasse van ijverige verdedigers van
bureaucratische privileges heeft gecreëerd. Ook de geriatrische
samenstelling van de huidige elite en de officiële ideologie, die
jarenlang is verschoven van het creëren van een nieuwe sociale orde naar
het verheerlijken van de bestaande, spelen hierin een rol.

Met andere woorden, de belangrijkste drijfveer van het huidige
Sovjetconservatisme is het behouden van wat men al heeft, zowel binnen
als buiten de grenzen, en niet het in gevaar brengen daarvan. Een
conservatieve regering kan natuurlijk betrokken raken bij gevaarlijke
militaristische acties, zoals we hebben gezien in Tsjecho-Slowakije. Dit
zijn echter daden van imperiaal protectionisme, een soort defensief
militarisme, en dus geen revolutionair of opruiend militarisme. Het is
zeker waar dat voor de meeste Sovjetleiders, en waarschijnlijk ook voor
de meeste Amerikaanse leiders, detente geen altruïstische onderneming
is, maar eerder het nastreven van nationale belangen. In dit opzicht is
dit triest. Maar het is waarschijnlijk ook zo dat wederzijds eigenbelang
een duurzaam fundament voor detente biedt, meer dan verheven en
uiteindelijk leeg altruïsme.
\end{quote}

Ook een onberispelijke anti-Sovjet bron, zoals voormalig CIA-directeur
William Colby, is van mening dat de grootste zorg van de Sovjets vooral
defensief van aard is: het voorkomen van een nieuwe catastrofale invasie
van hun grondgebied. Zoals Colby getuigde voor de Commissie Buitenlandse
Betrekkingen van de Senaat:

\begin{quote}
Je merkt een sterke bezorgdheid, zelfs paranoia, over hun eigen
veiligheid bij de Sovjets. Ze zijn vastberaden dat ze nooit meer zullen
worden binnengevallen of de soort onrust zullen meemaken die ze in het
verleden hebben ervaren tijdens verschillende invasies. Ik denk dat ze
zichzelf zo veel mogelijk willen beschermen om ervoor te zorgen dat hen
dit niet opnieuw overkomt.17
\end{quote}

Zelfs de Chinezen hebben, ondanks hun grootspraak, een conservatief en
pacifistisch buitenlands beleid gevoerd. Ze zijn er niet alleen niet in
geslaagd om Taiwan te veroveren, dat internationaal erkend is als
onderdeel van China, maar ze hebben ook toegestaan dat de kleine
eilanden voor de kust van Quemoy en Matsu in handen bleven van Chiang
Kai-shek. Ze hebben geen actie ondernomen tegen de door de Britten en
Portugezen bezette enclaves Hongkong en Macau. Bovendien nam China de
ongebruikelijke stap om een unilateraal staakt-het-vuren af te kondigen
en zijn troepen terug te trekken tot aan de grens, nadat het met gemak
had gewonnen van de Indiase strijdkrachten in hun geëscaleerde
grensoorlog.¹

\section{HET VERMIJDEN VAN A PRIORI
GESCHIEDENIS}\label{het-vermijden-van-a-priori-geschiedenis}

Er is één idee dat zowel Amerikanen als sommige libertariërs met elkaar
verbindt, en dat hen mogelijk weerhoudt om de analyse van dit hoofdstuk
serieus te nemen: de mythe die Woodrow Wilson verspreidde, namelijk dat
democratieën per definitie vreedzaam zijn, terwijl dictaturen
onvermijdelijk oorlogszuchtig zijn. Deze bewering kwam hem goed uit om
de verantwoordelijkheid voor het meeslepen van Amerika in een onnodige
en gruwelijke oorlog te verbergen. Maar er is simpelweg geen bewijs voor
deze veronderstelling. Veel dictaturen hebben zich op zichzelf gericht
en beperkt tot het uitbuiten van hun eigen bevolking. Voorbeelden
hiervan zijn het premoderne Japan, het communistische Albanië en talloze
dictaturen in de huidige Derde Wereld. Oeganda's Idi Amin, wellicht de
wreedste en meest repressieve dictator van deze tijd, vertoont geen
enkel teken dat hij zijn regime bedreigt door buurlanden binnen te
vallen. Aan de andere kant heeft een onbetwiste democratie als
Groot-Brittannië in de negentiende eeuw en daarvoor zijn imperialisme
met geweld over de wereld verspreid.

De theoretische reden waarom de focus op democratie of dictatuur de
plank misslaat, is dat staten - alle staten - hun bevolking besturen en
beslissen of ze al dan niet oorlog voeren. Alle staten, ongeacht of ze
formeel een democratie, dictatuur of een andere bestuursvorm zijn,
worden geleid door een heersende elite. Of deze elite zal besluiten om
oorlog te voeren tegen een andere staat, hangt af van een complex web
van oorzaken. Dit omvat het temperament van de heersers, de kracht van
hun vijanden, de prikkels om oorlog te voeren en de publieke opinie.
Hoewel de publieke opinie in beide gevallen peilingen behoeft, is het
belangrijkste verschil tussen een democratie en een dictatuur als het om
oorlog gaat, dat in het eerste geval meer propaganda nodig is om de
goedkeuring van de bevolking te krijgen. Intensieve propaganda is hoe
dan ook noodzakelijk, zoals blijkt uit het onvermoeibare opinievormende
gedrag van alle moderne oorlogvoerende staten. De democratische staat
moet daar echter harder en sneller voor werken. Ook moet de
democratische staat hypocrieter omgaan met retoriek die is bedoeld om de
waarden van de massa aan te spreken: rechtvaardigheid, vrijheid,
nationaal belang, patriottisme, wereldvrede, enzovoort. In democratische
staten is de kunst van het overtuigen van de bevolking dan ook wat
verfijnder en geraffineerder. Dit geldt echter, zoals we hebben gezien,
voor alle regeringsbeslissingen en niet alleen voor oorlog of vrede.
Alle regeringen -- en vooral democratische regeringen -- moeten zich
inspannen om hun onderdanen te overtuigen dat al hun onderdrukkende
daden werkelijk in het belang van de bevolking zijn.

Wat we hebben gezegd over democratie en dictatuur geldt ook voor de
afwezigheid van een relatie tussen de mate van interne vrijheid in een
land en zijn externe agressiviteit. Sommige staten zijn in staat om
intern een aanzienlijke hoeveelheid vrijheid te waarborgen terwijl ze in
het buitenland een agressieve oorlog voeren. Andere staten kunnen intern
een totalitair regime handhaven, terwijl ze een vreedzaam buitenlands
beleid voeren. De voorbeelden van Oeganda, Albanië, China en
Groot-Brittannië zijn allemaal relevant in deze vergelijking.

Kortom, libertariërs en andere Amerikanen moeten voorzichtig zijn met
een a priori geschiede­nis. Ze moeten zich vooral hoeden voor de aanname
dat in elk conflict de staat met een democratischer systeem of meer
interne vrijheid automatisch het slachtoffer is van agressie door de
meer dictatoriale of totalitaire staat. Er is simpelweg geen historisch
bewijs voor zo'n veronderstelling. Om te bepalen wat de relatieve
rechten en onrechtvaardigheden zijn en hoe de agressie in een
buitenlands geschil zich verhoudt, is er geen alternatief voor een
gedetailleerd empirisch historisch onderzoek van het conflict zelf. Het
zal dan ook geen grote verrassing zijn als zo'n onderzoek aantoont dat
de democratische en relatief vrijere Verenigde Staten agressiever en
imperialistischer zijn geweest in buitenlandse aangelegenheden dan een
relatief totalitair Rusland of China. Aan de andere kant betekent het
toejuichen van een land omdat het minder agressief is in buitenlandse
aangelegenheden geenszins dat de waarnemer ook maar enigszins sympathiek
staat tegenover de interne staat van dienst van dat land. Het is van
groot belang -- sterker nog, het is letterlijk levensbelangrijk -- dat
Amerikanen net zo koel, helder en zonder mythe kunnen kijken naar de
staat van dienst van hun regering in buitenlandse aangelegenheden, als
ze dat in toenemende mate kunnen doen in de binnenlandse politiek.
Oorlog en de valse `externe dreiging' zijn al lange tijd de
belangrijkste middelen waarmee de staat de loyaliteit van zijn
onderdanen probeert terug te winnen. Zoals we hebben gezien, waren
oorlog en militarisme de ondergang van het klassieke liberalisme. We
mogen niet toestaan dat de staat ooit nog met deze list wegkomt.

\section{\# EEN BUITENLANDS BELEID
PROGRAMMA}\label{een-buitenlands-beleid-programma}

De theoretische reden waarom de focus op democratie of dictatuur
tekortschiet, is dat staten - alle staten - hun bevolking besturen en
beslissen of ze oorlog voeren. Alle staten, ongeacht of ze nu een
democratie, dictatuur of een andere bestuursvorm zijn, worden geleid
door een heersende elite. Of deze elite ervoor kiest om oorlog te voeren
tegen een andere staat, hangt af van een complex web van factoren,
waaronder het temperament van de heersers, de kracht van hun vijanden,
de prikkels tot oorlog en de publieke opinie. Hoewel het in beide
gevallen belangrijk is om de publieke opinie te peilen, is het
belangrijkste verschil tussen een democratie en een dictatuur dat in een
democratie meer propaganda nodig is om de goedkeuring van de bevolking
te verkrijgen. Intensieve propaganda is in ieder geval onmisbaar, zoals
we kunnen zien aan het onvermoeibare opinievormende gedrag van alle
moderne oorlogvoerende staten. Democratische staten moeten echter harder
en sneller werken. Daarnaast moeten ze ook bedrevener zijn in het
gebruik van retoriek die inspeelt op de waarden van de massa, zoals
rechtvaardigheid, vrijheid, nationaal belang, patriottisme en
wereldvrede. De kunst van het overtuigen van de bevolking in
democratische staten is dus wat verfijnder en geraffineerder. Dit geldt
echter niet alleen voor oorlog of vrede, maar voor alle
regeringsbeslissingen. Immers, alle regeringen -- vooral democratische
-- moeten zich inspannen om hun onderdanen ervan te overtuigen dat al
hun daden van onderdrukking werkelijk in het belang van de bevolking
zijn. Wat we hebben besproken over democratie en dictatuur geldt ook
voor de afwezigheid van een relatie tussen de mate van interne vrijheid
in een land en zijn externe agressiviteit. Sommige staten zijn in staat
om intern een aanzienlijke vrijheid te waarborgen terwijl ze in het
buitenland agressieve oorlogen voeren. Andere staten kunnen intern een
totalitair regime hanteren en toch een vreedzaam buitenlands beleid
voeren. De voorbeelden van Oeganda, Albanië, China en Groot-Brittannië
zijn allen relevant in deze vergelijking. Kortom, libertariërs en andere
Amerikanen moeten voorzichtig zijn met een a priori historie. Ze moeten
zich vooral hoeden voor de aanname dat de meest democratische of vrije
staat in elk conflict noodzakelijkerwijs het slachtoffer is van agressie
door een meer dictatoriale of totalitaire staat. Er is simpelweg geen
historisch bewijs voor deze veronderstelling. Om te beoordelen wat de
relatieve rechten en onrechtvaardigheden zijn, en hoe de agressie in een
buitenlands geschil zich verhoudt, is er geen alternatief voor
gedetailleerd empirisch historisch onderzoek naar het conflict zelf. Het
mag dan ook geen grote verrassing zijn als zo'n onderzoek aantoont dat
de democratische en relatief vrije Verenigde Staten agressiever en
imperialistischer zijn geweest in buitenlandse aangelegenheden dan een
relatief totalitair Rusland of China. Aan de andere kant betekent het
toejuichen van een land omdat het minder agressief is in het buitenland
geenszins dat de waarnemer in enige mate sympathiek staat tegenover de
interne staat van dienst van dat land. Het is van groot belang --
sterker nog, het is letterlijk levensbelangrijk -- dat Amerikanen met
dezelfde koelheid, helderheid en zonder mythe kijken naar de staat van
dienst van hun regering in buitenlandse aangelegenheden, zoals ze dat in
toenemende mate kunnen in de binnenlandse politiek. Oorlog en de valse
`externe dreiging' zijn al lange tijd de belangrijkste middelen waarmee
de staat de loyaliteit van zijn onderdanen probeert terug te winnen.
Zoals we hebben gezien, waren oorlog en militarisme de ondergang van het
klassieke liberalisme. We moeten ervoor zorgen dat de staat nooit meer
met deze truc wegkomt.

Om onze discussie af te ronden: de belangrijkste pijler van een
libertair buitenlands beleidsprogramma voor Amerika moet zijn dat de
Verenigde Staten hun beleid van wereldwijde interventie opgeven. Ze
moeten zich onmiddellijk en volledig terugtrekken, zowel militair als
politiek, uit Azië, Europa, Latijns-Amerika en het Midden-Oosten; met
andere woorden, van overal. Amerikaanse libertariërs zouden nu moeten
pleiten voor het terugtrekken van de VS uit alle betrokkenheid. De
Verenigde Staten zouden hun basis moeten ontmantelen, hun troepen moeten
terughalen, hun voortdurende politieke inmenging moeten stoppen en de
CIA moeten afschaffen. Daarnaast zou er een einde moeten komen aan alle
buitenlandse hulp. Dit is namelijk slechts een manier om de Amerikaanse
belastingbetaler te dwingen Amerikaanse exportproducten en begunstigde
buitenlandse staten te subsidiëren, allemaal in de naam van `het helpen
van de hongerende volkeren van de wereld'. Kortom, de regering van de
Verenigde Staten moet zich volledig terugtrekken binnen haar eigen
grenzen en overal een strikt beleid van politieke isolatie of
neutraliteit handhaven.

De mentaliteit van dit ultra-`isolationalistische' libertarische
buitenlands beleid werd in de jaren dertig verwoord door de
gepensioneerde generaal-majoor Smedley D. Butler van het Korps
Mariniers. In de herfst van 1936 stelde generaal Butler een nu vergeten
grondwetswijziging voor, een wijziging die libertariërs zou verheugen
als deze opnieuw serieus overwogen zou worden. Hier is Butlers
voorgestelde grondwetswijziging in zijn geheel:

\begin{quote}
De verplaatsing van leden van de landstrijdkrachten binnen de grenzen
van de Verenigde Staten en de Panama Canal Zone is om welke reden dan
ook verboden.

De schepen van de marine van de Verenigde Staten, evenals die van andere
takken van de strijdkrachten, mogen om welke reden dan ook niet verder
dan vijfhonderd mijl van onze kust varen, tenzij het gaat om een
humanitaire missie.

Vliegtuigen van de landmacht, marine en het Korps Mariniers mogen om
welke reden dan ook niet verder dan zevenhonderdvijftig mijl van de kust
van de Verenigde Staten vliegen.20
\end{quote}

\section{\texorpdfstring{\textbf{ONTWAPENING}}{ONTWAPENING}}\label{ontwapening}

Het doel van ontwapening is om de wereld veiliger te maken. Door het
verminderen van wapens en militaire middelen verminderen we de kans op
conflicten en geweld. Ontwapening kan op verschillende manieren
plaatsvinden, zoals via internationale verdragen, onderhandelingen
tussen landen of een zelf opgelegd beleid. Het is belangrijk om een
evenwicht te vinden tussen veiligheid en de vermindering van wapens.
Ontwapening bevordert ook vertrouwen tussen landen. Wanneer staten zich
eraan committeren om hun arsenalen te verkleinen, kan dit leiden tot
positieve relaties en samenwerking. Dit is cruciaal voor wereldwijde
stabiliteit en vrede. De weg naar ontwapening is vaak complex en vol
uitdagingen. Het vergt inzet van zowel regeringen als de gemeenschap.
Echter, het is een noodzakelijke stap richting een veiligere wereld voor
toekomstige generaties.

Strikt isolationisme en neutraliteit vormen de eerste pijler van een
libertair buitenlands beleid. Dit gaat hand in hand met de erkenning van
de belangrijkste verantwoordelijkheid van de Amerikaanse staat voor de
Koude Oorlog en voor zijn betrokkenheid bij verschillende conflicten in
deze eeuw. Gezien het isolement is het echter de vraag welk wapenbeleid
de Verenigde Staten zouden moeten hanteren. Veel oorspronkelijke
isolationisten pleitten voor een beleid van `bewapening tot de tanden'.
Maar in een nucleair tijdperk brengt een dergelijk programma een ernstig
risico met zich mee: de kans op een wereldwijde holocaust, de opkomst
van een machtig bewapende staat, en bovendien de enorme verspilling en
verstoringen die onproductieve overheidsuitgaven veroorzaken voor de
economie.

Zelfs vanuit een puur militair perspectief hebben de Verenigde Staten en
de Sovjet-Unie de capaciteit om elkaar vele malen te vernietigen. De
Verenigde Staten zouden hun nucleaire vergeldingsmacht eenvoudig kunnen
inzetten door alle wapenprogramma's te stoppen, behalve de
Polaris-onderzeeërs. Deze onderzeeërs zijn namelijk onkwetsbaar en
uitgerust met kernraketten met meervoudig gerichte kernkoppen. Voor
libertariërs, of voor iedereen die bezorgd is over de massale nucleaire
vernietiging van mensenlevens, biedt zelfs het handhaven van alleen de
Polaris-onderzeeërs nauwelijks een bevredigende oplossing. De
wereldvrede zou blijven steunen op een wankele `balans van terreur', een
situatie die altijd verstoord kan worden door een ongeluk of door de
acties van waanzinnigen aan de macht. Nee, als men zich wil beschermen
tegen de nucleaire dreiging, is het van cruciaal belang om wereldwijde
nucleaire ontwapening te realiseren. De overeenkomst van SALT in 1972 en
de onderhandelingen voor SALT II zijn slechts een aarzelende eerste stap
in die richting.

Aangezien het in het belang is van iedereen, ook van alle leiders, om
niet vernietigd te worden in een nucleaire holocaust, biedt dit
wederzijds eigenbelang een sterke, rationele basis voor het overeenkomen
en uitvoeren van een beleid van gezamenlijke en wereldwijde `algemene en
volledige ontwapening' van nucleaire en andere moderne
massavernietigingswapens. Zo'n gezamenlijke ontwapening is haalbaar,
vooral sinds de Sovjet-Unie op 10 mei 1955 Westerse voorstellen in deze
richting accepteerde. Deze acceptatie resulteerde echter in een totale
en paniekerige terugtrekking van de Westerse landen van hun eigen
voorstellen! 21

De Amerikaanse versie is al lange tijd dat we ontwapening willen,
inclusief inspectie, terwijl de Sovjets alleen geïnteresseerd zijn in
ontwapening zonder inspectie. In werkelijkheid ligt het echter anders.
Sinds mei 1955 is de Sovjet-Unie voorstander van volledige ontwapening
en onbeperkte inspectie van alles wat ontmanteld is. Aan de andere kant
zijn de Amerikanen wel voor onbeperkte inspectie, maar dan met weinig of
geen echte ontwapening! Dit was het probleem met het spectaculaire maar
in feite oneerlijke `open skies'-voorstel van president Eisenhower. Dit
voorstel verving de ontwapeningsplannen die we snel introkken na de
acceptatie door de Sovjet-Unie in mei 1955. Zelfs nu open skies in wezen
is gerealiseerd door Amerikaanse en Russische ruimtesatellieten, houdt
het controversiële SALT-akkoord uit 1972 geen werkelijke ontwapening in,
maar slechts beperkingen op de verdere nucleaire uitbreiding. Omdat de
Amerikaanse strategische macht wereldwijd is gebaseerd op nucleaire
kracht en luchtmacht, is er bovendien reden om te twijfelen aan de
oprechtheid van de Sovjet-Unie met betrekking tot de afschaffing van
kernraketten of offensieve bommenwerpers.

Er zou niet alleen gezamenlijke ontwapening van kernwapens moeten zijn,
maar ook van alle wapens die massaal over nationale grenzen kunnen
worden ingezet, vooral bommenwerpers. Juist massavernietigingswapens
zoals raketten en bommenwerpers kunnen nooit zo worden ingezet dat
onschuldige burgers worden beschermd. Bovendien zou het volledig
afschaffen van raketten en bommenwerpers elke regering, ook de
Amerikaanse, dwingen tot een beleid van isolatie en neutraliteit. Alleen
als regeringen geen offensieve oorlogswapens meer hebben, zullen ze
gedwongen worden om een beleid van isolatie en vrede te voeren. Gezien
de slechte staat van dienst van alle regeringen, inclusief de
Amerikaanse, zou het onverstandig zijn om deze voorbodes van massamoord
en vernietiging in hun handen te laten en te vertrouwen op hun goede
gedrag. Als het illegitiem is voor regeringen om zulke wapens ooit te
gebruiken, waarom zouden ze dan volledig geladen in hun niet altijd
schone handen moeten blijven?

Het contrast tussen de conservatieve en libertarische opvattingen over
oorlog en Amerikaans buitenlands beleid kwam duidelijk naar voren in een
discussie tussen William F. Buckley Jr.~en de libertariër Ronald Hamowy
in de vroege dagen van de hedendaagse libertarische beweging. Buckley
had een hekel aan de libertarische kritiek op de conservatieve
standpunten over buitenlands beleid en schreef:

\begin{quote}
In elke samenleving is er ruimte voor mensen die zich alleen maar
bezighouden met het bijhouden van hun tablets. Maar laat hen beseffen
dat ze dit alleen kunnen doen dankzij de bereidheid van de
conservatieven om offers te brengen om de Sovjet-vijand te weerstaan.
Hierdoor kunnen ze genieten van hun monnikendom en doorgaan met hun
drukke seminars over het al dan niet demunicipaliseren van de
vuilnismannen.

Waarop Hamowy zonder omhaal antwoordde:

Het mag misschien ondankbaar lijken, maar ik wil Mr.~Buckley niet
bedanken voor het redden van mijn leven. Verder ben ik ervan overtuigd
dat, als zijn standpunt de overhand krijgt en hij volhardt in zijn
ongevraagde hulp, het resultaat vrijwel zeker mijn dood zal zijn --- en
die van tientallen miljoenen anderen --- in een nucleaire oorlog of mijn
dreigende gevangenneming als `on-Amerikaan'.

Ik hecht veel waarde aan mijn persoonlijke vrijheid. Daarom vind ik dat
niemand het recht heeft om zijn beslissingen aan een ander op te
dringen. Mr.~Buckley verkiest dood boven een leven in het rood. Ik ook.
Maar ik vind het belangrijk dat iedereen zelf die keuze kan maken. Een
nucleaire holocaust zal dat voor hen doen.
\end{quote}

Daarnaast kunnen we stellen dat iedereen die dat wil, het recht heeft om
de persoonlijke keuze te maken tussen `beter dood dan rood' of `geef me
vrijheid of geef me de dood'. Wat hij echter niet mag doen, is deze
beslissingen voor anderen nemen, zoals het pro-oorlogsbeleid van het
conservatisme dat doet. Wat conservatieven eigenlijk zeggen, is: `Beter
dood dan rood' en `geef me vrijheid of geef ze de dood.' Dit zijn geen
strijdkreten van nobele helden, maar van massamoordenaars.

Mr.~Buckley heeft in één opzicht gelijk: in het nucleaire tijdperk is
het belangrijker om ons zorgen te maken over oorlog en buitenlands
beleid dan over de demunicipalisering van de afvalverwerking, hoe
belangrijk dat laatste ook mag zijn. Maar als we ons daarop
concentreren, komen we onvermijdelijk tot een tegenovergestelde
conclusie dan Buckley. We beslissen dat, omdat moderne lucht- en
raketwapens niet zo kunnen worden ingezet dat ze geen schade toebrengen
aan burgers, hun bestaan in feite veroordeeld moet worden.
Kernontwapening en het ontwapenen van luchtwapens volgen wordt dan een
groot en allesoverheersend doel dat omwille van zichzelf nagestreefd
moet worden, zelfs nog nadrukkelijker dan de demunicipalisering van
vuilnis.

\textbf{Deel III: Epiloog} In deze epiloog willen we terugblikken op de
thema's en inzichten die we in dit boek hebben besproken. We hebben
gezien hoe persoonlijke vrijheid en het recht van individuen om hun
eigen keuzes te maken van groot belang zijn. Het is niet alleen een
kwestie van filosofische overtuigingen, maar heeft ook praktische
implicaties voor onze samenleving. Het idee dat niemand het recht heeft
om beslissingen voor een ander te nemen, blijft centraal staan. Dit
principe is cruciaal in een wereld waar ideologieën steeds vaker leiden
tot conflicten. De opmerking van Mr.~Buckley dat hij liever dood is dan
`rood', raakt een gevoelige snaar. Het is begrijpelijk dat hij deze
keuze voor zichzelf maakt, maar het is essentieel dat deze vrijheid ook
voor iedereen geldt. Niemand zou gedwongen moeten worden om een bepaalde
ideologie aan te nemen. We moeten ons bewust zijn van de gevolgen van
extremisme, of het nu aan de rechterkant of linkerkant van het politieke
spectrum voorkomt. De dilemma's die we tegenkomen zijn niet zwart-wit.
In de complexe wereld van vandaag verdienen onze keuzes aandacht en
reflectie. Daarnaast hebben we besproken hoe de hedendaagse
wapensystemen niet selectief kunnen worden ingezet zonder onschuldige
levens in gevaar te brengen. Dit maakt de roep om ontwapening alleen
maar dringender. Het is belangrijk dat we ons inzetten voor nucleaire
ontwapening, zelfs als dit voorafgaat aan andere urgente zaken zoals het
verbeteren van onze afvalverwerkingssystemen. Als we vooruitkijken, is
het onze verantwoordelijk om deze waarden hoog te houden en te vechten
voor een wereld waarin vrijheid en vrede centraal staan. Wetende dat
onze keuzes grote impact hebben, zowel lokaal als globaal, ligt de
volgende stap bij ons. Laten we altijd blijven streven naar een
rechtvaardige en vrije samenleving.

\bookmarksetup{startatroot}

\chapter{Een strategie voor vrijheid}\label{een-strategie-voor-vrijheid}

\phantomsection\label{educatie-theorie-en-beweging}
\section{EDUCATIE: THEORIE EN
BEWEGING}\label{educatie-theorie-en-beweging}

En zo hebben we het: een kern van waarheid die goed is in theorie en
toepasbaar op onze politieke problemen -- het nieuwe libertarisme. Maar
hoe behalen we, nu we de waarheid kennen, de overwinning? We staan voor
een groot strategisch probleem dat alle `radicale' overtuigingen door de
geschiedenis heen heeft gekenmerkt: hoe brengen we de overgang tot stand
van onze huidige, door de staat gecontroleerde wereld naar het
uiteindelijke doel van vrijheid?

Er bestaat geen toverformule voor strategie. Elke strategie voor sociale
verandering, die gebaseerd is op overtuiging en bekering, blijft een
kunst en is geen exacte wetenschap. Dit gezegd hebbende, zijn we nog
steeds niet zonder wijsheid in onze zoektocht naar de juiste doelen. Er
kan een vruchtbare theorie zijn, of tenminste een vruchtbare discussie,
over de effectieve strategie voor verandering.

Over één punt kunnen we het nauwelijks oneens zijn: een eerste en
noodzakelijke voorwaarde voor het succes van het libertarisme -- en voor
elke sociale beweging, of het nu het boeddhisme of vegetariërisme is --
is educatie. Dit houdt in dat we grote aantallen mensen moeten
overtuigen en aanspreken. Onderwijs heeft twee essentiële aspecten: het
vestigen van aandacht op het bestaan van een bepaald systeem en het
aanzetten van mensen om het libertarische systeem te omarmen. Als onze
beweging alleen zou bestaan uit slogans, publiciteit en andere middelen
om aandacht te trekken, dan zouden we misschien veel mensen bereiken.
Maar het zou al snel blijken dat we niets substantieels te zeggen
hebben, waardoor ons publiek vluchtig en onbetrouwbaar zou zijn.
Libertariërs moeten daarom grondig nadenken en onderzoek doen. Ze moeten
theoretische en systematische boeken, artikelen en tijdschriften
publiceren en deelnemen aan conferenties en seminars. Aan de andere kant
zal een louter theoretische benadering weinig opleveren als niemand ooit
van deze boeken en artikelen heeft gehoord. Daarom is publiciteit nodig:
slogans, studentenactivisme, lezingen, radio- en tv-spots, enzovoort.
Echt onderwijs kan niet zonder theorie en activisme, zonder ideologie en
mensen die die ideologie uitdragen.

Dus, net zoals de theorie onder de aandacht van het publiek moet komen,
heeft deze ook mensen nodig die het vaandel dragen, discussies aangaan,
actie ondernemen en de boodschap uitdragen. Nogmaals, zowel de theorie
als de beweging zijn nutteloos en steriel zonder elkaar. De theorie zal
verwelken als er geen zelfbewuste beweging is die zich inzet om deze te
verspreiden. Aan de andere kant wordt de beweging zinloos als ze de
ideologie en het doel uit het oog verliest. Sommige libertarische
theoretici vinden het onrein of ongepast dat er een actieve beweging met
handelende individuen bestaat. Maar hoe kan vrijheid worden bereikt
zonder libertariërs die de zaak bevorderen? Aan de andere kant verwerpen
sommige militante activisten, in hun haast om actie te ondernemen --
welke actie dan ook -- de vermeende discussies over theorieën. Toch
wordt hun actie zinloos en verspilde energie als ze maar een vaag idee
hebben van waar ze voor strijden.

Verder hoor je libertariërs (en ook leden van andere sociale bewegingen)
vaak klagen dat ze `alleen maar tegen zichzelf praten' met hun boeken,
tijdschriften en conferenties, en dat er maar weinig mensen van de
`buitenwereld' luisteren. Deze veelgehoorde klacht miskent echter het
veelzijdige doel van `educatie' in de breedste zin. Het is niet alleen
belangrijk om anderen te onderwijzen; ook voortdurende zelfeducatie is
cruciaal. Het corps van libertariërs moet altijd proberen anderen te
werven voor hun gelederen -- dat is zeker. Maar ze moeten ook hun eigen
groep levendig en gezond houden. Zelfeducatie heeft twee belangrijke
doelen. Ten eerste helpt het bij het verfijnen en bevorderen van de
libertarische `theorie' -- het doel van onze hele onderneming.
Libertarisme, hoe relevant en waar ook, kan niet alleen in stenen
tabletten worden gegrift. Het moet een levende theorie zijn, die zich
ontwikkelt door middel van schrijven en discussiëren, en door fouten te
weerleggen en te bestrijden wanneer die zich voordoen. De libertarische
beweging heeft tientallen kleine nieuwsbrieven en tijdschriften,
variërend van gestencilde bladen tot verzorgde publicaties, die
voortdurend opkomen en weer verdwijnen. Dit is een teken van een
gezonde, groeiende beweging, bestaande uit ontelbare individuen die
nadenken, argumenteren en bijdragen leveren.

Er is nog een andere belangrijke reden om `tegen onszelf te praten',
zelfs als dat het enige gesprek is dat plaatsvindt. Dat is de
versterking: de psychologisch noodzakelijke wetenschap dat er anderen
zijn die hetzelfde denken. Mensen met wie je kunt praten, discussiëren
en in het algemeen kunt communiceren en omgaan. Op dit moment is het
libertarische credo nog steeds dat van een relatief kleine minderheid,
en bovendien bepleit het radicale veranderingen in de status quo. Daarom
voelt het als een eenzaam geloof. De herbevestiging van het bestaan van
een beweging, van `met onszelf praten', kan dat isolement doorbreken en
overwinnen. De hedendaagse beweging is inmiddels oud genoeg om een
aanzienlijk aantal overlopers te hebben gekend. Analyse van deze
overlopers toont aan dat in bijna alle gevallen de libertariër
geïsoleerd was, afgesneden van gemeenschap en interactie met zijn
lotgenoten. Een bloeiende beweging met een sterk gevoel van gemeenschap
en saamhorigheid is de beste remedie tegen het opgeven van vrijheid als
een hopeloze of `onpraktische' zaak.

\section{\texorpdfstring{\textbf{ZIJN WIJ
`UTOPISTEN'?}}{ZIJN WIJ `UTOPISTEN'?}}\label{zijn-wij-utopisten}

Er zijn verschillende opvattingen over de term `utopisten'. Sommige
mensen gebruiken het als een scheldwoord, een manier om idealisten te
ridiculiseren. Andere zien het als een uitnodiging tot dromen en het
verkennen van nieuwe mogelijkheden. Hoe je ook kijkt naar de term, het
is belangrijk om te begrijpen dat idealisme en praktische realiteit hand
in hand kunnen gaan. Veel mensen denken dat utopische ideeën alleen maar
fantasieën zijn, ontsproten aan de geest van dromers. Maar deze ideeën
kunnen ook dienen als inspiratiebron voor sociale verandering. Ze kunnen
ons aanmoedigen om buiten de gebaande paden te denken en ons te storten
op het creëren van een betere wereld. Bijvoorbeeld, de libertarische
beweging is vaak afgeschilderd als een groep die zich richt op
onrealistische idealen. Maar als we de kern van deze ideeën onderzoeken,
zien we dat ze zijn geworteld in de waarden van vrijheid en
zelfbeschikking. Deze waarden zijn niet alleen theoretisch; ze zijn
essentieel voor een samenleving waarin individuen de ruimte krijgen om
te bloeien. Een ander punt van kritiek is dat utopisten de moeilijkheden
van het leven niet volledig onder ogen zien. Maar het omarmen van een
utopisch idee betekent niet dat we de realiteit negeren. Integendeel,
het kan ons helpen om constructieve oplossingen te vinden voor de
uitdagingen waarmee we worden geconfronteerd. Door te dromen van wat
mogelijk is, kunnen we actie ondernemen en verandering teweegbrengen. In
deze discussie over utopisme en realisme is het cruciaal om te erkennen
dat hoop en idealisme ons in staat stellen om vooruit te kijken. Ze
geven ons de motivatie om te vechten voor wat we geloven en om te
streven naar een wereld die rechtvaardiger en vrijer is voor iedereen.
Dus, zijn we utopisten? Misschien, maar dat hoeft niet negatief te zijn.
Het kan juist een kracht zijn die ons aanspoort tot actie en
verandering.

Goed, we moeten onderwijs hebben via zowel theorie als een beweging.
Maar wat moet de inhoud van dat onderwijs zijn? Elk `radicaal' credo
wordt ervan beschuldigd `utopisch' te zijn en de libertarische beweging
is daarop geen uitzondering. Sommige libertariërs beweren zelfs dat we
mensen niet moeten afschrikken door `te radicaal' te zijn. Daarom zouden
we de volledige libertarische ideologie en het complete libertarische
programma buiten het zicht moeten houden. Deze mensen pleiten voor een
`Fabiaans' programma van geleidelijkheid, dat zich alleen richt op het
stap voor stap afbouwen van de staatsmacht. Een voorbeeld hiervan is
belastingheffing. In plaats van de `radicale' maatregel om alle
belastingen af te schaffen, of zelfs de inkomstenbelasting te laten
vallen, zouden we ons moeten beperken tot een oproep voor kleine
verbeteringen. Laten we zeggen: een verlaging van de inkomstenbelasting
met twee procent.

Op het gebied van strategisch denken kunnen libertariërs veel leren van
de marxisten. Deze groep denkt al langer na over strategieën voor
radicale sociale veranderingen dan wie dan ook. Marxisten wijzen twee
cruciale strategische denkfouten aan die `afwijken' van het juiste pad.
De eerste noemen ze `links sektarisme'. De tweede, een tegengestelde
afwijking, is `rechts opportunisme'. Critici van libertarische
`extremistische' principes zijn vaak te vergelijken met de marxistische
`rechtse opportunisten'. Het grote probleem met deze opportunisten is
dat ze zich te veel richten op geleidelijke en `praktische' programma's,
die een goede kans maken om snel aangenomen te worden. Hierdoor lopen ze
het risico het uiteindelijke doel, namelijk de libertarische idealen,
volledig uit het oog te verliezen. Wie zich beperkt tot het voorstellen
van een belastingverlaging van twee procent, draagt bij aan het in de
vergetelheid raken van het uiteindelijke doel: de volledige afschaffing
van belastingen. Door de focus te leggen op onmiddellijke middelen,
zorgt hij ervoor dat het uiteindelijke doel wordt weggevaagd, en daarmee
de reden waarom je libertariër bent. Als libertariërs weigeren het
vaandel van pure principes en het uiteindelijke doel hoog te houden, wie
zal dat dan doen? Het antwoord is: niemand. Daarom is een belangrijke
bron van afvalligheid binnen de beweging in de afgelopen jaren het
verkeerde pad van opportunisme geweest.

Een opvallend voorbeeld van afvalligheid door opportunisme is iemand die
we `Robert' zullen noemen. Robert werd begin jaren vijftig een toegewijd
en militant libertariër. Al snel richtte hij zich op activisme en
directe resultaten. Hij kwam tot de conclusie dat het strategisch gezien
verstandig was om alle gesprekken over het libertarische doel te
minimaliseren. Dit gold in het bijzonder voor de libertarische
vijandigheid tegenover de overheid. Zijn focus lag op het benadrukken
van het `positieve' en de prestaties die mensen konden bereiken door
middel van vrijwillige actie. Naarmate zijn carrière vorderde, begon
Robert compromisloze libertariërs als een last te beschouwen. Daarom
besloot hij systematisch iedereen in zijn organisatie te ontslaan die
`negatief' over de overheid was. Het duurde niet lang voordat Robert
openlijk afstand deed van zijn libertarische ideologie. Hij pleitte voor
een `partnerschap' tussen overheid en particuliere ondernemingen --
tussen dwang en vrijwilligheid. Kortom, hij nam openlijk plaats binnen
het establishment. Toch zal Robert in zijn momenten van reflectie zelfs
naar zichzelf verwijzen als een `anarchist', maar dan alleen in een
abstracte, onrealistische context die weinig te maken heeft met de echte
wereld.

De vrijemarkteconoom F.A. Hayek, die zelf op geen enkel gebied als een
`extremist' beschouwd kan worden, heeft overtuigend geschreven over het
cruciale belang van het hooghouden van de pure en `extreme' ideologie
voor het succes van de vrijheid. Hij stelt dat een van de belangrijkste
aantrekkingskrachten van het socialisme altijd de voortdurende nadruk op
het `ideale' doel is geweest. Dit ideaal doordringt, informeert en leidt
de acties van iedereen die ernaar streeft. Hayek voegt daar aan toe:

\begin{quote}
We moeten de opbouw van een vrije samenleving opnieuw beschouwen als een
intellectueel avontuur, een daad van moed. Wat we missen is een liberaal
utopia; een programma dat niet alleen een verdediging is van de
bestaande situatie, noch een afgezwakte vorm van socialisme. We hebben
een echt liberaal radicalisme nodig dat de zwakheden van de machtigen,
waaronder de vakbond, zonder angst aanspreekt. Het moet ook niet te
praktisch zijn en zich niet alleen richten op wat momenteel politiek
haalbaar lijkt. We hebben intellectuele leiders nodig die de
verleidingen van macht en invloed kunnen weerstaan, en die bereid zijn
voor een ideaal te strijden, hoe klein de kans op vroege verwezenlijking
ook mag zijn. Het moeten mensen zijn die vasthouden aan hun principes en
vechten voor hun volledige realisatie, hoe ver weg dat ook lijkt. Vrije
handel en kansen zijn idealen die nog steeds veel mensen kunnen
inspireren. Maar een simpele `redelijke vrijheid van handel' of
`versoepeling van controles' is intellectueel niet respectabel en zal
waarschijnlijk weinig enthousiasme opwekken. De belangrijkste les die de
echte liberaal kan leren van het succes van socialisten is dat hun moed
om utopisch te zijn hen de steun van intellectuelen heeft opgeleverd.
Dit heeft hun invloed op de publieke opinie versterkt en mogelijk
gemaakt wat eerder volkomen onhaalbaar leek. Degenen die zich enkel
richtten op wat uitvoerbaar leek binnen de huidige opinie, hebben
gemerkt dat zelfs dat snel politiek onmogelijk werd door veranderingen
in de publieke opinie die zij niet hebben kunnen beïnvloeden. Tenzij we
de filosofische basis van een vrije samenleving opnieuw tot een
levendige intellectuele kwestie kunnen maken, en de uitvoering ervan tot
een uitdaging voor de slimste en meest creatieve geesten onder ons, zijn
de vooruitzichten voor vrijheid inderdaad somber. Maar als we het geloof
in de kracht van ideeën kunnen herwinnen, dat ooit zo kenmerkend was
voor het liberalisme, dan is de strijd nog niet verloren.
\end{quote}

Hayek wijst hier een belangrijke waarheid aan en geeft een goede reden
om het uiteindelijke doel te benadrukken: de opwinding en het
enthousiasme die een logisch samenhangend systeem kan oproepen. Wie zou
er daarentegen de barricades opgaan voor een belastingverlaging van twee
procent?

Er is nog een andere belangrijke reden om vast te houden aan zuivere
principes. Hoewel dagelijkse sociale en politieke gebeurtenissen vaak
het resultaat zijn van veel druk en het vaak onbevredigende resultaat
van het geduw en getrek van tegenstrijdige ideologieën en belangen, is
het juist daarom des te belangrijker voor de libertariër om de lat hoog
te leggen. De roep om een belastingverlaging van twee procent leidt
doorgaans slechts tot een lichte matiging van een geplande
belastingverhoging. De vraag om een aanzienlijke belastingverlaging kan
echter resulteren in een substantiële vermindering. Door de jaren heen
is het de strategische rol van de `extremist' geweest om de dagelijkse
agenda steeds verder in zijn richting te duwen. Socialisten zijn hier
bijzonder bedreven in. Als we kijken naar het socialistische programma
van 60 of zelfs 30 jaar geleden, zien we dat maatregelen die destijds
als gevaarlijk socialistisch werden beschouwd, inmiddels onmisbaar zijn
binnen de `mainstream' van het Amerikaanse erfgoed. Op deze wijze worden
de dagelijkse compromissen van zogenaamde `praktische' politiek
onverbiddelijk in de collectivistische richting getrokken. Er is geen
reden dat de libertariër niet hetzelfde zou kunnen bereiken. Een van de
redenen dat de conservatieve oppositie tegen collectivisme zo zwak is,
is dat het conservatisme van nature geen consistente politieke filosofie
biedt. Het is vooral een `praktische' verdediging van de bestaande
status quo, vaak neergezet als de belichaming van de Amerikaanse
`traditie'. Maar naarmate het statisme groeit en zich uitbreidt, raakt
het per definitie steeds meer verankerd en dus `traditioneel'. Hierdoor
kan het conservatisme geen intellectuele wapens vinden om deze
omverwerping tegen te gaan.

Vasthouden aan principes betekent meer dan alleen het hooghouden van het
ultieme libertarische ideaal en het niet tegenspreken ervan. Het houdt
ook in dat je ernaar streeft dat doel zo snel mogelijk te bereiken, voor
zover dat fysiek haalbaar is. Kortom, de libertariër mag nooit een
geleidelijke benadering van zijn doelen verkiezen boven een
onmiddellijke en snelle aanpak. Want daarmee ondermijnt hij het cruciale
belang van zijn eigen doelen en principes. Als hij zelf al zo weinig
waarde hecht aan zijn idealen, hoe hoog zullen anderen ze dan waarderen?

Kortom, om de vrijheid echt na te streven, moet de libertariër ernaar
verlangen dat deze op de meest effectieve en snelste manier wordt
bereikt. Vanuit deze mentaliteit verklaarde de klassieke liberaal
Leonard E. Read, die na de Tweede Wereldoorlog pleitte voor de
onmiddellijke en totale afschaffing van prijs- en looncontroles, in een
toespraak: `Als er hier op dit podium een knop zou zijn die, zodra je
erop drukt, alle loon- en prijscontroles onmiddellijk zou opheffen, dan
zou ik hem meteen indrukken!'

De libertariër zou dus iemand moeten zijn die op de knop zou drukken
voor de onmiddellijke afschaffing van alle inbreuken op de vrijheid, als
die knop zou bestaan. Natuurlijk is hij zich ervan bewust dat zo'n
magische knop niet echt bestaat, maar zijn fundamentele voorkeur
beïnvloedt en vormt zijn hele strategische visie.

Een `abolitionistisch' perspectief betekent niet dat de libertariër de
snelheid waarmee zijn doel feitelijk bereikt kan worden, onrealistisch
inschat. William Lloyd Garrison, de libertarische abolitionist van de
slavernij, was ook niet `onrealistisch' toen hij in de jaren 1830 de
glorieuze standaard van onmiddellijke emancipatie vooropstelde. Zijn
doel was moreel juist en zijn strategisch realisme kwam voort uit het
besef dat hij niet verwachtte zijn doel snel te bereiken. In hoofdstuk 1
zagen we dat Garrison zelf het volgende onderscheid maakte: `We dringen
aan op onmiddellijke afschaffing, hoe serieus we dat ook doen, maar
uiteindelijk zal het helaas een geleidelijke afschaffing zijn. We hebben
nooit gezegd dat slavernij in één klap omvergeworpen zou worden; dat het
zo zou moeten zijn, blijven we altijd beweren.' Anders, zoals Garrison
scherp waarschuwde: `Geleidelijkheid in theorie is eeuwigheid in de
praktijk.'

Geleidelijkheid ondermijnt in theorie het doel zelf. Het houdt in dat
het doel ondergeschikt wordt gemaakt aan andere niet- of antiliberale
overwegingen. Een voorkeur voor gradualisme impliceert dat deze andere
overwegingen belangrijker zijn dan vrijheid. Stel je voor dat een
voorstander van de afschaffing van slavernij zou zeggen: `Ik ben voor
het beëindigen van de slavernij, maar pas over tien jaar.' Dit zou
impliceren dat afschaffing over acht of negen jaar, of zelfs
onmiddellijk, verkeerd zou zijn. Het zou betekenen dat het beter is
slavernij nog een tijdje voort te zetten. Dit wijst erop dat
overwegingen van rechtvaardigheid zijn losgelaten en dat het doel voor
de abolitionist of libertariër niet langer prioriteit heeft. In feite
zou dit voor zowel de abolitionist als de libertariër betekenen dat ze
pleiten voor het voortzetten van misdaad en onrecht.

Hoewel het belangrijk is voor de libertariër om zijn ultieme en
`extreme' ideaal hoog te houden, maakt dit hem, in tegenstelling tot
Hayek, nog geen `utopist'. De echte utopist is iemand die een systeem
voorstaat dat in strijd is met de natuurlijke wetten van mensen en de
echte wereld. Een utopisch systeem kan niet functioneren, zelfs niet als
iedereen wordt overgehaald om het in praktijk te brengen. Dit systeem
zou zichzelf niet kunnen handhaven. Het utopische doel van links,
namelijk communisme --- de afschaffing van specialisatie en de
acceptatie van eenheid --- kan niet slagen, zelfs niet als iedereen
bereid zou zijn het onmiddellijk te omarmen. Het zou niet kunnen
functioneren omdat het inbreuk doet op de menselijke natuur en de
werkelijkheid, vooral op de uniciteit en individualiteit van elk
persoon, op hun capaciteiten en interesses. Bovendien zou het leiden tot
een drastische afname van de productie van rijkdom, zodat het merendeel
van de mensheid gedoemd zou zijn snel te verhongeren en uit te sterven.

Kortom, de term `utopisch' verwart in de volksmond twee soorten
obstakels die zich voordoen bij een programma dat radicaal afwijkt van
de status quo. Het eerste obstakel is dat zo'n programma in strijd is
met de menselijke natuur en de werkelijkheid, waardoor het niet zou
kunnen functioneren als het eenmaal is geïmplementeerd. Dit betreft het
utopisme van het communisme. Het tweede obstakel is de uitdaging om
voldoende mensen te overtuigen van de noodzaak van het programma. Het
eerste obstakel is een gebrekkige theorie, omdat het de natuur van de
mens negeert; het tweede is een kwestie van menselijke wil, namelijk het
overtuigen van genoeg mensen van de juistheid van de doctrine. In zijn
gebruikelijke negatieve betekenis is `utopisch' alleen van toepassing op
het eerste obstakel. In wezen is de libertarische doctrine dus niet
utopisch, maar juist realistisch, omdat het de enige theorie is die
daadwerkelijk consistent is met de menselijke natuur en de wereld. De
libertariër ontkent de diversiteit van de mens niet; hij viert deze en
streeft ernaar om die diversiteit volledig tot uiting te laten komen in
een wereld van volledige vrijheid. Door dit te doen, bevordert hij ook
een enorme toename in productiviteit en de levensstandaard van iedereen.
Dit is een uiterst `praktisch' resultaat dat doorgaans door echte
utopisten als kwaadaardig `materialisme' wordt afgewezen.

De libertariër is bij uitstek realistisch omdat hij de aard van de staat
en zijn streven naar macht volledig begrijpt. Daarentegen is de
schijnbaar veel realistischere conservatief, die gelooft in een
`beperkte overheid', de echte onpraktische utopist. Deze conservatief
blijft maar herhalen dat de centrale overheid sterk beperkt moet worden
door een grondwet. Maar terwijl hij zich verzet tegen de corruptie van
de oorspronkelijke grondwet en tegen de uitbreiding van de federale
macht sinds 1789, slaagt hij er niet in de juiste les uit deze
degeneratie te trekken. Het idee van een strikt beperkte constitutionele
staat was een nobel experiment, maar het mislukte zelfs onder de meest
gunstige omstandigheden. Als het toen al mislukte, waarom zou een
soortgelijk experiment het nu beter doen? Nee, de ware onpraktische
utopist is de conservatieve laissez-fairist, de man die alle wapens en
beslissingsbevoegdheid in handen van de centrale regering legt en
vervolgens zegt: `Beperk jezelf.'

Er is nog een andere belangrijke reden waarom libertariërs het bredere
utopisme van links verachten. Linkse utopisten postuleren steeds opnieuw
een drastische verandering in de menselijke aard; voor hen heeft de mens
eigenlijk geen vaste aard. Het individu wordt gezien als oneindig
kneedbaar door de maatschappelijke instellingen. Daarom wordt aangenomen
dat het communistische ideaal --- of het socialistische overgangssysteem
--- de Nieuwe Communistische Mens zal creëren. De libertariër gelooft
echter dat elk individu uiteindelijk een vrije wil heeft en zichzelf
vormt. Het is daarom naïef om te hopen op een uniforme en radicale
verandering van mensen door de beoogde Nieuwe Orde. Natuurlijk hoopt de
libertariër wel op een morele verbetering bij iedereen, al vallen zijn
morele doelen nauwelijks samen met die van socialisten. Zo zou hij
dolgelukkig zijn als alle verlangen naar agressie tussen mensen zou
verdwijnen. Maar hij is veel te realistisch om zijn vertrouwen te
stellen in dit soort veranderingen. In plaats daarvan vertrouwt het
libertarische systeem op het idee dat het op zichzelf al veel moreler
zal zijn en beter zal functioneren dan welk ander systeem dan ook, gelet
op de bestaande menselijke waarden en houdingen. Hoe meer het verlangen
naar agressie verdwijnt, des te beter zal elk sociaal systeem werken,
inclusief het libertarische. Hoe minder behoefte er bijvoorbeeld zal
zijn aan politie of rechtssystemen. Maar het libertarische systeem is
niet afhankelijk van dergelijke veranderingen.

Als de libertariër moet pleiten voor het direct bereiken van vrijheid en
het afschaffen van statisme, en als gradualisme in theorie in strijd is
met dit allesomvattende doel, welke strategische houding kan de
libertariër dan nog aannemen in de huidige wereld? Moet hij zich
beperken tot het pleiten voor onmiddellijke afschaffing? Zijn
`overgangseisen' --- stappen richting praktische vrijheid --- per
definitie onwettig? Nee, dat zou hem in een andere zelfvernietigende
strategische val doen trappen, namelijk die van `links sektarisme'.
Libertariërs zijn vaak opportunisten geweest die hun uiteindelijke doel
uit het oog hebben verloren of onderschatten. Maar sommigen maken de
tegenovergestelde fout: zij vrezen en veroordelen elke vooruitgang naar
het doel, omdat ze denken dat dit noodzakelijkerwijs het doel zelf
ondermijnt. De tragedie is dat deze sektaristen, door elke vooruitgang
die het uiteindelijke doel voorbijschiet te veroordelen, het gekoesterde
doel zelf ijdel en nutteloos maken. Natuurlijk zouden we allemaal
dolgelukkig zijn als we in één klap totale vrijheid zouden bereiken,
maar de realistische vooruitzichten voor zo'n grote sprong zijn beperkt.
Sociale verandering gebeurt meestal niet in één sprongetje, ook al is
deze niet altijd klein en geleidelijk. Door elke overgangsbenadering
naar het doel af te wijzen, maken deze sektarische libertariërs het
onmogelijk om het doel zelf ooit te bereiken. Zo kunnen ze net zo
volledig `liquidationistisch' zijn ten opzichte van het pure doel als de
opportunisten zelf.

Soms, vreemd genoeg, ondergaat een individu veranderingen van de ene
tegenovergestelde dwaling naar de andere, terwijl het het juiste
strategische pad minacht. Zo kan een linkse sektariër, wanhopig na jaren
van vruchteloze herhalingen van zijn zuiverheid zonder echte
vooruitgang, zich in het onstuimige struikgewas van rechtse
opportunisten storten. Hij is op zoek naar enige vooruitgang op korte
termijn, zelfs als dat ten koste gaat van zijn uiteindelijke doel. Aan
de andere kant kan een rechtse opportunist, die zich ongemakkelijk
begint te voelen bij het compromitteren van zijn eigen of de
intellectuele integriteit van zijn collega's en hun uiteindelijke
doelen, zich aansluiten bij het linkse sektarisme. Hij keurt dan elke
strategische prioriteitstelling af die hen dichter bij die doelen kan
brengen. Op deze manier voeden en versterken de twee tegengestelde
afwijkingen elkaar, wat destructief is voor de belangrijke taak om het
libertarische doel effectief te bereiken.

Hoe kunnen we dan bepalen of een halve maatregel of overgangseis moet
worden toegejuicht als een stap vooruit, of moet worden afgekeurd als
opportunistisch verraad? Er zijn twee belangrijke criteria om deze vraag
te beantwoorden: (1) het uiteindelijke doel van vrijheid moet altijd
hoog in het vaandel staan als het gewenste doel, ongeacht de
overgangseisen; en (2) geen enkele stap of middel mag ooit expliciet of
impliciet in tegenspraak zijn met dit uiteindelijke doel. Een eis op de
korte termijn gaat misschien niet zover als we zouden willen, maar moet
altijd in lijn zijn met het uiteindelijke doel. Als dat niet het geval
is, zal het kortetermijndoel tegen het langetermijndoel inwerken, en dan
is de opportunistische liquidatie van het libertarische principe een
feit.

Een voorbeeld van een contraproductieve en opportunistische strategie is
het belastingstelsel. De libertariër streeft naar de uiteindelijke
afschaffing van belastingen. Het is dan ook volkomen legitiem om als
strategische maatregel in die richting aan te dringen op een drastische
verlaging of zelfs afschaffing van de inkomstenbelasting. Echter, de
libertariër mag nooit een nieuwe belasting of belastingverhoging
steunen. Hij zou bijvoorbeeld niet moeten pleiten voor een grote
verlaging van de inkomstenbelasting en tegelijk voorstellen dat deze
vervangen wordt door een omzetbelasting of een andere vorm van
belasting. Het verlagen of, beter nog, afschaffen van een belasting
betekent altijd een legitime en niet-contradictore vermindering van de
staatsmacht, en vormt een belangrijke stap richting vrijheid. Maar het
vervangen van een belasting door een nieuwe of verhoogde belasting op
een ander terrein doet precies het tegenovergestelde. Dit betekent
immers een nieuwe en extra belastingdruk van de staat op een ander
front. Het opleggen van een nieuwe of hogere belasting staat volledig op
gespannen voet met het libertarische doel en ondermijnt dit zelfs.

In dit tijdperk van voortdurende federale tekorten worden we vaak
geconfronteerd met de vraag: moeten we instemmen met een
belastingverlaging, ook al kan dit leiden tot een groter
overheidstekort? Conservatieven, die de voorkeur geven aan een
evenwichtige begroting boven belastingverlaging, verzetten zich steevast
tegen elke belastingverlaging die niet meteen vergezeld gaat van een
gelijkwaardige of grotere verlaging van de overheidsuitgaven. Echter,
aangezien belastingheffing een onwettige daad van agressie is,
ondermijnt het algehele gebrek aan enthousiasme voor
belastingverlagingen -- welke vorm dan ook -- het libertarische doel.
Het is belangrijk om je tegen overheidsuitgaven te verzetten wanneer de
begroting wordt besproken of in stemming wordt gebracht. In dat geval
zou de libertariër ook moeten pleiten voor drastische bezuinigingen.
Kortom, overheidsactiviteiten moeten zoveel mogelijk worden
teruggedrongen. Elke vorm van verzet tegen een specifieke verlaging van
belastingen of uitgaven is ontoelaatbaar, omdat dit in strijd is met de
libertarische principes en het libertarische doel.

Een bijzonder gevaarlijke verleiding voor opportunisme is de neiging van
sommige libertariërs, vooral binnen de Libertarische Partij, om
`verantwoordelijk' en `realistisch' over te komen door een soort
`vierjarenplan' voor destatisering voor te stellen. Het belangrijkste
punt hierbij is niet het aantal jaren dat het plan beslaat, maar het
concept van een volledig en gepland programma voor de overgang naar
totale vrijheid. Bijvoorbeeld: in jaar 1 zou wet A moeten worden
ingetrokken, wet B gewijzigd, en belasting C met 10 procent verlaagd. In
jaar 2 zou wet D dan moeten worden ingetrokken en belasting C nogmaals
met 10 procent verlaagd, enzovoort. Het grote probleem met zo'n plan, en
de ernstige contradictie met het libertarische principe, is dat het
sterk impliceert dat wet D pas in het tweede jaar van het geplande
programma zou moeten worden ingetrokken. De valkuil van gradualisme in
theorie zou zo op grote schaal worden toegepast. De zogenaamde
libertarische planners zouden in een positie komen waarin het lijkt
alsof ze zich verzetten tegen een sneller tempo naar vrijheid dan hun
plan voorschrijft. En er is inderdaad geen legitieme reden om voor een
langzamer tempo te kiezen; het tegenovergestelde is zelfs waar.

Er is nog een andere ernstige tekortkoming aan het idee van een
alomvattend gepland programma voor vrijheid. De zorgvuldigheid en het
doordachte tempo, evenals de uitgebreide opzet van het programma,
suggereren dat de staat niet echt de vijand van de mensheid is. Daarmee
lijkt het mogelijk en zelfs wenselijk om de staat te gebruiken voor een
gecontroleerd pad naar vrijheid. Het inzicht dat de staat de grootste
vijand van de mensheid is, leidt echter tot een heel andere strategische
benadering. Libertariërs zouden elke vermindering van de macht of
activiteiten van de staat op elk vlak met enthousiasme moeten
aanmoedigen en accepteren. Zo'n vermindering zou op elk moment een
welkome afname van misdaad en agressie met zich meebrengen. Daarom zou
de zorg van de libertariër niet moeten zijn om de staat te gebruiken
voor een weloverwogen destabilisatie, maar om alle uitingen van statisme
waar en wanneer mogelijk weg te halen.

In lijn met deze analyse nam het Nationaal Comité van de Libertarische
Partij in oktober 1977 een strategieverklaring aan, die het volgende
inhield:

We moeten de vaandel van puur principe hooghouden en ons doel nooit in
gevaar brengen. De morele imperatief van het libertarische principe eist
dat tirannie, onrecht, de afwezigheid van volledige vrijheid en
schending van rechten niet langer geaccepteerd kunnen worden.

Elke intermediaire eis moet worden behandeld, zoals in het platform van
de Libertarische Partij, en moet worden gezien als een tussenstap naar
ons ultieme doel, dat als superieur wordt beschouwd. Daarom dient elke
eis gepresenteerd te worden als een stap richting ons einddoel, en niet
als een doel op zich.

Onze principes hooghouden betekent dat we het moeras van zelfopgelegd,
verplicht geleidelijkheid volledig moeten vermijden. We moeten voorkomen
dat we, uit naam van eerlijkheid, het verminderen van lijden of het
voldoen aan verwachtingen, uitstelgedrag vertonen op de weg naar
vrijheid. Het bereiken van vrijheid moet ons allesomvattende doel zijn.

We mogen ons niet vastleggen op een specifieke volgorde van
destatisering, omdat dit kan worden opgevat als het goedkeuren van het
voortzetten van statisme en schending van rechten. Omdat we nooit in de
positie willen zijn dat we de voortzetting van tirannie steunen, moeten
we alle destatiseringsmaatregelen accepteren, waar en wanneer dat
mogelijk is.

De libertariër mag zich dus nooit laten verleiden tot voorstellen voor
`positieve' overheidsmaatregelen. Vanuit zijn perspectief zou de rol van
de overheid alleen moeten zijn om zich zo snel mogelijk uit alle
maatschappelijke domeinen terug te trekken, zodra ze daaronder wordt
gebracht.

Er mogen geen tegenstrijdigheden in de retoriek zijn. Een libertariër
moet geen argumenten of beleidsaanbevelingen formuleren die in strijd
zijn met het uiteindelijke doel. Stel je voor dat een libertariër om
zijn mening over een specifieke belastingverlaging wordt gevraagd. Zelfs
als hij denkt dat hij op dit moment niet krachtig kan pleiten voor
belastingafschaffing, moet hij absoluut vermijden om aan zijn steun voor
de belastingverlaging onoprechte opmerkingen toe te voegen, zoals:
`Natuurlijk is enige belasting essentieel.' Dergelijke retorische trucs
kunnen het publiek in verwarring brengen en de principes ondermijnen.
Dit kan uiteindelijk alleen maar schadelijk zijn voor het doel dat we
nastreven.

\section{IS ONDERWIJS GENOEG?}\label{is-onderwijs-genoeg}

Alle libertariërs, ongeacht hun factie of overtuiging, hechten veel
waarde aan onderwijs. Ze willen steeds meer mensen overtuigen om
libertariër te worden, bij voorkeur met een sterke toewijding. Het
probleem is echter dat de meeste libertariërs een erg simplistische kijk
hebben op de rol en de reikwijdte van dat onderwijs. Ze doen, kortom,
niet eens een poging om de vraag te beantwoorden: en daarna? Wat gebeurt
er nadat een bepaald aantal mensen overtuigd is? En hoeveel mensen
moeten er overtuigd zijn om naar de volgende fase over te gaan?
Iedereen? Een meerderheid? Veel mensen?

De impliciete opvatting van veel libertariërs lijkt te zijn dat alleen
onderwijs voldoende is, omdat iedereen een gelijkwaardige kans heeft om
overtuigd te worden. Op zich klopt dat, maar sociologisch gezien is het
een zwakke strategie. Juist libertariërs zouden moeten erkennen dat de
staat een parasitaire vijand van de samenleving is. De staat creëert een
elite die de rest van ons onderdrukt en hun inkomen door dwang verwerft.
Het overtuigen van de heersende groepen van hun onrechtvaardigheid is,
hoewel theoretisch mogelijk (en misschien in één of twee gevallen
haalbaar), in de praktijk vrijwel onwerkelijk. Hoe groot is bijvoorbeeld
de kans dat we de leidinggevenden van General Dynamics of Lockheed ervan
kunnen overtuigen geen overheidssubsidies aan te nemen? En hoe
waarschijnlijk is het dat de president van de Verenigde Staten dit boek
of een ander stuk libertarische literatuur leest en dan zegt: `Ze hebben
gelijk. Ik heb ongelijk. Ik neem ontslag.'? Het is duidelijk dat de kans
om diegenen te overtuigen die profiteren van de uitbuiting door de
staat, op zijn minst verwaarloosbaar is. Onze hoop ligt bij het
overtuigen van de massa mensen die lijden onder de staatsmacht, niet bij
degenen die er beter van worden.

Maar als we dit stellen, erkennen we ook dat het probleem niet alleen
ligt bij het onderwijs, maar ook bij de macht. Nadat een aanzienlijk
aantal mensen zich bekeerd heeft, is er een belangrijke bijkomende taak:
manieren en middelen vinden om de macht van de staat uit onze
samenleving te verwijderen. De staat zal zich immers niet vrijwillig
terugtrekken uit de macht. Daarom moeten wij, naast onderwijs, ook
andere middelen inzetten. Welk middel of welke combinatie van middelen
we gebruiken --- of dat nu stemmen, alternatieve instellingen die niet
door de staat worden aangetast, of het massaal weigeren om met de staat
samen te werken is --- hangt af van de omstandigheden van dat moment en
wat wel of niet effectief blijkt te zijn. In tegenstelling tot
theoretische vraagstukken zijn de specifieke tactieken die we moeten
toepassen, zolang ze in lijn zijn met de principes en het uiteindelijke
doel van een echt vrije samenleving, een kwestie van pragmatisme,
beoordelingsvermogen en de soms onnauwkeurige `kunst' van de tacticus.

\section{WELKE GROEPEN?}\label{welke-groepen}

Maar onderwijs vormt het huidige strategische probleem voor de nabije en
onbepaalde toekomst. Een belangrijke vraag is: wie zijn de meest
waarschijnlijke kandidaten voor bekering als we er niet op kunnen
rekenen onze heersers in grote aantallen te overtuigen? Welke sociale,
beroepsmatige, economische of etnische klassen komen hiervoor in
aanmerking?

Conservatieven hebben vaak hun hoop gevestigd op grote zakenmensen. Deze
opvatting over grote bedrijven kwam het meest tot uiting in Ayn Rand's
uitspraak: `Big Business is Amerika's meest vervolgde minderheid.'
Vervolgd? Met enkele eervolle uitzonderingen daargelaten, staan grote
bedrijven gretig in de rij voor staatssteun. Voelt Lockheed, General
Dynamics, AT\&T of Nelson Rockefeller zich echt vervolgd?

De steun van grote bedrijven voor de welzijns- en oorlogsstaat is zo
overduidelijk en ingrijpend, van lokaal tot federaal niveau, dat zelfs
veel conservatieven het nu moeten erkennen, al is het maar tot op zekere
hoogte. Hoe is het mogelijk dat ze zo'n fervente steun geven aan
`Amerika's meest vervolgde minderheid'? Voor conservatieven rest dan
alleen de optie om aan te nemen dat (a) deze zakenmensen dom zijn en
geen inzicht hebben in hun eigen economische belangen, en/of (b) dat ze
gehersenspoeld zijn door links-liberale intellectuelen, die hen hebben
vervuld met schuldgevoelens en misplaatst altruïsme. Toch zullen beide
verklaringen niet opgaan, zoals slechts een blik op AT\&T of Lockheed
zal aantonen. Grote zakenmensen zijn vaak bewonderaars van het statisme,
`bedrijfsliberalen', niet omdat hun ziel vergiftigd is door
intellectuelen, maar omdat ze er persoonlijk voordeel uit halen. Sinds
de opkomst van het statisme aan het begin van de twintigste eeuw hebben
zij de grote macht van staatscontracten, subsidies en kartelvorming
aangegrepen om privileges te verwerven ten koste van de rest van de
samenleving. Het is niet vergezocht om aan te nemen dat Nelson
Rockefeller veel meer door eigenbelang wordt gedreven dan door nobel
altruïsme. Zelfs liberalen erkennen dat het enorme netwerk van
regelgevende overheidsinstanties ervoor zorgt dat elke industrie ten
gunste van grote bedrijven wordt gekarteliseerd, ten koste van het
publiek. Maar om hun New Deal-wereldbeeld te redden, troosten liberalen
zichzelf met de gedachte dat deze agentschappen en andere
`hervormingen', die in de tijd van de Progressieve, Wilson- of
Rooseveltiaanse periodes zijn ingevoerd, met de beste bedoelingen zijn
gelanceerd, met het `publieke welzijn' voor ogen. Ze geloven dat het
idee en de ontstaansgeschiedenis van deze agentschappen oprecht waren;
alleen in de praktijk zijn ze afgedwaald naar zonde en onderdanigheid
aan particuliere, zakelijke belangen. Echter, wat Kolko, Weinstein,
Domhoff en andere revisionistische historici duidelijk hebben
aangetoond, is dat dit deel uitmaakt van een liberale mythe. In
werkelijkheid zijn al deze hervormingen, zowel op nationaal als lokaal
niveau, bedacht, geschreven en gelobbyd door de zeer bevoorrechte
groepen zelf. Het werk van deze historici onthult onomstotelijk dat er
geen Gouden Eeuw van Hervorming was voordat de zonde toesloeg; de zonde
was er vanaf het begin, vanaf het moment van conceptie. De liberale
hervormingen van de Progressieve-New Deal-Welzijnsstaat waren ontworpen
om te creëren wat ze ook daadwerkelijk hebben gecreëerd: een wereld van
gecentraliseerd statisme, van `partnerschap' tussen overheid en
industrie, een wereld die bestaat uit subsidies en monopolie-privileges
voor bedrijven en andere bevoorrechte groepen.

Verwachten dat de Rockefellers of een legioen andere bevoordeelde
zakenlieden zich zullen bekeren tot een libertarische of zelfs
laissez-faire visie is ijdel. Dit betekent echter niet dat alle grote
ondernemers of zakenlieden moeten worden afgeschreven. In tegenstelling
tot de marxisten vormen niet alle zakenlieden, en zelfs niet alle grote
zakenlieden, een homogene economische klasse met identieke belangen.
Wanneer de CAB monopolieprivileges verleent aan enkele grote
luchtvaartmaatschappijen of wanneer de FCC een monopolie toekent aan
AT\&T, zijn er talloze andere bedrijven en ondernemers, groot en klein,
die hierdoor benadeeld worden. Het toekennen van een
communicatiemonopolie aan AT\&T door de FCC heeft er bijvoorbeeld lange
tijd voor gezorgd dat de snelgroeiende datacommunicatie-industrie in de
kinderschoenen bleef staan. Pas toen de FCC besloot om concurrentie toe
te staan, kon de industrie met sprongen vooruitgaan. Privilegies
impliceren uitsluiting, en daardoor zullen er altijd veel bedrijven en
zakenlieden zijn, groot en klein, die een sterk economisch belang hebben
bij het beëindigen van staatscontrole over hun sector. Er zijn dus veel
ondernemers, vooral degene die ver verwijderd zijn van de bevoorrechte
`Oosterse gevestigde orde', die potentieel ontvankelijk zijn voor
vrijemarkt- en libertarische ideeën.

Van welke groepen kunnen we dan verwachten dat ze bijzonder ontvankelijk
zijn voor libertarische ideeën? Waar, zoals de marxisten zouden zeggen,
is ons voorgestelde `agentschap voor sociale verandering'? Dit is een
belangrijke strategische vraag voor libertariërs, omdat het ons inzicht
geeft in waar we onze educatieve inspanningen op moeten richten.

Campusjongeren spelen een belangrijke rol in de opkomende libertarische
beweging. Dit is niet verrassend. De universiteit is een periode waarin
mensen openstaan voor reflectie en het overdenken van fundamentele
vragen over onze samenleving. Deze jongeren verlangen naar consistentie
en eerlijke waarheden. Als studenten, die zijn opgegroeid in een wereld
van wetenschap en abstracte ideeën, en die nog niet belast zijn met de
verantwoordelijkheden en soms beperkte perspectieven van volwassenheid,
vormen zij een vruchtbare basis voor libertarische overtuigingen. In de
toekomst kunnen we een aanzienlijke groei van het libertarisme op
nationale campussen verwachten. Deze groei gaat gepaard met de
toenemende steun van een groeiend aantal jonge geleerden, professoren en
afgestudeerde studenten.

Jongeren zouden zich ook aangetrokken moeten voelen tot het
libertarische standpunt over onderwerpen die vaak het meest aansluiten
bij hun zorgen. Dit betreft in het bijzonder onze oproep tot volledige
afschaffing van de dienstplicht, terugtrekking uit de Koude Oorlog,
burgerlijke vrijheden voor iedereen en de legalisering van drugs en
andere slachtoffervrije overtredingen.

Ook de media zijn een rijke bron gebleken van positieve aandacht voor
het nieuwe libertarische credo. Dit is niet alleen te danken aan de
publiciteitswaarde, maar ook aan de consistentie van het libertarisme.
Dit trekt een groep mensen aan die scherp is op nieuwe sociale en
politieke trends. Hoewel zij van oorsprong liberaal zijn, merken ze
steeds meer de mislukkingen en ineenstortingen van het gevestigde
liberalisme op. Mediamensen voelen zich over het algemeen niet
aangetrokken tot een vijandige conservatieve beweging die hen
automatisch afschrijft als links en onwelkom is tegenover hun
standpunten over buitenlands beleid en burgerlijke vrijheden. Maar
dezezelfde mediamensen kunnen wel openstaan voor een libertarische
beweging. Deze beweging sluit aan bij hun instincten voor vrede en
persoonlijke vrijheid en koppelt hun verzet tegen de grote overheid aan
de overheidsbemoeienis met de economie en eigendomsrechten. Een groeiend
aantal mensen in de media legt deze nieuwe en verhelderende verbanden.
Hun invloed is natuurlijk uiterst belangrijk en heeft een krachtige
hefboomwerking op de rest van het publiek.

Hoe staat het met `Midden-Amerika'? Dit is de enorme middenklasse en
arbeidersklasse die het grootste deel van de Amerikaanse bevolking
uitmaakt en vaak lijnrecht tegenover de campusjeugd staat. Hebben we
enige aantrekkingskracht op hen? Logischerwijs zou onze
aantrekkingskracht op Midden-Amerika nog groter moeten zijn. We richten
ons namelijk vol op de groeiende en aanhoudende ontevredenheid die de
massa van het Amerikaanse volk teistert: stijgende belastingen,
inflatie, fileproblemen, criminaliteit en welvaartsschandalen. Alleen
libertariërs beschikken over concrete en consistente oplossingen voor
deze dringende kwesties. Onze oplossingen richten zich op het
verminderen van de rol van de overheid in al deze gebieden door deze
verantwoordelijkheden over te dragen aan particuliere en vrijwillige
initiatieven. We kunnen aantonen dat de overheid en het statisme
verantwoordelijk zijn voor deze problemen, en dat het wegnemen van
dwingende overheidsmaatregelen de nodige oplossingen biedt.

Kleine ondernemers kunnen we een echte vrije ondernemingswereld beloven,
vrij van monopolieprivileges, kartels en staatsubsidies van de
gevestigde orde. Ook aan de grote zakenmensen buiten de monopolistische
gevestigde orde kunnen we een wereld beloven waarin hun individuele
talenten en inzet eindelijk volop de ruimte krijgen om te groeien. Dit
zal leiden tot verbeterde technologie en hogere productiviteit, voor
henzelf en voor ons allemaal. Daarnaast kunnen we verschillende etnische
en minderheidsgroepen laten zien dat alleen onder vrijheid elke groep in
staat is om zijn eigen belangen te cultiveren en eigen instellingen te
runnen, zonder hinder van meerderheidsoverheersing en zonder dwang.

Kortom, de aantrekkingskracht van het libertarisme spreekt verschillende
klassen aan. Het is een aantrekkingskracht die de grenzen van ras,
beroep, economische klasse en generaties overschrijdt. Alle mensen die
niet tot de heersende elite behoren, zijn mogelijk ontvankelijk voor
onze boodschap. Iedereen of elke groep die waarde hecht aan vrijheid of
welvaart kan een potentiële aanhanger van het libertarische credo zijn.

Vrijheid heeft de potentie om alle groepen in het publieke spectrum aan
te spreken. Toch geldt dat wanneer alles goed gaat, de meeste mensen
weinig interesse tonen in publieke zaken. Voor radicale sociale
verandering - een verschuiving naar een ander sociaal systeem - is er
vaak een zogenaamde `crisissituatie' nodig. Dit betekent dat er een
breuk in het bestaande systeem moet zijn die een algemene zoektocht naar
alternatieve oplossingen oproept. Wanneer zo'n zoektocht naar sociale
alternatieven op gang komt, moeten activisten van een andersdenkende
beweging klaarstaan om dat radicale alternatief te bieden. Ze moeten de
crisis in verband brengen met de inherente tekortkomingen van het
systeem zelf en laten zien hoe het alternatieve systeem de bestaande
crisis kan oplossen, en toekomstige ineenstortingen kan voorkomen.
Hopelijk hebben de andersdenkenden ook een staat van dienst waarin ze de
huidige crisis hebben voorspeld en ervoor hebben gewaarschuwd.

Bovendien is een kenmerk van crisissituaties dat zelfs de heersende
elites hun steun voor het systeem beginnen te verzwakken. Door de crisis
verliest zelfs een deel van de staat zijn motivatie en enthousiasme om
te regeren. Kortom, belangrijke segmenten binnen de staat falen. In
situaties van ineenstorting kunnen zelfs leden van de heersende elite
zich zonder meer openstellen voor een alternatief systeem, of op zijn
minst hun enthousiasme voor het bestaande systeem verliezen.

Zo benadrukt historicus Lawrence Stone dat een verval in de wil van de
heersende elite een vereiste is voor radicale verandering. 'De elite kan
haar manipulatieve vaardigheden verliezen, of haar militaire
superioriteit, of haar zelfvertrouwen of cohesie. Ze kan vervreemd raken
van de niet-elite of overweldigd worden door een financiële crisis. Ook
kan ze incompetent, zwak of wreed zijn.'5

\section{\texorpdfstring{\textbf{WAAROM VRIJHEID ZAL
WINNEN}}{WAAROM VRIJHEID ZAL WINNEN}}\label{waarom-vrijheid-zal-winnen}

Vrijheid heeft de potentie om alle groepen in het publieke spectrum aan
te spreken. Toch blijkt dat wanneer alles goed gaat, de meeste mensen
weinig interesse tonen in publieke zaken. Voor radicale sociale
verandering -- een verschuiving naar een ander sociaal systeem -- is
vaak een zogenaamde `crisissituatie' nodig. Dit betekent dat er een
breuk in het bestaande systeem moet zijn, die een algemene zoektocht
naar alternatieve oplossingen oproept. Wanneer zo'n zoektocht naar
sociale alternatieven op gang komt, moeten activisten van een
andersdenkende beweging klaarstaan om dat radicale alternatief te
bieden. Ze moeten de crisis koppelen aan de inherente tekortkomingen van
het systeem zelf. Daarnaast moeten ze laten zien hoe het alternatieve
systeem niet alleen de huidige crisis kan oplossen, maar ook
soortgelijke ineenstortingen in de toekomst kan voorkomen. Hopelijk
hebben deze andersdenkenden ook een staat van dienst waarin ze de
huidige crisis hebben voorspeld en ervoor hebben gewaarschuwd. Een
belangrijk kenmerk van crisissituaties is dat zelfs de heersende elites
hun steun voor het systeem beginnen te verzwakken. Door de crisis
verliest zelfs een deel van de staat zijn motivatie en enthousiasme om
te regeren. Kortom, belangrijke segmenten binnen de staat falen. In
situaties van ineenstorting kunnen zelfs leden van de heersende elite
zich openstellen voor een alternatief systeem, of op zijn minst hun
enthousiasme voor het bestaande systeem verliezen. Historicicus Lawrence
Stone benadrukt dat een verval in de wil van de heersende elite cruciaal
is voor radicale verandering. 'De elite kan haar manipulatieve
vaardigheden verliezen, of haar militaire superioriteit, of haar
zelfvertrouwen en cohesie. Ze kan vervreemd raken van de niet-elite,
overweldigd worden door een financiële crisis, of incompetent, zwak of
wreed zijn.'5

Na het uiteenzetten van het libertarische credo en de toepassing ervan
op belangrijke hedendaagse problemen, evenals het schetsen van welke
groepen in de samenleving zich naar verwachting tot dit credo zullen
aangetrokken voelen en op welke momenten, is het tijd om de
toekomstperspectieven voor vrijheid te beoordelen. We moeten in het
bijzonder de sterke en groeiende overtuiging van de auteur bespreken.
Niet alleen dat het libertarisme uiteindelijk op de lange termijn zal
zegevieren, maar ook dat het binnen een opmerkelijk korte tijd als
winnaar uit de bus zal komen. Ik ben ervan overtuigd dat de donkere
nacht van tirannie ten einde loopt en dat een nieuwe dageraad van
vrijheid voor de deur staat.

Veel libertariërs zijn pessimistisch over de vooruitzichten voor
vrijheid. Als we kijken naar de groei van het statisme in de twintigste
eeuw en het verval van het klassieke liberalisme, zoals we in het
inleidende hoofdstuk hebben beschreven, is het begrijpelijk om een
pessimistische kijk te hebben. Dit pessimisme kan verergeren als we
terugblikken op de mensheid en de donkere geschiedenis van despotisme,
tirannie en uitbuiting in verschillende beschavingen zien. We zouden
onszelf kunnen verontschuldigen als we dachten dat de klassiek-liberale
opleving van de zeventiende tot en met de negentiende eeuw in het Westen
slechts een atypische uitbarsting van glorie was in de lugubere
kronieken van de geschiedenis, zowel het verleden als de toekomst.
Echter, dit zou getuigen van een misvatting die door marxisten
`impressionisme' wordt genoemd: een oppervlakkige focus op historische
gebeurtenissen zonder een diepere analyse van de causale wetten en
trends die daarachter werken.

Het pleidooi voor libertarisch optimisme kan worden gepresenteerd in wat
je concentrische cirkels zou kunnen noemen. We beginnen met de breedste
en langste-termijn overwegingen en gaan vervolgens naar de scherpere
focus op kortetermijntrends. In de breedste zin en op de lange termijn
zal het libertarisme uiteindelijk zegevieren. Dit komt omdat alleen het
libertarisme verenigbaar is met de aard van de mens en de wereld. Alleen
vrijheid kan leiden tot welvaart, vervulling en geluk voor de mens.
Kortom, het libertarisme zal winnen omdat het waar is. Het biedt het
juiste beleid voor de mensheid, en uiteindelijk zal de waarheid altijd
overwinnen.

Maar zulke lange-termijnoverwegingen kunnen inderdaad erg uitgebreid
zijn. Vele eeuwen wachten tot de waarheid zegeviert, en dat kan een
schrale troost zijn voor degenen die op een bepaald moment in de
geschiedenis leven. Gelukkig zijn er redenen voor hoop op kortere
termijn. In het bijzonder is er een reden die ons in staat stelt om het
sombere beeld van de geschiedenis vóór de achttiende eeuw als niet
langer relevant voor de toekomstige vooruitzichten van vrijheid te
beschouwen.

Onze stelling is dat de geschiedenis een grote sprong heeft gemaakt, een
keerpunt, toen de klassiek-liberale revoluties ons naar de industriële
revolutie van de achttiende en negentiende eeuw leidden. In de
pre-industriële wereld, de wereld van de Oude Orde en de boereneconomie,
was er geen reden waarom het despotisme niet eindeloos zou kunnen
voortduren, zelfs gedurende vele eeuwen. De boeren produceerden het
voedsel, terwijl koningen, edelen en feodale landheren het overschot van
de boeren inzamelden, boven wat nodig was om hen in leven en aan het
werk te houden. Hoe wreed en uitbuitend het agrarisch despotisme ook
was, het kon voortbestaan om twee belangrijke redenen: (1) de economie
kon zelfs op een ondermaats niveau makkelijk in stand worden gehouden;
en (2) de massa's wisten niet beter, hadden nooit een beter systeem
meegemaakt en konden daardoor worden overgehaald om als lastdieren voor
hun heren te blijven werken.

Maar de Industriële Revolutie was een beslissende sprong in de
geschiedenis. Ze bracht omstandigheden en verwachtingen met zich mee die
onomkeerbaar waren. Voor het eerst in de wereldgeschiedenis leidde de
Industriële Revolutie tot een maatschappij waarin de levensstandaard van
de massa's steeg van het bestaansminimum naar ongekende hoogten. De
bevolking in het Westen, die eerder stagneerde, begon nu te groeien en
te profiteren van de sterk toegenomen kansen op werk en een beter leven.

De klok kan niet worden teruggedraaid naar een pre-industriële tijdperk.
Niet alleen zouden de massa's zo'n ingrijpende verandering in hun
verwachtingen van een stijgende levensstandaard niet accepteren, maar
een terugkeer naar een agrarische wereld zou ook leiden tot hongerdood
en de dood van het grootste deel van de huidige bevolking. We zijn
gevangen in het industriële tijdperk, of we dat nu leuk vinden of niet.

Maar als dat zo is, dan is de zaak van de vrijheid veiliggesteld. De
economische wetenschap heeft aangetoond, zoals we in dit boek
gedeeltelijk hebben laten zien, dat alleen vrijheid en een vrije markt
in staat zijn om een industriële economie te laten draaien. Kortom,
terwijl een vrije economie en een vrije maatschappij wenselijk en
rechtvaardig zouden zijn in een pre-industriële wereld, zijn ze in een
industriële wereld ook van essentieel belang. Zoals Ludwig von Mises en
andere economen hebben aangetoond, werkt statisme simpelweg niet in een
industriële economie. Daarom zal het, met een universele toewijding aan
een industriële wereld, uiteindelijk - en veel sneller dan alleen door
de opkomst van de waarheid - duidelijk worden dat de wereld vrijheid en
de vrije markt moet omarmen als voorwaarden voor het overleven en
bloeien van de industrie. Dit inzicht werd door Herbert Spencer en
andere negentiende-eeuwse libertariërs erkend in hun onderscheid tussen
de `militaire' en de `industriële' samenleving, tussen een samenleving
van `status' en een samenleving van `contract'. In de twintigste eeuw
toonde Mises aan dat (a) alle staatsinterventie de markt verstoort en
verlamt, en als dit niet wordt teruggedraaid, leidt het tot socialisme;
en (b) dat socialisme een rampspoed is omdat het geen industriële
economie kan plannen zonder winst- en verliesprikkels, en zonder een
echt prijssysteem of eigendomsrechten op kapitaal, land en andere
productiemiddelen. Kortom, zoals Mises voorzag, kunnen noch socialisme,
noch de verschillende tussenvormen van statisme en interventionisme
functioneren. Daarom moeten, gezien de algemene verschuiving naar een
industriële economie, deze vormen van statisme worden afgewezen en
vervangen door vrijheid en vrije markten.

Dit was een veel kortere aanloop dan simpelweg wachten op de waarheid.
Toch leek het voor de klassieke liberalen rond de eeuwwisseling van de
twintigste eeuw - de Sumners, Spencers en Paretos - een ondraaglijk
lange weg. Dat is hen niet kwalijk te nemen, want zij zagen het verval
van het klassieke liberalisme en de opkomst van nieuwe, despotische
vormen waartegen ze zich zo krachtig en vasthoudend verzetten. Helaas
waren zij getuige van deze ontwikkeling. De wereld zou moeten wachten,
zo niet eeuwen, dan toch minstens decennia, voordat socialisme en
corporatief statisme ontmaskerd zouden worden als volslagen
mislukkingen.

Maar de lange termijn is nu aangebroken. We hoeven de verwoestende
effecten van het statisme niet langer te voorspellen; ze zijn al
zichtbaar. Lord Keynes lachte ooit de kritiek van vrijemarkteconomen weg
dat zijn inflatoire beleid op de lange termijn desastreus zou zijn. In
zijn bekende antwoord gniffelde hij dat `we op de lange termijn allemaal
dood zijn'. Maar nu is Keynes dood, en leven wij in zijn lange termijn.
De statistische kippen zijn thuisgekomen om te scharrelen.

Aan het begin van de twintigste eeuw, en decennia daarna, waren de zaken
lang niet zo duidelijk. Statistische interventie, in al zijn vormen,
probeerde een centraal geleide economie in stand te houden en zelfs uit
te breiden. Dit gebeurde echter ten koste van de voorwaarden voor een
vrij leven en een vrije markt, die op de lange termijn essentieel zijn
voor overleving. Gedurende een halve eeuw kon statistisch ingrijpen zijn
ondermijnende acties uitvoeren door middel van planning, controles, hoge
en verlammende belastingen, en inflatie van papiergeld, zonder dat dit
direct leidde tot duidelijke crises of ontwrichtingen. De
industrialisatie van de vrije markt in de negentiende eeuw had een grote
buffer van `vet' in de economie opgebouwd, die deze verstoringen kon
opvangen. De overheid kon het systeem belastingen, beperkingen en
inflatie opleggen zonder daar snel of duidelijk negatieve gevolgen van
te ondervinden.

Maar nu is het statisme zo ver gevorderd en is het al zo lang aan de
macht dat de buffer dun is. Zoals Mises al in de jaren veertig aangaf,
is het `reservefonds' dat door laissez-faire was gecreëerd, `uitgeput'.
Alles wat de overheid nu doet, leidt onmiddellijk tot negatieve
neveneffecten die voor iedereen duidelijk zijn, zelfs voor veel van de
meest fervente verdedigers van het statisme.

In de communistische landen van Oost-Europa hebben de communisten steeds
meer ingezien dat socialistische centrale planning niet werkt voor een
industriële economie. Dit heeft geleid tot een snelle terugtrekking van
centrale planning naar vrije markten, vooral in Joegoslavië. Ook in de
Westerse wereld verkeert het staatskapitalisme in een crisis. Het wordt
steeds duidelijker dat de overheid geen geld meer heeft. Het verhogen
van de belastingen zal de industrie en stimuleringsmaatregelen
onherstelbaar verlammen. Aan de andere kant zal het creëren van meer
nieuw geld leiden tot rampzalige, ongecontroleerde inflatie. Daarom
horen we steeds vaker de `noodzaak van lagere verwachtingen van de
overheid' terug van eens zo enthousiaste voorstanders van de staat. In
West-Duitsland heeft de sociaal-democratische partij de roep om
socialisme al lang laten varen. In Groot-Brittannië, waar de economie
gebukt gaat onder hoge belastingen en toenemende inflatie---een situatie
die de Britten zelfs de `Engelse ziekte' noemen---is de Tory-partij,
jarenlang in handen van toegewijde statisten, nu overgenomen door een
vrijemarktgerichte fractie. Zelfs de Labor-partij heeft zich
teruggetrokken uit de chaos van het op hol geslagen statisme.

In de Verenigde Staten kunnen we bijzonder optimistisch zijn, omdat we
de cirkel van optimisme kunnen verkleinen tot een kortetermijndimensie.
We kunnen met gerustgesteld zeggen dat de Verenigde Staten zich in een
permanente crisissituatie bevinden, en we kunnen zelfs de oorsprong van
die crisis aanwijzen: 1973-1975. Gelukkig voor de zaak van de vrijheid
is er niet alleen een crisis van het statisme in de Verenigde Staten
ontstaan, maar heeft deze zich ook breed in de samenleving verspreid, op
verschillende terreinen van het leven en bijna gelijktijdig. Deze
ineenstortingen van het statisme hebben een synergetisch effect gehad;
ze hebben elkaar versterkt in hun cumulatieve impact. Bovendien worden
deze crises door iedereen gezien als een gevolg van het statisme, en
niet van de vrije markt, publieke hebzucht of iets anders. Uiteindelijk
kunnen deze crises alleen worden verlicht door de overheid uit het spel
te halen. We hebben libertariërs nodig om de weg te wijzen.

Laten we de verschillende gebieden van systemische crisis op een rijtje
zetten en bekijken hoe veel daarvan samenhing in de periode van 1973 tot
1975 en daarna. Van de herfst van 1973 tot 1975 verkeerden de Verenigde
Staten in een inflatoire depressie, ondanks veertig jaar van vermeende
Keynesiaanse fine-tuning die beide problemen voor altijd zou moeten
oplossen. Ook in deze tijd bereikte de inflatie angstaanjagende
proporties met dubbele cijfers.

Bovendien beleefde New York City in 1975 zijn eerste grote
schuldencrisis, die leidde tot een gedeeltelijke wanbetaling. De
gevreesde term `wanbetaling' werd echter vermeden; men sprak eerder van
een `stretchout', waarbij kortetermijnschuldeisers gedwongen werden om
langetermijnobligaties van New York City te accepteren. Deze crisis was
de eerste van vele wanbetalingen op staats- en lokale obligaties in het
hele land. Staats- en lokale overheden komen steeds vaker voor
onaangename `crisis'-keuzes te staan: radicaal snijden in de uitgaven,
hogere belastingen die bedrijven en middenklasse-burgers uit het gebied
zullen verdrijven, of in gebreke blijven bij het aflossen van schulden.

Sinds het begin van de jaren zeventig is steeds duidelijker geworden dat
hoge belastingen op inkomen, spaargeld en investeringen de
bedrijfsactiviteit en productiviteit hebben stilgelegd. Accountants
beginnen nu pas te beseffen dat deze belastingen, vooral in combinatie
met inflatoire verstoringen in de zakelijke besluitvorming, hebben
geleid tot een groeiende schaarste aan kapitaal. Hierdoor dreigt de
vitale kapitaalvoorraad van Amerika geleidelijk uitgeput te raken,
zonder dat we dat opmerken.

Het land wordt gekweld door belastingoproeren tegen de hoge belastingen
op eigendom, inkomen en omzet. Het is dan ook veilig om te stellen dat
verdere belastingverhogingen politiek zelfmoord zouden betekenen voor
politici op elk bestuursniveau.

Het Sociale Zekerheidssysteem, dat ooit zo heilig was in de Amerikaanse
opinie dat het letterlijk boven kritiek stond, wordt nu gezien als
volledig in verval. Libertarische en vrije-markt schrijvers wijzen hier
al lange tijd op. Zelfs de gevestigde orde erkent inmiddels dat het
Sociale Zekerheidssysteem failliet is en dat het in geen enkel opzicht
een echt `verzekeringssysteem' is.

Regulering van de industrie wordt steeds meer beschouwd als een falen.
Zelfs figuren zoals senator Edward Kennedy pleiten inmiddels voor
deregulering van de luchtvaartmaatschappijen. Daarnaast is er veel
discussie over de afschaffing van de ICC en de CAB.

Op sociaal vlak komt het ooit zo heilige openbare schoolsysteem steeds
meer onder vuur te liggen. Openbare scholen, die onderwijsbeslissingen
voor de hele gemeenschap moeten nemen, veroorzaken intense sociale
conflicten over ras, geslacht, religie en onderwij inhoud. Ook de
overheidspraktijken rond criminaliteit en opsluiting staan steeds meer
onder druk. Libertarische denker Dr.~Thomas Szasz heeft bijna
alleenhandig veel mensen bevrijd van onvrijwillige opsluiting. Inmiddels
erkent de overheid dat haar beleid om criminelen te `rehabiliteren'
volledig is mislukt. De handhaving van drugswetten, zoals het verbod op
marihuana en wetten tegen verschillende vormen van seksuele relaties, is
compleet ingestort. In het hele land groeit de roep om alle wetten tegen
misdaden zonder slachtoffers af te schaffen. Dit zijn wetten die
misdaden aanduiden waarbij er geen slachtoffers zijn. Het wordt steeds
duidelijker dat de pogingen om deze wetten te handhaven enkel leiden tot
ontberingen en een soort politiestaat. De tijd nadert snel dat het
prohibitionisme op het gebied van persoonlijke moraliteit net zo
ineffectief en onrechtvaardig zal blijken te zijn als bij alcohol.

Samen met de desastreuze gevolgen van het statisme op economisch en
sociaal vlak volgde de traumatische nederlaag in Vietnam, die
culmineerde in 1975. De volslagen mislukking van de Amerikaanse
interventie in Vietnam heeft geleid tot een groeiende heroverweging van
het hele interventionistische buitenlandse beleid dat de Verenigde
Staten sinds Woodrow Wilson en Franklin D. Roosevelt voeren. Er heerst
steeds meer de opvatting dat de Amerikaanse macht moet worden
teruggedrongen en dat de regering de wereld niet succesvol kan besturen.
Dit is de `neoisolationistische' reactie op het terugdringen van de
interventies van de Grote Regering in eigen land. Hoewel het Amerikaanse
buitenlandse beleid nog steeds agressief globalistisch is, slaagde dit
neo-isolationistische sentiment er in 1976 in om de Amerikaanse
interventie in Angola te beperken.

Misschien wel het meest sprekende teken van de afbraak van de mystiek
rond de Amerikaanse staat, en van zijn morele basis, was de onthulling
van Watergate in 1973-1974. Watergate biedt ons de grootste hoop op een
korte termijn overwinning voor de vrijheid in Amerika. Politici wijzen
ons sindsdien op de vernietiging van het `vertrouwen in de overheid' bij
het publiek -- iets dat ook hoog tijd was. Watergate leidde tot een
radicale verandering in de diepgewortelde attitudes van iedereen,
ongeacht hun ideologie, ten opzichte van de overheid. Watergate maakte
ons in de eerste plaats wakker voor de invasies van persoonlijke
vrijheid en privébezit door de overheid: afluisteren, drugsgebruik, het
onderscheppen van post, provocateurs en zelfs moordaanslagen. Het
schandaal ontheiligde eindelijk onze voormalige iconen, de FBI en CIA,
en zorgde ervoor dat ze met een nuchtere blik werden bekeken. Maar
belangrijker nog, door de afzetting van de president te bewerkstelligen,
ontheiligde Watergate voorgoed een functie die door het Amerikaanse
publiek vrijwel als soeverein werd beschouwd. De president zal niet
langer boven de wet staan; hij zal niet langer zonder gevolgen kunnen
handelen.

Maar het allerbelangrijkste is dat de overheid in Amerika grotendeels
ontheiligd is. Niemand vertrouwt politici of de regering nog; alle
overheden worden met blijvende vijandigheid bekeken. Dit brengt ons
terug naar de staat van gezond wantrouwen jegens de regering die het
Amerikaanse publiek en de revolutionairen van de achttiende eeuw
kenmerkte.

Even leek het erop dat Jimmy Carter zijn doel om het vertrouwen van de
mensen in de overheid terug te brengen, zou kunnen realiseren. Maar door
het fiasco met Bert Lance en andere misstappen is hij daar gelukkig niet
in geslaagd. De voortdurende regeringscrisis gaat nog altijd door.

De omstandigheden in de Verenigde Staten zijn rijp voor de triomf van de
vrijheid, zowel nu als in de toekomst. Wat we nodig hebben, is een
groeiende en levendige libertarische beweging die deze systeemcrisis
duidt en ons door de overheid gecreëerde moeras de libertarische weg
wijst. Zoals we aan het begin van dit werk hebben gezien, is dat precies
wat er gaande is. En nu komen we eindelijk bij het beloofde antwoord op
de vraag die we in het inleidende hoofdstuk stelden: Waarom nu? Amerika
heeft een diepgewortelde erfenis van libertarische waarden, maar waarom
komen ze dan pas de laatste vier of vijf jaar naar voren?

Ons antwoord is dat het ontstaan en de snelle groei van de libertarische
beweging geen toeval zijn. Het is een gevolg van de crisissituatie die
Amerika tussen 1973 en 1975 trof en sindsdien aanhoudt. Crisissituaties
wekken altijd interesse en zetten mensen aan tot het zoeken naar
oplossingen. Deze crisis heeft veel nadenkende Amerikanen geïnspireerd
om te beseffen dat de overheid ons in deze puinhoop heeft gebracht.
Alleen vrijheid -- het terugdringen van de overheid -- kan ons eruit
helpen. We groeien omdat de omstandigheden gunstig zijn. In zekere zin
heeft, net als op de vrije markt, de vraag zijn eigen aanbod gecreëerd.

En daarom kreeg de Libertarische Partij in 1976, bij haar eerste poging
voor een nationaal ambt, 174.000 stemmen. Een recent nummer van The
Baron Report, een gezaghebbende nieuwsbrief over politiek in Washington
die bepaald niet libertarisch is, ontkende beweringen van de media over
een huidige trend richting conservatisme onder kiezers. In plaats
daarvan wijst het rapport erop dat `als er al een trend in de opinie
zichtbaar is, deze richting het libertarisme gaat -- de filosofie die
tegen overheidsbemoeienis is en pleit voor persoonlijke rechten.' Het
rapport voegt eraan toe dat libertarisme een aantrekkingskracht heeft
aan beide uiteinden van het politieke spectrum: `Conservatieven
verwelkomen deze trend omdat het aangeeft dat het publiek sceptisch is
over federale programma's; liberalen zien het als een groeiende
acceptatie van individuele rechten op gebieden zoals drugs, seksueel
gedrag, enzovoort, en als een toenemende terughoudendheid van het
publiek om buitenlandse interventie te steunen.'

\section{Naar een vrij Amerika}\label{naar-een-vrij-amerika}

De omstandigheden in de Verenigde Staten zijn zowel nu als in de
toekomst ideaal voor de triomf van de vrijheid. Wat we nodig hebben, is
een groeiende en levendige libertarische beweging die deze systeemcrisis
toelicht en ons door het door de overheid gecreëerde moeras naar de
libertarische weg leidt. Zoals we aan het begin van dit werk hebben
gezien, is dat precies wat er gebeurt. Nu komen we eindelijk bij het
beloofde antwoord op de vraag die we in het inleidende hoofdstuk
stelden: Waarom nu? Amerika heeft een diepgewortelde erfenis van
libertarische waarden, maar waarom komen deze pas de laatste vier of
vijf jaar naar voren? Het antwoord is dat het ontstaan en de snelle
groei van de libertarische beweging geen toeval zijn. Dit is een gevolg
van de crisissituatie die Amerika tussen 1973 en 1975 trof en die nog
steeds voortduurt. Crisissituaties wekken altijd belangstelling en
zetten mensen aan tot het zoeken naar oplossingen. Deze crisis heeft
veel nadenkende Amerikanen geïnspireerd om te beseffen dat de overheid
ons in deze puinhoop heeft gebracht. Alleen vrijheid -- het terugdringen
van de overheid -- kan ons eruit helpen. We groeien omdat de
omstandigheden gunstig zijn. In zekere zin heeft, net als op de vrije
markt, de vraag zijn eigen aanbod gecreëerd. Daarom kreeg de
Libertarische Partij in 1976, bij haar eerste poging voor een nationaal
ambt, 174.000 stemmen. Een recent nummer van The Baron Report, een
invloedrijke nieuwsbrief over politiek in Washington die allesbehalve
libertarisch is, ontkende de mediaberichten over een trend richting
conservatisme onder de kiezers. Het rapport wijst er namelijk op dat,
`als er al een trend in de opinie zichtbaar is, deze richting het
libertarisme gaat -- de filosofie die oproept tot minder
overheidsinterventie en meer persoonlijke rechten.' Tevens voegt het
rapport toe dat libertarisme aantrekkelijk is voor zowel conservatieven
als liberalen: `Conservatieven verwelkomen deze trend omdat het toont
dat het publiek sceptisch staat tegenover federale programma's;
liberalen zien het als een groeiende acceptatie van individuele rechten
op gebieden zoals drugs en seksueel gedrag, en een toenemende
terughoudendheid van het publiek om buitenlandse interventie te
steunen.'

Het libertarische credo belichaamt eindelijk het beste van het
Amerikaanse verleden, samen met de belofte van een veel betere toekomst.
Libertariërs staan, meer dan conservatieven, die vaak vastklampen aan
monarchale tradities uit een vervlogen Europees verleden, stevig in de
grote klassiek-liberale traditie. Deze traditie heeft de Verenigde
Staten gevormd en ons het erfgoed van individuele vrijheid, een
vreedzaam buitenlands beleid, minimale overheidsinterventie en een vrije
markteconomie gegeven. Libertariërs zijn de enige echte erfgenamen van
Jefferson, Paine, Jackson en de abolitionisten.

En toch, terwijl we meer traditioneel en geworteld Amerikaans zijn dan
de conservatieven, zijn we in sommige opzichten radicaler dan de
radicalen. Niet omdat we de wens of hoop hebben om de menselijke natuur
door politiek te herschikken, maar omdat alleen wij een echte, scherpe
en oprechte breuk aanbieden met het oprukkende statisme van de
twintigste eeuw. Oud Links vraagt enkel om meer van datgene waar we nu
onder lijden, terwijl Nieuw Links uiteindelijk slechts nog ernstiger
statisme of verplicht egalitarisme en uniformiteit voorstelt.
Libertarisme is de logische uitkomst van het inmiddels vergeten `Oud
Rechts' uit de jaren '30 en '40, dat zich verzette tegen de New Deal,
oorlog, centralisatie en staatsinterventie. Wij willen breken met alle
aspecten van de liberale staat: met zijn welvaart en oorlogvoering, zijn
monopolieprivileges en egalitarisme, alsook met de onderdrukking van
slachtofferloze misdaden, of deze nu persoonlijk of economisch van aard
zijn. Enkel wij bieden technologie zonder technocratie, groei zonder
vervuiling, vrijheid zonder chaos, wet zonder tirannie en de verdediging
van eigendomsrechten op zowel iemands persoon als zijn materiële
bezittingen.

Overblijfselen van libertarische doctrines zijn overal om ons heen te
vinden, zowel in delen van ons glorieuze verleden als in waarden en
ideeën van het verwarrende heden. Maar alleen het libertarisme slaagt
erin deze elementen te integreren tot een krachtig, logisch en
consistent systeem. Het enorme succes van Karl Marx en het marxisme komt
niet door de geldigheid van zijn ideeën -- die zijn immers allemaal
misleidend -- maar doordat hij het aandurfde om de socialistische
theorie in een krachtig systeem te verweven. Vrijheid kan niet slagen
zonder een gelijkwaardige en contrasterende theoretische basis. Tot de
laatste paar jaren hadden we, ondanks onze rijke erfenis van economisch
en politiek denken, geen volledig geïntegreerde en consistente theorie
van vrijheid. Die systematische theorie hebben we nu wel. We zijn
volledig gewapend met kennis en klaar om onze boodschap te verspreiden
en de verbeelding te prikkelen van alle groepen in de samenleving.
Andere theorieën en systemen hebben duidelijk gefaald: het socialisme is
overal op zijn retour, vooral in Oost-Europa. Het liberalisme heeft ons
vastgezet in een scala aan onoplosbare problemen. Het conservatisme
biedt niets meer dan een steriele verdediging van de status quo.
Vrijheid is in de moderne wereld nog nooit volledig uitgetest.
Libertariërs stellen nu voor om de Amerikaanse droom en de wereldwijde
droom van vrijheid en welvaart voor iedereen waar te maken.


\backmatter


\end{document}
